Introduced in the early 90s by Boser, Guyon and Vapnik \cite{BGV92},
\emph{Support Vector Machines} (SVMs) are a class of machine learning
algorithms deeply rooted in Statistical Learning Theory
\cite{v-edbed-82}, able to classify data taken from an unknown
probability distribution, given a set of training examples. As opposed
to analogous methods such as, e.g., artificial neural networks, they
have the main advantages that $(a)$ training is guaranteed to end up
in a global minimum, $(b)$ their genralisation power is theoretically
well founded, $(c)$ they can easily work with highly dimensional,
non-linear data, and $(d)$ the solution achieved is sparse. Due to
these good properties, they have been now extensively used in, e.g.,
speech recognition, object classification and function approximation
\cite{Cristianini00}. On the other hand, one of their main drawbacks
is their alleged inability to cope with large datasets
\cite{KeerthiCDC06}.

Yet, in most real-life applications, datasets \emph{are} large, for
example when online learning must be performed. Online learning is a
scenario in which training data is provided one example at a time, as
opposed to the batch mode in which all examples are available at once
(see \cite{Laskov2006} and citations therein). In the case of, e.g.,
non-stationary data, online algorithms will generally perform better
since ambiguous information (i.e., whose distribution varies over
time) is present, and couldn't possibly be taken into account by the
batch algorithm. Online algorithms allow to incorporate additional
training data, when it is available, without re-training from scratch.

In an online setting there is no guarantee that the flow of data will
\emph{ever} cease; therefore, applying SVMs here looks appealing but
we need a way to cope with large datasets and slash the training and
testing times. One of the keys to the problem seems to lie in the the
sparseness of their solution. That an SVM solution is \emph{sparse}
means that usually just a few samples can account for its complexity;
in fact, SVMs can be seen as a way of compressing data by selecting
``the most important'' samples (\emph{support vectors}) among those in
the training set. Keeping the number of support vectors small without
losing accuracy is therefore a major challenge, also since a recent
result \cite{Steinwart03} shows that the number of support vectors
grows indefinitely with, namely proportionally to, the number of
training samples.

In this paper we propose a method of incrementally selecting support
vectors based upon \emph{linear independence in the feature space}:
support vectors which are linearly dependent on already stored ones
are rejected, and a smart, incremental minimisation algorithm is
employed to find the new minimum of the cost function. The size of the
kernel matrix (the core of an SVM and its major bottleneck) is
therefore kept small. Our experiments indicate that SVMs employing
this idea, that we call \emph{Online Independent Support Vector
Machines} (OISVMs), do not grow linearly with the training set, as it
was the case in \cite{Steinwart03}, but reach a limit size and then
stop growing \cite{engel2004}. In the case of finite-dimensional
feature spaces they also \emph{keep the full accuracy of standard
SVMs}; whereas in the infinite-dimensional case, at the price of a
negligible loss in accuracy, one can tune the growing rate of the
machine.

To support this claim, we show an extensive set of experimental
results obtained by comparing SVMs and OISVMs on standard benchmark
databases as well as on a real-world, online application coming from
robotic vision: place recognition in an indoor environment, from
sequences acquired by robot platforms under different weather
conditions and across a time span of several months.
{\bf after the experiments are done, place some stunning numbers here.}

The paper is structured as follows: after a review of the relevant
literature, Section \ref{sec:bg} gives an overview of background
mathematics proper to SVMs; in Section \ref{sec:opt} OISVMs are
described; in Section \ref{sec:exp} we show the experimental results;
and lastly in Section \ref{sec:concl} conclusions are drawn and future
work is outlined.
