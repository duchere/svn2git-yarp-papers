Introduced in the early 90s by Boser, Guyon and Vapnik \cite{BGV92},
\emph{Support Vector Machines} (SVMs) are a class of machine learning
algorithms deeply rooted in statistical learning theory
\cite{v-edbed-82}, able to classify data taken from an unknown
probability distribution, given a set of training examples. As opposed
to analogous algorithms such as, e.g., artificial neural networks,
they have the main advantages that training is guaranteed to end up in
a global solution, that they can easily work in highly dimensional,
non-linear feature spaces, and that the solution achieved is
sparse. Due to these good properties, they have been now extensively
used in, e.g., speech recognition, object classification and function
approximation \cite{Cristianini00}. On the other hand, one of their
main drawbacks is their alleged inability to cope with large datasets,
since their space requirement is quadratic and the training time is
cubic in the number of training samples \cite{KeerthiCDC06}.

Yet, in most real-life applications, online learning must be
performed. Online learning is a scenario in which training data is
provided one example at a time, as opposed to the batch mode in which
all examples are available at once (see
\cite{Laskov2006} and citations therein, such as
\cite{Munro1951,lecun98efficient,Murata2002}). In the case of
non-stationary data, online algorithms will generally perform better
if ambiguous information, e.g. different distributions varying over
time, is present, and couldn't possibly be taken into account by the
batch algorithm. Online algorithms allow to incorporate additional
training data, when it is available, without re-training from
scratch.

Therefore, applying SVMs in an on-line setting looks appealing but
requires the ability to cope with large data-sets and to slash the
training and testing times. One of the keys to the problem seems to
lie in the the sparseness of their solution. That an SVM solution is
\emph{sparse} means that usually just a few samples can account for
most of the complexity of the approximated function; in fact, SVMs can
be seen as a way of compressing data by selecting ``the most
important'' samples (\emph{support vectors}) among those in the
training set. Keeping the number of support vectors small without
losing accuracy of the solution is therefore a major issue, also since
a recent result \cite{Steinwart03} shows that the number of support
vectors grows indefinitely as more training samples are acquired and
used for training.

Following related literature, in this paper we propose a method of
selecting support vectors based upon \emph{linear independence in the
feature space}: support vectors which are linearly dependent on
already stored ones are rejected, and a smart, incremental
minimisation algorithm is employed to find the new minimum of the cost
function. The size of the kernel matrix (the core of an
SVM and its major bottleneck) is therefore kept small. Our
experiments indicate that SVMs employing this idea, that we call
\emph{Online Independent Support Vector Machines} (OISVMs), do
not grow linearly with the training set, as it was the case in
\cite{Steinwart03}, but reach a limit size and then stop growing in
the case of finite-dimensional feature spaces, \emph{while keeping the
full accuracy of standard SVMs}; and that they grow dramatically less
in the infinite-dimensional case, at the price of a negligible loss in
accuracy.

To support this claim, we show an extensive set of experimental
results obtained by comparing SVMs and OISVMs on standard benchmark
databases as well as on a real-world, online application coming from
robotic vision: place recognition in an indoor environment, from
sequences acquired by robot platforms under different weather
conditions and across a time span of several months.

The paper is structured as follows: in Section \ref{sec:bg} we give an
overview of background mathematics proper to SVMs; in Section
\ref{sec:opt} OISVMs are described; in Section
\ref{sec:exp} we show the experimental results; and lastly in Section
\ref{sec:concl} conclusions are drawn and future work is outlined.
