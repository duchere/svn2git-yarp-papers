The exploitation of SVM solution sparseness probably appears first in
\cite{DownsGM01}: a simplification of the decision function is therein
proposed, based upon linear independence of the support vectors in the
feature space, performed \emph{after} the training is done. This is a
simple consequence of the fact that, if the feature space has
dimension $n$, at most $n+1$ independent SVs are required to build the
solution \cite{PontilV98}. Downs et al.'s idea is useful in reducing
the testing time, but not the training time, since every time new data
is available training must be performed from scratch and taking into
account all SVs found so far. Discarding from the sample set the
linearly dependent SVs won't work, unless one is prepared to lose
accuracy --- in fact, other methods to heuristically select a subset
of the support vectors have been proposed,
e.g. \cite{LeeM01,schoel06,KeerthiCDC06}.

In order to keep the solution compact without losing accuracy, the key
is to keep the size of the kernel matrix small, i.e., reducing its
rank. Unsupervised rank reduction methods have been proposed, e.g., in
\cite{KeerthiCDC06}, as well as supervised ones
\cite{Baudat03,BachJordan2005}, but no application of these ideas
appears so far, to the best of our knowledge, in online settings. This
is particularly important since it has been shown \cite{Steinwart03}
that the SV set grows linearly with the sample set; therefore, in an
online setting, a SVM would grow indefinitely.

The exact solution to online SVM learning was given by Cauwenberghs
and Poggio in 2000 \cite{CauwenberghsP00}, but their idea has received
little attention in the community so far \cite{Laskov2006}, probably
due to the lack of a detailed analysis of its complexity.
