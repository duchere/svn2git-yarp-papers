The exploitation of sparseness in SVMs probably appears first in
\cite{DownsGM01}: a simplification of the decision function is therein
proposed, based upon linear independence of the support vectors in the
feature space, performed \emph{after} the training is done. This is a
simple consequence of the fact that, if the feature space has
dimension $n$, at most $n+1$ independent SVs are required to build the
solution \cite{PontilV98}. The idea is useful in reducing the testing
time, but not the training time, since every time new data is
available training must be performed from scratch and taking into
account all SVs found so far. Discarding from the sample set the
linearly dependent SVs won't work, unless one is prepared to lose
accuracy --- in fact, other methods to heuristically select a subset
of the support vectors have been proposed,
e.g. \cite{LeeM01,schoel06,KeerthiCDC06}.

In order to keep the solution compact without losing accuracy, the key
is to keep the size of the kernel matrix small, i.e., reducing its
rank. Unsupervised rank reduction methods have been proposed, e.g., in
\cite{KeerthiCDC06}, as well as supervised ones
\cite{Baudat03,BachJordan2005}, but no application of these ideas
appears so far, to the best of our knowledge, in online settings. This
is particularly important since it has been shown \cite{Steinwart03}
that the SV set grows linearly with the sample set; therefore, in an
online setting, a SVM would grow indefinitely.

The exact solution to online SVM learning was given by Cauwenberghs
and Poggio in 2000 \cite{CauwenberghsP00}, but their idea has received
little attention in the community so far \cite{Laskov2006}, probably
due to the lack of a detailed analysis of its complexity.

Visual place recognition for robot platforms is a widely researched
topic in which online learning is critical. Incremental learning
approaches had been so far mostly used for constructing the
geometrical map, or the environment representation, online.  Brunskill
et~al. \cite{emma:irca05} proposed a model using incremental PCA for
simultaneous localization and mapping (SLAM). A similar approach was
used in the only work we are aware of that uses an incremental method
in the context of place recognition. In \cite{ljubjiana:icra02}
incremental PCA was used to update low-dimensional representations of
images taken by a mobile robot as it moved around in an
environment. They also tested repetitive learning of their model with
the same training images several times. While they obtained impressive
results in term of reconstruction, their method was not tested for
classification.
