\subsection*{Experiments with Standard Benchmark Database}

In order to test the effectiveness of OISVMs with respect to standard
SVMs, we first show some numerical results on standard benchmarks for machine 
learning methods. We have chosen finite- and infinite-dimensional kernels, namely
polynomial kernels of degree $1$ (linear) and cubic, and
Gaussian kernel. In the finite-dimensional case, $\eta$ is
essentially irrelevant, and we have set it to machine precision. 
In the case of the infinite-dimensional kernel, we have run the OISVM with
$\eta$ at different values, expecting, as foretold, bigger values of
$\eta$ to cause the accuracy to degrade, but also the size of the
machine to remain smaller than with smaller values.
For each benchmark, we display the mean number of retained support vectors
on $5$ random $75\%/25\%$ train/test runs as well as the mean performance loss.

\subsection*{Experiments with Real-world Application}

We performed another series of experiments, namely place recognition in an indoor
environment \cite{pronobis:iros06, luo:icra07}, to evaluate our algorithm. 
\textbf{Fixed me: If it is necessary to say more about the importance of doing incremental
learning for place recognition.}
We considered a realistic scenario
where the algorithms had to incrementally update the model to adapt to the variations
in an indoor environment introduced by human activities over a long time spans. These
variations includes people appear in different rooms during working time, objects such as
cups are moved or taken in/out of the drawers, pieces of furniture are pushed around, and
decoration changes.   

For experiments, we used a newly introduced database called IDOL2 (Image Database
for rObot Localization 2, \cite{luo:idol2}), which contains 24 image sequences acquired
using a perspective camera mounted on two mobile robot platforms. The acquisition was
performed with an indoor laboratory environment consisting of five rooms of different
functionality. The sequences were acquired under various weather and illumination conditions
(sunny, cloudy, and night) and across a time span of six months. Thus, this data capture
natural variability that occurs in real-world environments because of both natural changes
in the illumination and human activity. The image sequences in the database are divide as
follows: for each robot platform and for each type of illumination conditions, there were
four sequences recorded. Of these four sequences, the first two were acquired six months
before the last two. This means that, for each robot and for every illumination condition,
there are always two sequences acquired under similar conditions, and two sequences acquired
under very different conditions. It makes the database suitable for different kinds of evaluation
on the adaptability of an incremental algorithm. For further details about the database, we
refer the readers to \cite{luo:idol2}.

The evaluation was performed using composed receptive field histograms (CRFH)
\cite{Linde:Lindeberg:ICPR04} as global image features and SIFT \cite{lowe99object}
for extracting local features. In the experiments, we consider
both $chi^2$ kernel for SVM (when use CRFH), and local kernels \cite{wallraven:iccv03} (SIFT).
\textbf{Fixed me: Shall we write something about the SVM library.} We have run the OISVM with
$\eta$ at different values. 

...