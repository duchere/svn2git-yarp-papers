\documentclass[10pt,twocolumn,letterpaper]{article} 

\usepackage{cvpr}
\usepackage{times}
\usepackage{epsfig}
\usepackage{graphicx}
\usepackage{amsmath}
\usepackage[psamsfonts]{amssymb}
\usepackage{url}
\usepackage[pagebackref=true,breaklinks=true,letterpaper=true,colorlinks,bookmarks=false]{hyperref}

% ------------------------------------------------------------------------ 

\cvprfinalcopy
\def\cvprPaperID{****}
\def\httilde{\mbox{\tt\raisebox{-.5ex}{\symbol{126}}}}
\ifcvprfinal\pagestyle{empty}\fi

\def\RR{\mathbb{R}}
\def\NN{\mathbb{N}}
\def\xx{\mathbf{x}}
\def\ww{\mathbf{w}}
\def\aa{\mathbf{\alpha}}
\def\bb{\mathbf{\beta}}
\def\ee{\mathbf{e}}
\def\dd{\mathbf{d}}
\def\mdd{\tilde{\dd}}
\def\b{\mathcal{B}}
\def\d{\mathcal{D}}

\begin{document}

% ------------------------------------------------------------------------ 

\title{Online Independent Support Vector Machines}

\author{
Francesco Orabona, Claudio Castellini\\
LIRA-Lab, University of Genova\\
viale F. Causa, 13\\
16145 Genova, Italy\\
{\tt\small \{bremen,drwho\}@liralab.it}
\and
Barbara Caputo, Jie Luo\\
IDIAP Research Institute\\
rue du Simplon 4\\
P.O. Box 592 --- CH-1920 Martigny, Switzerland\\
{\tt\small \{bcaputo,jiel\}@idiap.ch}
\and
Giulio Sandini\\
Italian Institute of Technology\\
via Morego, 30\\
16100 Genova, Italy\\
{\tt\small giulio.sandini@iit.it}
}

\maketitle

% ------------------------------------------------------------------------ 

\begin{abstract}

  Support Vector Machines (SVMs) are a machine learning method widely
  employed in, e.g., visual recognition, medical diagnosis and robotic
  control. One of their most interesting characteristics is that the
  solution achieved is sparse: a few samples (support vectors) usually
  account for most of the complexity of the classification task. Both
  the training and testing time crucially depend on the number of
  support vectors, and it is then very important to keep their number
  small, trying at the same time to retain the accuracy of the
  solution. This is especially evident in online settings such as
  topological mapping for a robotic platform, where a large amount of
  unknown data is expected to be available as the robot adapts to the
  changing environment.

  In this paper we introduce a novel approach to the problem, called
  Online Independent SVMs, in which the solution is built online using
  only those support vectors which are linearly indepedent in the
  feature space. We then show that OISVM achieve an excellent accuracy
  vs. compactness trade-off in general, while retaining the full
  accuracy of ordinary SVMs in case a finite-dimensional kernel is
  used. This statement is supported by experiments on standard
  benchmark databases as well as on a real-world application, namely
  place recognition in an indoor environment, from sequences acquired
  by robot platforms under different weather conditions and across a
  time span of several months.

\end{abstract}

% ------------------------------------------------------------------------ 

\section{Introduction}

\subsection*{Related work}

\subsection*{Background}
\label{sec:bg}

\section{Online Independent Support Vector Machines}
\label{sec:opt}

\subsection*{Linear independence}

\subsection*{Training the machine}

\section{Experimental Results}
\label{sec:exp}

\section{Conclusions}
\label{sec:concl}

% ------------------------------------------------------------------------ 

{\small
\bibliographystyle{ieee}
\bibliography{paper}
}

\end{document}
