In this section we report the experimental evaluation of our method.
We first test it on a set of databases commonly used in the machine 
learning community for assessing new algorithms (section \ref{exp:ml});
we then apply it to a realistic scenario of recognizing indoor places
across different viewpoints and illumination conditions 
(section \ref{exp:idol2}).

\subsection{Experiments with Standard Benchmark Database}
\label{exp:ml}

In order to test the effectiveness of OISVMs with respect to standard
SVMs, we first show some numerical results on standard benchmarks for machine 
learning methods. We have chosen finite- and infinite-dimensional kernels, namely
polynomial kernels of degree $1$ (linear) and cubic, and
Gaussian kernel. In the finite-dimensional case, $\eta$ is
essentially irrelevant, and we have set it to machine precision. 
In the case of the infinite-dimensional kernel, we have run the OISVM with
$\eta$ at different values, expecting, as foretold, bigger values of
$\eta$ to cause the accuracy to degrade, but also the size of the
machine to remain smaller than with smaller values.
For each benchmark, we display the mean number of retained support vectors
on $10$ random $75\%/25\%$ train/test runs as well as the mean performance loss.
For the sake of comparison we have also used the more common norm-1
SVM formulation, that is with $p=1$ in formula (\ref{eqn:svm_primal}), because
it is known to produce sparser solutions that the norm-2 formulation.

\begin{table}
\begin{center}
\begin{tabular}[!h]{|l|c|c|c|}
\hline
Database&$\Delta$ in&\% of the SVs&\% of the SVs\\
name&test err&vs norm-2&vs norm-1\\
\hline
Breast&$-0.47\pm0.82$&$10.2\pm0.87$&$22.1\pm1.77$\\
\hline
Diabetes&$0.52\pm2.1$&$40.2\pm2.1$&$55.2\pm2.73$\\
\hline
German&$-0.40\pm1.15$&$6.1\pm0.23$&$9.2\pm0.35$\\
\hline
Heart&$0.45\pm1.01$&$10.3\pm0.56$&$15.5\pm0.94$\\
\hline
\end{tabular}
\end{center}
\label{table:t1}
\caption{Comparison of OISVM on different standard dataset. The kernel used
 is gaussian and the $\eta$ values are optimized to get the best trade-off on
 each database.}
\end{table}
