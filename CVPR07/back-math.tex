

Assume $\{\xx_i,y_i\}_{i=1}^l$, with $\xx_i \in \RR^m$ and $y_i \in
\{-1,1\}$, is a set of samples drawn from an unknown probability
distribution. The problem is to approximate it in order to classify
more data coming from the same source to the best extent
possible. Assuming the data are linearly separable, according to the
standard approach, a \emph{separating hyperplane} in $\RR^m$ is sought
for:

\begin{equation} \label{eqn:dec}
  f(\xx) = sgn(\ww\cdot\xx + b)
\end{equation}

with $\ww \in \RR^m$ and $b \in \RR$. In this case, the hyperplane
must respect the constraints $y_i(\ww\cdot\xx_i + b)-1\geq 0$, for all
$i = 1,\ldots,l$ (from now on, this will be implicit whenever a
subscript $i$ appears free in a formula). In the general, more likely
and realistic case in which the data are not linearly separable, we
introduce $l$ slack variables $\xi_i$ and rather require that
$y_i(\ww\cdot\xx_i + b)-1+\xi_i\geq 0$, with $\xi_i \geq 0$. In order
to find such a hyperplane, we wish to maximise the hyperplane's
distance from both groups of samples (\emph{margin}). The margin is
easily determined to be $\frac{2}{||\ww||}$, so we are left with the
problem of minimising $||\ww||$ subject to the above constraints. The
problem is then usually solved minimising the following expression:

\begin{equation} \label{eqn:svm_primal}
  \min_{\ww} \left( ||\ww||^2 + C \sum_{i=1}^l \xi_i \right)
\end{equation}

subject to the constraints

\begin{eqnarray} \label{eqn:svm_constr}
  y_i (\ww\cdot\xx_i + b) & \geq & 1-\xi_i \\
                    \xi_i & \geq & 0 \nonumber
\end{eqnarray}

where $C \in \RR$ is a positive weight coefficient. Since both the
problem and the constraints are convex, (\ref{eqn:svm_primal}) and
(\ref{eqn:svm_constr}) can be compactly expressed in Lagrangian form
by introducing $l$ pairs of coefficients $\alpha_i, \mu_i$ and then
minimising the objective function

\begin{eqnarray} \label{eqn:lp1}
  L_P =
      \frac{1}{2} ||\ww||^2
    - \sum_{i=1}^l \alpha_i \left(y_i (\ww\cdot\xx_i+b) - 1 + \xi_i \right) \\
    + C \sum_{i=1}^l \xi_i - \sum_{i=1}^l \mu_i \xi_i \nonumber
\end{eqnarray}

subject to the constraints that $\alpha_i,\mu_i\geq 0$. By using the
extremum conditions for $\ww$ and $b$, that is, $\nabla_{\ww,b} L_P =
0$, one finds that

\begin{equation} \label{eqn:w1}
  \ww = \sum_{i=1}^l \alpha_i y_i \xx_i
\end{equation}

which, substituted in Equation (\ref{eqn:dec}), gives

\begin{equation} \label{eqn:lin_sol}
  f(\xx) = sgn \left( \sum_{i=1}^l \alpha_i y_i \xx \cdot \xx_i + b \right)
\end{equation}

Notice that, in the last Equation, the $\xx$'s only appear in the form
of inner products; in order to boost the expressive power of SVMs
then, the $\xx_i$s are usually mapped to a highly, possibly
infinite-dimensional space (the \emph{feature space}) via a
non-linear mapping $\Phi(\xx)$; the core of the SVM becomes then the
so-called \emph{kernel function} $K$ such that $K(\xx_1,\xx_2) =
\Phi(\xx_1)\cdot\Phi(\xx_2)$. This idea is called \emph{kernel trick}
and is standard in literature; it avoids the necessity of explicitly
knowing $\Phi$. Equation (\ref{eqn:lin_sol}) then becomes

\begin{equation} \label{eqn:sol}
  f(\xx) = sgn \left( \sum_{i=1}^l \alpha_i y_i K(\xx,\xx_i) + b \right)
\end{equation}

After training, that is after the minimisation of $L_P$, some of the
$\alpha_i$s (actually most of them in many practical applications) are
zero; those $\xx_i$s for which this does \emph{not} hold are somehow
crucial to the solution and are called \emph{support vectors}, hence
the name of the approach. This phenomenon is known as
\emph{sparseness} of the solution, meaning that only a subset of the
training data is usually really needed to build it.
