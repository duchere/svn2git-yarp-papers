Due to space limitations, this is a very quick account of SVMs --- the
interested reader is referred to \cite{Burges98,SmolaTut2004} for a
tutorial, and to \cite{Cristianini00} for a comprehensive introduction
to the subject. Assume $\{\xx_i,y_i\}_{i=1}^l$, with $\xx_i \in \RR^m$
and $y_i \in \{-1,1\}$, is a set of samples and labels drawn from an
unknown probability distribution; we want to find a function $f(\xx)$
best approximating it, such that $sign(f(\xx))$ determines the
category of any future sample $\xx$. In the most general setting,

\begin{equation} \label{eqn:sol}
  f(\xx) = \sum_{i=1}^l \alpha_i y_i K(\xx,\xx_i) + b
\end{equation}

\noindent where $b \in \RR$ and $K(\xx_1,\xx_2) = \Phi(\xx_1)
\cdot \Phi(\xx_2)$, the \emph{kernel function}, evaluates inner
products between samples through a non-linear mapping $\Phi$. The
$\alpha_i$s are Lagrangian coefficients obtained by solving (the
Lagrangian form of) the problem

\begin{eqnarray} \label{eqn:svm_primal}
  \min_{\ww,b}      & \frac{1}{2} ||\ww||^2 + C \sum_{i=1}^l \xi_i \\
  \mbox{subject to} & y_i (\ww\cdot\xx_i + b) \geq 1-\xi_i         \nonumber \\
                    & \xi_i \geq 0                                 \nonumber
\end{eqnarray}

\noindent where $\ww \in \RR^m$ defines a \emph{separating hyperplane}
in the feature space, i.e., the space where $\Phi$ lives, whereas
$\xi_i \in \RR$ are slack variables and $C \in \RR$ is a positive
weight coefficient. The condition $0 \leq \alpha_i \leq C$ must
hold.

Thanks to the introduction of $K(\cdot,\cdot)$ and the associated
\emph{kernel matrix} $K$, with $K_{ij} = K(\xx_i,\xx_j)$, SVMs are
able to find $f(\xx)$, complex as it may be, by solving a linear
separation problem in a highly-dimensional space. This idea is often
called \emph{kernel trick}. In general, the size of $K$ is $l \times
l$, and the time required by an SVM to train and predict is, in turn,
cubic and linear in it \cite{KeerthiCDC06}; but, in practice, most of
the $\alpha_i$ are found to be zero after training and therefore can
be neglected in evaluating the solution (those $\xx_i$s which cannot
be neglected are called \emph{support vectors}). This phenomenon is
known as \emph{sparseness} of the solution, meaning that only a subset
of the training data is usually really needed to build it.
