Object recognition is a key problem of artificial vision; in robotics, it
is strongly connected to that of grasping. In fact, there is so far no
general solution to either problem. Traditionally,
visual features are evaluated from camera images and statistical methods
are then trained on huge visual datasets in order to obtain a robust object
classifier. The knowledge so obtained is then used to choose a model to
perform a grasping action.

Inspired, among others, by the neuroscientific framework of mirror neurons,
we hereby propose to enhance the model of an object by adding to its visual
features a probabilistic description of the grasps chosen by human subjects
to grasp it. %--- in other words, a model of its affordances.
Since in a standard setting the grasps are not directly
available to the system, they must be reconstructed from the visual features,
and then used to augment the recognition system's input space. We achieve
this by building a map from visual to motor features, which
we call a Visuo-Motor Map (VMM), practically enforced via regression on a
human grasping database.

We experimentally show that such a technique improves the
recognition rate of a standard object classifier: in case the original motor
features are used, the improvement is dramatic, whereas when we reconstruct
them via the VMM we still obtain a statistically significant improvement.
%The proposed system can be seen as an instance of a general framework for
%multi-model learning, in which an artificial system learns to reconstruct
%active sensory patterns (``how do I grasp a mug'') from passive ones (the
%visual appearance of a mug), and then to use the former together with the
%latter to improve its understanding of the environment.
