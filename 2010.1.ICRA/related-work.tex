The capability to recognise and categorise objects is a crucial
ability for an autonomous agent; and in robotics, it is inextricably
woven with the ability of grasping an object. In cognitive science, the
theoretical link between vision and manipulation was provided by 
Gibson, according to whom an object is characterized by three
properties: (1) it has a certain minimal and maximal size related to the
body of an agent, (2) it shows temporal stability, and (3) it is manipulable
by the agent. These properties imply that the object is defined in relation
to an embodied agent able to manipulate the object. Therefore the set of
possible manipulation actions are a crucial part of the object definition
itself. 

Interestingly, the theory of affordances has recently found neurological evidence,
it is claimed, in the mirror neurons paradigm \cite{gallese-96,rizzolatti-04}.
According to it, structures exist in the high primates' brain which will fire
if, and only if, an object is grasped (which mainly involves the sensorimotor
system) or is seen grasped by an external agent (involving the visual system only,
\cite{umilta-01}). In addition to the original findings in monkeys, very recent
evidence has been produced for the existence of such structures in humans
\cite{friston09}. If this is true, then the human object classification is
so robust exactly because we \emph{know what to do} with the
objects we see --- a capability which machines lack, so far.

This idea has so far been little exploited; among the positive cases there
are \cite{metta-06,lopes-05} who take an exquisitely robotic perspective,
letting their systems acquire motor information about objects by
having a humanoid robot actually manipulating them. On the other hand,
the vast majority of work on object recognition and categorization
models objects starting from static images, without taking into account
their 3D structure and their manipulability \cite{griffin_perona_cvpr2008,leibe_etal_ijcv2008}.
Few very recent attempts try to capture the Gibson's view. The approach
proposed in \cite{gupta_davis_cvpr2008} presents a Bayesian framework that
unifies the inference processes involved in object categorization and
localization, action understanding and perception of object reaction.
The joint recognition of objects and actions is based on shape and motion,
and the models take as input video data. In \cite{kjellstrom_etal_eccv2008},
the authors consider objects as contextual information for recognizing
manipulation actions and vice versa. The action-object dependence is
modelled with a factorial conditional random field with a hierarchical
structure. In both approaches, objects and their affordances are first
modelled separately, and combined together in a second step. This does
not consider the embodiment of the agent manipulating the objects.
