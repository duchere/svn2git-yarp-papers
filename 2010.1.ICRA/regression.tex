The VMM is supposed to be a regression strategy from visual to motor features, 
as defined above. Since the output is multivariate
(the motor features, consisting of $22$ numbers) and the input is very highly dimensional (the visual features, consisting of $200$
numbers), we decided
to enforce the VMM using neural networks. Each network was kept as
simple as possible: one hidden layer with $20$ neurons,
log-sigmoid transfer function and scaled conjugate gradient 
backpropagation. The training procedure used the early stopping
strategy, i.e. the training set was divided in a new training
and validation set. The network is evaluated on the validation set: 
when the performance stops improving, the algorithm halts. 

Most of these settings are inspired by the work of Richmond and others
\cite{papcun,richmond2007} on audio-to-motor mapping. In fact,
since each object may correspond to several grasps as it happens in reality
(recall the Section above), the relationship between the visual  and the motor features
is highly non-functional and it is in general hard, if not pointless, to
model it using a single NN. Richmond's idea was to model a \emph{probability
distribution} rather than a functional map; here we follow a somewhat more naive approach:
we define an ``archetipal grasp'' related to the specific
object observed. In the case of an object that can be grasped in only one way 
(for instance ``pig''), then the archetipal grasp will correspond 
to it. In case of an object graspable in different ways, then the 
archetipal grasp will correspond to an ``average'' grasp between those 
possible. We expect that this reconstructed grasp will have a 
positive effect on the overall performance of our object recognition system;
at the same time we hope that such representation won't get messed up
with other ones since the output space is also rather high-dimensional.


%The VMM Trying to map visual to motor information
%actually means defining a regression strategy. In our setting,
%we need a method which receive in input a SIFT feature vector
%and produces in output a sensorymotor feature vector. 
%A reasonable hypothesis is that every time we see an object 
%we are able to associate to it all the possible ways in
%which we know it can be grasped. The idea can be simplified
%supposing to define an ``archetipal grasp'',related to the specific
%object observed. In the case of an object that can be grasped in only one way 
%(for instance ``pig''), then the archetipal grasp will correspond 
%to it. In case of an object graspable in different ways, then the 
%archetipal grasp will correspond to an ``average'' grasp between those 
%possible. We expect that this reconstructed grasp will have a 
%positive effect on the overal performance of our object recognition system.
%
%We implemented this strategy using 7 neural
%networks, one for each object in the VMGdb database. All the
%neural networks are equal: 200 input values corresponding
%to the SIFT feature vector elements; 20 neurons in the hidden layer;
%22 output values corresponding to the sensorimotor feature vector 
%elemets; log-sigmoid transfer function; scaled conjugate gradient 
%backpropagation. The training procedure used the early stopping
%strategy, i.e. the training set was divided in a new training
%and validation set. The network is evaluated on the validation set: 
%when the performance stops improving, the algorithm halts. 
%
%\textbf{Solo qualche idea rozza: nella discussione finale andrebbe 
%detto che la rete neurale cosi' definita e' debole...la scelta del 
%numero nei neuroni nell'hidden layer e' 'occhiometrica': ...la teoria 
%suggerisce che dovrebbero essere di piu'...anche l'addestramento per ogni
%oggetto e' fatto usando circa 200 campioni, ma per come e' fatta la rete
%numero dei parametri che volgiamo stimare e' maggiore di 200...,
%avremmo bisogno di piu' campioni o diridurre la dimensione dei
%vettori di input ed output (sappiamo che non tutti gli elementi dei vettori
%SIFT e motor sono utili)}
