This paper presented a theoretical framework for joint modelling of visual and motor data for multimodal object recognition.
The key feature of our approach is the learning of a Visuo-Motor Map between the two modalities during training.
The existence of this map makes it possible to benefit from the multimodal nature of the model even when the motor data
is not perceived by the system. Experiments confirm the validity of our approach, showing a gain in performance
of up to 7.6\% and 2.4\% when using both modalities, compared to results achieved using vision only.

%In this paper we have shown that even a simple approach to sensorimotor learning
%can significantly improve the performance of a traditional classifer. The performance
%of our VMM-enhanced object classification system is such that .... \textbf{numeri dagli
%esperimenti}. The VMM has been obtained so far in the most straightforward way, that
%is, by applying a standard neural network to visual features of objects, and having
%it map onto motor features of an associated grasp.

The data upon which our experiments have been carried on are collected in the
CONTACT Visuo-Motor Grasping dataBase (VMGdB), which we envision as a testbed and
a benchmark for all researchers interested in investigating the nature of (human and
robotic) grasping, and its ties to object recognition. %In fact, in this work we have
%neglected a lot of potentially useful information contained in the database, for instance
%the dynamics associated with the reaching phase, prior to grasping, which is well-known
%to carry substantial information about it \cite{174427,santello}.

\noindent
{\bf Future Work.} The current implementation of the framework contains several simplifying assumptions, each corresponding to 
ongoing and future research directions:
\begin{enumerate}
\item {\em Dynamic of the data.}
In this work we have
neglected a lot of potentially useful information coming from the dynamics associated with the reaching phase, prior to grasping.
This is well-known
to carry substantial information about it \cite{174427,santello}. We plan to include the dynamic in the representation of motor and 
also visual features: indeed the dynamic changes in the object state associated with its manipulation are an important cue on the
object's identity \cite{gupta_davis_cvpr2008,kjellstrom_etal_eccv2008}. This research direction will likely lead us to move
from grasping postures to grasping actions, and therefore affordance-based object representations.

\item {\em Shape-based visual representations.}
While here we used an appearance-based visual representation for the object, we are fully aware that this visual information
is weakly correlated with the grasping and therefore makes the life of the mapping function much harder. We plan in the future to
represent objects based on shape information. This will lead to visual information complementary to the grasp hand posture (the
configuration of the hand at the moment of the grasp can be seen as a motor-based information regarding the shape of the object).
We also expect that a shape-based visual representation will make it possible to build categorical object models based on their
shape, i.e. graspability. This might lead to better defined VMMs, and it would greatly help in the case of large number
objects. 


\item {\em Learning of the Visuo-Motor Map.}
Experimental results  indicate that the ``real'' motor features, that is, the
grasping hand postures as recorded by the data glove, contain much more information
than the visual features alone. Therefore, if one were able to extract more (or better)
motor information from the sight of an object, that is, to build a better VMM,
the situation could improve further. One immediate direction for this line of research
is that of abandoning the somewhat artificial notion of an archetypal grasp associated
with an object, and start training a VMM in order for it to reconstruct a
\emph{probability distribution} over grasps. This would correspond to enhancing an
object model with a possible set of grasps, rather than with one grasp only; a
further, important step toward the re-definition of an object in terms of its
affordances.
We are currently studying an extended version of the VMM architecture based on vector-valued regression to explicitly take into account the many-to-many relationship between objects and grasps. In this case, instead of learning the map between an object and an average grasp, we would learn the map between an object and a vector of the possible grasps. At run time this would allow us to associate the most probable grasp to the object under consideration.


\end{enumerate}
