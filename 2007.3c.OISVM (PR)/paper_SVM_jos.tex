
%This section contains the required mathematical background. We
%introduce SVMs both in the batch and on-line versions. The interested
%reader is referred to, e.g., \cite{Cristianini00,Herbrich01} for a
%comprehensive introduction to the subject.

%\subsection{Support Vector Machines}

Assume that we are given $l$ observations. Each observation is a pair of 
an instance, $\xx_i\in\RR^m$, and a label, $y_i\in\{-1,1\}$, for $1 \le i \le l$.  
The training set, $\{\xx_i,y_i\}_{i=1}^l$, is a set of samples drawn from 
an unknown fixed probability distribution. In binary classification our goal
is to predict the label of a given an instance drawn from the same distribution as the training set. Formally, we would like to find a function
$f(\xx)$ such that given a new instance $\xx$,
the prediction of its label is sign of the function, namely, $sgn(f(\xx))$.

Assuming the
data are linearly separable, an SVM will look for the \emph{separating
hyperplane} in $\RR^m$, $f(\xx) = \ww\cdot\xx + b$, such that
$\|\ww\|$ is minimal and all the constraints
$y_i(\ww\cdot\xx_i + b)-1\geq 0$, for $i = 1,\ldots,l$ are fulfilled. 
%(from now on, the range of $i$ will be implicit whenever a subscript $i$ appears
free in a formula). 
In the more general case where the data is not
linearly separable, the problem is relaxed by introducing $l$ slack
variables $\xi_i$ and constraints are $y_i(\ww\cdot\xx_i +
b)-1+\xi_i\geq 0$, with $\xi_i \geq 0$.
% In order
%to find such a hyperplane, we wish to maximize the hyperplane's
%distance from both groups of samples (\emph{margin}), minimizing at
%the same time the values of the slack variables. The margin is
%easily determined to be $\frac{2}{\|\ww\|}$, so we are left with the
%problem of minimizing $\|\ww\|$ and $\xi_i$ subject to the above constraints.
This problem is then usually solved by minimizing the following convex
function:
\begin{equation} \label{eqn:svm_primal}
  \min_{\ww,b} \left( \frac{1}{2} \|\ww\|^2 + C \sum_{i=1}^l \xi_i^p \right)
\end{equation}
\noindent subject to the constraints $y_i (\ww\cdot\xx_i + b) \geq
1-\xi_i$ and $\xi_i \geq 0$, where $C \in \RR^+$ is an error penalty
coefficient and $p$ is usually $1$ or $2$ \cite{Cristianini00}. This
minimization problem can be compactly expressed in Lagrangian form by
introducing $l$ pairs of coefficients $\alpha_i, \mu_i$ and then
minimizing
\begin{equation} \label{eqn:lp1}
  L_P = \frac{1}{2} \|\ww\|^2 - \sum_{i=1}^l \alpha_i \left(y_i (\ww\cdot\xx_i+b) - 1 +
        \xi_i \right) + C \sum_{i=1}^l \xi_i^p - \sum_{i=1}^l \mu_i \xi_i \nonumber
\end{equation}
subject to the constraints that $\alpha_i,\mu_i\geq 0$ for all $i$.
Karush-Kuhn-Tucker (KKT) conditions \cite{Cristianini00},
give the necessary and sufficient conditions for $\ww, b$ and
$\alpha_i$ to be a feasible solution. From these conditions we have
\begin{equation}
  \frac{\partial L_P}{\partial \ww}= \ww - \sum_{i=1}^l \alpha_i y_i \xx_i = 0 \label{eqn:kkt1}
\end{equation}
%$\alpha_i$ to be be a solution, we obtain, for $p=1$,
%\begin{eqnarray}
%  \frac{\partial L_P}{\partial \ww}= \ww - \sum_{i=1}^l \alpha_i y_i \xx_i & = & 0 \Rightarrow \ww = \sum_{i=1}^l \alpha_i y_i \xx_i \label{eqn:kkt1} \\
%  \frac{\partial L_P}{\partial \xi_i}= C-\alpha_i-\mu_i & = & 0  \label{eqn:kkt2} \\
%  \frac{\partial L_P}{\partial b}= \sum_{i=1}^l \alpha_i y_i & = & 0 \label{eqn:kkt3} \\
%  \alpha_i \left(y_i (\ww\cdot\xx_i+b) - 1 + \xi_i \right) & = & 0 \label{eqn:kkt4} \\
%  \xi_i ( \alpha_i - C ) & = & 0 \label{eqn:kkt5}
%\end{eqnarray}
%Whereas for $p=2$ condition (\ref{eqn:kkt5}) disappears and condition
%(\ref{eqn:kkt2}) becomes
%\begin{equation}
%	\frac{\partial L_P}{\partial \xi_i}= 2 C \xi_i - \alpha_i = 0
%	\label{eqn:kkt2b}
%\end{equation}
%Taking (\ref{eqn:kkt1}) into account, $f(\xx)$ can be expressed as
or $\ww = \sum_{i=1}^l \alpha_i y_i \xx_i$. Hence the inference function $f(\xx)$ can be expressed as
\begin{equation} \label{eqn:lin_sol}
  f(\xx) = \sum_{i=1}^l \alpha_i y_i \xx \cdot \xx_i + b
\end{equation}
where $\xx$ is a query instance for which we would like to predict the label.
Note that in the last equation, the query instance $\xx$
appears only in the form of inner products with the instances in the training set $\xx_i$.
In order to improve the discriminative power of SVMs, the instances $\xx_i$ are  mapped into
a high (possibly infinite) dimensional space (the \emph{feature
space}) via a non-linear mapping $\Phi(\xx)$. Each inner product is replaced by the \emph{kernel function} $K(\xx_i,\xx_j) = \Phi(\xx_i)\cdot \Phi(\xx_j)$. We define the \emph{kernel matrix} $K$ such that the $ij$-th element is $K_{ij} = K(\xx_i,\xx_j)$.
%This idea is called
%\emph{kernel trick} and is standard in SVM literature; it avoids the
%necessity of explicitly knowing $\Phi$. 
In the following, the term
\emph{kernel dimension} will refer, as is customary, to the dimension
of the feature space.
%The kernel dimension is related to the
%generalization power of the machine, and it depends on the choice of
%the kernel itself.
Widely used kernels include the \emph{polynomial}
kernel, which has finite kernel dimensional and the \emph{Gaussian} kernel
which has infinite kernel dimensional. Equation (\ref{eqn:lin_sol}) then becomes
\begin{equation} \label{eqn:sol}
  f(\xx) = \sum_{i=1}^l \alpha_i y_i K(\xx,\xx_i) + b
\end{equation}

After training, that is after the minimization of $L_P$, some of the coefficients 
$\alpha_i$ are
zero. The instances $\xx_i$ for which the corresponding $\alpha_i$ are not zero called
\emph{support vectors}. 

%This phenomenon is known as
%\emph{sparseness} of the solution, meaning that only a subset of the
%training data is usually really needed to build it.

%Using the KKT conditions in (\ref{eqn:lp1}) we obtain that to train a
%SVM we must solve a quadratic programming problem (QP) with as many
%unknowns as training samples.
%State-of-the-art QP solvers decompose
%the problem into manageable subproblems or, in the limit, perform
%iterative pairwise optimization \cite{Platt99}. This approach is
%essentially batch, that is, all the training samples must be available
%from the start.

%\subsection{On-line Learning with SVMs}

While in the batch setting we assume to have all the training observations are
available in hindsight, in on-line learning the training observations are
revealed one at time and no a-priori knowledge of the full training
set can be assumed. In general, an on-line learning framework works in rounds.
On round $i$ an instance $\xx_i$ is presented to the learning algorithm. 
The algorithm predicts a label using the previous hypothesis $h_{i-1}$. 
Then the correct label $y_i$ is revealed and the algorithm suffers an instantaneous
loss. Based on the loss incurred the algorithm updates its hypothesis
%iteratively by refining a hypothesis $h_i$. In particular, at any
%point in time a new sample $\xx_{l+1}$ is received, it is predicted
%using the current hypothesis $h_l$.  The true label $y_{l+1}$ is
%matched against its prediction, 
resulting a new hypothesis $h_i$ \cite{Herbrich01}.  
In other words, for any given sequence of
instance-label pair, $(\xx_1,y_1),\cdots,(\xx_l,y_l)$, a sequence of
hypotheses $h_1,\cdots,h_{l}$ is generated, such that $h_{i}$ depends
only on $h_{i-1}$ and $(\xx_i,y_i)$.
%In the case of kernel based algorithms
%such as SVMs, the representer theorem \cite{CoxOS90,ScholkopfHS01}
%states that, under broad hypotheses, the solution to a problem such as
%(\ref{eqn:svm_primal}) can be expressed as a linear combination of
%kernel functions evaluated on the training points. Hence for any $l$,
%$h_{l}$ is a linear combination of kernel functions evaluated on $\xx_1,
%\cdots,\xx_l$.
%In a sense, $h_{l}$ takes into account all samples up to
%$\xx_l$, so there is no real difference in calculating the solution
%using all the samples received $\xx_1,
%\cdots,\xx_l$ or only using $h_{l-1}$ and $(\xx_l,y_l)$.
Note that, unlike in the batch setting, here we are also interested
in the intermediate hypotheses $h_i, i=1,\ldots,l$. In fact the
hypothesis is used every time a new sample is acquired, for training and
testing.

The standard training algorithms for SVMs are meant to be used in the
batch setting and we can distinguish two different approaches to extend
them to the online setting. In the first one the batch algorithm is
adapted to examine one sample at the time and produce an exact or
approximate solution. Examples of this approaches are respectively \cite{CauwenberghsP00}
and \cite{SyedLS99}. A second approach is based on
gradient-descent methods, in which the solver slowly converge to the batch
solution. This approach has been followed in
\cite{KivinenSW04,ChengVSWC07}.
