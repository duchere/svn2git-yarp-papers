Support Vector Machines (SVMs) \cite{BGV92} are one of the most popular and
promising classification algorithms. As opposed to other learning methods such
as, e.g., neural networks, they are strongly theoretically founded, and have
been shown to enjoy excellent performance in several applications (see, e.g.,
\cite{Cristianini00}). One framework, though, in which their power has not yet
been fully developed is \emph{on-line} learning. A system involved in on-line
learning must face a potentially endless flow of training data, updating
its knowledge after each new sample.
This setting is particularly difficult for SVMs as the size of an SVM solution
grows linearly with the number of training samples taken into account \cite{Steinwart03}.
Since any real system
has access to finite resources (e.g., finite computational power, memory etc.),
a strategy to limit the number of data points is required, and a trade-off to accuracy
must be accepted. So, it becomes
crucial to find a way to save resources while obtaining an acceptable
approximation of the ideal (batch) system.

In this paper we describe a new on-line algorithm, On-line Independent SVMs
(OISVMs), which approximately converges to the ideal, batch SVM solution.
Similar to other kernel-based discriminative on-line algorithms, OISVMs
construct the hypothesis via a subset of the samples seen so far called
\emph{basis}; new samples are put in the basis only if they are linearly
independent in the feature space from the current basis.
The approximate solution obtained can be controlled via a user-defined
parameter. As opposed to similar algorithms,
the size of the solution found in our case is \emph{always} bounded, implying a
bounded testing time. The reduction in time and space requirement is on
average dramatic, at the price of a negligible loss of accuracy. We
show both theoretically and empirically that the size of the basis \emph{does
not grow} linearly with the training set, but converges to a limit size
and then stops growing.
%Notice that the price to
%pay for this is that SVMs typically require a long training %time --- an SVM can
%be up to $50$ times slower than other specialized approaches with
%similar performances \cite{BurgesS96}
%This makes the approach unsuitable, in its current formulation, for
%on-line learning; whereas OISVMs exactly solve this problem,
%incrementally selecting a subset of support vectors (SVs), %based upon
%\emph{linear independence in the feature space}.
OISVMs produce smaller models compared to the standard SVMs, with a
training complexity which is asymptotically quadratic in the number of
samples, and with bounded testing time. Moreover, they reach
near-optimal performance while retaining the good generalization power
of standard SVMs.
%In the case of
%finite-dimensional feature spaces they also \emph{keep the full
%accuracy of standard SVMs}; whereas in the infinite-dimensional case,
%at the price of a negligible loss in accuracy, one can tune the
%growing rate of the machine.

These statements are supported by an extensive evaluation on standard
benchmark databases as well as on two real-world applications, namely
$(a)$ place recognition by a mobile robot in an indoor environment,
and $(b)$ human grasping posture classification.

The paper is organized as follows. Section \ref{sec:SVM} sets up the problem
and describes our method. 
%In Section \ref{sec:comparison} we give a detailed
%comparison with similar approaches, and how machine learning has so
%far been applied to the problems we have tackled. 
Section
\ref{sec:exp} describes the experimental results, and Section
\ref{sec:conclusions} concludes the paper and outlines some future
work.

\subsection*{Related work}

%In the field of
%robot navigation, on-line methods have been used for building
%topological maps and detect loop closure
%\cite{Tapus:Siegwart:ICRA2006}, to learn variability of environments
%due to illumination changes and natural dynamic of rooms
%\cite{luo07iros}, or for adaptive obstacle avoidance in dynamic indoor
%environments like corridors \cite{Zeng:Weng:ICRA04}. On-line methods
%have been applied both to indoor
%\cite{luo07iros,Tapus:Siegwart:ICRA2006} and outdoor navigation
%\cite{ljubjiana:icra02}.
%%, mostly within a probabilistic framework \cite{ljubjiana:icra02,emma:irca05,Tapus:Siegwart:ICRA2006}.
%With the notable exception of \cite{luo07iros}, all on-line approaches
%proposed so far have been tested on a very limited temporal domain, of
%few hours if not of few minutes. Thus, it is not clear if these
%methods are able to provide high accuracy, combined with
%controlled computing resources, in case of on-line learning across a
%time span of several months.
%Regarding grasping recognition, machine learning has never
%been applied on-line to the best of our knowledge. In \cite{AR}
%\emph{regression}, rather than classification, is used to predict the
%grasping configuration. Batch approaches have been used to classify
%grasps, e.g., in \cite{ekvall} (using hidden Markov models),
%\cite{degranville} (Gaussian mixtures) and \cite{friedrich} (neural
%networks). In \cite{heumer}, a comparison of classification systems
%for grasp recognition is presented, the main outcome of which being
%that (uncalibrated) human grasping data gathered from a CyberGlove can
%be used to an excellent extent to recognize the grasp type, although
%the performance can be heavily influenced by multiple user
%analysis. In all works examined, the emphasis is really on affordances
%\cite{gibson} rather than on objects or hand configurations; in other
%words, one is generally interested in the types of grasps and what can
%be done using them.

Attempts to cast the SVMs as an on-line learning algorithm have already
appeared (e.g., \cite{CauwenberghsP00,DomeniconiG01,SyedLS99,BordesEWB05}), but
in all cases there is no attempt to reduce the growth of the solution.

After-training simplification methods aiming to ``shrink'' the obtained solution
(e.g. \cite{DownsGM01,NguyenH05}, chapter $18.3$ in \cite{SmolaS02}) are
too computationally expensive
in an on-line setting where a new solution must be produced after each new
sample.
%In fact the idea therein is useful in reducing the testing time, but
%it is unfeasible in an on-line setting, since the simplification
%should be performed every time a new sample is acquired.
Also, the lack of knowledge of the complete
data set typical of on-line learning rules out a number of methods, 
such as, e.g., kernel matrix low rank approximation methods
based on Incomplete Cholesky Factorization
\cite{FineS02,Baudat03,BachJordan2005}.

Other methods have been proposed to heuristically select a subset of the
vectors as basis, e.g., in \cite{LeeM01,KeerthiCDC06}. In
\cite{WuSB06}, instead, a method to directly build a ``vocabulary'' of
vectors is proposed, but the formulation is not convex.
%On the other hand, a solution with \emph{bounded complexity} must be
%produced. 
%As a matter of fact, the number of support vectors retained
%by an SVM is proportional to the number of training samples
%\cite{Steinwart03}, and the testing time is in turn proportional to
%the number of support vectors; so in an on-line setting the machine
%will eventually become unfeasibly slow. 
On the other hand, several on-line kernel-based methods have been proposed 
a priori bounding the complexity of the predictor
\cite{EngelMM04,KivinenSW04,WestonBB05,ChengVSWC07}.
All of them use a sort of stochastic gradient descent procedure to train a
classifier with a fixed size. However the resulting loss of performance compared
to standard SVM makes these algorithms unsuitable for most real-world applications.
A strategy similar to the one we propose is used in \cite{CsatoO01}, but
again the training is achieved through a variant of the stochastic gradient
descent procedure.
Another possibility is to directly optimize the primal formulation of SVM,
but this strategy cannot be used with infinite dimensional kernels.
%Moreover in many of them
%(e.g. \cite{CsatoO01,KivinenSW04,WestonBB05,ChengVSWC07}) only an
%approximate solution is found after each update, and the algorithm
%will slowly converge to the true solution using a
%gradient-descent-like procedure.
% and the SVM
%feature of having a unique solution is lost.
%Besides all of this, these
%methods require the knowledge of the full training set too, and,
%again, are not suited for on-line learning.

%Experiment $(a)$ is concerned with a mobile robot moving around in an
%indoor environment subject to high variability due to human activities
%over long time spans. These variations include people appearing in
%different rooms during working time, objects such as cups moved or
%taken in/out of the drawers, pieces of furniture pushed around, and so
%forth. The robot is then required to localize its own position,
%despite all these changes.  In experiment $(b)$ the system tries to
%classify the ways a number of human subjects grasp several different
%objects over a reasonably long time span, therefore being independent
%of the subjects' styles and the object's shapes.
