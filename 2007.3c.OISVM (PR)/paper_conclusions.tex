In this paper we have presented, implemented and tested a novel system
for on-line learning based upon Support Vector Machines. The method,
On-line Independent SVMs (OISVMs), avoids using in the solution those
support vectors which are linearly dependent of previous ones in the
feature space; linear independence is checked incrementally every time
a new sample is made available to the system; the optimization problem
is solved via an incremental algorithm which benefits of the small size
of the basis set. A parameter called $\eta$ is employed to trade
better/worse accuracy for more/fewer support vectors in the basis.

We tested the method both on a standard set of benchmark databases and
on two real-world case studies, namely: $(a)$ place recognition in an
indoor environment and $(b)$ human grasping classification. Both
real-world experiments have been carefully crafted in order to take
into account changing conditions in the environment: robot images were
acquired under different weather conditions and across a time span of
several months; grasping was done across a time span of several days,
with several different objects and by several human subjects with
little or no prior knowledge about the experiment.

The experimental results, carried out for various values of $\eta$,
show that OISVMs enjoy an excellent accuracy/basis size trade
off. Moreover, a deeper analysis of the value of the objective
function at the end of the training reveals that our method
selects the basis vectors in such a way as to better minimize
it.

%% %In particular, in the case of
%% %robot localisation, we can reduce the number of SVs up to $4.5$ times
%% %with a negligeble loss in performance; whereas, in grasping classification
%% %we can reduce more than $13$ times.
%% Furthermore, our analysis clearly shows that OISVMs improve on SVMs,
%% proving to be highly adaptive to environmental changes (Introduction,
%% requirement $1$) and accurate (Introduction, requirement $2$), by
%% keeping the number of support vectors extremely small, thus minimizing
%% the need for storage and making testing and training significantly
%% faster (Introduction, requirements $3$ and $4$).

An open issue is how to choose $\eta$ to achieve an optimal trade-off
between accuracy and speed for a given task. Both theory and experiments
clearly point out the relevance of $\eta$ for fully exploiting the power of
OISVMs, but its choice is today one of the heuristics of the method, along with
the choice of $C$, the kernel function and the kernel parameters. Therefore,
the only solution we can point to the reader is the use of a validation set,
as it is customary for the other previously mentioned SVM parameters.
How to determine $\eta$ in a more principled way will be the focus of future
work.

Another research direction we plan to pursue is to extend the method so to
handle an underlying dynamic distribution of the data, as opposed to the static
scenario considered in this paper. This problem has been already tackled in 
generative frameworks (i.e. \cite{YoonRDG08}) but it is still far from being solved
in discriminative settings.

Lastly we plan to use OISVM in a life-long learning scenario. So far OISVMs are
able to perform continuous learning on data collected on a span of
time of up to several months, but it is still not possible to use them
with a truly never-ending stream of data, since all training samples
must anyway be retained. To achieve this goal we plan to extend
the method so to include a forgetting mechanism.
%%  Another important
%% issue is multimodality: systems typically acquire
%% inputs from different modalities that must be combined together during
%% learning so to build a rich internal model. Thus, we plan to extend
%% OISVMs so to work on multimodal data streams.
%Also, so far we have considered a purely supervised learning
%framework, but while it is reasonable to have a privileged teacher for
%some time, in general the system will have to act autonomously. Thus,
%OISVMs should be extended to the semi-supervised framework. Future
%work will address these issues.
