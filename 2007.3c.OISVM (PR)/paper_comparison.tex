In on-line training, samples are received by the system with no
statistical guarantee on the future; therefore, on-line learning rules
out all methods relying on the knowledge of the complete training
set. This includes a number of methods which might appear akin to
ours, such as, e.g., kernel matrix low rank approximation methods
based on Incomplete Cholesky Factorization
\cite{FineS02,Baudat03,BachJordan2005}. For the same reason, any
after-training simplification procedure cannot be used
(e.g. \cite{DownsGM01,NguyenH05}, chapter $18.3$ in \cite{SmolaS02}).
In fact the idea therein is useful in reducing the testing time, but
it is unfeasible in an on-line setting, since the simplification
should be performed every time a new sample is acquired.

Other methods to heuristically select a subset of the support vectors
have been proposed, e.g., in \cite{LeeM01,KeerthiCDC06}. In
\cite{WuSB06}, instead, a method to directly build a ``vocabulary'' of
vectors is proposed, but the formulation is not convex.
% and the SVM
%feature of having a unique solution is lost.
Besides this, these
methods require the knowledge of the full training set too, and,
again, are not suited for on-line learning.

On the other hand, a solution with \emph{bounded complexity} must be
produced. As a matter of fact, the number of support vectors retained
by an SVM is proportional to the number of training samples
\cite{Steinwart03}, and the testing time is in turn proportional to
the number of support vectors; so in an on-line setting the machine
will eventually become unfeasibly slow. Obviously, while bounding the
complexity of the solution, one also wants to have the best possible
solution.
To bound the number of operations during testing, the complexity of
the predictor must be somehow bounded a priori. Training in the primal
with a linear kernel can be a way to have a small solution, but it
cannot be used with infinite dimensional kernels. On the other hand
several on-line kernel-based methods have been proposed to bound
complexity of predictor
\cite{CsatoO01,EngelMM04,KivinenSW04,WestonBB05,ChengVSWC07}. For all
of them the price to pay is a loss in prediction performance.
Moreover in many of them
(e.g. \cite{CsatoO01,KivinenSW04,WestonBB05,ChengVSWC07}) only an
approximate solution is found after each update, and the algorithm
will slowly converge to the true solution using a
gradient-descent-like procedure.

It must be remarked that, on the other hand, OISVMs always find
the best possible solution, given the subset of vectors selected as
basis vectors to build it. This is due to the fact that all the
received training samples are stored, like in the on-line methods in
\cite{CauwenberghsP00}. In the long run this is problematic (for
instance, if we move towards life-long learning \cite{Thrun95k}), but
it can be solved using a forgetting strategy like, e.g., those
proposed in \cite{WestonBB05}.
%% QUELLA CHE SEGUE E` UNA CACCHIATA... CHI L'HA SCRITTA, LAPALISSE? :-)
%% ; alternatively, one could use
%% out-of-core storage of the data (i.e., storage on the hard disk) in
%% order to be able to deal with big training sets.

The sparsification method presented here has been already proposed in
\cite{CsatoO01,EngelMM04}, but we use the original SVM loss function,
more suited for classification tasks. In fact the maximization of the
margin and the concept of the margin itself can be taken into account
during training only using this loss function \cite{Cristianini00}.
Both the above cited papers employ the squared error as loss function,
which is more suited for regression problems, implicitly assuming an
additive gaussian noise on the output values. On the other hand the
exact solution to on-line SVM learning in \cite{CauwenberghsP00}
cannot be used to reduce the number of support vectors. A different
method has been proposed by Liu et al. \cite{LiuSD05} and rediscovered
by Collobert et al. \cite{CollobertSWB06}: they have used a non-convex
formulation of the learning problem where training errors are no
longer support vectors, thus dramatically reducing the growth rate of
the support vectors with the training samples. However, in the paper
it is not clear if the number of support vectors reaches a limit or if
it will grow indefenitely, even if less than with standard SVM.

As far as our experimental test-beds are concerned: in the field of
robot navigation, on-line methods have been used for building
topological maps and detect loop closure
\cite{Tapus:Siegwart:ICRA2006}, to learn variability of environments
due to illumination changes and natural dynamic of rooms
\cite{luo07iros}, or for adaptive obstacle avoidance in dynamic indoor
environments like corridors \cite{Zeng:Weng:ICRA04}. On-line methods
have been applied both to indoor
\cite{luo07iros,Tapus:Siegwart:ICRA2006} and outdoor navigation
\cite{ljubjiana:icra02}.
%, mostly within a probabilistic framework \cite{ljubjiana:icra02,emma:irca05,Tapus:Siegwart:ICRA2006}.
With the notable exception of \cite{luo07iros}, all on-line approaches
proposed so far have been tested on a very limited temporal domain, of
few hours if not of few minutes. Thus, it is not clear if these
methods are able to provide high accuracy, combined with
controlled computing resources, in case of on-line learning across a
time span of several months.

Lastly, regarding grasping recognition, machine learning has never
been applied on-line to the best of our knowledge. In \cite{AR}
\emph{regression}, rather than classification, is used to predict the
grasping configuration. Batch approaches have been used to classify
grasps, e.g., in \cite{ekvall} (using hidden Markov models),
\cite{degranville} (Gaussian mixtures) and \cite{friedrich} (neural
networks). In \cite{heumer}, a comparison of classification systems
for grasp recognition is presented, the main outcome of which being
that (uncalibrated) human grasping data gathered from a CyberGlove can
be used to an excellent extent to recognise the grasp type, although
the performance can be heavily influenced by multiple user
analysis. In all works examined, the emphasis is really on affordances
\cite{gibson} rather than on objects or hand configurations; in other
words, one is generally interested in the types of grasps and what can
be done using them.
