%\title{\huge Preparation of Papers in Two-Column Format\\
%for BioRob Conference Proceedings Published by
%IEEE$^{*}$\footnoterule\thanks{$^{*}$This work is partially
%supported by NSF Grant \#234567890 to M. Burns and CNSF Grant
%\#123456789 to N. Flanders}}

%% This skeleton file requires IEEEtran.cls version 1.6 or later.
%%
\documentclass[conference,a4paper]{./sty/IEEEtran}
% If the IEEEtran.cls has not been installed into the LaTeX system files,
% manually specify the path to it:
% \documentclass[conference]{../sty/IEEEtran}
\IEEEoverridecommandlockouts
\overrideIEEEmargins

% some very useful LaTeX packages include:
\usepackage{cite}
%\usepackage{cite}      % Written by Donald Arseneau
                        % V1.6 and later of IEEEtran pre-defines the format
                        % of the cite.sty package \cite{} output to follow
                        % that of IEEE. Loading the cite package will
                        % result in citation numbers being automatically
                        % sorted and properly "ranged". i.e.,
                        % [1], [9], [2], [7], [5], [6]
                        % (without using cite.sty)
                        % will become:
                        % [1], [2], [5]--[7], [9] (using cite.sty)
                        % cite.sty's \cite will automatically add leading
                        % space, if needed. Use cite.sty's noadjust option
                        % (cite.sty V3.8 and later) if you want to turn this
                        % off. cite.sty is already installed on most LaTeX
                        % systems. The latest version can be obtained at:
                        % http://www.ctan.org/tex-archive/macros/latex/contrib/supported/cite/

\usepackage[dvips]{graphicx}  % Written by David Carlisle and Sebastian Rahtz
                        % Required if you want graphics, photos, etc.
                        % graphicx.sty is already installed on most LaTeX
                        % systems. The latest version and documentation can
                        % be obtained at:
                        % http://www.ctan.org/tex-archive/macros/latex/required/graphics/
                        % Another good source of documentation is "Using
                        % Imported Graphics in LaTeX2e" by Keith Reckdahl
                        % which can be found as esplatex.ps and epslatex.pdf
                        % at: http://www.ctan.org/tex-archive/info/

%\usepackage{amsmath}   % From the American Mathematical Society
                        % A popular package that provides many helpful commands
                        % for dealing with mathematics. Note that the AMSmath
                        % package sets \interdisplaylinepenalty to 10000 thus
                        % preventing page breaks from occurring within multiline
                        % equations. Use:
\usepackage{multirow}
\usepackage[left=0.71in,top=0.94in,right=0.71in,bottom=1.18in]{geometry}
\setlength{\columnsep}{0.24in}
% correct bad hyphenation here
%\hyphenation{op-tical net-works semi-conduc-tor IEEEtran}
\usepackage{subfigure}

\begin{document}
% paper title
\title{\huge Unmodelled Grasping}

% author names and affiliations
\author{\authorblockN{Lorenzo Natale}
\authorblockA{\textit{Computer Science and Artificial Intelligence}\\
\textit{Massachusetts Institute of Technology}\\
\textit{32 Vassar St. 32-380, Cambridge, MA 02139, USA}\\
\textit{lorenzo@csail.mit.edu}\\}%
\and
\authorblockN{Eduardo Torres-Jara}
\authorblockA{\textit{Computer Science and Artificial Intelligenc}\\
\textit{Massachusetts Institute of Technology}\\
\textit{32 Vassar St. 32-380, Cambridge, MA 02139, USA}\\
\textit{etorresj@csail.mit.edu}\\}}%

% sections

We describe YARP, Yet Another Robot Platform, an open-source project
that encapsulates lessons from our experience in building humanoid
robots.  The goal of YARP is to minimize the effort
devoted to infrastructure-level software development
 by facilitating code reuse, 
modularity and so maximize research-level development and collaboration. Humanoid robotics is a ``bleeding edge'' field of research, with constant flux in sensors, actuators, and 
processors.  Code reuse and maintenance is therefore a significant 
challenge. We describe the main problems we faced and the 
solutions we adopted. 
In short, the main features of YARP include support for inter-process
communication, image processing as well as a class hierarchy
to ease code reuse across different hardware platforms. YARP
is currently used and tested on Windows, Linux and QNX6 which are common 
operating systems used in robotics. 

%With YARP, we lay the ground-work for long-term
%software development. [need to review this]


\section{Introduction}

YARP is written by and for researchers in humanoid robotics, who find
themselves with a complicated pile of hardware to control with an
equally complicated pile of software.

%YARP includes modules to facilitate software development on
%humanoid robots, including abstractions for the operating system,
%image processing, physical devices, mathematical operations, etc.
%
At the time of writing, running decent visual, auditory, and tactile
perception while performing elaborate motor control in real-time
requires a lot of processor cycles. The only practical way to get
those cycles at the moment is to have a cluster of computers. Every
year what one machine can do grows, but so also do our demands~--
humanoid robots stretch the limits of current technology, and are
likely to do so for the foreseeable future.
Moreover, software is in general tight to the hardware on which it runs.
This limits modularity and code reuse which, in turn, complicates software 
development and mantainability. In the last few years we have been developing
a software platform to ease these tasks and improve the software quality on 
our robotic platforms. 
We want to reduce the effort devoted to programming to increase the 
time spent doing research. At the same time, we would like to have 
stable robotic platforms to work with.
Today YARP is a platform for long-term software 
development for applications that are real-time, computation-intensive, 
and involve interfacing with diverse and changing hardware. It is is 
successfully used on several platforms in our research Laboratories. 
The diversity of contexts on which it has been applied
show that our efforts have been somehow successful [reference table?].


\begin{table}
\centerline{\small
\begin{tabular}{|l|c|c|c|}
\hline
Robot&Laboratory&size&OS\\
\hline
Babybot&LIRA-Lab&13&Win/QNX6\\
Eurobot&LIRA-Lab&11&Win/QNX6\\
RobotCub&LIRA-Lab&3&Win\\
Obrero&MIT-CSAIL&4&Linux/OSX\\
Mertz&MIT-CSAIL&4&Linux\\
Domo&MIT-CSAIL&6&Linux\\
COG&MIT-AILab&30&QNX4/Linux\\
Kismet&MIT-AILab&12&Linux/Win/QNX4\\
\hline
\end{tabular}
}
\caption {
Robots using YARP.
}
\end{table}


\section{Motivation}

Let us now introduce the main features of YARP by describing the general lessons
we have learned and applied to YARP.
%
%
%Here are some general lessons we have learned, and apply to YARP.
%
%\begin{itemize} \pflist
%\item {\bf One processor is never enough.}

\textit{\textbf{One processor is never enough.}}
Designing a robot control system as a set of processes running on a
set of computers is a good way to work. It minimizes time spent wrestling with
code optimization, rewriting other people's code, and maximizes
time spent actually doing research.  The heart of YARP is a
communications mechanism to make writing and running such processes as
easy as possible. Even where mobility is required this is not a limiting
factor if teathers or wireless communication are practical.

%\item {\bf Modularity.} 
\textit{\textbf{Modularity.}}
Code is better maintainaed and reused if it is organized in small processes 
each one performing a simple task. In a cluster of computers some processes 
are bound to specific machines (usually when they require particular hardware 
device), but most of the times they can run on any of the available computers. 
With YARP it is easy to write processes that are location independent and 
that can run on different machines without code changes. This allows to move 
processes across the cluster at runtime to redistribute the computational 
load on the CPUs or to recover from a hardware failure. 
YARP does not contain any means of automatically allocating processes as in 
some approaches like GRID \cite{grid}. Our apporach is that of leaving this
task to the user to act sensibly and allocate the processes. The rationale is that: i)
special interface hardware is necessarily to be controlled by the appropriate piece of 
software, and ii) in an etherogeneous network of processors, faster processors might 
need to be allocated differently from slower processors. The final behavior is that of 
a sort of ``soft real-time'' parallel computation cluster without the more demanding
requirements of a real-time operating system.

\textit{\textbf{Minimal interference.}}
%Ports were designed with the two-fold goal of reducing the interactions at large between 
%the various components of the robot controller and, simultaneously, to allow efficient 
%communication between interacting parts of the system. The bottleneck in this approach
%would eventually be the available bandwidth on the network. 
As long as enough resources are available, the addition of new components 
should minimally interfere with existing processes. This is important, since often 
the actual performance of a robotic controller depends on the timing of various signals. 
While this is not strictly guaranteed by the YARP infrastructure, the problem is in 
practice alleviated computationally by allowing the inclusion of more processors to 
the network, and from the communication point of view by the buffer policy.
%isolating sub-components.

%\item {\bf Stopping hurts.}
\textit{\textbf{Stopping hurts.}}
It is a commonplace that human cycles are much, much more expensive
than machine cycles.  In robotics, it turns out that the human
cost of stopping and restarting a process can be very high.
For example, that process may interface with some
custom hardware which requires a physical reset.  
That reset many need to be carefully ordered with respect to when the
process is stopped and started.
%
There may be other dependent processes that need to be restarted in
turn, and other dependent hardware. 
%
%
%
These ordering constraints are time-consuming to satisfy.
%
YARP does its part to minimize dependencies between processes,
% so only true physically required dependencies remain.  
communication channels between processes can come and go. 
A process that is killed or dies
unexpectedly does not require processes to which it connects to be
restarted. This also simplify cooperation between people, as
it minimizes the need to synchronize development on different  
parts of the system.
%
%In complex systems, with dozens of processes and hundreds of connections, it might become
%unpractical to shut down and restart the whole system every time a module is even slightly 
%changed. YARP allowing the run-time connection of channels permits the disconnection of 
%only those parts of the system that need to be, for instance, rebuilt. 
%
%\item {\bf Humility helps.}

\textit{\textbf{Humility helps.}}
Over time, sofware for a sophisticated robot needs to 
aggregate code written by many different people in many
different contexts.  Doubtless that code will have
dependencies on various communication, image processing,
and other libraries. Very often the operating system on which
software is developed pose similar constraints. This is especially
true with code that relies heavily on the services offered by the 
operating system (such as communication, scheduling, synchronization primitives, 
and device driver interface).
%
Any component that tries to place itself ``in control'' and has strong
constraints on what dependencies are permissible will not be tolerated
for long.  It certainly cannot co-exist with another component
with the same assumption of ``dominance''. 
Although YARP offers support for communication, image processing,
interfacing to hardware etc., it is written with an {\em open world}
mindset.  We do not assume it will be the only library used, and
endeavor to be as friendly to other libraries as possible.
%
YARP allows interconnecting many modules seamlessly without subscribing
to any specific programming style, language interface, 
or demanding specifications as for instance in CORBA~\cite{vinoski97corba}
or DCOM~\cite{dcom}. Such systems, although far more powerful than YARP,
require a much tighter link between the general algorithmic code and the 
communication layer.
We have taken a more lightweight approach: YARP is a plain library linked
to uses-level code that can be used directly just by instantiating appropriate classes.
%
% and communication does not require any diversion
% of pre-existing threads. That is, YARP is a plain library linked to user-level
% code and as such migration to YARP can be easily carried out a posteriori. 
%
% Systems such as CORBA~\cite{vinoski97corba}, although far more powerful than YARP, require 
%
% adhering to well-defined interface specificiations (nothing bad as such) but consequently 
%
% a much tighter link between the general algorithmic code and the communication layer.
%
% is much 
% stricter. 
%
%
%
Finally, other programming languages can access YARP as well, provided they
have means of linking and calling C++ code. We have successfully used
YARP from within Matlab or L~\cite{brooks90behavior}.

%It is userful to reserve
%that role for the occasional poorly-designed hardware device that
%assumes it is the center of the universe.

%\end{itemize}

\noindent
%
%The OS library contains the communication facilities described in
%section \ref{sec:communication} and classes implementing
%synchronization routines (like mutexes and semaphores) and
%threads. 
%
\textit{\textbf{Exploit diversity.}}
%
Different operating systems offer different
features. Sometimes it is easier to write code to perform a given task
on a platform as opposed to another. This can happen for example if 
device drivers for a given board are provided only on a specific platform or
if an algorithm is available open source on another. We decided to reduce the
dependencies with the operating system. For this we
use ACE~\cite{ACEBook}, an open source library providing a framework for
concurrent programming across a very wide range of operating
systems. YARP inherits the portability of ACE and has indeed been used and
tested on Windows, Linux and QNX 6.



YARP's core communication model was the survivor from an early humanoid robot
controlled by a set of Motorola 68332 processors, an Apple Mac, and a loose network
of PCs running QNX, Linux, and Microsoft Windows.  Communication was a
hodge-podge of dual-port RAM, QNX message passing, CORBA, and raw
sockets.  At one point, three incompatible communication protocols
layered over QNX message passing were in use simultaneously.  This
variety was a consequence of organic growth, as developers added new
modules to the robot.  YARP began as one of the communication
protocols built on QNX message passing.  A key, defining, feature of
YARP was that it was {\em broad-minded}: it was
implemented in the form of a library which placed minimal constraints
on user code; communication resources did not need to be allocated at
any particular time or place in a program; reading messages could be
blocking, polling, or callback based, etc. This meant it could be
easily added without disturbing existing code, and communication could
be moved across to the new protocol piece by piece.

%The basic YARP module is an IPC infrastructure that supports communication across a
%network exploiting different protocols. 
%

%The mathematical library provides classes and functions to handle
%vectors and matrices, together with a few algebraic routines like
%single value decomposition, QR and LU factorization.

%More details about the image processing library and the device driver
%library can be found in the next sections.

\section{Related Work}
\label{sec:relatedwork}

When acting the robot can do a lot of things. But getting to act
is a problem. Many examples.
\section{Control of the Robot}
\label{sec:constrol}
Exploration. Get rid of the precision. Replaced by the sensing.
Smooth approach to the objects. Controllers. Trajectories. Paul's
hand localization.


\subsection{Simulation analysis}
\label{sec:analysis}

Tactile sensors. Exploration of the object.


\subsection{Design}

\subsection{Material and molding}



\section{Discussion and Conclusions}

In this paper, we showed how causality can be probed at different
levels by the robot.  Initially the environment was the body of the
robot itself, then later a carefully circumscribed interaction with
the outside world.  This is reminiscent of Piaget's distinction
between primary and secondary circular
reactions~\cite{ginsburg78piaget}.  Objects are central to interacting
with the outside world.  We raised the issue of how an agent can
autonomously acquire a working definition of objects. 

In computer vision there is much to be gained by bringing a
manipulator into the equation.  Many variants and extensions to the
experimental ``poking'' strategy explored here are possible.  For
example, a robot might try to move an arm around {\em behind} the
object.  As the arm moves behind the object, it reveals its occluding
boundary.  This is a precursor to visually extracting shape
information while actually manipulating an object, which is more
complex since the object is also being moved and partially occluded by
the manipulator.  Another possible strategy that could be adopted as a
last resort for a confusing object might be to simply hit it firmly,
in the hopes of moving it some distance and potentially overcoming
local, accidental visual ambiguity.  Obviously this strategy cannot
always be used!  But there is plenty of room to be creative here.
%
%
\ifrev
%
There are also limitations in our current implementation that could
usefully be addressed.
%
The robot itself is not mobile, so its workspace is limited.  
%
There are also many constraints on the arm that make fine 
motor control impossible -- it cannot maintain all reachable 
poses indefinitely, and there is significant noise and some 
hysteresis in its analog sensors.
%
The robot will only attempt to reach towards a target that is 
actually accessible to its arm -- not too close, not too far, 
as determined using visual disparity.  In practice, this 
means that the ideal workspace is a table in front of the 
robot, and the motor control of the robot has been 
specifically tuned to work well in that situation.
%
A simple attention system and tracking mechanism are used to 
bring the robot's attention to a target.  This phase can fail 
if the robot gets distracted by some more salient (but 
unreachable) part of the scene.
%
Objects that move together are not individually segmented.
%
And segmentation does not always succeed, due to shadows,
or strong nearby edges.
%
\fi

The robotic experiments support the view that reaching, grasping, and recognition
can be learned by following a particular ontogenetic pathway without the
intervention of an external teacher.
This pathway is consistent with and inspired by what is
known of this process in biological systems (primates/mammals).
%although this evidence is rather sparse.
%
We have endeavored to build from as few innate components as possible, to
elucidate the visual and motor challenges faced by a learning robot rather
than simply solving them by fiat.
%
%There is relatively little evidence to work with, but it is at least clear that the 
%sequence of events leading to object manipulation/recognition cannot take
%an arbitrary form unless we assume that some/many of its components are innate.
Although newborns show amazing abilities \cite{spelke-2000} such as early imitation 
\cite{meltzoff-moore-1977}, face detection, etc, there is also evidence 
that the maturation of the brain is far from complete at birth and
complex perceptual abilities require a long time to emerge \cite{kovacs00human}.
%
We have given a simple existence proof that object segmentation,
recognition and localization can develop without any prior knowledge
of visual appearance.  We have also shown that, without any prior
knowledge of the human form, the robot can identify episodes when a
human is manipulating objects that are familiar to the robot purely by
the operational similarity of the human arm and its own manipulator in
this situation.  We believe such demonstrations are important both in
their own right, and in their elucidation of a concrete series of
steps that lead to a desired behavior.  This may serve a useful
reference point from which to investigate the biological solution to
the same problem -- although it can't provide the answers, it can at
least suggests useful questions.

%
%This is important, since segmentation and recognition as usually expressed 
%suffer from a chicken-and-egg problem, where views of an object must be
%segmented before its appearance can be learned...

%We cannot claim that this is the only possible view but it is certainly one worth
%investigating. Rephrasing Berkeley we can say:
%\begin{quote}
%...objects can only be known by
%\emph{action}. Vision is subject to illusions, 
%which arise from \emph{many different} problems...
%\end{quote}
%that AI guys know far too well

%Could relate some of this to the embodied intelligence ideas
%of Brooks... particularly the working hypothesis.



\section{Acknowledgments}

YARP and ICub make heavy use of software released under free and open
licenses -- thank you world.
%
The authors would like to gratefully acknowledge contributions to YARP
from Alessandro Scalzo, Francesco Nori, Radu Bogdan Rusu, 
Alexis Maldonado, Eric Mislivec, Christopher
Prince, Charles C. Kemp, Julio Gomes, Alexandre
Bernardino, Carlos Beltran, Jonas Ruesch, Assif Mirza, Hatice
Kose-Bagci, Mike P. Blow, Lars Olsson,
Jose Gaspar, Claudio Castellini, Michael Bucko, Nelson
Gon\c calves, Marco Barbosa, Tomassino Ferrauto, Boris Duran, Mattia
Castelnovi, and Giacomo Spigler 
 (if we missed anyone, plese let us
know).  
%
The ``yarpimage'' driver mentioned in Section~\ref{sect:interop}
was written by Radu Bogdan Rusu.

We are happy to be just one set of contributors to the ICub platform
amongst the larger team of the RobotCub Partners and the RobotCub
Consortium.
%
ICub software contributers include (with some overlap with YARP):
Alessandro Scalzo, Alexandre Bernardino, Alexis Maldonado, Basilio
Noris, David Vernon, Eric Sauser, Micha Hersch, Fabio Berton, Giorgio
Metta, Jonas Hornstein, Jonas Ruesch, Lijin Aryananda, Lorenzo Natale,
Ludovic Righetti, Manuel Lopes, Paul Fitzpatrick, Julio Gomes, Plinio
Moreno, and Vadim Tikhanoff (if you are not listed, please check that
you have correctly placed a copyright and copy-policy statement in
your source code).
%
%
The simulators shown in Figure~\ref{fig:simulators} were developed 
by Ludovic Righetti (Webots-based simulator) and
Vadim Tikhanoff (ODE-based simulator).
%
Our thanks to the reviewers for their constructive feedback, which
made this a better paper.
%
The authors were supported by European Union
grant RobotCub (IST-2004-004370).





% optional entry into table of contents (if used)
%\addcontentsline{toc}{section}{Acknowledgment}

% trigger a \newpage just before the given reference
% number - used to balance the columns on the last page
% adjust value as needed - may need to be readjusted if
% the document is modified later
%\IEEEtriggeratref{8}
% The "triggered" command can be changed if desired:
%\IEEEtriggercmd{\enlargethispage{-5in}}

\bibliographystyle{plain}
\bibliography{obmanip}


\end{document}
