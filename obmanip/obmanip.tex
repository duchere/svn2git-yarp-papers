%\title{\huge Preparation of Papers in Two-Column Format\\
%for BioRob Conference Proceedings Published by
%IEEE$^{*}$\footnoterule\thanks{$^{*}$This work is partially
%supported by NSF Grant \#234567890 to M. Burns and CNSF Grant
%\#123456789 to N. Flanders}}

%% This skeleton file requires IEEEtran.cls version 1.6 or later.
%%
\documentclass[conference,a4paper]{./sty/IEEEtran}
% If the IEEEtran.cls has not been installed into the LaTeX system files,
% manually specify the path to it:
% \documentclass[conference]{../sty/IEEEtran}
\IEEEoverridecommandlockouts
\overrideIEEEmargins

% some very useful LaTeX packages include:
\usepackage{cite}
%\usepackage{cite}      % Written by Donald Arseneau
                        % V1.6 and later of IEEEtran pre-defines the format
                        % of the cite.sty package \cite{} output to follow
                        % that of IEEE. Loading the cite package will
                        % result in citation numbers being automatically
                        % sorted and properly "ranged". i.e.,
                        % [1], [9], [2], [7], [5], [6]
                        % (without using cite.sty)
                        % will become:
                        % [1], [2], [5]--[7], [9] (using cite.sty)
                        % cite.sty's \cite will automatically add leading
                        % space, if needed. Use cite.sty's noadjust option
                        % (cite.sty V3.8 and later) if you want to turn this
                        % off. cite.sty is already installed on most LaTeX
                        % systems. The latest version can be obtained at:
                        % http://www.ctan.org/tex-archive/macros/latex/contrib/supported/cite/

\usepackage[dvips]{graphicx}  % Written by David Carlisle and Sebastian Rahtz
                        % Required if you want graphics, photos, etc.
                        % graphicx.sty is already installed on most LaTeX
                        % systems. The latest version and documentation can
                        % be obtained at:
                        % http://www.ctan.org/tex-archive/macros/latex/required/graphics/
                        % Another good source of documentation is "Using
                        % Imported Graphics in LaTeX2e" by Keith Reckdahl
                        % which can be found as esplatex.ps and epslatex.pdf
                        % at: http://www.ctan.org/tex-archive/info/

%\usepackage{amsmath}   % From the American Mathematical Society
                        % A popular package that provides many helpful commands
                        % for dealing with mathematics. Note that the AMSmath
                        % package sets \interdisplaylinepenalty to 10000 thus
                        % preventing page breaks from occurring within multiline
                        % equations. Use:
\usepackage{multirow}
\usepackage[left=0.71in,top=0.94in,right=0.71in,bottom=1.18in]{geometry}
\setlength{\columnsep}{0.24in}
% correct bad hyphenation here
%\hyphenation{op-tical net-works semi-conduc-tor IEEEtran}
\usepackage{subfigure}

\begin{document}
% paper title
\title{\huge Unmodelled Grasping}

% author names and affiliations
\author{\authorblockN{Lorenzo Natale}
\authorblockA{\textit{Computer Science and Artificial Intelligence}\\
\textit{Massachusetts Institute of Technology}\\
\textit{32 Vassar St. 32-380, Cambridge, MA 02139, USA}\\
\textit{lorenzo@csail.mit.edu}\\}%
\and
\authorblockN{Eduardo Torres-Jara}
\authorblockA{\textit{Computer Science and Artificial Intelligenc}\\
\textit{Massachusetts Institute of Technology}\\
\textit{32 Vassar St. 32-380, Cambridge, MA 02139, USA}\\
\textit{etorresj@csail.mit.edu}\\}}%

% sections

\iflong
\begin{abstract}
\else
\begin{Abstract}
\fi
 
Vision and manipulation are inextricably intertwined in the primate
brain.  Tantalizing results from neuroscience are illuminating the
mixed representations used by the brain in reaching, grasping, and
object recognition.  We wish to instantiate these results in robotic
form to probe their technical advantages and verify that the
associated models are at least consistent and without lacunae.

We believe it would be missing the point to investigate this on a
platform where dextrous manipulation and sophisticated machine vision
are already implemented (if such a platform existed).

In this paper, we show how we can take a simple precursor to manipulation,
namely poking and prodding, and already realize significant advantages in
visual processing, and make enough progress to develop a system that
is functionally analogous to models coming out of neuroscience.

We show how operational concepts can actually lead to well grounded
objects.


\ifverbose
For the purposes of manipulation, we would like to know what parts of
the environment are physically coherent ensembles -- that is, which
parts will move together, and which are more or less independent.  It
takes a great deal of experience before this judgement can be made
from purely visual information.  This paper develops active strategies
for acquiring that experience through experimental manipulation, using
tight correlations between arm motion and optic flow to detect both
the arm itself and the boundaries of objects with which it comes into
contact.  We argue that following causal chains of events out from
the robot's body into the environment allows for a very natural
developmental progression of visual competence, and relate this idea 
to results in neuroscience.
\fi

\ifverbose
For the purpose of understanding development we would like to present
causality as a possible principle to frame a number of neural science
results coherently. We will show how this can lead also to an
implementation in an artificial system following the epigenetic
approach. To this purpose we will show different levels of causal
linkages, or instances of the general principle, which allow tasks of
increasing complexity to be implemented.  Action and the physical
interaction of the robot with the environment play a fundamental role.
In an ecological perspective, the role of this physical interaction
for developing categorization and object undestanding is emphasized.
\fi

%%{\bf \em
%%\iflong
%%(long version)
%%\else
%%(short version)
%%\fi
%%}

\iflong
\end{abstract}
\else
\end{Abstract}
\fi


\section{Outline}

The sad fate of most robot software

Modularity in robotics

YARP: Yet Another Robot Platform

Excising communication ``plumbing'' from code

Excising device dependencies from code

Conclusions


\section{Sad fate}

Many robot projects are ``black holes'', in terms of software.  A lot
of software gets sucked in, but very little comes out.  Once a piece
of software has been adapted to a particular robot, it takes a lot
of work to extricate it again and apply it to another.

Obviously the answer to this problem is modularity.  So there are 
now many architectures/frameworks/... for modular robot systems.
The prime concern for any such system should be that it is not
a ``black hole'' -- that once a piece of software has been adapted
to a particular framework, it takes a lot of work to extricate it
again and apply it to another.  That would be a bit self-defeating.

We study YARP from this perspective.  How sticky is resultant user
code to the robot and to the framework itself?


\section{Free and Open Source}

Useful, more malleable.

Has the pragmatic benefit that a user of the software can
modify and integrate it to their hearts content without the 
pain of dealing with opaque binaries.

Has the revolutionary benefit that the user is not trapped in the role
of being a ``consumer'' of software, but can also be a publisher of
the changes, additions, and integrative work they do in an effective
form.  This is achieved by explicitly granting far more rights to
users than they have under the law of most countries, contrasting with
agreeably with the formerly more common practice of attempting to
minimize user rights.  These rights are typically granted
conditionally; a user may only make use of these extended rights if
(for example) distributed code is always available in its most useful
original (source) form, with compatible freedoms attached to it.  This
condition seeks to balance freedoms of individuals versus benefit to
the group.  The freedom to distribute code in obscure (compiled) forms



Split between people who emphasize pragmatic concerns and those
who emphasize freedom.  Just cite the issue, no need to revisit
it here.


\section{CMake}

Open-source, deals well with various IDEs and command-line development.

Not as familiar as autoconf/automake/... etc.

Has the excellent property of being simpler than making Makefiles
or configuring a project, when external libraries are involved.

The big downside is that the language is unfamiliar and a bit ugly.
It is simple and well-documented, but quirky.  An alternative with
some similar properties, scons, uses python instead.  The ant system
uses with java also seems cleaner.  However, it gets the job
done, and has the huge advantage of not being dependent on an
external language being installed.

CMake is free and open-source, with a healthy community of 
developers.



\section{Call for publication}

As a research community, we both read and produce papers, building on
each others' work.

We also both acquire and produce software

Our software tends to die with our projects

Sad!  Software collaboration speeds things up

Research groups that all use a specific robot (Khepera, Pioneer, AIBO,
...) often form a natural software community

But each alone is a small subset of robotics

Groups developing new robots face obstacles

Differences in sensors, actuators, bodies...

Differences in processors, operating systems, libraries, frameworks,
languages, compilers...

Big barriers to software collaboration


\section{Modularity}

Constant hardware flux

Parts change rapidly

Interfaces change slowly

Lots of software grew and evolved alongside the changing hardware

Parts change rapidly

Interfaces change slowly

``Modularity'' is rewarded


\subsection{Broom}

The way parts interact can last longer than the parts themselves

E.g. an eternal broom

replace broom head

replace broom handle


\subsection{Theseus}

Long-lived software is like the Ship of Theseus

The mast gets replaced

The planks get replaced

Over time, everything may get replaced

In philosophy, this is a ``paradox of identity''

For us, it's just our job


\subsection{dark path}

The opposite of a modular system is a coupled one.

In a ``coupled'' system, changes in one part trigger changes in another.

Coupling leads to complexity

Complexity leads to confusion

Confusion leads to suffering

This is the path to the Dark Side

\subsection{modular robots}

Robot software is notoriously hardware-specific and task-specific

Both hardware and target tasks change quickly, even within the
lifetime of one project

Our humanoid robots are far more complex than one person can build and
maintain, both in terms of hardware and software

They need to be modular


\subsection{YARP}

YARP is an open-source software library  for humanoid robotics

History

An MIT / LIRA-Lab collaboration

Born on Kismet, grew on COG

With a major overhaul, now used by RobotCub consortium

Exists as an independent open source project

C++ source code


\subsection{Things we use}

CMake slide.

SWIG slide.


\subsection{What is YARP for}


Factor out details of data flow between programs from program source code

Data flow is very specific to robot platform, experimental setup,
network layout, communication protocol, etc.

Useful to keep ``algorithm'' and ``plumbing'' separate

Factor out details of devices used by programs from program source code

The devices can then be replaced over time by comparable alternatives;
code can be used in other systems


\section{Literature}

The literature of a research community both expresses its ideas, and
aids in their evolution

Published ideas are read, evaluated, and built upon

Useful advances get published

Publication of software can speed progress

Facilitates evaluating and comparing approaches

Brings new research topics into reach

Publish or perish!



\section{Related Work}
\label{sec:relatedwork}

When acting the robot can do a lot of things. But getting to act
is a problem. Many examples.
\section{Control of the Robot}
\label{sec:constrol}
Exploration. Get rid of the precision. Replaced by the sensing.
Smooth approach to the objects. Controllers. Trajectories. Paul's
hand localization.


\subsection{Simulation analysis}
\label{sec:analysis}

Tactile sensors. Exploration of the object.


\subsection{Design}

\subsection{Material and molding}



\section{Conclusions}

A bunch of stuff we should talk about...

A survey of robot development environments
\cite{kramer2007development} (p.s. rules YARP out, claim it has
no coherent program, think they were seeing YARP1).

% \cite{gerkey03player} % redundant to newer paper

% Not sure why this is here \cite{natale05developmental}

Nesnas \cite{nesnas2006claraty} - ``coping with hardware and software 
heterogeneity''.  It is useful for the list of problems it
works through (solutions are not that exciting).
%
Mentions danger of overgeneralizing interfaces, argues for 
``multi-level abstraction models, object-oriented methodologies
and design patterns''.  A bit vacuous.

Vaughan \cite{vaughan2006reusable} - this is a good player/stage
paper.

von Krogh \cite{vonkrogh2006promise} - looking at open source
software from a (management) research perspective.

Our previous comments on YARP1 \cite{metta2006yarp} which
describe its history and some usage information.

The case of embedded Linux \cite{henkel2006selective} --
interesting overlaps with robotics.

See Bill Gates article in Scientific American, January 2007

Missing infrastructure.
Need a lot of software.
Ideally (for researchers starting out) should be commoditized.


comparison with PC:

Difference: now we have the network.  Go the player route, rather than
the single IDE.  Transform from hardware to open protocols as first
step.  Then whole ecology of computation is available.

Robots aren't that special.  webcam/microphone/games...

The resources available to us are generally lower.  Smaller communities.
Less software expertise.  This could chage.







\subsection{YARP Network Scraps}

Not localization friendly, English bias at a quite low level.
Probably localization would not be practical at the network
protocol level.  Also approach to character encoding is
primitive and ad-hoc; no guarantee that encoded data
will survive in text-mode.



YARP is a free and open source project.  Since its source code is
released under a free and open license, useful parts of it can be used
by other systems, and its operation can be studied.

YARP has a large quantity of documentation (although we always need
more).  The communication protocol it uses is documented, and can be
interfaced with without using the YARP code-base.

Beyond just documenting the communication protocol, particular attention
has been devoted to make sure that that reading and writing data to a
YARP port can be done with very little effort.  YARP ports will 
accept and make connections of any of several different forms;
for a program build without the YARP code-base, it suffices
to implement just one of those connection types in order to
get basic connectivity.  If bandwidth requirements are not
excessive, the very simplest connection type can be implemented:
a very basic text-mode protocol.



\section{Acknowledgments}

YARP and ICub make heavy use of software released under free and open
licenses -- thank you world.
%
The author would like to gratefully acknowledge contributions to YARP
from Radu Bogdan Rusu, Alexis Maldonado, Eric Mislivec, Christopher
Prince, Charles C. Kemp, Francesco Nori, Julio Gomes, Alexandre
Bernardino, Carlos Beltran, Jonas Ruesch, Assif Mirza, Hatice
Kose-Bagci, Jose Gaspar, Claudio Castellini, Michael Bucko, Nelson
Gon\c calves, Marco Barbosa, Tomassino Ferrauto, Boris Duran, Mattia
Castelnovi, and Alessandro Scalzo (if we missed anyone, plese let us
know).  We'd also like to thank contributors to the ICub project.
%
The authors were supported by European Union
grant RobotCub (IST-2004-004370).





% optional entry into table of contents (if used)
%\addcontentsline{toc}{section}{Acknowledgment}

% trigger a \newpage just before the given reference
% number - used to balance the columns on the last page
% adjust value as needed - may need to be readjusted if
% the document is modified later
%\IEEEtriggeratref{8}
% The "triggered" command can be changed if desired:
%\IEEEtriggercmd{\enlargethispage{-5in}}

\bibliographystyle{plain}
\bibliography{obmanip}


\end{document}
