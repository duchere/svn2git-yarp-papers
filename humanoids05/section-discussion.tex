\section{Discussion}
We have shown results on two phases of the acquisition of sensorimotor coordination in a upper body humanoid robot. The system includes a visual attention system employing top-down and bottom-up information. The former is introduced in the system beforehand, whereas the latter is modulated by the robot's interaction with the environment. 

We have shown the importance of the interaction between the environment and the robot for learning. This was demonstrated indirectly, when the robot exploited self-produced actions to explore its own body, and directly when the robot actively explored the visual properties of the objects it grasped.

In the final experiment we start to link different actions to different objects to investigate the possibility for the robot to autonomously learn what actions are more suitable for different contexts (different objects or environment). In this way we hope to give our robot the possibility to autonomously enrich its knowledge of the world. This is not only relevant for action, but also from a perceptual point of view. Indeed, actions establish a link between events and the causes that have generated them. In other words by acting in the world an active agent can link the actions it performs with their consequences. This link can be used in two ways. For planning, to select a particular action required to bring about a desired consequence. The advantage to use such a representation is that usually goals are more clearly expressed in perceptual terms. For interpretation, to understand the meaning of an attended event. In this case the only information available is the sensorial experience associated with the event. In this case the robot can search its own experience for an event that closely match what it is observing and select the action(s) that generated them. For example the sound of an object that hits the floor can be associated with the action of dropping it. Both problems are interesting and challenging, luckily the solution to both problems appears to be tightly intertwined with sensorimotor development.

