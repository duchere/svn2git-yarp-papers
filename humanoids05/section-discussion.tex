\section{Discussion}
We have shown results on two phases of the acquisition of sensorimotor coordination in a upper body humanoid robot. The system includes a visual attention system employing top-down and bottom-up information. We have shown the importance of the interaction between the environment and the robot for learning. This was demonstrated indirectly, when the robot exploits self-produced actions to explore its own body, and directly when the robot actively explores the visual properties of the objects it grasps. In the final experiment we start to link different actions to different objects to investigate the possibility for the robot to autonomously learn what actions are more suitable for different contexes (different objects or environment). In this way we hope to give our robot the possibility to autonomously enrich its knowledge of the world. 