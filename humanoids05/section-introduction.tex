\section{Introduction}
Manipulation in robotics is a unique opportunity to study the interaction between an artificial system and the environment. We focus on manipulation not only as a means to perform useful practical tasks but also and in particular because it offers the unique possibility to investigate active learning. Active learning refers to the ability for an agent to autonomously perform and guide the exploration of the environment. In this context manipulation allows the agent to collect information about objects by performing action on them. Even very simple forms of actions like poking or pushing and object can be sufficient for this purpose (cite poking).
The sensorial experience of a humanoid robot can be quite rich if compared, for example, to the one of a simple vision system; moreover some features of objects are more naturally perceived through senses other than vision. For example the smoothness of a surface, the weight of an object and its three dimensional structure are naturally determined through tactile experience. These information are extracted by appropriate exploratory actions (procedures).
The importance of motor activity for perceptual development has been emphasized in developmental psychology(cite Piaget, Gibson). Many researchers agree on the fact that motor development during infancy determines the timing of perceptual development (for a review see cite Bushnell Child Development). For example perception of object features such as volume, hardness, texture and weight are unlikely to emerge before 6/9 months during development. Haptic sensitivity of three dimensional shapes might appear even later (12/15 months). This \'timetable\'






