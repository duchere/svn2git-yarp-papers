\section{Introduction}
\label{sect:intro}
Manipulation is a unique opportunity to study the interaction between an artificial system and the environment. We focus on manipulation not only as a means to perform useful practical tasks but also and in particular, because it offers the possibility to investigate active learning. Active learning refers, for example, to the ability of an agent to autonomously perform and guide the exploration of the environment. At a deeper level, action can contribute to change the environment in a direction that is best suited to the agent's goal, for example, to facilitate perception. In this context manipulation allows the agent to collect information about objects by performing specific actions on them \cite{metta03early}. Even very simple forms of action like poking or pushing an object can be sufficient to this purpose \cite{fitzpatrick03learning}.

The sensorial experience of a humanoid robot can be quite rich including, for example, the possibility of multi-sensory perception: some features of objects are more naturally perceived through senses other than vision. For example the smoothness of a surface, the weight of an object and its three dimensional structure are naturally determined through tactile experience. This information is extracted by appropriate exploratory actions.
The importance of motor activity for perceptual development has been emphasized in developmental psychology \cite{hofsten04motor,gibson88explore}. Many researchers agree on the fact that motor development during infancy determines the timing of perceptual development (for a review see \cite{bushnell93motor}). For example perception of object features such as volume, hardness, texture and weight are unlikely to emerge before 6/9 months of age. Haptic sensitivity of three dimensional shapes appears even later at around 12/15 months. It is perhaps illuminating that this timetable fits surprisingly well with the development of actions in infants: the ability to move the hand is required for infants to begin manipulating objects and consequently perceiving certain properties. Lederman and Klatzky \cite{lederman87hand} have shown that adults make use of stereotyped hand movement (\emph{exploratory procedures}) to determine certain properties of objects; different procedures were employed by subjects to assess different properties. The ability of infants to correctly execute these procedures can determine their ability to perceive the associated object characteristic. Infants ability to interact with objects is indeed quite limited at birth; early reaching in newborns is pretty inaccurate and only rarely result in actual contact with the object \cite{hofsten82eye-hand}. At the age of three months infants are more reliable in grasping objects, although grasp is usually with the full hand open (as in the power grasp). Only later on at 6/9 months of age infants become skilled in handling objects and grasping them with differentiated grasp types \cite{hofsten93thestructuring}. Accordingly, perception of properties like temperature, size and hardness can occur relatively early in development, whereas properties requiring more dexterous actions like texture or three dimensional shape would emerge only later on.

Similarly, we are pursuing a developmental approach for the design of a humanoid robot. Development of the robot unfolds along three phases: learning about the body, learning to interact with the environment, and learning to interpret events. In the first phase the robot learns properties of its body, which allow recognizing and controlling movement. For example, the robot controls reaching movements by first learning the weight of its arm and to recognize the hand. These abilities are used in the next phase to initiate the interaction with the environment and to learn about it. The robot begins this exploration by reaching for objects and learning their physical properties when it manages to grasp them \cite{natale04learning,natale05exploring,torres-jara05tapping}. The robot's experience acquired during this interaction is used in the third phase to interpret events around the robot by matching expectations and perceptions (as for example in \cite{metta03early}). We focus here on the two first phases: learning about the body and learning to interact with the environment. We describe how the robot builds an internal model of its hand which allows localizing it in the visual scene. The hand internal model is used to direct gaze towards the hand and to learn an inverse model which can be used to control how to reach a point in space. The robot uses these abilities to build a visual model of the objects it happens to grasp. The robot's ability to interact with the environment influences the visual attention system; the visual model of the object grasped by the robot, in fact, is then used as a top-down primer during the search of graspable objects.

The rest of the paper is organized as follow. Section \ref{sect:robot} describes the robotic platform. Section \ref{sect:vision} describes the attention system of the robot, the used model of objects and the method to extract three-dimensional information about objects. Section \ref{sect:body} describes how the robot acquires its motor skills. Section \ref{sect:exp} presents an experiment where we show how the modules described in the paper are integrated in a complete behavioral system. Finally in Section \ref{sect:conclusion} we draw the conclusions.
