\section{Introduction}
Manipulation is a unique opportunity to study the interaction between an artificial system and the environment. We focus on manipulation not only as a means to perform useful practical tasks but also and in particular because it offers the unique possibility to investigate active learning. Active learning refers to the ability for an agent to autonomously perform and guide the exploration of the environment. In this context manipulation allows the agent to collect information about objects by performing action on them. Even very simple forms of actions like poking or pushing and object can be sufficient to this purpose \cite{fitzpatrick03learning}.

The sensorial experience of a humanoid robot can be quite rich if compared, for example, to the one of a simple vision system; moreover some features of objects are more naturally perceived through senses other than vision. For example the smoothness of a surface, the weight of an object and its three dimensional structure are naturally determined through tactile experience. These information are extracted by appropriate exploratory actions (procedures).
The importance of motor activity for perceptual development has been emphasized in developmental psychology(cite Piaget, Gibson). Many researchers agree on the fact that motor development during infancy determines the timing of perceptual development (for a review see \cite{bushnell93motor}). For example perception of object features such as volume, hardness, texture and weight are unlikely to emerge before 6/9 months during development. Haptic sensitivity of three dimensional shapes might appear even later (12/15 months). This timetable fits well with development of actions in infants. The idea is that the ability to move the hand is required for infants to begin handling objects and perceiving certain properties. Lederman and Klatzky have shown that adults make use of stereotyped hand movement (\emph{exploratory procedures}) to determine certain properties of objects; different procedures are employed by subjects to assess different properties \cite{lederman87hand}. The ability of infants to correctly execute these procedures can determine their ability to perceive the associated characteristic. Infants ability to interact with objects is indeed quite limited at birth; early reaching in newborns is pretty inaccurate and only rarely result in actual contact with the object \cite{hofsten82eye-hand}. At the age of three months infants are more reliable in grasping objects, although grasp is usually with the full hand open (as in the power grasp). Only later on at 6/9 months of age infants become skilled in handling objects and grasping them with differentiated grasp types \cite{hofsten93thestructuring}. Accordingly, perception of properties like temperature, size and hardness can occur relatively early in development, whereas properties requiring more dexterous actions with the hand like texture or three dimensional shape would emerge only later on.

We are pursuing a deveolpmental approach for the design of a humanoid robot. Development for the robot unfolds by following three phases: learning about the body, learning to interact, learning to interpret. In the first phase the robot learns properties of its own body allowing it to recognize and control it. For example the robot learns the weight of its arm and to recognize the hand. These abilities are used in the next phase to initiate interaction with the environment and learn about it. The robot begins reaching for objects and learns some properties of the objects it manages to grasp \cite{natale04learning,natale05exploring,torres-jara05tapping}. The robot experience of interaction is used in the third phase to interpret events around the robot by matching expectations and perceptions (poking/pushing). We focus here on the two initial phases: learning about the body and learning to interact. We describe how the robot tunes an internal model of its hand which allows it to localize it in the visual scene. The hand internal model is used to direct gaze towards the hand and to learn to reach a point in space with the arm. The robot uses these abilities to build a visual model of the objects it grasps. The robot's ability to interact with the environment influences the visual attention system; the visual model of the object grasped by the robot, in fact, is used as a top-down primer during an object search task.

The rest of the paper is organized as follow. Section 2 describes the robotic platform. Section 3 describes the attention system of the robot and introduces the visual features used to build the object model. Section 4 describes how the robot acquires its motor skills. Section 5 presents an experiment were we show how the modules described in the paper are integrated together in the robot. Finally in Section 6 we draw the conclusions.
