This paper desribes a developmental sequence that allows a humanoid robot to learn about its own body and the environment. We acquiped the robot with an initial set of motor and perceptual competencies ranging from simple reflex to more sophisticated visual routines providing a bottom-up attention system. This gives the robot an initial form of sensorimotor coordination sufficient to initiate interaction with the environment. This in turn allows the robot to improve its motor and perceptual skills by first constructing a ``body-schema'' and later on learning about the objects in the world. The body-schema allows the robot to properly control the body to fixate, reach and touch objects in the workspace. This interaction is used to form a visual model of the objects grasped by the robot which modulate the attention system in a top-down way. In the final experiment we show an initial effort to study the acuisition of object affordances. We discuss the importance of sensorimotor coordination as a required step not only for the control of action but also, and more importantly, for perceptual development.

