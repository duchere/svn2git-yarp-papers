This paper describes a developmental sequence that allows a humanoid robot to learn about the shape of its body and successively about certain parts of the environment. We equipped the humanoid robot with an initial set of motor and perceptual competencies ranging from simple stereotyped actions to more sophisticated visual routines providing a bottom-up attention system. This initial form of sensorimotor coordination is sufficient to initiate the interaction with the environment and allows the robot to improve its motor and perceptual skills by first constructing a ``body-schema'' and later by learning about objects. The body-schema allows controlling movements to fixate, reach and touch objects in the environment. The interaction is further used to form a visual model of the objects grasped by the robot which eventually modulate the attention system in a top-down way. In another experiment we show an initial effort to study the acquisition of object affordances. We discuss the importance of sensorimotor coordination as a required step not only for the control of action but also, and more importantly, for perceptual development.