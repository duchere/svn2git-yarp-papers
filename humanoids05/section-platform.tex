\section{The robotic platform}
\label{sec-platform}
The experiments reported in this paper were performed by using an upper torso humanoid robot called Babybot (Figure~\ref{fig-platform}). The Babybot is an upper torso humanoid robot consisting of a head, arm and a hand. The head has five degress of freedom, and is equipped with two cameras, microphones and a gyro. The cameras pan indipendenly and tilt around a common axis; the remaining degrees of freedom of the head allows it to pan and tilt at the level of the neck. The arm is an industrial PUMA 260 manipulator. The hand is attached to the arm end point and a total of 16 degrees of freedom actuated only by 6 motors. The five fingers of the hand are thus largely underactuated: the thumb and index fingers are controlled independently by two motors each, whereas the remaining two motors are connected to the middle, ring and small fingers which form a single virtual joint. The couling between each joint and the motors is achieved by means of springs which give the hand a certain degree of compliance and elasticity. Magnetic potentiometers provide position and force feedback at each joint whereas force sensing resistors on the palm and fingers provide tactile feedback (see Figure~\ref{fig-platform}). A more detailed description of the hand can be found here \cite{natale04thesis}.

\begin{figure}
\centering
\includegraphics[width=3in]{platform}
\caption{The robotic platform: The Babybot}
\label{fig-platform}
\end{figure}
