\clearpage
\newpage


\section{Known carriers}



\begin{tabular}{|l|p{7cm}|}
\hline
\multicolumn{2}{|c|}{YARP carriers} \\
\hline
tcp & Regular tcp \\
fast\_tcp & Variant that drops flow control \\
udp & UDP \\
http & Basic web interface \\
mcast & Multicast - avoid repeating the same data going
to many clients ``on the wire''  \\
local & local \\
mpi & delegate to MPI (Daniel Krieg) \\
xmlrpc & translate messages into XML/RPC compatible form \\
tcpros & interoperate with ROS publisher/subscribers \\
mjpeg & receive/transmit images in mjpeg-over-http format \\
text & send messages in human readable plain text form \\
shmem & use shared memory \\
\hline
\end{tabular}

Similarly, many devices.


\section{YARP plus/minus}


\section{Robot evolution}

Robot technology and information technology in general is a
fast-moving target.  With YARP, we've taken the following steps to
help our users lead:

\begin{itemize}

\item YARP's implementations of different connection types, called
carriers, are designed as replaceable plugins.  New network types can
be accommodated through new carriers.  New versions of a particular
carrier can be evolved over time, and co-exist with older versions,
without sudden breaks of backwards compatibility.  Interoperation
with other middleware or data sources/sinks can be systematically
added.

\item YARP strongly encourages users to access devices via sets of
interfaces that allow for similar evolution.

\item We rely on carefully chosen dependencies, namely ACE and CMake,
to help us keep up with churn in operating system releases, IDEs, and
other parts of our users' toolchains.

\end{itemize}

\section{Scrap}


Ideally, users of robot middleware would cleanly separate out their
algorithmic work from the messy plumbing associated with particular
robot setups.  But in practice, it is quite uncommon for users to do
this, and it causes problems later as setups change or they try to
collaborate with others.  YARP tries to save users from themselves
by reducing the problems it causes.  Specifically:

\begin{itemize}

\item No constraint on future OS and IDE.

\item LPGL licensing.  Not complete freedom, but not bad.

\item A collaborator need not be using YARP.  It is practical to
communicate with a YARP-using program without using YARP.

\item If a collaborator is willing to link extra libraries in their
programs, they can use YARP without disturbing their existing
middleware.

\end{itemize}

It is easy to interoperate with YARP-using programs
without yourself necessarily having to use YARP. 
YARP is written in C++, with a core that uses
no external libraries, not even the standard template libraries, with
the exception of a small portion of ACE for portability (and this
portion can easily be embedded). YARP is free and open.  The
core YARP libraries are released under the LGPLv2.1.


\section{Acing TELNET}

Some keys to acing the TELNET test:

\begin{itemize}

\item TCP connections should be supported.  
  This is true of most middleware operating on regular
  networks. Some just support UDP, so this test doesn't
  make sense for them.

\item The basic communication model shouldn't stray too far from the
  notion of making ``connections'' to a named destination.

\item The direction in which a connection is initiated shouldn't
  determine the direction of data flow.

\item Human readable/writable.

\end{itemize}




\section{The origin of YARP}


YARP worked very well on QNX, and decently on Linux.  For Microsoft
Windows, Macs, and other platforms an important development in YARP's
history was to take on a dependency on the ACE library [ref].  This
happened around 2003, with the immediate goal of simplifying the
addition of multicast support.  ACE has proven its worth to us many
times over for cross-platform networking, although we needed
to carefully keep any reference to ACE out of public header files
in order to avoid inheriting some of its less desirable properties
(a somewhat unstable API, and constraints on header file inclusion
order).

When YARP grew an image processing library, care was taken to make the
data structures compatible with the IPL library (a non-free image
processing library from Intel).  IPL became the seed for OpenCV, which
also remained compatible with IPL. YARP has therefore somewhat
accidentally played nicely with OpenCV, which has grown to be a very
popular library, from the start.

YARP developers were early and enthusiastic adopters of CMake
\cite{fitzpatrick10cmaking}.  In 2006, we happily dumped
earlier custom build scripts in favor of CMake project descriptions,
and never looked back.  CMake was this missing piece for having a
truly comfortable cross-platform experience.

About the same time, we started using SWIG [ref] to support languages
other than C++.

Use of CMake and SWIG has become common in open source libraries.
The help users exploit software on many operating systems and 
languages.  By keeping YARP as ``just a library,'' there is no
obstacle to doing the same with YARP.  But since YARP facilitates
intercommunication, this has the powerful effect of allowing 
programs written in all these environments to communicate with
each other, and easily.



\section{Building on YARP}

Others have built on YARP.


For YARP, we try to make it easy to users to redirect data streams to
non-YARP based programs.  This lowers the cost of ``boundary
problems'' such as hooking up an unsupported device, computer,
network, or collaboration between people from different planets
(e.g. computer science versus developmental psychology versus
neuroscience).

The protocol used for a YARP connection is decided at connection-time.
The initiator of the connection is free to choose from a wide set of
protocols, and so can optimize for simplicity (using a plain text
protocol), speed (udp), scalability (multicast), etc.

A YARP protocol need not be specific to YARP.  For example, YARP
supports mjpeg-over-http.  This means that a YARP image source can be
viewed directly from a browser without bridging, or a YARP image sink
can receive data directly from a jpeg-streaming IP camera without
bridging.  YARP also supports XML/RPC, so can connect to certain
websites (or act as a webserver).

YARP connections can be initiated by either the "sender" or
"receiver", since the logical flow of data can be freely reversed.
This is important for supporting a wide range of protocols, which may
be "pull" or "push" in nature.  It also makes connecting to YARP
programs without using the YARP libraries even easier.  With other
middleware, you get stuck having to make a server for at least one of
the directions of data flow.

A YARP network is designed to be usable without YARP.  Examples of
making YARP connections without using the YARP library are available
in C and Python.


So in YARP, marshalling/demarshalling for user classes
is not automated.  This works
just fine for streams of data (images, vectors, nested lists etc.)
where users are likely to find built-in classes in YARP with
marshalling already taken care of.  It is, however, somewhat tedious
for RPC-style communication representing function call names,
arguments, and replies.  

The {\tt roscore} middleware \cite{quigley2009ros} is an
interesting alternative to YARP.  
It is part of the integrated ROS ``Robot Operating
System.''  ROS developers clearly understand in principle the
importance of having a thin middleware, but in practice seem to have
produced a rather tightly-coupled architecture.  This is good in that
it gets a lot closer to a complete solution for robot software than
anyone else has achieved.  It is bad in that it seems likely to create
a lot of re-writing churn in user code over the coming years, as
environments and assumptions change.

This
flexibility means that upgrades don't have to be ``big bang''
events, but can be gradual since new and old interfaces/protocols can
live side by side.


 having to change that program's
peers at the same time.  Upgrades don't have to be ``big bang''
events, but can be gradual since new and old interfaces/protocols can
live side by side.

  Having such 
a project is a way to keep themselves honest, and avoid the 
temptation of using control of data as a way to lock people in to
using their services.

Applying this test to
a middleware shows
to see if YARP-based programs are inherently easy or hard to
communicate with across the network, without the use of the YARP
libraries or programs.  This is an important question for
interoperation.

 in
developer time than one might think, given the variety of languages,
operating systems, development environments, and
versions-of-everything in use.


key to making
connecting to YARP programs without using the YARP libraries easy.
With other middleware, you can get stuck having to make a server for
