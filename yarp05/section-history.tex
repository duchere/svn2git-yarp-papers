
\section{History}

(what about the prehistory: the ``pizza boxes'')

YARP was born on the humanoid robot Kismet.  Kismet was controlled by
a set of Motorola 68332 processors, an Apple Mac, and a loose network
of PCs running QNX, Linux, and Microsoft Windows.  Communication was a
hodge-podge of dual-port RAM, QNX message passing, CORBA, and raw
sockets.  At one point, three incompatible communication protocols
layered over QNX message passing were in use simultaneously.  This
variety was a consequence of organic growth, as developers added new
modules to the robot.  YARP began as a one of the communication
protocols built on QNX message passing.  A key, defining, feature of
YARP was that it is designed to be {\em broad-minded}: it was
implemented in the form of a library which placed minimal constraints
on user code; communication resources did not need to be allocated at
any particular time or place in a program; reading messages could be
blocking, polling, or callback based, etc. This meant it could be
easily added without disturbing existing code, and communication could
be moved across to the new protocol piece by piece (if that were
necessary at all).  An image processing library grew out
of a corresponding hodge-podge of vision code which was
similarly broad-minded, and easy to insinuate into the system.

Having many incompatible communication protocols and image 
processing libraries sounds undesirable, and in the 
steady state, it is.  But in a development environment
it is nice to have the option not to choose when to
rewrite, rather than being forced into it.




