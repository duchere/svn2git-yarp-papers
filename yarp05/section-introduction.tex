\section{Introduction}

YARP is written by and for researchers in humanoid robotics, who find
themselves with a complicated pile of hardware to control with an
equally complicated pile of software.
%
At the time of writing, running decent visual, auditory, and tactile
perception while performing elaborate motor control in real-time
requires a lot of processor cycles. The only practical way to get
those cycles at the moment is to have a cluster of computers. Every
year what one machine can do grows, but so also do our demands~--
humanoid robots stretch the limits of current technology, and are
likely to do so for the foreseeable future.
%

YARP is a platform for long-term software development for applications
that are real-time, computation-intensive, and involve interfacing 
with diverse and changing hardware.
%
Here are some general lessons we have learned, and apply to YARP:

\begin{itemize} \pflist

\item {\bf Humility helps.}
%
Over time, sofware for a sophisticated robot needs to 
aggregate code written by many different people in many
different contexts.  Doubtless that code will have
dependencies on various communication, image processing,
and other libraries.
%
Any component that tries to place itself ``in control'' and has strong
constraints on what dependencies are permissible will not be tolerated
for long.  It certainly cannot co-exist with another component
with the same assumption of ``dominance''.  It is userful to reserve
that role for the occasional poorly-designed hardware device that
assumes it is the center of the universe.



\item {\bf One processor is never enough.}
%
Designing a robot control system as a set of processes running on a
set of computers is a good way to work, especially in a basic research
environment where if mobility is required, tethers or wireless
communication are practical.  It minimizes time spent wrestling with
code optimization, rewriting other people's code, etc., and maximizes
time spent actually doing research.  The heart of YARP is a
communications mechanism to make writing and running such processes as
easy as possible.


\item {\bf Stopping hurts.}
%
It is a commonplace that human cycles are much, much more expensve
than machine cycles.  In robotics, it turns out that the human
cost of stopping and restarting a process can be very high.
For example, that process may interface with some
custom hardware which requires a physical reset.  
That reset many need to be carefully ordered with respect to when the
process is stopped and started.
%
There may be other dependent processes that need to be restarted in
turn, and other dependent hardware.
%
These ordering constraints are time-consuming to satisfy.
%
YARP does its part to minimize dependencies between processes, so only
true physically required dependencies remain.  Communication channels
between processes can come and go.  A process that is killed or dies
unexpectedly does not require processes to which it connects to be
restarted.


\end{itemize}



\noindent
Although YARP offers support for communication, image processing,
interfacing to hardware etc., it is written with an {\em open world}
mindset.  We do not assume it will be the only library used, and
endeavor to be as friendly to other libraries as possible.

YARP includes a few modules to facilitate software development on
humanoid robots. They consist in the following libraries:

\begin{itemize}
\item{OS lib}
\item{mathematical lib}
\item{image processing lib}
\item{device drivers lib}
\end{itemize}

The OS library contains the communication facilities described in
section \ref{sec:communication} and classes implementing
synchronization routines (like mutexes and semaphores) and
threads. The OS library is built on top of ACE (Adaptive Communication
Environment, http://www.cs.wustl.edu/~schmidt/ACE.html or cite the
book), which is an open source library providing a framework for
concurrent programming across a very wide range of operating
systems. YARP inherits this portability and has indeed been used and
tested on Windows, Linux and QNX 6.

The mathematical library provides classes and functions to handle
vectors and matrices, together with a few algebraic routines like
single value decomposition, QR and LU factorization.

More details about the image processing library and the device driver
library can be found in the next sections.
