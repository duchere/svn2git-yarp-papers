\section{Introduction}

YARP is written by and for researchers in robotics, particularly
humanoid robotics, who find themselves with a complicated pile of
hardware to control with an equally complicated pile of software. At
the time of writing (2005), running decent visual, auditory, and
tactile perception while performing elaborate motor control in
real-time requires a lot of computation. The easiest and most scalable
way to do this right now is to have a cluster of computers. Every year
what one machine can do grows, but so do our demands. YARP is a set of
tools we have found useful for meeting our computational needs for
controlling various humanoid robots.

\subsection*{Stopping the pain}

YARP is designed with robotics research groups in mind.  Here are the
assumptions we make, distilled from our own repeated experience.
We express the assumptions in terms of {\em pain} -- situations
we would like to avoid.

\begin{itemize}

\item {\bf It hurts to be cramped.}
%
Designing a robot control system as a set of processes running on
  a set of computers is a good way to work, especially if the robot
  is not mobile.  It minimizes time spent wrestling with code
  optimization, rewriting other people's code, etc., and maximizes
  time spent actually doing research.
  The heart of YARP is a communications mechanism to make writing
  and running such processes as easy as possible.


\item {\bf It hurts to stop.}
%
The longer a process can stay running, the better.  Restarting
  processes often has a cost.  For example, it may talk to some
  custom hardware which requires a physical reset.  There may
  be other dependent processes that need to be restarted in turn.

  YARP does its part to minimize dependencies between processes.
  Communication channels between processes can come and go.
  A process that is killed or dies unexpectedly does not
  require processes to which it connects to be restarted


\item {\bf You're not special.}
%
Humility is important.  There is a lot going on.
  YARP does not attempt to take over control of the processes,
  but rather leaves process control completely in the hands of
  the OS.


\end{itemize}


+ Multiple people, shared resources, maximally independent projects.

+ We make it a rule that a process should never need to be restarted
  because of anything YARP does

+ Processes come and go.

+ Make streaming communication easy.

+ Name to ...




YARP client code is easy.

The package duration paradigm.

Giving control over buffer policy, but avoiding making user
think about it.

Gotchas:
+ pointers in structures
+ OnRead doesn't often get called.
+ sometimes expect both that all messages will get through, and
  that messages will get dropped for timeliness.

YARP

The principles of YARP:

+ Politeness.

+ Openness.

+ Playing well with others.

Motivations.

What type of robots we're dealing with.

History.


\subsection{Bozo the clown}

