\section{Introduction}

YARP is written by and for researchers in robotics, particularly
humanoid robotics, who find themselves with a complicated pile of
hardware to control with an equally complicated pile of software. 
%
At the time of writing, running decent visual, auditory, and tactile
perception while performing elaborate motor control in real-time
requires a lot of computation. The easiest and most scalable way to do
this at the moment is to have a cluster of computers. Every year what
one machine can do grows, but so do our demands~-- especially for
research groups which by definition operate at the limits of current
technology. YARP is a set of tools we have found useful for meeting
our computational needs for controlling various humanoid robots.

Although YARP offers support for communication, image processing,
interfacing to hardware etc., it is written with an {\em open world}
mindset.  We do not assume it will be the only library used, and
endeavor to be as friendly to other libraries as possible.


YARP is designed with robotics research groups in mind. 
%
It is targetted primarily at supporting long-term software development.
%
Here are some general lessons learned:

\begin{itemize} \pflist

\item {\bf Humility helps.}
%
Over time, sofware for a sophisticated robot needs to 
aggregate code written by many different people in many
different contexts.  Doubtless that code will have
dependencies on various communication, image processing,
and other libraries.
%
Any component that tries to place itself ``in control'' and has strong
constraints on what dependencies are permissible will not be tolerated
for long.  It certainly cannot co-exist with another component
with the same assumption of ``dominance''.  In robotics, it is useful to
reserve the ...


\item {\bf One processor is never enough.}
%
Designing a robot control system as a set of processes running on
  a set of computers is a good way to work, especially if the robot
  is not mobile.  It minimizes time spent wrestling with code
  optimization, rewriting other people's code, etc., and maximizes
  time spent actually doing research.
  The heart of YARP is a communications mechanism to make writing
  and running such processes as easy as possible.


\item {\bf Stopping hurts.}
%
The longer a process can stay running, the better.  Restarting
  processes often has a cost.  For example, it may talk to some
  custom hardware which requires a physical reset.  There may
  be other dependent processes that need to be restarted in turn.

  YARP does its part to minimize dependencies between processes.
  Communication channels between processes can come and go.
  A process that is killed or dies unexpectedly does not
  require processes to which it connects to be restarted



\end{itemize}

