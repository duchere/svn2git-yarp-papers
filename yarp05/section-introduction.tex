\section{Introduction}

YARP is written by and for researchers in robotics, particularly
humanoid robotics, who find themselves with a complicated pile of
hardware to control with an equally complicated pile of software. 
%
At the time of writing (2005), running decent visual, auditory, and
tactile perception while performing elaborate motor control in
real-time requires a lot of computation. The easiest and most scalable
way to do this right now is to have a cluster of computers. Every year
what one machine can do grows, but so do our demands. YARP is a set of
tools we have found useful for meeting our computational needs for
controlling various humanoid robots.

\subsection*{Stopping the pain}

YARP is designed with robotics research groups in mind.  
%
%
It is targetted primarily at supporting software development.
%
%
Here are the assumptions we make, distilled from our own repeated
experience.  
%
%We express the assumptions in terms of {\em pain} --
%situations we would like to avoid.

\begin{itemize}

\item {\bf Humility is key.}
%
Over time, sofware for a sophisticated robot needs to 
aggregate code written by many different people in many
different contexts.  No doubt that code will have
dependencies on various communication, image processing,
and other libraries.
%
YARP needs to play well with these other libraries.
It should not try to ``take control'' of processes,
hardware management, or any kind of service.


\item {\bf Computing power should be scalable.}
%
Designing a robot control system as a set of processes running on
  a set of computers is a good way to work, especially if the robot
  is not mobile.  It minimizes time spent wrestling with code
  optimization, rewriting other people's code, etc., and maximizes
  time spent actually doing research.
  The heart of YARP is a communications mechanism to make writing
  and running such processes as easy as possible.


\item {\bf Never stop.}
%
The longer a process can stay running, the better.  Restarting
  processes often has a cost.  For example, it may talk to some
  custom hardware which requires a physical reset.  There may
  be other dependent processes that need to be restarted in turn.

  YARP does its part to minimize dependencies between processes.
  Communication channels between processes can come and go.
  A process that is killed or dies unexpectedly does not
  require processes to which it connects to be restarted



\end{itemize}


\subsection*{Zone of proximal development}

(this section is very vague)

For infants, the zone of proximal development is the
difference between what they can accomplish on their
own compared to what they can accompish with an
adult's support [CITE].

By loose analogy, we label a robot control system's ``zone of proximal
development'' to be the set of modules being actively added or worked
on by the programmer, against a background of pre-existing, operating
modules.  

YARP helps insulate existing modules from changes in this zone,
and leaves them in a good form for when the zone moves on.
