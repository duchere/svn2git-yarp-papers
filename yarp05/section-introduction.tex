\section{Introduction}

YARP is written by and for researchers in humanoid robotics, who find
themselves with a complicated pile of hardware to control with an
equally complicated pile of software.
%
At the time of writing, running decent visual, auditory, and tactile
perception while performing elaborate motor control in real-time
requires a lot of processor cycles. The only practical way to get
those cycles at the moment is to have a cluster of computers. Every
year what one machine can do grows, but so also do our demands~--
humanoid robots stretch the limits of current technology, and are
likely to do so for the foreseeable future.
%

YARP is a platform for long-term software development for applications
that are real-time, computation-intensive, and involve interfacing 
with diverse and changing hardware.
%
Here are some general lessons we have learned, and apply to YARP:

\begin{itemize} \pflist
\item {\bf One processor is never enough.}
%
Designing a robot control system as a set of processes running on a
set of computers is a good way to work, especially in a basic research
environment where if mobility is required, tethers or wireless
communication are practical.  It minimizes time spent wrestling with
code optimization, rewriting other people's code, etc., and maximizes
time spent actually doing research.  The heart of YARP is a
communications mechanism to make writing and running such processes as
easy as possible.

\item {\bf Modularity.}
Code is better maintainaed and reused if it is organized in small processes each one performing a simple task. This approach is convenient during development because it allows to stop and restart small pieaces of the system at the time. In some cases processes are bound to specific machines (usually when they require particular hardware device), but most of the times they can run on any of the available machines. With YARP it is easy to write processes that are location independent and that can run on different machines without changes in the code. This allows to move the processes across the cluster at runtime to redistribute the computational load on the CPUs or to recover from a hardware failure.

\item {\bf Stopping hurts.}
%
It is a commonplace that human cycles are much, much more expensve
than machine cycles.  In robotics, it turns out that the human
cost of stopping and restarting a process can be very high.
For example, that process may interface with some
custom hardware which requires a physical reset.  
That reset many need to be carefully ordered with respect to when the
process is stopped and started.
%
There may be other dependent processes that need to be restarted in
turn, and other dependent hardware.
%
These ordering constraints are time-consuming to satisfy.
%
YARP does its part to minimize dependencies between processes,
% so only true physically required dependencies remain.  
communication channels
between processes can come and go. A process that is killed or dies
unexpectedly does not require processes to which it connects to be
restarted. This also simplify cooperation between people, as
it minimizes the interfearences between different parts of the system.

\item {\bf Humility helps.}
%
Over time, sofware for a sophisticated robot needs to 
aggregate code written by many different people in many
different contexts.  Doubtless that code will have
dependencies on various communication, image processing,
and other libraries. Very often the operating system on which
software is developed pose similar constraints. This is especially
true with code that relies heavily on the services offered by the 
operating system (such as communication, scheduling, synchronization primitives, 
and device driver interface).
%
Any component that tries to place itself ``in control'' and has strong
constraints on what dependencies are permissible will not be tolerated
for long.  It certainly cannot co-exist with another component
with the same assumption of ``dominance''.  
%It is userful to reserve
%that role for the occasional poorly-designed hardware device that
%assumes it is the center of the universe.







\end{itemize}



\noindent
Although YARP offers support for communication, image processing,
interfacing to hardware etc., it is written with an {\em open world}
mindset.  We do not assume it will be the only library used, and
endeavor to be as friendly to other libraries as possible.

YARP includes modules to facilitate software development on
humanoid robots, including abstractions for the operating system,
image processing, physical devices, mathematical operations, etc.
%
%The OS library contains the communication facilities described in
%section \ref{sec:communication} and classes implementing
%synchronization routines (like mutexes and semaphores) and
%threads. 
%
YARP uses ACE~\cite{ACEBook}, an open source library providing a framework for
concurrent programming across a very wide range of operating
systems. YARP inherits this portability and has indeed been used and
tested on Windows, Linux and QNX 6.

%The mathematical library provides classes and functions to handle
%vectors and matrices, together with a few algebraic routines like
%single value decomposition, QR and LU factorization.

%More details about the image processing library and the device driver
%library can be found in the next sections.
