
\section{Introduction}

YARP is written by and for researchers in humanoid robotics, who find
themselves with a complicated pile of hardware to control with an
equally complicated pile of software.
%
%YARP includes modules to facilitate software development on
%humanoid robots, including abstractions for the operating system,
%image processing, physical devices, mathematical operations, etc.
%
Achieving visual, auditory, and tactile
perception while performing elaborate motor control in real-time
requires a lot of processor cycles. The only practical way to get
those cycles at the moment is to have a cluster of computers. Every
year the capabilities of an individual machine grows, but so also do our demands~--
humanoid robots stretch the limits of current technology, and are
likely to do so for the foreseeable future.
Moreover, software easily becomes entangled with the hardware on which it runs and the
devices that it controls.
This limits modularity and code reuse which, in turn, complicates software 
development and mantainability. In the last few years we have been developing
a software platform to ease these tasks and improve the software quality on 
our robot platforms. 
We want to reduce the effort devoted to infrastructure-level programming to increase the 
time spent doing research-level programming. At the same time, we would like to have 
stable robot platforms to work with.
Today YARP is a platform for long-term software 
development for applications that are real-time, computation-intensive, 
and involve interfacing with diverse and changing hardware. It is
successfully used on several platforms in our research laboratories
(see Table~\ref{tab:robots}).
%
%The diversity of contexts on which it has been applied
%show that our efforts have been somehow successful [reference table?].

We begin the paper by summarizing the lessons we have learned over
the years while working on various robots, some of which are 
software engineering commonplaces and some of which are more
specific to long-term robotic research.  The bulk of the paper 
discusses the communication model supported by YARP.  We then
briefly mention other components of the library, in particular
image processing and device drivers.

\begin{table}
\centerline{\small
\begin{tabular}{|l|c|c|c|}
\hline
Robot&Laboratory&size&OS\\
\hline
Babybot&LIRA-Lab&13&Win/QNX6\\
Eurobot&LIRA-Lab&11&Win/QNX6\\
RobotCub&LIRA-Lab&3&Win\\
Obrero&MIT-CSAIL&4&Linux/OSX\\
Mertz&MIT-CSAIL&4&Linux\\
Domo&MIT-CSAIL&6&Linux\\
COG&MIT-AILab&30&QNX4/Linux\\
Kismet&MIT-AILab&12&Linux/Win/QNX4\\
\hline
\end{tabular}
}
\caption {
Robots using YARP.
}
\label{tab:robots}
\end{table}


\section{Motivation}

Let us now introduce the main features of YARP by describing the general lessons
we have learned and applied to YARP.
%
%
%Here are some general lessons we have learned, and apply to YARP.
%
%\begin{itemize} \pflist
%\item {\bf One processor is never enough.}

\textit{\textbf{One processor is never enough.}}
Designing a robot control system as a set of processes running on a
set of computers is a good way to work. It minimizes time spent wrestling with
code optimization, rewriting other people's code, and maximizes
time spent actually doing research.  The heart of YARP is a
communications mechanism to make writing and running such processes as
easy as possible. Even where mobility is required this is not a limiting
factor if teathers or wireless communication are practical.

%\item {\bf Modularity.} 
\textit{\textbf{Modularity.}}
Code is better maintainaed and reused if it is organized in small processes 
each one performing a simple task. In a cluster of computers some processes 
are bound to specific machines (usually when they require particular hardware 
device), but most of the times they can run on any of the available computers. 
With YARP it is easy to write processes that are location independent and 
that can run on different machines without code changes. This allows to move 
processes across the cluster at runtime to redistribute the computational 
load on the CPUs or to recover from a hardware failure. 
YARP does not contain any means of automatically allocating processes as in 
some approaches like GRID \cite{grid}. Our apporach is that of leaving this
task to the user to act sensibly and allocate the processes. The rationale is that: i)
special interface hardware is necessarily to be controlled by the appropriate piece of 
software, and ii) in an etherogeneous network of processors, faster processors might 
need to be allocated differently from slower processors. The final behavior is that of 
a sort of ``soft real-time'' parallel computation cluster without the more demanding
requirements of a real-time operating system.

\textit{\textbf{Minimal interference.}}
%Ports were designed with the two-fold goal of reducing the interactions at large between 
%the various components of the robot controller and, simultaneously, to allow efficient 
%communication between interacting parts of the system. The bottleneck in this approach
%would eventually be the available bandwidth on the network. 
As long as enough resources are available, the addition of new components 
should minimally interfere with existing processes. This is important, since often 
the actual performance of a robotic controller depends on the timing of various signals. 
While this is not strictly guaranteed by the YARP infrastructure, the problem is in 
practice alleviated computationally by allowing the inclusion of more processors to 
the network, and from the communication point of view by the buffer policy.
%isolating sub-components.

%\item {\bf Stopping hurts.}
\textit{\textbf{Stopping hurts.}}
It is a commonplace that human cycles are much, much more expensive
than machine cycles.  In robotics, it turns out that the human
cost of stopping and restarting a process can be very high.
For example, that process may interface with some
custom hardware which requires a physical reset.  
That reset many need to be carefully ordered with respect to when the
process is stopped and started.
%
There may be other dependent processes that need to be restarted in
turn, and other dependent hardware. 
%
%
%
These ordering constraints are time-consuming to satisfy.
%
YARP does its part to minimize dependencies between processes,
% so only true physically required dependencies remain.  
communication channels between processes can come and go. 
A process that is killed or dies
unexpectedly does not require processes to which it connects to be
restarted. This also simplify cooperation between people, as
it minimizes the need to synchronize development on different  
parts of the system.
%
%In complex systems, with dozens of processes and hundreds of connections, it might become
%unpractical to shut down and restart the whole system every time a module is even slightly 
%changed. YARP allowing the run-time connection of channels permits the disconnection of 
%only those parts of the system that need to be, for instance, rebuilt. 
%
%\item {\bf Humility helps.}

\textit{\textbf{Humility helps.}}
Over time, sofware for a sophisticated robot needs to 
aggregate code written by many different people in many
different contexts.  Doubtless that code will have
dependencies on various communication, image processing,
and other libraries. Very often the operating system on which
software is developed pose similar constraints. This is especially
true with code that relies heavily on the services offered by the 
operating system (such as communication, scheduling, synchronization primitives, 
and device driver interface).
%
Any component that tries to place itself ``in control'' and has strong
constraints on what dependencies are permissible will not be tolerated
for long.  It certainly cannot co-exist with another component
with the same assumption of ``dominance''. 
Although YARP offers support for communication, image processing,
interfacing to hardware etc., it is written with an {\em open world}
mindset.  We do not assume it will be the only library used, and
endeavor to be as friendly to other libraries as possible.
%
YARP allows interconnecting many modules seamlessly without subscribing
to any specific programming style, language interface, 
or demanding specifications as for instance in CORBA~\cite{vinoski97corba}
or DCOM~\cite{dcom}. Such systems, although far more powerful than YARP,
require a much tighter link between the general algorithmic code and the 
communication layer.
We have taken a more lightweight approach: YARP is a plain library linked
to uses-level code that can be used directly just by instantiating appropriate classes.
%
% and communication does not require any diversion
% of pre-existing threads. That is, YARP is a plain library linked to user-level
% code and as such migration to YARP can be easily carried out a posteriori. 
%
% Systems such as CORBA~\cite{vinoski97corba}, although far more powerful than YARP, require 
%
% adhering to well-defined interface specificiations (nothing bad as such) but consequently 
%
% a much tighter link between the general algorithmic code and the communication layer.
%
% is much 
% stricter. 
%
%
%
Finally, other programming languages can access YARP as well, provided they
have means of linking and calling C++ code. We have successfully used
YARP from within Matlab or L~\cite{brooks90behavior}.

%It is userful to reserve
%that role for the occasional poorly-designed hardware device that
%assumes it is the center of the universe.

%\end{itemize}

%
%The OS library contains the communication facilities described in
%section \ref{sec:communication} and classes implementing
%synchronization routines (like mutexes and semaphores) and
%threads. 
%
\textit{\textbf{Exploit diversity.}}
%
Different operating systems offer different
features. Sometimes it is easier to write code to perform a given task
on a platform as opposed to another. This can happen for example if 
device drivers for a given board are provided only on a specific platform or
if an algorithm is available open source on another. We decided to reduce the
dependencies with the operating system. For this we
use ACE~\cite{ACEBook}, an open source library providing a framework for
concurrent programming across a very wide range of operating
systems. YARP inherits the portability of ACE and has indeed been used and
tested on Windows, Linux and QNX 6.



YARP's core communication model was the survivor from an early humanoid robot
controlled by a set of Motorola 68332 processors, an Apple Mac, and a loose network
of PCs running QNX, Linux, and Microsoft Windows.  Communication was a
hodge-podge of dual-port RAM, QNX message passing, CORBA, and raw
sockets.  At one point, three incompatible communication protocols
layered over QNX message passing were in use simultaneously.  This
variety was a consequence of organic growth, as developers added new
modules to the robot.  YARP began as one of the communication
protocols built on QNX message passing.  A key, defining, feature of
YARP was that it was {\em broad-minded}: it was
implemented in the form of a library which placed minimal constraints
on user code; communication resources did not need to be allocated at
any particular time or place in a program; reading messages could be
blocking, polling, or callback based, etc. This meant it could be
easily added without disturbing existing code, and communication could
be moved across to the new protocol piece by piece.

%The basic YARP module is an IPC infrastructure that supports communication across a
%network exploiting different protocols. 
%

%The mathematical library provides classes and functions to handle
%vectors and matrices, together with a few algebraic routines like
%single value decomposition, QR and LU factorization.

%More details about the image processing library and the device driver
%library can be found in the next sections.
