
+ Lorenzo's comments

-- I would stress the fact that modularity is a requirement and that processes need not to be aware of the machine on which they are running (the reason why we use names to identify the connections as opposed to static ip numbers).

-- We also keep GUI separate from the rest of the code

-- I tried to add something on OS independencies (and ACE, maybe we should say something more in the introduction or in the communication section) ?

+ Multiple people, shared resources, maximally independent projects.

+ We make it a rule that a process should never need to be restarted
  because of anything YARP does

+ Processes come and go.

+ Make streaming communication easy.

+ Name to ...




YARP client code is easy.

The package duration paradigm.

Giving control over buffer policy, but avoiding making user
think about it.

Gotchas:

+ pointers in structures

+ OnRead doesn't often get called.

+ sometimes expect both that all messages will get through, and
  that messages will get dropped for timeliness.

YARP

The principles of YARP:

+ Politeness.

+ Openness.

+ Playing well with others.

Motivations.

What type of robots we're dealing with.

History.

forgot to mention, important to AVOID UNNECESSARY COPIES


Adaptive scheduling?  Could be difficult to reason about.


The fundamental image class in YARP can easily become a {\em Proxy}
to image data stored externally or in an alternate representation.



\section{other projects}

IPC by Christopher Fedor and Reid Simmons, used in Carmen.
Check this: \cite{roy03IROS}

