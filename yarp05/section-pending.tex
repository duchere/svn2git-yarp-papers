
+ Lorenzo's comments

-- I would stress the fact that modularity is a requirement and that processes need not to be aware of the machine on which they are running (the reason why we use names to identify the connections as opposed to static ip numbers).

-- We also keep GUI separate from the rest of the code

-- I tried to add something on OS independencies (and ACE, maybe we should say something more in the introduction or in the communication section) ?

+ Multiple people, shared resources, maximally independent projects.

+ We make it a rule that a process should never need to be restarted
  because of anything YARP does

+ Processes come and go.

+ Make streaming communication easy.

+ Name to ...




YARP client code is easy.

The package duration paradigm.

Giving control over buffer policy, but avoiding making user
think about it.

Gotchas:

+ pointers in structures

+ OnRead doesn't often get called.

+ sometimes expect both that all messages will get through, and
  that messages will get dropped for timeliness.

YARP

The principles of YARP:

+ Politeness.

+ Openness.

+ Playing well with others.

Motivations.

What type of robots we're dealing with.

History.

forgot to mention, important to AVOID UNNECESSARY COPIES


Adaptive scheduling?  Could be difficult to reason about.


The fundamental image class in YARP can easily become a {\em Proxy}
to image data stored externally or in an alternate representation.



\section{other projects}

IPC by Christopher Fedor and Reid Simmons, used in Carmen.
Check this: \cite{roy03IROS}



\section{Device Driver Example: YARPGenericFrameGrabber}

As an exemple we report here the structure of the generic frame grabber. The first layer is the YARPDeviceDriver which defines the methods open(), close() and IOCtl(). It also stores the function table that implements the interface of the drivere (m\_cmds); this table is allocated and initialized in the YARPDeviceDriver constructor. This table is correcly filled by the DERIVED class (see below).

{\small \begin{verbatim}
template <class DERIVED>
class YARPDeviceDriver
{
public:
	YARPDeviceDriver(int n_cmds);
	virtual ~YARPDeviceDriver();
protected:
	typedef int (DERIVED::*cmd_function_t)(void *);
	// function table
	cmd_function_t *m_cmds;

public:
	virtual int open(void *p) = 0;
	virtual int close() = 0;

	int IOCtl(int cmd, void *data)
	{
	  int ret = ((DERIVED *)
	            this->*m_cmds[cmd])(data);
	  return ret;
	}
};
\end{verbatim} }

In addition we defined the following messages:

{\small \begin{verbatim}
enum FrameGrabberCmd
{
  FCMDWaitNewFrame,
  FCMDAcquireBuffer,
  FCMDReleaseBuffer,
  FCMDGetSizeX,
  FCMDGetSizeY,
  FCMDSetContrast,
  ...
};
\end{verbatim} }

The first message waits for a new frame to be acquired. FCMDAcquireBuffer reserves the most recent frame and returns a pointer to it; the frame is released by the application by calling FCMDReleaseBuffer. The other messages are simple commands to set/get general parameters.

The YARPGenericGrabberAdapter is a virtual class which defines the interface for the adapter. In this case it is quite simple and consists only in the initialize() and uninitialize() methods. 

The last layer is a template class YARPGenericGrabber whose parameter is the adapter of a specific board. Part of the implementation of the YARPGenericGrabber is reported here:

{\small \begin{verbatim}
template <class ADAPTER>
class YARPGenericGrabber
{
protected:
	ADAPTER _adapter;
public:
	YARPGenericGrabber ();
	~YARPGenericGrabber ();

	int initialize (...)
	{
	  return  _adapter.initialize(...);
	}

	int uninitialize ()
	{
	  return _adapter.uninitialize();
	}

	int acquireBuffer (unsigned char **buffer)
	{
	  return _adapter.IOCtl(FCDMAcquireBuffer, 
	                        buffer);
	}

	int releaseBuffer (void)
	{
	  return _adapter.IOCtl(FCMDReleaseBuffer);
	}

	int setContrast(unsigned int contrast)
	{
	  return _adapter.IOCtl(FCMDSetContrast, 
	                        &contrast);
	}

	... //other methods
}
\end{verbatim} }

To instantiate and use the YARPGenericGrabber in our code we need to define the classes implementing the device driver and the adapter for the particular board we intend to use (respectively MyDeviceDriver and MyGrabberAdapter). MyDeviceDriver derives from YARPDeviceDriver. It implements open() and close() methods. The frame grabber interface is implemented in the form of a set of functions whose pointers are stored in m\_cmds. MyDeviceDriver does not need to implement all messages but only the subset of the ones that are meaningful for the board actually in use. This is perfectly safe because by default each entry of the table is initialized to point to an empty (but valid) function.

{\small \begin{verbatim}
class MyDeviceDriver : 
	public YARPDeviceDriver<MyDeviceDriver>
{
  MyDeviceDriver()
  {
    // fils function table
    m_cmds[FCMDAcquireBuffer] =
                     &MyDeviceDriver::acquireBuffer;
    m_cmds[FCMDReleaseBuffer] =
                     &MyDeviceDriver::releaseBuffer;
    m_cmds[FCMDFaitFrame] = 
                     &MyDeviceDriver::waitOnFrame;
    ...
  }

  // open and initialize the device
  int open(void *d);

  // close the device
  int close(void);

protected:
  // messages:
  int waitOnFrame(void *cmd);
  int acquireBuffer(void *buffer);
  int releaseBuffer(void *cmd);
  ...
}
\end{verbatim}}

 The MyGrabberAdapter implements only and all the methods defined in the YARPGenericGrabberAdapter. It also derives from MyGenericDeviceDriver to allow accessing the device driver from the higher level layer. A possible implementation is reported here:

{\small \begin{verbatim}
class MyGrabberAdapter: 
	public MyDeviceDriver,
	public YARPGenericGrabberAdapter
{
  int initialize(...)
  {
    MyDeviceDriver::open();
    MyDeviceDriver::IOCtl(FCMSetContrast, ...);
    ... // other initializations
    return YARP_OK;
  }

  int unitialize()
  {
    MyDeviceDriver::close();
    ...
  }
}
\end{verbatim}}

Having implemented the adapter and the device driver for our frame grabber, it is now sufficient to instantiate and use the YARPGenericGrabber as follows:

{\small
\begin{verbatim}

typedef YARPGenericGrabber<MyGrabberAdapter> 
                          YARPGrabber;

int main()
{
  YARPGrabber _grabber;
  // initialize device
  _grabber.initialize(..);
  
  bool done = false;

  while(!done)
  {
    // wait until a new frame is ready
    _grabber.waitOnNewFrame();
    // lock most recent frame and
    // get a pointer to it
    _grabber.acquireBuffer(&buffer);
    // ...
    // ...
    // release frame
    _grabber.releaseBuffer();
  }

  // close the device
  _grabber.uninitialize();
}
\end{verbatim}
}


\section{Zone of proximal development}

(this section is very vague)

For infants, the zone of proximal development is the
difference between what they can accomplish on their
own compared to what they can accompish with an
adult's support [CITE].

By loose analogy, we label a robot control system's ``zone of proximal
development'' to be the set of modules being actively added or worked
on by the programmer, against a background of pre-existing, operating
modules.  

YARP helps insulate existing modules from changes in this zone,
and leaves them in a good form for when the zone moves on.

