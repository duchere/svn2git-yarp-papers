\section{More libraries}
A few extra components were developed to facilitate software development on humanoid robots. These libraries are described in the following sections.

\subsection{Image processing}
Support for visual processing is almost a mandatory requirement for a software library designed to be applied to humanoid robotics. Vision is criticial for real time systems, as computer vision algorithms require the elaboration of a large quantity of data.

To help developers write efficient visual processing routines, Intel released the Image Processing Library (IPL). This library in optimized to provide high performance on machines which employ Intel processors, especially if equipped with MMX(TM) technology. The IPL library is a set of C functions which implement basic operations on images (from simple algebric operations on pixels to color conversions and convolutions). The libary is optimized for the CPU on which it is running (ar run-time the library automatically detects the CPU type and load the module that is more suitable). Another advantage of using the IPL library is that it is at the core of the opencv library which provides more sophisticated routines for image manipulation such as filtering, face tracking and optic flow (to mention just a few).

Unfortunatly the IPL library is a set of C functions which do not provide support for object oriented programming. We decided to write a library which implements a set of classes to store and manipulate the pixels of an image. To be more precise the set of classes is in the form of a C++ template that can be instantiated for each pixel type (for example at the moment RGB, grayscale and floating point are implemented). The YARPImageOf template defines an interface for all the image classes in the library; as in other parts of the library we decided to use templates for efficiency reasons. The internal structure of the image is identical to the one used inside the IPL library. This allows any user to take full advantage of the IPL and openCV libraries.

The image class provides support to transmit the internal data between two YARP ports (see section).

\subsection{Hardware Interface}
Another frequent problem during development in robotics is that it is very hard to reuse code in different platforms. In some cases this cannot be avoided, especially when the two platforms are mechanically different. In other cases, however, the platform is mechanically similar, but mounts different boards. For example this happens when two robots have different frame grabbers, or different control boards. In these situations it is not possible to keep the same code. However something can be done to reduce the number of changes that are required and to localize them to specific components. The idea is that high level software modules are not (or should not be) concerned with the low level details of the underlying hardware platform. 

To solve these problems we defined a virtual device driver interface into YARP and encapsulated the control parts of the robot into a standardized template class hierarchy. The structure of the YARP virtual device driver resambles the structure of UNIX device drivers. It has three main metohds: Open, Close and IOCtl. Open and Close execute code to initialize and quit the device, whereas the IOCtl is the core of the interface and consits in a set of messages. Each message defines an index in a table of functions. The advantage of using this structure as opposed to a virtual class is that it is not mandatory to implement all methods if a given function is not realized by the hardware. 
The low level software should capture the essential functioning of the particular class of device and hide the details of the implementation. For example the device driver for a control board should provide methods for moving a joint by specifying the desired position, velocity and acceleration. Other common functionalities are methods to read position, speed and torque (if available). In practice the main differences between cards lay on the steps required for the initialization of the device.

[add picture with example]

In some cases the same board (with the same device driver) might be connected to different setups. This is again the case of control board, but another example could be that of analog to digital converters. Changing the setup might force to change the initialization procedure or the configuration of the setup. This for example happens if the wiring of the hardware differs, or if it requires a different calibration procedure. These dissimilarities are taken into account in the next layer. 

The communication with the robot is cared out through an instance of a class for each component. The interface of this class is indipendent of the particular setup. The access to the hardware passes through two components: the virtual device driver the adapter. The virtual device driver was described above and implements most of the external interface directly. The adapter class is required to take into account the characteristics typical of the setup like the wiring.

[maybe we need an example or a picture]

\subsection{Robot interface}
The co
