\section{Introduction}
To reach for a visually identified object the robot has to solve
an inverse kinematic problem. This 
problem has been studied extensively in the robotic literature 
and many solutions exist. None of these solutions is easy to 
implement in practice. The main limitation is that they usually require 
some knowledge of the robot kinematics and that the camera parameters 
are either available ``a priori'' or estimated by means of a specific
calibration procedure. The problem become even more complicated 
in the case of humanoid robots which are often characterized by 
a large number of degrees of freedom (DOF) and redundancy. The 
kinematics of the robot in these systems becomes a complicated 
function whose estimation is not, in practice, a trivial task.

The traditioanl approach separates reaching in two
problems: 1) estimate the location of the target and
2) solve the arm inverse kinematics to compute the set of joint
angles which brings the hand to the target. Assuming we know the 
parameters of the stereo vision system, standard vision techniques
allow to solve the first problem and compute the position of the 
target in the 3D Cartesian space \cite{Soatto03vision}. These methods 
however require that the parameters of the cameras are known with 
good accuracy: this is a well studied problem in computer 
vision \cite{Soatto03vision}. Once the 
position of the target in the Cartesian space is available 
the second problem (i.e. the computation of the arm joint vector) is 
easily solved if the direct kinematics of the arm is available. 
Imposing additional constraints on the task helps to deal with 
redundancies \cite{liegeois77automatic}.

In general solving the reaching task requires a sensory-motor map
transforming the position of the target in the visual domain into the 
arm joint vector which solves the task. Extensive literature describe 
calibration procedures for retrieving at least part of the parameters 
which characterize this mapping 
\cite{Holerbach96calibration,Tsai88Calibration}. 
These techniques can be quite accurate but require the robot to operate
in somewhat structured environment, typically involving a known object (or
a calibration grid), and a calibrated hand pose sensor. Some procedures 
have been proposed to relax part of these assumptions. In particular 
\cite{AHE01} avoids the need for a calibrated object, whereas 
\cite{Bennett91calibration} exploits specific kinematic constraints 
to avoids an explicit measure of the hand pose.

Depending on how we decide to encode the target position (and the 
particular sensory-motor transformation which solves the reaching task)
these calibration procedures might not be necessary. Assuming that 
the robot is fixating the target, the joints of the head implicitly 
encode the position of the target in the world. In this case a simpler 
appraoch could be to directly map the joint vector defining the current 
head posture to the arm joint vector corresponding to the configuration 
of the arm that brings the hand to the fixation point. We call this map 
a forward ``motor-motor'' map because it is a direct link between the motor 
angle of the head and those of the arm 
\cite{blackburn94learning,metta99developmental}. 

Reaching can be performed by inverting this motor-motor 
map to retrieve the arm command which brings the hand to the fixation 
point. An extension of this approach to redundant manipulators has
been proposed in \cite{lopes06learning}. What is interesting in this 
approach is that it avoids esplicit computation of the 3D position 
of the target; what the robot needs, is the ability to control gaze
so that both eyes verge on the target. The position of 
this target is thus implicitly coded by the posture of the whole head (the 
angle of vergence between the eyes provide a measure of distance). 
This solution is not only convenient from an engeneering perspective. Studies 
on humans suggest that this might indeed be a biologically plausible 
solution (\cite{flanders-daghestani-berthoz-1999}).

The procedure to learn the motor-motor map is strightforward and can be 
easily aquired in a ``motor babbling'' phase in which the robot moves 
the arm randomly while at the same time maintains fixation on the hand. 
The samples aquired in this way are relayed to a modules which 
approximate the function relating arm and head joint vectors. The 
limitation of these approaches is that reaching
is intrinsically ballistic and does not use visual feedback. The precision 
with which the forward map is learned poses an intrinsic bound on
the accuracy of the reaching task. Although this accuracy varies with the
system, it might not be satisfactory for practical tasks.

This problem can be solved by using a visual servoing control schema (for a 
review: \cite{hutchinson96tutorial)}). These techniques use the 
Jacobian matrix to control the arm so that the visual distance between the 
hand and the target is progressively reduced to zero. The advantage is that 
even an inaccurate Jacobian matrix assures convergece 
of the hand to the desired location (trajectory). When continuous visual 
feedback is available closed loop techniques probably provide the best 
results, provided the Jacobian matrix is known with 
adequate accuracy. \cite{Hosoda94versatile,Mansard06jacobian,Lapreset04efficient} 
describe procedures to estimate the Jacobian matrix of a robot. Unfortunately 
these techniques do not suggest how to deal with the 
degrees of freedom problem. The Jacobian matrix, in fact, is a function
of bith the arm and the head joints, and its estimation become 
quickly hard as the number of joints increase.

A somewhat intermediate approach is described in \cite{scaz07fast}. In 
this case the robot uses stereo vision to compute the position
of the hand in the 3D space; reaching is here accomplished by inverting the 
forward mapping between the arm joints and the 3D position of the hand. This 
solution employs a visuomotor map, which naturally maps vision to motor space. 
Indeed from the forward map the authors derives the eye-to-hand visual 
Jacobian of the arm which allows the robot to drive the hand to the target with 
arbitrary precision. To deal with the degrees of freedom problem the authors
propose an approximate solution where the head joint vector is substituted 
for by the output of the robot gyroscope. The drawback of this approach is that 
it requires prior calibration of the visual system and that the relative position 
of the two cameras does not change. In practice this forces the robot to maintain 
the eyes stationary to a fixed position with respect of the head. 

The technique we propose in this paper integrates the open loop and the 
closed loop approaches. We use a motor-motor map to direct 
the hand close to the fixated object. The closed loop controller is activated 
as soon as visual feedback from the hand becomes available. Together 
these strategies enable the robot to initiate a reaching action regardless
of whether sight of the hand is available or not (e.g. when the hand is 
out of view or in the dark). The closed loop controller allows taking
advantage of the visual feedback to arbitrarely reduce the positioning
error. The robot autonomously learns the eye-to-hand Jacobian matrix and 
the motor-motor map for an area within the arm workspace and for different
head postures. The method we proposes does not rely on any prior information
about the robot kinematic structure nor it requires camera calibration 
or solving the esplicit 3D problem.
