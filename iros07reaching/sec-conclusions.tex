\section{Conclusions}
We have described the implementation of a reaching controller which 
enables a humanoid robot to reach for visually identified objects 
in the 3-dimensional space. Our solution integrates an open loop and 
a closed loop controller. The former allows the robot to perform 
ballistic movements toward the target and does not require visual feedback.
This strategy is important not only to improve the velocity of the arm, but 
also to initiate a reaching movement when the hand is not visible (e.g. 
when the hand is out of sight). The latter employs visual feedback to guarantee precise positioning of the hand in the image plane.

During an exploration phase the robot autonomously acquires the required 
sensory-motor transformations. Learning does not rely on prior information 
about the kinematic structure of the robot. The only simplification was 
that we used a color marker to visually localize the hand of the robot. 
Our assumption is that the hand localization/identification is a separate 
problem that is already solved when the robot starts learning to reach. 
Some solutions to this problem have been proposed in the literature, 
see for example in \cite{Natale05,edsinger06what}.

The proposed learning strategy allows the autonomous estimation of the 
forward motor-motor map and of the eye-to-hand visual 
Jacobian. The estimation of the Jacobian is a task for which 
several solutions have been proposed 
\cite{Hosoda94versatile,Mansard06jacobian,Lapreste04efficient,scaz07fast}.
Unfortunately these techniques do not address the degrees-of-freedom problem 
and the problem of redundancy of the head and the arm. In the solution we
propose the estimation of the Jacobian is fully autonomous and 
does not impose any constraints on the number of the degrees of freedom
that are actuated (the estimation of the Jacobian is performed on a subset 
of the arm workspace and for different head postures). We show that 
choosing the fixation point as a reference frame for encoding the target 
position and the Jacobian matrix leads to a simple solution to an otherwise 
complex kinematic task.

