
\section{Motivation}

A truly autonomous robot needs to be able to explore and learn from
its environment, since it cannot rely on receiving all the information
it needs passively~\cite{whaite97autonomous}.
It is telling that some of the earliest autonomous robots ever built,
the tortoises of W. Grey Walter, were given the mock-Linnean
designation ``Machina Speculatrix'' by their creator to emphas\ize{} their exploratory
\ahhbehavior{}, described as: ``it explores its environment actively,
persistently, systematically as most animals do''
\cite{walter50imitation}.
These robots had very simple control circuitry, and their \ahhbehavior{}
depended greatly on the morphology and dynamics of their own bodies.
%
This observation of the utility of a robot's body has recurred over
the years, perhaps most notably in the work of
Brooks~\cite{group-AAAI-98}.  It has also played a role in active
approaches to machine vision, where sensors are embedded in a robotic
platform and moved in a manner that simplifies visual
processing~\cite{ballard91animate}.
%
Since correctly perceiving the world comes so naturally to humans, and
appears so free of effort, the motivation for this work 
can be
difficult 
\ifrevised
for those outside the field of vision research
\fi
to grasp at an intuitive level.  
%
\ifrevised
For this reason, we begin our paper by
\else
We begin our paper by
\fi
seeking to clarify the difficulties a robot faces in perceiving the
world, and how its body can come to the rescue.



