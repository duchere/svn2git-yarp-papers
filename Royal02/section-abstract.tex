
\iflong
\begin{abstract}
\else
\begin{Abstract}
\fi

Experimentation is crucial to human progress at all scales, from
society as a whole to a young infant in its cradle.  It allows us to
elicit learning episodes suited to our own needs and limitations.
This paper develops active strategies for a robot to acquire visual
experience through simple experimental manipulation.  The experiments
are oriented towards determining what parts of the environment are
physically coherent -- that is, which parts will move
together, and which are more or less independent.  We argue that
following causal chains of events out from the robot's body into the
environment allows for a very natural developmental progression of
visual competence, and relate this idea to results in neuroscience.

\ifverbose
For the purpose of understanding development we would like to present
causality as a possible principle to frame a number of neural science
results coherently. We will show how this can lead also to an
implementation in an artificial system following the epigenetic
approach. To this purpose we will show different levels of causal
linkages, or instances of the general principle, which allow tasks of
increasing complexity to be implemented.  Action and the physical
interaction of the robot with the environment play a fundamental role.
In an ecological perspective, the role of this physical interaction
for developing categorization and object undestanding is emphasized.
\fi

%%{\bf \em
%%\iflong
%%(long version)
%%\else
%%(short version)
%%\fi
%%}

\iflong
\end{abstract}
\else
\end{Abstract}
\fi
