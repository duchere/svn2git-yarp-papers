
\ifroyal
\begin{abstract}{key1, key2}
\else
\begin{abstract}
\fi

Experimentation is crucial to human progress at all scales, from
society as a whole to a young infant in its cradle.  It allows us to
elicit learning episodes suited to our own needs and limitations.
This paper develops active strategies for a robot to acquire visual
experience through simple experimental manipulation.  The experiments
are oriented towards determining what parts of the environment are
physically coherent -- that is, which parts will move
together, and which are more or less independent.  We argue that
following causal chains of events out from the robot's body into the
environment allows for a very natural developmental progression of
visual competence, and relate this idea to results in neuroscience.

\end{abstract}
