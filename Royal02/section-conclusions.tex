
\section{Discussion and Conclusions}

In this paper, we showed how causality can be probed at different
levels by the robot.  Initially the environment was the body of the
robot itself, then later a carefully circumscribed interaction with
the outside world.  This is reminiscent of Piaget's distinction
between primary and secondary circular
reactions~\cite{ginsburg78piaget}.  Objects are central to interacting
with the outside world.  We raised the issue of how an agent can
autonomously acquire a working definition of objects. 

The number of papers written on techniques for visual segmentation is
vast.  Methods for \ahhcharacterizing{} the shape of an object through
tactile information are also being developed, such as shape from
probing 
\cite{paulos99fast} 
%%\cite{cole87shape,paulos99fast} 
or pushing
\cite{moll01reconstructing}.  
%%\cite{moll01reconstructing,jia98observing}.  
But while it has long
been known that motor strategies can aid
vision~\cite{ballard91animate}, work on active vision has focused
almost exclusively on moving cameras.  There is much to be gained by
bringing a manipulator into the equation~\cite{tsikos91segmentation}.
The implications may be far reaching.  For example,
we have shown that, without any prior knowledge of the human form, 
the robot can identify episodes when a human is manipulating objects
that are familiar to the robot purely by the operational similarity 
of the human arm and its own manipulator in this situation.

We have related our work to some very interesting results from
neurobiology that have implications for sensorimotor integration, such
as the discovery of mirror neurons.  Our view is that while biologists
are doing a good job of elucidating {\em what} mirror neurons are and
how they operate, work like ours can more readily clarify {\em why}
they are useful in \ahhpractice{}.  We believe that answer lies in the
developmental process.  A creature created fully formed could perhaps
operate just fine without mirror neurons, but to reach adult
competence from a more primitive stage requires continuously
interleaving perception with experimental action -- a situation that
seems to call for mirror neurons and similar machinery.  Our goal is
to build a robot capable of such experimentation, and to identify
specific functional advantages of mirror-like representation in the
development of its visual competence.  Knowledge of functional
advantages could suggest new and interesting relationships for biologists to
look for that they may not have thought of (since they have never
tried to actually build a vision system themselves).


%%In computer vision there is much to be gained by bringing a
%%manipulator into the equation.  

\ifverbose
Many variants and extensions to the
experimental ``poking'' strategy explored here are possible.  For
example, a robot might try to move an arm around {\em behind} the
object.  As the arm moves behind the object, it reveals its occluding
boundary.  This is a precursor to visually extracting shape
information while actually manipulating an object, which is more
complex since the object is also being moved and partially occluded by
the manipulator.  Another possible strategy that could be adopted as a
last resort for a confusing object might be to simply hit it firmly,
in the hopes of moving it some distance and potentially overcoming
local, accidental visual ambiguity.  Obviously this strategy cannot
always be used!  But there is plenty of room to be creative here.
\fi



\ifverbose
The number of papers written on techniques for visual segmentation is
vast.  Methods for \ahhcharacterizing{} the shape of an object through
tactile information are also being developed, such as shape from
probing 
%%\cite{cole87shape,paulos99fast} 
\cite{paulos99fast} 
or pushing
\cite{moll01reconstructing}.
%%,jia98observing}.  
But while it has long
been known that motor strategies can aid
vision~\cite{ballard91animate}, work on active vision has focused
almost exclusively on moving cameras.  There is much to be gained by
bringing a manipulator into the equation, as we have shown in this
paper.  

Many variants and extensions to the experimental ``poking''
strategy explored here are possible.  For example, a robot might try
to move an arm around {\em behind} the object.  As the arm moves
behind the object, it reveals its occluding boundary.  This is a
precursor to visually extracting shape information while actually
manipulating an object, which is more complex since the object is also
being moved and partially occluded by the manipulator.  Another
possible strategy that could be adopted as a last resort for a
confusing object might be to simply hit it firmly, in the hopes of
moving it some distance and potentially overcoming local, accidental
visual ambiguity.  Obviously this strategy cannot always be used!  But
there is plenty of room to be creative here.
\fi


