\section{Active Vision}

A vision system is said to be {\em active} if it is embedded within a
physical platform that can act to improve perceptual performance.  For
example, a robot's cameras might servo a rapidly moving target to
stabilize the image and keep the target in view.  In fact, active
vision is often equated with moving cameras, although in this paper we
use it in a broader sense of any controllable resource recruited to
serve vision (including, in our case, arm motion).

Historically, a number of logically distinct ideas are often
associated with active vision.  The first is that vision should be
approached within the the context of an overall task or
purpose~\cite{aloimonos87active}.  If an observer can engage in
controlled motion, it can integrate data from frame to frame to solve
problems that are ill-posed statically.  Well-chosen motion can
simplify the computation required for widely studied vision problems,
such as stereo matching~\cite{bajcsy88active,ballard91animate}.  These
interwoven ideas about active vision are teased apart
in~\cite{tarr94computational}.


Pending citations: \cite{tsikos91segmentation}.
