One of the most distinguishing features of cognitive systems is the
ability to predict the future course of actions and the results of
ongoing behaviours, and in general to plan actions well in advance.
Neuroscience has started examining the neural basis of these skills
with behavioural or animal studies, and it is now relatively well
understood that the brain builds models of the physical world through
learning. These models are sometimes called ``internal models'',
meaning that they are the internal rehearsal (or simulation) of the
world enacted by the brain.

In this paper we investigate the possibility of building internal
models of human behaviours with a learning machine that has access to
information in principle similar to that used by the brain when
learning similar tasks. In particular, we concentrate on models of
reaching and grasping and we report on an experiment in which
biometric data collected from human users during grasping was used to
train a Support Vector Machine.

We then assess to what degree the models built by the machine are
faithful representations of the actual human behaviours. The results
indicate that the machine is able to predict reasonably well human
reaching and grasping, and that prior knowledge of the object to be
grasped improves the performance of the machine, while keeping the
same computational cost.
