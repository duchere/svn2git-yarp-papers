It is believed that one of the most distinguishing features of cognitive
systems is the ability to foresee the future course of action, the results
of ongoing behaviors, and in general to plan actions well in advance.
Neuroscience has started examining the neural basis of these skills with 
behavioral or animal studies. It is now relatively well understood that
the brain builds models of the physical world through learning. These
models are sometimes called ``internal models'' meaning that they are
the internal rehearsal (or simulation) of the world enacted by the brain.

In this paper we investigate the construction of models of human behaviors
by a machine that has access to information that is in principle similar
to that used by the brain when learning similar tasks. In particular, we
concentrate on models of reaching and grasping and we explore to what degree
they are faithful representations of actual human behaviors.
