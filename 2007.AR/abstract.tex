One of the most distinguishing features of cognitive systems is the
ability to predict the future course of actions and the results of
ongoing behaviours, and in general to plan actions well in advance.
Neuroscience has started examining the neural basis of these skills
with behavioural or animal studies. It is now relatively well
understood that the brain builds models of the physical world through
learning. These models are sometimes called ``internal models''
meaning that they are the internal rehearsal (or simulation) of the
world enacted by the brain.

In this paper we investigate the construction of models of human
behaviours by a machine that has access to information that is in
principle similar to that used by the brain when learning similar
tasks. In particular, we concentrate on models of reaching and
grasping and we explore to what degree they are faithful
representations of the actual human behaviours.
