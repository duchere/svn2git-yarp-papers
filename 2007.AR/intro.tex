It is reasonably well established that the brain stores internal models of action
either because they are required to control movement or, as it has been determined
more recently, to interpret the movement of others \cite{kawato-99}. There is a large body of 
literature that links observation of action to action execution itself as for 
example the study of the motor system conducted by Rizzolatti and colleagues in 
relatively recent years \cite{rizzolatti-04,gallese-96,rizzolatti-01}. 

In the monkey, premotor area F5 has been particularly well studied and it is in fact the 
location where ``mirror neurons'' were first identified. In this respect, mirror neurons 
are the quintessential correlate of internal models since they are activated 
both when executing a specific grasping action and when observing a congruent action 
being executed by another individual (or the experimenter) \cite{fadiga-00}.

In a study by Umilt\'a et al. \cite{umilta-01} the response of mirror neurons to the observation of
actions that terminate behind a screen has been investigated. In this case, the authors analyzed 
mirror neurons in situations where the final part of the action was occluded by an opaque 
screen with the monkey knowing of the presence/absence of an object to be grasped. As long
as the object was shown to the monkey, the brain could easily supply the missing
visual information by rehearsing the internal model of the action. The control experiment was that
of an identical hand kinematics, an identical screen but the absence of the target object.
Elsewhere it has been also shown that the presence of an object is important in eliciting the 
mirror neurons response in the monkey \cite{gallese-96}.

{\em A posteriori}, it is easy to imagine that the presence of a target object and its geometrical 
properties strongly constrain the type of grasp and the approach to the object, and that as a consequence 
the brain might need to include this information when planning appropriate transitive actions. 
As we mentioned earlier, in the monkey these constraints are so strong that mirror neurons are not 
activated unless the goal of the action is clearly perceivable. The brain codes for the object-motor 
identity in part via another class of F5 visuomotor neurons called ``canonical neurons'' (for a 
discussion see for example \cite{metta-06}). 

In this respect, the work of Fogassi et al. \cite{fogassi-05} contributed to the identification of mirror
neurons in the parietal cortex (inferior parietal lobule), which are thought to be related to the 
decoding of the intentions of others. Contextual information which links the enacted action to its
final goal seems to be implicated in this type of neural response. The presence of objects is 
a clear contextual cue.

In humans, it has been demonstrated that the activation of brain areas correlated to 
action observation is not simply a perceptual effect but rather the activation of
a precise sensorimotor model which includes for example the hand kinematics \cite{pozzo-06}.
 
Accordingly, Fadiga et al. \cite{} have shown that motor imagery changes the excitability
of the cortico-spinal connections specifically to the imagined action, that is, 
imagining a motor task causes the under-threshold activation of the same neural pathways 
required to execute the task. This under-threshold activation was revealed in \cite{} 
by transcranial magnetic stimulation. In a conceptually similar experiment \cite{}, the 
excitability of cortico-spinal pathways was also examined as a consequence of the actual 
sensory input. In summary, the motor system is similarly activated when acting in first
person, when imagining an action, or when watching somebody else's action.

Jeannerod \cite{} has thoroughly investigated motor imagery, which can be thought of as the 
activation of internal models of action, in various situations {\bf [NEED TO WORK ON THIS]}.

[SAY SOMETHING ABOUT IMITATION, TELEOPERATION and PROSTHETIC]

VISION -> MOTOR, shown by ourselves.