It is reasonably well established that the brain stores internal models of action
either because they are required to control movement or, as it has been determined
more recently, to interpret the movement of others \cite{kawato-99, wolpert-03, 
mussaivaldi-00, lackner-98}. 
There is a large body of literature that links observation of action to action 
execution as for example the study of the motor system conducted by Rizzolatti and 
colleagues in relatively recent years \cite{rizzolatti-04,gallese-96,rizzolatti-01}. 

In the monkey, premotor area F5 has been particularly well studied and it is in fact the 
location where ``mirror neurons'' were first identified. In this respect, mirror neurons 
are the quintessential correlate of internal models since they are activated 
both when executing a specific grasping action and when observing a congruent action 
being executed by another individual (or the experimenter) \cite{fadiga-00}.

In a study by Umilt\'a et al. \cite{umilta-01} the response of mirror neurons to the observation of
actions that terminate behind a screen has been investigated. In this case, the authors analyzed 
mirror neurons in situations where the final part of the action was occluded by an opaque 
screen with the monkey knowing of the presence/absence of an object to be grasped. As long
as the object was shown to the monkey, the brain could easily supply the missing
visual information by rehearsing the internal model of the action. The control experiment, 
in this case, was that of an identical hand kinematics, an identical screen but the absence of 
the target object, that is, identical visual stimulation apart from the knowledge of the 
presence of the object. Elsewhere it has been also shown that the presence of an object is 
required to elicit the mirror neurons response in the monkey \cite{gallese-96}.

{\em A posteriori}, given these results, it is easy to see how the presence of a target 
object and its geometrical properties strongly constrain the type of grasp and the approach 
to the object, and that, as a consequence, the brain might need to include this information when 
planning an appropriate course of action. 
As we mentioned earlier, in the monkey these constraints are so strong that mirror neurons do not 
fire unless the goal of the action is clearly perceivable. The brain codes for the object-motor 
identity in part via another class of F5 visuomotor neurons called ``canonical neurons'' (for a 
discussion see for example \cite{metta-06}). 
To complete the ``picture'', the work of Graziano, Hu, and Gross \cite{graziano-97} has shown that 
the presence of object is coded in the ventral premotor cortex and maintained even when the 
object is no longer visible as long as there is evidence for its presence at a particular
location.

Relevant to this discussion, the work of Fogassi et al. \cite{fogassi-05} contributed to the 
identification of mirror neurons in the parietal cortex (inferior parietal lobule), which are 
thought to be related to the decoding of the intentions of others. Contextual information 
which links the enacted action to its final goal seems to be implicated in this type of neural 
response. The presence of objects is a clear contextual cue.

In humans, it has been demonstrated that the activation of brain areas correlated to 
action observation is not simply a perceptual effect but rather the activation of
a precise sensorimotor model which includes for example the hand kinematics \cite{pozzo-06}.
 
Accordingly, Fadiga et al. \cite{fadiga-99,vargas-04} have shown that motor imagery changes
the excitability of the cortico-spinal connections specifically to the imagined action, that is, 
imagining a motor task causes the under-threshold activation of the same neural pathways 
required to execute the task. This under-threshold activation was revealed 
by transcranial magnetic stimulation. In a conceptually similar experiment 
\cite{fadiga-05}, the excitability of cortico-spinal pathways was also examined as a consequence 
of the actual sensory input. In summary, the motor system is similarly activated 
when acting in first person, when imagining an action, or when watching somebody else's action.

% note: it'd be nice to examine the difference with presence/absence of the object in
% humans using the TMS.

Jeannerod \cite{jeannerod-88}, for example, goes in great length in showing how plausible is 
the fact that mental imagery uses the same internal models used by actual action generation. 
It is known in this respect that the time required to simulate an action is the same that 
is required to execute that action \cite{sirigu-96}. For a review refer to \cite{jeannerod-99}.

In this study, we set forth to investigate whether a machine equipped with enough sensory 
information about human movement could acquire a similar ``internal model'' via machine 
learning methods. Internal models of this type can potentially be used in various ways
including the control of robotic artifacts, the interpretation of human movements, and 
clearly they can be used to reproduce and mimic human movement. 

- previous experiment: mirror

- current goal: prosthetic, teleoperation

- future goals: robotics

- no vision: justified by the fact that motor+vision can learn a model that maps vision into motor



