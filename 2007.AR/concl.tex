With this initial experiment we really pose further questions and
sketch future research rather than draw definite conclusions. The
machine learning questions addressed in this paper do indeed have an
answer, albeit partial; on the other hand, it remains difficult to say
something other than speculations when comparing these results to
neuroscience.

In short, the answer to the two questions posed in Section
\ref{sec:exp_desc} is that we can predict well given that we have
access to motor information at least during learning, and that knowing
the objects to be grasped improves the ability to predict the outcome
of an action. There are many caveats in this experiment, as for
example, the question on whether a pre-processing of the data through
clustering could improve performance further: i.e., given that objects
afford certain grasping postures and they are executed with high
probability. In humans the quality of the prediction of grasping is a
function of the expectancies of the various possible grasp types which
are in turn determined by the past experience of manipulation of the
target object\footnote{personal communication with Luciano Fadiga.}.

The solution found by the SVMs detailed in the previous Section is
optimal, since the dependence from hyperparameters has been optimized
out in our case by grid search and cross-validation that although
expensive is known to provide good results. An analysis of the
solution should thus provide an accurate characterization of the
problem qua the data set that has been collected.

In this sense (and only in this sense) we have shown that by
partitioning the training set per object provides a general
improvement of the quality of the solution and simultaneously of the
training time (worst case $O(l^3)$ versus $O(3 \cdot (l/3)^3)$ in our
case with $3$ objects and $l$ the total number of samples). This can
be an effective strategy when the world affords such an intuitive
partitioning as for objects (seen as discrete entities).

This is also true from what is known about the brain structures that
control grasping where the presence of a target object, its shape and
affordance, and in general any contextual cue, are coded separately by
different populations of neurons and influence simultaneously the
response of the neurons that enact specific motor plans. After motor
prediction is in place, the next step, that of recognizing the action
of another individual is conceptually simple since it amounts to
building a classifier on highly predictable motor trajectories.

Another interesting question that is left to future research is
whether we can investigate the complexity of the controllers of
reaching and grasping (which are known to develop separately in
humans) from the complexity of the learned models or as a
consequence of the prediction error.

Clearly, the fact that we can train such models is prone to
be applied in various contexts, as we mentioned, ranging from control
of robots through interpretation and prediction of human behavior in
particular for man-machine communication.
