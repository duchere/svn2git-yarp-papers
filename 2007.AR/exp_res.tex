In order to understand how far our system could predict the grasping 
hand position and posture, we followed this methodology. First of all,
for each regression experiment, we defined an \emph{input sequence} as
a sequence of time sequences grouped by subject or by object or
both. Then from each input sequence we extracted the related grasp
sequences according to the methodology described in the previous
Section. Lastly, we fixed a \emph{blind percentage} $B$ of the
grasp sequence.

For each grasp sequence, we considered the values of the sensors at
the ending point of the sequence, that is, at the time of contact with
the object, as the target value. The input given to the Support Vector
Machine was then $(1-B)\%$ starting from the starting point of the
sequence. In other words, the last $B\%$ of the sequence was hidden to
the system. We expected that bigger $B$s would lead to a bigger
regression error, and tried to find what was the maximum $B$ for which
an acceptable error was achieved.

We grouped the $28$ sensors this way: the $3$ numbers representing the
position of the hand (coming from the FoB), the $3$ numbers
representing the orientation of the hand (coming from the FoB too),
and the $22$ numbers representing the posture of the hand (coming from
the DataGlove). According to the device resolutions detailed in the
previous Section, we defined an ``acceptable'' error like this: $1$
centimeter for each component of the hand position, $1$ degree for
each component of the hand orientation and $3$ units for each
component of the hand posture.

As far as the Support Vector Machines are concerned, a Gaussian kernel
was chosen, and its hyperparameters ($C$ and $\gamma$) were found by
grid search \cite{...}. The error on regression was obtained by
$5$-fold cross-validation on the grasp sequences obtained from the
target input sequence. This ensures that the system is able to
generalise well.

\subsubsection*{Accuracy on a single subject}

\subsubsection*{Accuracy on a single object}

\subsubsection*{Overall accuracy}

\subsubsection*{Discussion}

