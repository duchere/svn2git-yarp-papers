Our machine learning system is based upon Support Vector Machines
(SVMs). Introduced in the early 90s by Boser, Guyon and Vapnik
\cite{BGV92}, SVMs are a class of kernel-based learning algorithms
deeply rooted in Statistical Learning Theory \cite{v-edbed-82}. As
opposed to analogous algorithms such as, e.g., artificial neural
networks, they have the main advantages that their training is
guaranteed to end up in a global solution, that they can easily work
in highly dimensional, non-linear feature spaces, and that the
solution achieved is sparse. Due to these good properties, they have
been now extensively used in, e.g., speech recognition, object
classification and function approximation with good results
\cite{Cristianini00}. For an extensive introduction to regression
based upon SVMs, see, e.g., \cite{SmolaTut2004}. Our system employs
LIBSVM v2.82 \cite{ChangL01}, a standard, efficient implementation of
Support Vector Machines.

We were mainly interested in answering two questions:

\begin{enumerate}

  \item how far in the future can our system predict well?

  \item how does the knowledge of the grasped object affect the error?

\end{enumerate}

In order to answer the questions, we have set up a regression
experiment in which a \emph{blind fraction} $0 \leq B \leq 1$ of the
grasp was taken as main parameter. $B$ indicates what fraction of the
grasp, from the contact point backwards, is hidden to the system. In
practice, since the sampling frequency was $50$Hz and each grasp was
stretched to $1$ second, a sequence of $(1-b) \cdot 50$ samples was
fed to the SVM, along with the desired value as regression target,
i.e., the actual position/orientation/posture of the hand at the time
of contact. This procedure was repeated independently for each single
sensor. At the end, the overall error was obtained by grouping the
errors for each sensor this way: the position of the hand ($3$
sensors), the hand orientation ($2$ sensors), and the posture of the
hand ($22$ sensors). According to the device resolutions detailed in
the previous Section, we set the $\epsilon$ value related to the SVM
\cite{SmolaTut2004} like this: $0.1$ inches for each component of the
hand position, $0.5$ degrees for each component of the hand
orientation and $1$ degree for each component of the hand posture. As
an ``acceptable'' error, we considered approximately two or three
times this value.

As far as the Support Vector Machines are concerned, a Gaussian kernel
was chosen, and its hyperparameters ($C$ and $\gamma$) were found by
grid search. The error on regression was obtained by $5$-fold
cross-validation. This ensures a good degree of generalisation.

\subsubsection*{Overall accuracy}

In order to answer the first question above, we collected all sessions
together and checked the overall regression error. Results are visible
in Figure \ref{gino} for various values of $B$...

\subsubsection*{Accuracy given an object}

In order to answer the second question, we ran three separate
regression experiments by collecting, in turn, all sessions obtained
for the beer can, for the scotch roll and for the mug. Results are
visible in Figure \ref{gino2}.
