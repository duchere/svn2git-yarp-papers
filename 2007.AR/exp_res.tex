Our machine learning system is based upon Support Vector Machines
(SVMs). Introduced in the early 90s by Boser, Guyon and Vapnik
\cite{BGV92}, SVMs are a class of kernel-based learning algorithms
deeply rooted in Statistical Learning Theory \cite{v-edbed-82}. As
opposed to analogous algorithms such as, e.g., artificial neural
networks, they have the main advantages that their training is
guaranteed to end up in a global solution, that they can easily work
in highly dimensional, non-linear feature spaces, and that the
solution achieved is sparse. Due to these good properties, they have
been now extensively used in, e.g., speech recognition, object
classification and function approximation with good results
\cite{Cristianini00}. For an extensive introduction to regression
based upon SVMs, see, e.g., \cite{SmolaTut2004}. Our system employs
LIBSVM v2.82 \cite{ChangL01}, a standard, efficient implementation of
Support Vector Machines.

We were mainly interested in answering two questions:

\begin{enumerate}

  \item how far in the future can our system predict well?

  \item how does the knowledge of the grasped object affect the error?

\end{enumerate}

In order to answer the first question, we have set up a regression
experiment in which a \emph{blind fraction} $0 \leq B \leq 1$ of the
grasp was taken as main parameter. $B$ indicates what fraction of the
data associated to the grasp, from the contact point backwards, is
hidden to the system. In practice, since the sampling frequency was
$50$Hz and each grasp was stretched to $1$ second, a sequence of
$(1-b)*50$ samples was fed to the SVM, along with the desired value as
target. This procedure was repeated for each single sensor. At the
end, the overall error was obtained by grouping the $28$ sensors this
way: the $3$ numbers representing the position of the hand (coming
from the FoB), the $3$ numbers representing the orientation of the
hand (coming from the FoB too), and the $22$ numbers representing the
posture of the hand (coming from the DataGlove). According to the
device resolutions detailed in the previous Section, we set the
$\epsilon$ value related to the SVM \cite{SmolaTut2004} like this:
$0.1$ inches for each component of the hand position, $0.5$ degrees
for each component of the hand orientation and $1$ degree for each
component of the hand posture. As an ``acceptable'' error, we
considered approximately two or three times this value.

As far as the Support Vector Machines are concerned, a Gaussian kernel
was chosen, and its hyperparameters ($C$ and $\gamma$) were found by
grid search. The error on regression was obtained by $5$-fold
cross-validation. This ensures a good degree of generalisation.

\subsubsection*{Overall accuracy}

\subsubsection*{Accuracy given an object}

\subsubsection*{Discussion}

