The dexterity of active hand prosthetics is limited not only due
to the limited availability of dexterous prosthetic hands, but
mainly due to limitations in interfaces. How is an amputee
supposed to command the prosthesis what to do (i.e., how to grasp
an object) and with what force (i.e., holding a hammer or grasping
an egg)? So far, in literature, the most interesting results have
been achieved by applying machine learning to forearm surface
electromyography (EMG) to \emph{classify} finger movements; but
this approach lacks, in general, the possibility of quantitatively
determining the force applied during the grasping act.

In this paper we address the issue by applying machine learning to the
problem of \emph{regression} from the EMG signal to the force a human
subject is applying to a force sensor. A detailed comparative analysis
among three different machine learning approaches (Neural Networks,
Support Vector Machines and Locally Weighted Projection Regression)
reveals that the type of grasp can be reconstructed with an average
accuracy of $90\%$, and the applied force can be predicted with an
average error of $10\%$, corresponding to about $5$N over a range of
$50$N. None of the tested approaches clearly outperforms the others,
which seems to indicate that machine learning as a whole is a viable
approach.
%
%Notwithstanding the well-known bad conditioning of the surface EMG
%signal then, this looks highly encouraging in applying machine
%learning to enable amputees gain a fine control over advanced
%prosthetic hands, also since a surface EMG setup can be cheaply and
%easily realised and it is totally non-invasive.
