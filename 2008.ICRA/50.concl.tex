In this paper we have presented a machine learning approach to
joint classification of grasping and regression on the applied force,
using forearm surface electromyography. The approach is totally
non-invasive and easy to set up and use, and it can require as little
as about $15$ minutes of training to achieve good results. (This does
not include the time required for uniformisation, which was not
optimised at this stage.)

Our experiments, carried out using a Support Vector Machine with
Gaussian kernels, a Neural Network with sigmoidal activation
function and Locally Weighted Projection Regression, indicate that
the approach achieves an average accuracy of around $90\%$ in
classification of grasp types and a normalised root MSE of $10\%$
in prediciton of the force applied during the grasp. This makes it
suitable for driving a force-controlled robotic hand, and opens a
new field of applications in prosthetic hands.

The approach has recently been applied to the DLR-II Hand (see Figure
\ref{fig:DLRHandII} and \cite{ButFisGre2004}) with good results, which will be
shortly made available to the scientific community. Future work mainly
includes, beside improving the accuracy of the present system via a
better experimental setup, the implantation of the hand on a disabled
person with extensive experiments to test the real usability of the
proposed approach.
