\section{The robot Obrero}
\label{sec:platform}

Obrero \cite{obrero} 
consists of a hand, an arm and a head (Figure~\ref{fig:RobotObrero}). 
Obrero was designed to approach manipulation as a task manly guided by
tactile and force feedback.
We use the robot's limb as a sensing/exploring device as opposed 
to a pure acting device. This is a convenient approach to operate 
in unstructured environments, on unmodeled objects. 
Obrero's limb is sensor-rich and
safe, it is designed to reduce the risk of damages upon contact
with objects. The head consists of a commercial camcorder (SONY) 
that can move along the pan and tilt directions. The arm has 6 
Degrees of Freedom (DOF) distributed in this way: 3 in the shoulder, 1 at the level of 
the elbow and 2 in the wrist. The arm mounts Series Elastic Actuators 
\cite{williamson95series} which provide low-impedance and force feedback 
at each joints (\cite{domo}). Position feedback is provided by 
analog encoders (potentiometers).

The hand consists of a palm, a thumb, a middle and an
index finger. Each one of the fingers has two links that can be
opened and closed. The thumb and the middle finger can also
rotate. These rotations allow the the thumb to oppose to either 
the index or the middle fingers.

%By rotating the thumb it can be opposed to the index
%finger and by rotating the thumb and the middle, they can oppose
%to each other.

The total number of degrees of freedom in the hand is 8. The hand 
is underactuated, only 5 motors are connected to the fingers. Three motors control
the opening and closing of the fingers. The remaining two motors actuate
the rotation of the thumb and middle finger. The phalanges in each fingers 
are mechanicaly coupled and actuated by a common motor. The link between 
the two joints is realized by means of Series 
Elastic Actuators, to allow independent motion 
whenever one of the phalanges blocks (for example as a result of contact 
with an object). This elastic coupling allows the hand to automatically 
adapt to the object it grasps. All joints in the hand are equipped with an 
optimized version of the Series Elastic Actuators \cite{actuator} which provide 
force feedback and reduce the mechanical impedence of the fingers. Finally 
position feedback is obtained through analog encoders mounted in all joints. 

%However, they can decouple thanks to the SEA's in each
%joint. All the joints of the hand are controlled using an
%optimized design for a series elastic actuator \cite{actuator}.
%There are a total of 8 SEA's int the hand. Series elastic
%actuators\cite{williamson95series} reduce their mechanical
%impedance and provide force sensing.

The tactile sensors mounted on the robot were designed for robotic tasks. 
Their design was inspired by the dome-like shape of the ridges that have been 
observed in the human skin, where, the innervations at the base of the ridges 
detect the deformation of the skin. 

Inspired by this mechanism we realized a tactile sensor made of silicon rubber 
and with a dome-like shape (see figure\ref{}). At the tip of the dome, in the 
internal part, we mounted a small magnet, whose position is measured by four 
hall-effect sensors placed at the base. The hall-effect sensors in this way 
measure the deformation of the dome by sensing the position of the magnet at the 
tip. The sensors are very sensitive and have been tested to detect up to the 
minimum normal force of 10g. 

The shape the sensors favors contact with the environment from any direction, 
as opposed to most of the tactile sensors which are flat. 
This high deformability and the properties of the silicon rubber allow the 
sensors to conform to the objects, thus increasing friction and improving 
contact detection. 

In the particular implementation used in this paper, we used the ``magnetic'' 
version of the tactile sensors, however, a optical version has been also tested. 
An analysis and description of the design of these sensors can be
found in \cite{etorresjSoft}. Groups of tactile sensors were
placed in parts of the hand. Two groups of four
were placed on each finger (a group in each of the two falanges)
and 16 on the palm. A detail of the palm and fingers can be
observed in figure\ref{}. Each one of these tactile sensors uses 4
sensors to determine the contact forces. That means that overall the 
tactile feedback consist of 160 sensors. At the base of the
palm, where for practical reasons, we were not able to mount 
these tactile sensors, we placed a smaller infrared proximity sensor. 
To summarize, the hand has 5 motors, 8 DOF, 8 force sensors, 10 position
sensors, 160 tactile sensors and a infrared proximity sensor.

