\section{Conclusions}
\label{sec:conclusions}

We have described the design of a behavior that allows a humanoid
robot to grasp objects without prior knowledge about their shape
and location. We summarize here our approach and the lessons 
we learned:
%
\begin{itemize}
%
\item Give up precision, explore instead. Sometime in robotics
we struggle to have robots as precise as possible in performing
the tasks for which we program them. We found that exploration
can be more effective in dealing with uncertainties.
%
\item Be soft. Exploration must be gentle if we want to avoid
catastrophic effects on either the robot or the
objects/environment. The mechanical design of the robot proved
helpful in this respect.
%
\item Sense and exploit the environment. If inquired the world
can provide useful feedback; however the robot must be able
to ask the right questions (interact) and interpret the answers
(have appropriate sensors).
%
\end{itemize}

We endowed the robot with the minimum capabilities required to explore
the environment. These include a simple ability to detect visual
motion, a way to control the arm to roughly reach for objects and a
set of explorative primitives. Haptic feedback drives the
exploration and allows the robot to successfully grasp objects on
a table. We show that the information generated in this ways
can be potentially used to learn physical properties of objects like shape.


In the context of epigenetic robotics we are interested in studying
methods to improve the perceptual abilities of robots by exploiting the
physical interaction with the environment. In this paper we have shown
how haptic feedback can significantly improve this interaction thereby
enhancing the robot's ability to learn about the environment.

Finally, it is worth saying that, to better illustrate our point, we
deliberately took a somewhat extreme approach. We certainly believe 
that future robots will have to take advantage of the integration of
all sensory modalities.


