\section{Conclusions}
\label{sec:conclusions}

We have described the design of a behavior that allows a humanoid
robot to grasp objects without prior knowledge about their shape
and location. We summarize here the lessons we learnt:[\textbf{X
maybe saying approach and lessons to avoid a cut in the flow after
the bullet points X]}
\begin{itemize}

\item Give up precision, explore instead. Sometime in robotics
we struggle to have robots as precise as possible in performing
the tasks for which we program them. We found that exploration
can be more effective in dealing with uncertainties.

\item Be soft. Exploration must be gentle \textbf{[X That does not
mean very slow. maybe saying this X]}if we want to avoid
catastrophic effects on either the robot or the
objects/environment. The mechanical design of the robot proved
helpful in this respect.

\item Sense and exploit the environment. If inquired the world
can provide useful feedback; however the robot must be able
to ask the right questions (interact) and interpret the answers
(have appropriate sensors).

\end{itemize}

We endowed the robot with the minimum capabilities required to explore
the environment. These include a simple ability to detect visual
motion, a way to control the arm to roughly reach for objects and a
set of explorative primitives. Haptic feedback drives the
exploration and allows the robot to successfully grasp objects on
a table. We show that the information generated in this ways
can be potentially used to learn physical properties of objects like shape.

In the context of epigenetic robotics we are interested in studying
methods to improve the perceptual abilities of robots by exploiting the
physical interaction with the environment. In this paper we have shown
how haptic feedback can significantly improve this interaction thereby
enhancing the robot's ability to learn about the environment.

Finally, it is worth saying that, to prove \textbf{[X show is
better than prove if the reviewer is picky :) X]} our point, we
deliberately took a somewhat extreme approach. \textbf{[X maybe
change to we certainly believe X]} Of course we believe that
future robots will have to take advantage of the integration of
all sensory modalities.

%We have shown that a compliant robot hand is capable of safely
%reaching toward a object with only a rough idea of its location.
%% We have demonstrated that a this reaching behavior together with a
%% exploratory behavior can use the tactile sensing information to
%% successfully guide the motion of the hand toward a position
%% adequate for grasping and lifting. We have also shown that the
%% sensors used respond to the requirements of the task. They detect
%% the contact with different parts of different objects and increase
%% the friction for helping the lifting. Finally, we have
%% demonstrated that the whole behavior does not rely in a prior
%% knowledge of the object. In short, a robot with the capability of
%% real interaction with its environment, that is which can gently
%% act on it and feel it, allows us to successful interact with
%% unknown objects.



%We have demonstrated a compliant robot hand capable of safely
%coming into contact with a variety of objects without any prior
%knowledge of their presence or location the safety is built into
%the mechanics and the low level control, rather than into careful
%trajectory planning and monitoring. We have shown that, once in
%contact with these objects, the robot can perform a useful
%exploratory procedure: tapping. The repetitive, redundant,
%cross-modal nature of tapping gives the robot an opportunity to
%reliably identify when the sound of contact with the object
%occurs, and to collect samples of that sound. We demonstrated the
%utility of this exploratory procedure for a simple object
%recognition scenario. This work fits in with a broad theme of
%learning about objects through action that has motivated the
%authors' previous work (Fitzpatrick et al., 2003). We wish to
%build robots whose ability to perceive and act in the world is
%created through experience, and hence robust to environmental
%perturbation. The innate abilities we give our robots are not
%designed to accomplish the specific, practical, useful tasks which
%we (and our funders) would indeed like to see, since direct
%implementations of such behaviors are invariably very brittle;
%instead we concentrate on creating behaviors that give the robot
%robust opportunities for adapting and learning about its
%environment. Our gamble is that in the long run, we will be able
%to build a more stable house by building the ground floor first,
%rather than starting at the top.
