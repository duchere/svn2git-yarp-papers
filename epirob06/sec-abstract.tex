\begin{abstract}
Experimental results in psychology have shown the important role
of manipulation in guiding infant development. This has inspired
work in developmental robotics as well. In this case, however, the
benefits of this approach has been limited by the intrinsic
difficulties of the task. Controlling the interaction between the
robot and the environment in a meaningful and safe way is hard
especially when little prior knowledge is available. We push the
idea that haptic feedback can enhance the way robots interact with
unmodeled environments. We approach grasping and manipulation as
tasks driven mainly by tactile and force feedback. We implemented
a grasping behavior on a robotic platform with sensitive tactile
sensors and compliant actuators; the behavior allows the robot to
grasp objects placed on a table. Finally, we demonstrate that the 
haptic feedback originated by the interaction with the objects carries
implicit information about their shape and can be useful for learning.
\end{abstract}