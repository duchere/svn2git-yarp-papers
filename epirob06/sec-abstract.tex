\begin{abstract}

Experimental results in psychology have shown the important role
of manipulation in guiding infant development. This has inspired
work in developmental robotics as well. In this case, however, the
benefits of this approach has been limited by the intrinsic
difficulties of the task. Controlling the interaction between the
robot and the environment in a meaningful and safe way is hard
especially when little prior knowledge is available. We push the
idea that haptic feedback can enhance the way robots interact with
unmodeled environments. We approach grasping and manipulation as
tasks driven mainly by tactile and force feedback. We implemented
a grasping behavior on a robotic platform with sensitive tactile
sensors and compliant actuators; the behavior allows the robot to
grasp objects placed on table. We demonstrate that the haptic
feedback originated by the interaction with the objects carries
implicit information about their shape and could\textbf{[X can
instead of could X]} be useful for learning.

%% the advantage of
%% interacting with the environment. The ways to interact were very
%% simple but very powerful for learning about the objects with which
%% the robot came in contact. More sophisticated ways to interact
%% with objects, such as grasping, offer the possibility of
%% increasing the benefits of interaction. In this paper, we present
%% an approach to grasping and manipulation based mainly in tactile
%% and force feedback which allows the robot to gently deal with
%% unmodeled environments. We use a robotic platform with highly
%% sensitive tactile sensors and highly compliant actuators to
%% interact safely and gently with objects. We show that this
%% approach is able to successfully grasp unknonw objects and
%% differentiate them by shape.

%that The importance of interacting with the environment has been
%stated in psycological studies an din roby the embodied
%intelligence field and for development. There are many ways to
%interact with environment but the physical contact is he most
%common in biological agents. That is why animals are provided by
%limbs. pwas, etc. All of them covered by a higly inervated skin%
%that allows them to get intimate informatino about their phyisical
%enviroment. This interaction helps them to develop the skills
%needed for their survival and is done in a safe way. Interaction
%with the enviroment is key for the robtos, with outhe posibility
%of the interactions very little can be done.  In this paper we
%show how a robot can explore its enviromante, feel objects and
%grasp them only by using tactile and force feedback information.
%The interaction is safe and useful. In other words, the robot can
%come in actual physical contact with its environment. Many efforts
%in robotics has been done about learning from the environment by
%interacting with it. However, the interaction has been very rough.
%This works will present an alternative to fill this grasp.
%ALthough this is not the ultimate goal.


%Humans use a set of exploratory procedures to examine object
%properties through grasping and touch. Our goal is to exploit
%similar methods with a humanoid robot to enable developmental
%learning about manipulation. We use a compliant robot hand to find
%objects with very rough estimation about their location, and then
%tap grab them. This behavior lets the robot collect samples of the
%grasping of that object.
%
%An important property of embodied agents is their ability to
%interact with the environment in which they operate. This is
%considered of fundamental importance for the emergence of
%intelligent behavior. Recent work in robotics has shown how simple
%actions (like poking and prodding) can facilitate perception and
%learning1. Grasping is particularly appealing in this context
%because provides direct access to physical properties of objects
%(like shape, volume and weight) that are difficult to perceive
%otherwise. Unfortunately this aspect has rarely been investigated,
%with a few exceptions2,3. In part this is because current robots
%have very limited perceptual capabilities. In particular, tactile
%sensing is often inadequate or inexistent. For this reason most of
%the research on manipulation has focused on vision and has left
%haptic sensing overlooked. This paper pushes the idea that
%sensitive haptic feedback dramatically simplifies manipulation and
%improves the ability of robots to successfully interact with
%unknown objects. In this paper we report a series of experiments
%recently performed on Obrero. The robot exploits its sensing
%capabilities to grasp a number of objects individually placed on a
%table. No prior information about the objects is available to the
%robot. Vision is used at the beginning of the task to direct the
%attention of the robot and to give a rough estimation of the
%position of the object. Tactile feedback allows the robot to
%refine this initial estimation during the task. The robot reaches
%for the object and explores with the hand the area around it.
%During exploration the robot exploits tactile feedback to find the
%actual position of the object and grasp it. The mechanical
%compliance of the robot and the control facilitate the exploration
%by allowing a smooth and safe interaction with the object. In
%figure 2, we observe a sequence of the robot grasping one of the
%objects. Preliminary analysis of the data collected in these
%experiments shows that the haptic feedback originated by the
%interaction between the objects and the robot carries information
%useful for learning.


\end{abstract}
