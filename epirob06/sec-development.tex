\section{Overall developmental approach}
 \label{sec:approach}
[Should we keep this?]

 We wish to give our robots many ways to learn about
objects through action (Fitzpatrick et al., 2003). This
contributes to perceptual development, where the robot's
experience of the world is filtered by prior experience. This
process can be broken down into four steps:

\begin{itemize}

\item Identification of an opportunity to reliably extract some
object features.

\item Exploitation of that opportunity to extract those features.

\item Use careful generalization to transform the robot's
perception of its environment.

\item Transformation of the robot's activity, enabled by its
extended perceptual abilities.

\end{itemize}

In previous work, we have demonstrated this process. In (Arsenio
et al., 2003), we showed that poking an object gives us the
opportunity to reliably extract visual features of its appearance.
By carefully choosing features that generalize, the robot's
perception of its environment is transformed, and new activities
are enabled (Fitzpatrick, 2003b). Other opportunities we have
explored include the use of grasping (Natale et al., 2005) and the
integration of multi-modal cues across sound, vision, and
proprioception (Fitzpatrick et al., 2005, Arsenio and Fitzpatrick,
2005). Having established this process, we are now seeking to
broaden the range of opportunities that can be identified and
exploited (steps 1 and 2 above). In the current work, we identify
(and in fact create) an opportunity to reliably extract examples
of contact sounds involving an object (by tapping that object). We
build the appropriate robot behavior and data collection
infrastructure to gather those features.
