\section{Background}
\label{sec:background}

Little is known concerning how infants \textbf{[X the talk about
newborns 2 months and 5 months. may be newborn instead of infants
X}} use tactile sensing for manipulation \cite{streri93Seeing}.
%Tactile sensing in neonates is very poorly known
%in contrast to the visual mechanism\cite{streri93Seeing}.
However, tactile sensing provides information to children since
his time in the wound. Motor abilities of children are quite
limited during the first months of development. This does not
prevent infants from using their hand to engage interaction
with the world.
%Hands and arms are intensively used to
%learn about their environment even though their motion
%capabilities are not completely developed at an early stage.
In some circumstances children exploit tactile feedback to learn
about objects \cite{streri86Habituation}. Streri and P\^{e}cheux
measured the habituation time of newborns (2 months and 5 months
old) during tactile exploration of objects placed in their hands.
In this experiment children spent more time exploring novel rather
than familiar objects, even when they did not make visual contact
with the hand. This experiment was repeated with visual feedback
showing the same behavior.

%% For example, \cite{streri86Habituation} placed different objects in
%% the hand of children who had no visual contact with their hand.
%% The time that the children hold the object was used to determine
%% when they habituated to the object. After hand in the object a
%% number of times the holding time is reduced. When new objets where
%% presented the holding time increased. This shows learning and
%% adaptation to novelty. This experiment was repeated with visual
%% feedback showing the same behavior.
% As shown in the experiment: a
%3D projection of an object was shown to children form....
Later on in development the integration between visual and tactile
modalities during reaching tasks has been found. Tactile
information is, for example, used to confirm visual stimulus as
described in \cite{bower70Coordination}. In this experiment an
image of an object was projected used polarized lenses so that
children wearing ``stereo'' glasses perceived a 3D object.
Children of different ages were tested in two conditions: when the
object was actually present and when it was not. Older children (5
to 6 months-old) showed surprise when the did not feel the contact
when expected, whereas the younger ones closed their hand to grasp
the object \textbf{[X and  did not show any surprise. * This is
not true. Both showed surprise. Also, I am not sure if its later
because the experiment was done with children from 6 days to 6
months. And we do not know if they were tested in the both
conditions. X]}

In adults, several studies have revealed the importance of
somatosensory input (force and touch). For example
\cite{Johansson90Liffiing} has studied in detail what feedback
is provided by the skin during object lifting tasks and how it is
used to control the movements of the fingers. The results of these
experiments proved the importance of somatosensory feedback: they
showed that human subjects had difficulties avoid object slipping when
they had their fingertips anesthetized, even with full vision
\cite{johansson91how}. Haptic feedback has an important role for
object perception as well. Lederman and Klatzky \cite{klatzky87Hand}
identified and described a set of motor strategies
\emph{exploratory procedures} used by humans to determine properties
of objects such as shape, texture, weight or volume.

%Tactile sensing gets information from the object and triggers
%motor responses. This is because the mechanoreceptors, being
%closer to manipulate the object, provide timely mechanical
%information of the interaction between the hand and the
%object\cite{Johansson90Liffiing}. In \cite{Johansson90Liffiing} an
%analysis of the forces observed in the grasp-and-lift task is
%presented. The analysis of many transition phase is mentioned but
%not much data is provided about the initial contact. Which is a
%big problem in robotics\cite{volpe90real}.
%Humans use a set of strategies collectively called exploratory
%procedures (Lederman and Klatzky, 1987) in their perception of the
%world around them, such as tracing object outlines with a finger.
%This has inspired work on robotics.

%We address the task of approaching and lifting an object using
%tactile feedback to modulate the motor control of the hand and the
%arm.


%In robotics exploratory robotics has been developed to do the
%same. Active exploration is needed in unstructured environments.
%(Bajcsy 1989). Humans are very good at extracting propreties of 3D
%objects. Very bad at 2-D compared to Vision. (cite).

%An analog of human sensitivity to thermal diffusivity was
%developed by (Campos et al., 1991), allowing a robot to
%distinguish metal (fast diffusion) from wood (slow diffusion).

%Tactile information gets information from the object and triggers
%motor responses. This is because the mechanoreceptors, being
%closer to manipulate the object, provide timely mechanical
%information of the interaction between the hand and the
%object\cite{Johansson90Liffiing}. In \cite{Johansson90Liffiing} an
%analysis of the forces observed in the grasp-and-lift task is
%presented. The analysis of many transition phase is mentioned but
%not much data is provided about the initial contact. Which is a
%big problem in robotics\cite{volpe90real}.

%Infromation of the tactile sensors during manipulative finger
%movements and its utilization in motion control. Known in flexing
%movements of teh receptor-bearing fingers.

%\cite{Johansson90Liffiing} states the importance of teh teactile
%feedback in the motion control of hand. Tactile feedback afferents
%resopnd during mnaipulative tasks (suach lifting). These
%activation are important for the control of manipulation.





%Experimental results suggest that from a very early age, arm
%movements in infants are influenced by vision. For example, van
%der Meer and colleagues found that sight of the hand allows
%infants to maintain the posture of the hand when pulled by an
%external force (van der Meer et al., 1995). Von Hofsten compared
%the arm movements of two groups of infants in the presence and
%absence of an object and found that in the former case arm
%movements were significantly more frequent. When the infants were
%fixating the objects the movements were directed closer to it (von
%Hofsten, 1982). Taken together, these results suggest that in
%children some sort of eye-hand coordination is already present
%soon after birth. But on the other hand, continuous visual
%feedback from the hand is not required for infants to reach for an
%object (Clifton and D.W. Muir, 1993, Clifton et al., 1994). Indeed
%it is only at 9 months of age that children seem to be able to
%exploit visual feedback from the hand during the approach phase
%(Ashmead et al., 1993). A possible explanation for this could be
%that in the first months of development the visual system of
%infants is still rather immature: visual acuity is limited and
%perception of depth has not developed yet (Bushnell and Boudreau,
%1993). Later on during development the role of vision is certainly
%crucial to control the correct preshape of the hand according to
%the object's shape and orientation; however, tactile feedback from
%the contact with an object is an alternative source of information
%that could initially substitute for the visual feedback. In
%adults, several studies have revealed the importance of
%somatosensory input (force and touch); for example human subjects
%with anesthetized fingertips have difficulty in handling small
%objects even with full vision (Johansson, 1991). Humans use a set
%of strategies collectively called exploratory procedures (Lederman
%and Klatzky, 1987) in their perception of the world around them,
%such as tracing object outlines with a finger. This has inspired
%work on robotics. An analog of human sensitivity to thermal
%diffusivity was developed by (Campos et al., 1991), allowing a
%robot to distinguish metal (fast diffusion) from wood (slow
%diffusion).

%The importance of interacting with the world to learn form it has
%been repeatly stated. .. The development of the hand-eye
%coordination has been reported as important for adquiring ... and
%for interacting in the world.


%Contact with objects to extract properties is important. However,
%actual contact with the world has been limited in robotics.
%\cite{volpe90real}, for example, has analyzed the problem of
%impact. That is the first contact were usually the robot switches
%form force control to position control and inestabilyt can be
%presetn. Especailly becaus teh impact produces oscillation that
%make the robto  to operate between the condition s of contact and
%not contact withte table. This nono-linearity makes even more
%possible the inestability.






%In adults, several studies have revealed the importance of
%somatosensory input (force and touch); for example human subjects
%with anesthetized fingertips have difficulty in handling small
%objects even with full vision(Johansson, 1991). Humans use a set
%of strategies collectively called exploratory procedures (Lederman
%and Klatzky, 1987) in their perception of the world around them,
%such as tracing object outlines with a finger. This has inspired
%work on robotics. An analog of human sensitivity to thermal
%diffusivity was developed by (Campos et al., 1991), allowing a
%robot to distinguish metal (fast diffusion) from wood (slow
%diffusion).

%Skin feedback has not been considerd in  conotrol, for example the
%possible feedback of postion no  joints.
