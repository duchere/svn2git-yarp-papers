\section{The importance of haptic feedback for perception and action}
\label{sec:background}

In adults, several studies have revealed the importance of
somatosensory input (force and touch). For example Johansson and Westling
\cite{Johansson90Tactile} have studied in detail what feedback
is provided by the skin during object lifting tasks and how it is
used to control the movements of the fingers. The results of these
experiments proved the importance of somatosensory feedback: they
showed that human subjects had difficulties avoid object slipping when
they had their fingertips anesthetized, even with full vision
\cite{johansson91how}. 

Haptic feedback has an important role for
object perception as well. Lederman and Klatzky \cite{klatzky87Hand}
identified and described a set of motor strategies
\emph{exploratory procedures} used by humans to determine properties
of objects such as shape, texture, weight or volume.

Little is known concerning how infants use 
tactile sensing for manipulation \cite{streri93Seeing}.
%Tactile sensing in neonates is very poorly known
%in contrast to the visual mechanism\cite{streri93Seeing}.
%However, tactile sensing provides information to children since
%his time in the wound. 
In some circumstances children exploit tactile feedback to learn
about objects \cite{streri86Habituation}. Streri and P\^{e}cheux
measured the habituation time of newborns (2 months and 5 months
old) during tactile exploration of objects placed in their hands.
In this experiment children spent more time exploring novel rather
than familiar objects, even when they did not make visual contact
with the hand. This experiment was repeated with visual feedback
showing the same behavior.

Motor abilities of children are quite
limited during the first months of development. This does not
prevent infants from using their hand to engage interaction
with the world. The importance of motor activity for perceptual 
development has been emphasized in developmental psychology 
\cite{hofsten04motor,gibson88explore}. Researchers agree on
the fact that motor development determines the timing of
perceptual development. In other words the ability of infants 
to explore the environment would determine their capacity to 
perceive certain properties. 
Accordingly, perception of object features like temperature, 
size and hardness is likely to occur relatively early in development, 
whereas properties requiring more dexterous actions like texture or 
three dimensional shape would emerge only later on (see \cite{bushnell93motor} 
for a review).
%% Later on in development the integration between visual and tactile
%% modalities during reaching tasks has been found. Tactile
%% information is, for example, used to confirm visual stimulus as
%% described in \cite{bower70Coordination}. In this experiment an
%% image of an object was projected used polarized lenses so that
%% children wearing ''stereo'' glasses perceived a 3D object.
%% Children of different ages were tested in two conditions: when the
%% object was actually present and when it was not. Older children (5
%% to 6 months-old) showed surprise when the did not feel the contact
%% when expected, whereas the younger ones closed their hand to grasp
%% the object \textbf{[X and  did not show any surprise. * This is
%% not true. Both showed surprise. Also, I am not sure if its later
%% because the experiment was done with children from 6 days to 6
%% months. And we do not know if they were tested in the both
%% conditions. X]}
%