\section{Background}
\label{sec:background}
%Experimental results suggest that from a very early age, arm
%movements in infants are influenced by vision. For example, van
%der Meer and colleagues found that sight of the hand allows
%infants to maintain the posture of the hand when pulled by an
%external force (van der Meer et al., 1995). Von Hofsten compared
%the arm movements of two groups of infants in the presence and
%absence of an object and found that in the former case arm
%movements were significantly more frequent. When the infants were
%fixating the objects the movements were directed closer to it (von
%Hofsten, 1982). Taken together, these results suggest that in
%children some sort of eye-hand coordination is already present
%soon after birth. But on the other hand, continuous visual
%feedback from the hand is not required for infants to reach for an
%object (Clifton and D.W. Muir, 1993, Clifton et al., 1994). Indeed
%it is only at 9 months of age that children seem to be able to
%exploit visual feedback from the hand during the approach phase
%(Ashmead et al., 1993). A possible explanation for this could be
%that in the first months of development the visual system of
%infants is still rather immature: visual acuity is limited and
%perception of depth has not developed yet (Bushnell and Boudreau,
%1993). Later on during development the role of vision is certainly
%crucial to control the correct preshape of the hand according to
%the object's shape and orientation; however, tactile feedback from
%the contact with an object is an alternative source of information
%that could initially substitute for the visual feedback. In
%adults, several studies have revealed the importance of
%somatosensory input (force and touch); for example human subjects
%with anesthetized fingertips have difficulty in handling small
%objects even with full vision (Johansson, 1991). Humans use a set
%of strategies collectively called exploratory procedures (Lederman
%and Klatzky, 1987) in their perception of the world around them,
%such as tracing object outlines with a finger. This has inspired
%work on robotics. An analog of human sensitivity to thermal
%diffusivity was developed by (Campos et al., 1991), allowing a
%robot to distinguish metal (fast diffusion) from wood (slow
%diffusion).


Active exploration is needed in unstructured environments. (Bajcsy
1989) Humans are very good at extracting propreties of 3D objects.
Very bad at 2-D compared to Vision. (cite). Tactile information
gets information from the object and triggers motor responses.
This is because the mechanoreceptors, being closer to manipulate
the object, provide timely mechanical information of the
interaction between the hand and the
object\cite{Johansson90Liffiing}. In \cite{Johansson90Liffiing} an
analysis of the forces observed in the grasp-and-lift task is
presented. The analysis of many transition phase is mentioned but
not much data is provided about the initial contact. Which is a
big problem in robotics\cite{volpe90real}.





In adults, several studies have revealed the importance of
somatosensory input (force and touch); for example human subjects
with anesthetized fingertips have difficulty in handling small
objects even with full vision(Johansson, 1991). Humans use a set
of strategies collectively called exploratory procedures (Lederman
and Klatzky, 1987) in their perception of the world around them,
such as tracing object outlines with a finger. This has inspired
work on robotics. An analog of human sensitivity to thermal
diffusivity was developed by (Campos et al., 1991), allowing a
robot to distinguish metal (fast diffusion) from wood (slow
diffusion).


We address the task of approaching an object modulating the motor
control by using tactile feedback.
