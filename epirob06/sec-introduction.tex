\begin{figure}[tbp]
\centerline{
\includegraphics[width=2.5in, angle=270 ]{./figures/ObreroHands.eps}
} \caption{Robot Obrero. The robot has a highly sensitive and
force controlled hand, a single force controlled arm and a
camcorder as a head. Obrero's hand  has three fingers, 8 DOF, 5
motors, 8 force sensors, 10 position sensors and 160 tactile
sensors.} \label{fig:RobotObrero}
\end{figure}

\section{Introduction}

Recent work in developmental robotics has emphasized the role of action
during perception and learning (cite). Developmental pshycology on the 
other hand, recognize that motor activity is of paramount importance 
for the correct emergence of cognition and intelligent behavior (cite). 
Be them artificial or natural all embodied agents have numerous ways 
to exploit the physical interaction with the environment to their advantage. 
In robotics actions like pushing, prodding, and tapping have been used for 
visual and auditory perception respectively (cite). 
More articulated explorative actions and grasping might increase these benefits, 
as they give direct access to physical properties of objects like shape, volume
and weight.

%The capability of interacting with the environment is considered
%of fundamental importance for the emergence of intelligent
%behavior. Recent work in robotics has shown how simple actions
%(like poking and prodding) can facilitate the development of
%perception \cite{Natale05},\cite{metta03early}. More controlled
%actions, such as a gentle exploration and grasping, might increase
%these benefits. Gently exploration allows the robot to deal safely
%even with delicate objects and enables grasping. Grasping provides
%direct access to physical properties of objects (like shape,
%volume and weight) that are difficult to perceive otherwise and
%allows the robot to remain in contact with the object for further 
% inspection.

Unfortunately all these aspects have not been extensively investigated yet. The 
reason for this is that controlling the interaction between the robot and 
the environment is a difficult problem, especially in absence of accurate 
models either of the robot or the environment (as it is often the case in 
developmental robotics). As discussed in \cite{volpe90real} the 
\emph{impact problem} affect the mechanical design of the robot as well as
its control.

The design of the robot can alleviate these problems. On the one hand, we know 
for example that having a certain degrees of elasticity in the limbs helps to 
``smooth''  and control the forces that originate upon contact.  
On the other hand, we can work to enhance the perceptual abilities of the robot.
Traditional robotic systems in fact have perceptual systems that do not seem
adequate for grasping. Haptic feedback in particular in often quite limited or 
completely absent. This is because, unfortunately, most of the tactile 
sensors commercially available are inadequate for robotics tasks: they are only 
sensitive to forces coming from a specific angle of incidence, rigid 
and almost frictionless. 

Obrero\cite{obrero} in a upper body humanoid robot, designed to overcome these
limitations. It is equipped with series elastic actuators, 
which provide intrinsic elasticity and force feedback at each joint. The hand 
is equipped with tactile sensors (cite) which provide a deformable and sensitive 
interface between the fingers and the objects.

We report a series of experiments where Obrero exploits its
sensing capabilities to grasp a number of objects individually
placed on a table. No prior information about the objects is
available to the robot. Vision is used at the beginning of the
task to direct the attention of the robot and to give a rough
estimation of the position of the object. Next, the robot moves
its limb towards the object and explores with the hand the area
around it. During exploration, the robot exploits tactile feedback
to find the actual position of the object and grasp it. The
mechanical compliance of the robot and the control facilitate the
exploration by allowing a smooth and safe interaction with the
object. Results show that the haptic information acquired by
the robot during grasping codes information about the shape of 
the objects.

The paper is organized as follows. Section~\ref{sec:background}
briefly reviews XXX and introduces the notion of exploratory
procedures in humans and robots. Section~\ref{sec:platform}
describes our robotic platform, designed to enable sensor-rich
reaching and grasping (sensitive manipulation).In
section~\ref{sec:approach}, we review our general developmental
approach to robot perception. This motivates us to develop a robot
behavior (described in Section~\ref{sec:behavior}) which gives the
robot a way to actively probe of objects in its environment in a
robust way. Section~\ref{sec:results} present an evaluation of
method for grasping. Finally, section ~\ref{sec:conclusions}.
concludes with a discussion about the results.

