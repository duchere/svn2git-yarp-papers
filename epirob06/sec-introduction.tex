\begin{figure}[tbp]
\centerline{
\includegraphics[width=2.5in, angle=270 ]{./figures/ObreroHands.eps}
} \caption{Robot Obrero. The robot has a highly sensitive and
force controlled hand, a force controlled arm and a
camcorder as a head. Obrero's hand  has three fingers, 8 DOF, 5
motors, 8 force sensors, 10 position sensors and 160 tactile
sensors.} \label{fig:RobotObrero}
\end{figure}

\section{Introduction}

Recent work in developmental robotics has emphasized the role of
action for perception and learning 
\cite{metta03early,natale04learning,natale05from}. Developmental
psychology on the other hand, recognize that motor activity is of
paramount importance for the correct emergence of cognition and
intelligent behavior \cite{streri93Seeing,bushnell93motor}. 
Being them either artificial or natural, all embodied agents have 
numerous ways to exploit the physical interaction with the environment 
to their advantage. In robotics actions like pushing, prodding, 
and tapping have been used for visual and auditory perception 
respectively \cite{metta03early,etorresjara05tapping}. More articulated 
explorative actions or grasping might increase these benefits, as they give
direct access to physical properties of objects like shape, volume
and weight.

%The capability of interacting with the environment is considered
%of fundamental importance for the emergence of intelligent
%behavior. Recent work in robotics has shown how simple actions
%(like poking and prodding) can facilitate the development of
%perception \cite{Natale05},\cite{metta03early}. More controlled
%actions, such as a gentle exploration and grasping, might increase
%these benefits. Gently exploration allows the robot to deal safely
%even with delicate objects and enables grasping. Grasping provides
%direct access to physical properties of objects (like shape,
%volume and weight) that are difficult to perceive otherwise and
%allows the robot to remain in contact with the object for further
% inspection.

Unfortunately all these aspects have not been extensively investigated yet. One
of the reasons for this is that controlling the interaction between the robot and
the environment is a difficult problem \cite{volpe90real}, especially in absence of accurate
models either of the robot or the environment (as it is often the case in
developmental robotics). %As discussed in \cite{volpe90real} the
%\emph{impact problem} affect the mechanical design of the robot as well as
%its control.

The design of the robot can ease these problems. We know
for example that having a certain degrees of elasticity in the limbs helps to
``smooth''  and control the forces that originate upon contact.
Another approach is to to enhance the perceptual abilities of the robot.
Traditional robotic systems in fact have perceptual systems that do not seem
adequate for grasping. Haptic feedback in particular in often quite limited or
completely absent. This is because, unfortunately, most of the tactile
sensors commercially available are inadequate for robotics tasks: they are only
sensitive to forces coming from a specific angle of incidence, rigid
and almost frictionless.

Obrero\cite{obrero} is an upper body humanoid robot designed to
overcome these limitations. It is equipped with series elastic
actuators, which provide intrinsic elasticity and force feedback
at each joint. The hand is equipped with tactile sensors
\cite{etorresjSoft} which provide a deformable and sensitive
interface between the fingers and the objects.

We report a series of experiments where Obrero exploits its
sensing capabilities to grasp a number of objects individually
placed on a table. No prior information about the objects is
available to the robot. The use of visual feedback was voluntarely 
limited. Vision is used at the beginning of the task to direct 
the attention of the robot and to give a rough
estimation of the position of the object. Next, the robot moves
its limb towards the object and explores with the hand the area
around it. During exploration, the robot exploits tactile feedback
to find the actual position of the object and grasp it. The
mechanical compliance of the robot and the control facilitate the
exploration by allowing a smooth and safe interaction with the
object. Results show that the haptic information acquired by
the robot during grasping carries information about the shape of
the objects.

The paper is organized as follows. Section~\ref{sec:background}
briefly reviews the importance of haptic feedback 
for manipulation in infants and humans. Section~\ref{sec:platform}
describes our robotic platform. Section~\ref{sec:controlling} provides
some implementation details and described the grasping behavior. The latter
is evaluated in section~\ref{sec:results}. Finally, section ~\ref{sec:conclusions}.
draws the conclusions of this work.