Our results clearly indicate that the problem, at least from a
theoretical point of view, is solved for healthy subjects. The
indicated approach, that is, Support Vector Machines, obtains
excellent results in all variants hereby presented. This does not
mean, of course, that SVMs are the one and only good method; actually,
Neural Networks are LWPR (see \cite{lwpr}) have already been employed
with only slightly worse results, and probably even simpler approaches
would get an acceptable level of performance.

This paper, as already said, is a companion to, and the natural
extension of, \cite{2008.ICRA,2008.BioCyb}. Here we show that the
approach described therein can be applied with sucess to any healthy
subject, and in a non-controlled, DLA-like framework. We also show
that the uniformisation procedure produces remarkably small training
sets, which nevertheless attain an excellent accuracy, and we confirm
the phenomenon therein described, consisting in a linear degradation
of the performance, as the minimum distance $d$ is linearly changed,
whereas the training set size changes polynomially.

We are now ready to claim that machine learning can be used in a
practical application of this framework, at least on healthy subjects,
to realise an adaptive approach to EMG-based control of mechanical
hands. As far as ``real'' prosthetics is concerned, we have already
carried on experiments on amputees, revealing that a surprisingly fine
residual muscular activity can be found in stumps, even several
\emph{decades} after the operation (see our poster
\cite{2008.Neurorob}; a full report is in preparation). Muscular
plasticity means highly discernible signals, and a system such as the
one described here is then able to solve the problem in that case,
too.

The possibility of building a common model to aid reciprocal learning
is fascinating. In this paper we have shown that it should be possible
to build one without much effort; a combination of data taken from
other patients, together with the uniformisation strategy, could
actually give a head-start to the prosthesis, therefore shortening the
patient's training time.

All in all, we believe that the new, exciting scenario of
\emph{adaptive prosthetics} is ready to appear. In such a scenario,
the patient and the prosthesis are involved in a loop of
\emph{reciprocal learning} and the training time is dramatically
shorter than it is now. At the same time, the amputee can control the
prosthesis in a much finer way, allowing for a better restoration of
pre-operation DLA capacity. This will change in a better way the
amputees' life. The method can be, in principle, extended to other
kinds of amputation and disability.
