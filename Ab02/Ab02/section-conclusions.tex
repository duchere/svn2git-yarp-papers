
\section{Discussion and Conclusions}

In this paper, we showed how causality can be probed at different
levels by the robot.  Initially the environment was the body of the
robot itself, then later a carefully circumscribed interaction with
the outside world.  This is reminiscent of Piaget's distinction
between primary and secondary circular
reactions~\cite{ginsburg78piaget}.  Objects are central to interacting
with the ouside world.  We raised the issue of how an agent can
autonomously acquire a working definition of objects. 

\ifverbose
According to
recent neuroscience results the problem is ``well posed'' only if the
agent is embodied.
\fi

\ifverbose
We also 
the concept of causality and to a certain extent
our particular implementation fit well with a bulk of neural science
results. Implementing a model of development might help to 
understand ``how the brain does it''.
\fi


In computer vision there is much to be gained by bringing a
manipulator into the equation.  Many variants and extensions to the
experimental ``poking'' strategy explored here are possible.  For
example, a robot might try to move an arm around {\em behind} the
object.  As the arm moves behind the object, it reveals its occluding
boundary.  This is a precursor to visually extracting shape
information while actually manipulating an object, which is more
complex since the object is also being moved and partially occluded by
the manipulator.  Another possible strategy that could be adopted as a
last resort for a confusing object might be to simply hit it firmly,
in the hopes of moving it some distance and potentially overcoming
local, accidental visual ambiguity.  Obviously this strategy cannot
always be used!  But there is plenty of room to be creative here.



%The number of papers written on techniques for visual segmentation is
%vast.  Methods for characterizing the shape of an object through
%tactile information are also being developed, such as shape from
%probing \cite{cole87shape,paulos99fast} or pushing
%\cite{moll01reconstructing,jia98observing}.  But while it has long
%been known that motor strategies can aid
%vision~\cite{ballard91animate}, work on active vision has focused
%almost exclusively on moving cameras.  There is much to be gained by
%bringing a manipulator into the equation, as we have shown in this
%paper.  Many variants and extensions to the experimental ``poking''
%strategy explored here are possible.  For example, a robot might try
%to move an arm around {\em behind} the object.  As the arm moves
%behind the object, it reveals its occluding boundary.  This is a
%precursor to visually extracting shape information while actually
%manipulating an object, which is more complex since the object is also
%being moved and partially occluded by the manipulator.  Another
%possible strategy that could be adopted as a last resort for a
%confusing object might be to simply hit it firmly, in the hopes of
%moving it some distance and potentially overcoming local, accidental
%visual ambiguity.  Obviously this strategy cannot always be used!  But
%there is plenty of room to be creative here.



