
\iflong
\begin{abstract}
\else
\begin{Abstract}
\fi
 
Vision and manipulation are inextricably intertwined in the primate
brain.  Tantalizing results from neuroscience are illuminating the
mixed representations used by the brain in reaching, grasping, and
object recognition.  We wish to instantiate these results in robotic
form to probe their technical advantages and verify that the
associated models are at least consistent and without lacunae.

We believe it would be missing the point to investigate this on a
platform where dextrous manipulation and sophisticated machine vision
are already implemented (if such a platform existed).

In this paper, we show how we can take a simple precursor to manipulation,
namely poking and prodding, and already realize significant advantages in
visual processing, and make enough progress to develop a system that
is functionally analogous to models coming out of neuroscience.

We show how operational concepts can actually lead to well grounded
objects.


\ifverbose
For the purposes of manipulation, we would like to know what parts of
the environment are physically coherent ensembles -- that is, which
parts will move together, and which are more or less independent.  It
takes a great deal of experience before this judgement can be made
from purely visual information.  This paper develops active strategies
for acquiring that experience through experimental manipulation, using
tight correlations between arm motion and optic flow to detect both
the arm itself and the boundaries of objects with which it comes into
contact.  We argue that following causal chains of events out from
the robot's body into the environment allows for a very natural
developmental progression of visual competence, and relate this idea 
to results in neuroscience.
\fi

\ifverbose
For the purpose of understanding development we would like to present
causality as a possible principle to frame a number of neural science
results coherently. We will show how this can lead also to an
implementation in an artificial system following the epigenetic
approach. To this purpose we will show different levels of causal
linkages, or instances of the general principle, which allow tasks of
increasing complexity to be implemented.  Action and the physical
interaction of the robot with the environment play a fundamental role.
In an ecological perspective, the role of this physical interaction
for developing categorization and object undestanding is emphasized.
\fi

%%{\bf \em
%%\iflong
%%(long version)
%%\else
%%(short version)
%%\fi
%%}

\iflong
\end{abstract}
\else
\end{Abstract}
\fi
