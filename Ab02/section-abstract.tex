
\iflong
\begin{abstract}
\else
\begin{Abstract}
\fi

Vision and manipulation are inextricably intertwined in the primate
brain.  Tantalizing results from neuroscience are illuminating the
mixed motor and sensory representations used by the brain during
reaching, grasping, and object recognition.  We now know a great deal
about {\it what} happens in the brain during these activities, but not
necessarily {\it why}.  Is the integration we see functionally
important, or just a reflection of evolution's lack of
enthusiasm for sharp
modularity?  We wish to instantiate these results in robotic form to
probe their technical advantages and to find any lacunae that exist in
the models.  We believe it would be missing the point to investigate
this on a platform where dextrous manipulation and sophisticated
machine vision are already implemented in their mature form, and 
instead follow a developmental approach from simpler primitives.

We begin with a precursor to manipulation, simple poking and prodding,
and show how it facilitates object segmentation, a long-standing
problem in machine vision.  The robot can familiarize itself with the
objects in its environment by acting upon them.  It can then recognize
other actors (such as humans) in the environment through their effect
on the objects it has learned about.  We argue that following causal
chains of events out from the robot's body into the environment allows
for a very natural developmental progression of visual competence, and
relate this idea to results in neuroscience.


\iflong
\end{abstract}
\else
\end{Abstract}
\fi
