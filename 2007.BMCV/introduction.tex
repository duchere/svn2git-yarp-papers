
Place recognition is an open and highly challenging problem in
computer vision, especially when applied to mobile robotics in indoor
environments. Simply stated, the problem is that of determining what
room of a house or office a mobile robot is in, based upon what the
robot sees through one or more cameras. The problem is made very
difficult by at least three factors:

\begin{enumerate}

  \item the input space is huge, since we deal with images, usually at
    a reasonable resolution and in colour;

    \item images of the same place can be quite different as $(i)$
    illumination conditions change according to the time of day and
    the season, $(ii)$ places can be cluttered with objects shifted
    around, and $(iii)$ people of other moving obstacles can get in
    the way in real time.


  \item recognition must be done on-line and possibly in real time,
    since the robot is moving around, and knowing what place the robot
    is in would improve its stability against motion errors and
    obstacle avoidance. The amount of data to take into account is
    really massive, potentially endless;

\end{enumerate}

Recent work \cite{pronobis:iros06} showed that a pure learning approach can be very effective
for tackling the first two issues: indeed it was demonstrated that an SVM-based approach could achieve a remarkable robustness
to illumination changes and variability due to the normal use of the environments, while running 
in real time.

Still, it is well known that 
%Quite clearly, only a machine learning approach can potentially solve
%the problem, given the amazing variability of the input data. A
%learning machine is adaptive in nature and, if properly trained, can
%tackle the variability of the input. Nevertheless, machine learning
%usually suffers when confronted with massive amounts of data and when
%asked to work on-line.
%A striking example of this is represented by Support Vector Machines
%(SVM), one of the most promising machine learning approaches
%nowadays. Introduced in the early 90s by Boser, Guyon and Vapnik
%\cite{BGV92}, SVMs are able to classify data drawn from an unknown
%probability distribution, given a set of training examples. As opposed
%to analogous methods such as, e.g., artificial neural networks, they
%have the main advantages that $(a)$ training is guaranteed to end up
%in a global minimum, $(b)$ their generalization power is theoretically
%well founded, $(c)$ they can easily work with highly dimensional,
%non-linear data, and $(d)$ the solution achieved is sparse. Due to
%these good properties, they have been now extensively used in, e.g.,
%speech recognition, object classification and function approximation
%\cite{Cristianini00}. On the other hand they have the disadvantage to
%``grow'' for ever \cite{Steinwart03}, that is 
both
the complexity of the
support vector solution and the testing time grow proportionally with the number of
training samples. SVMs can be up to $50$ times slower of other
specialized approaches with similar performances
\cite{BurgesS96}. This represents a serious drawback for SVMs in the robotic vision 
domain: assuming to adopt an incremental learning approach, the continuous growth of the training data would
soon make the resulting algorithm too slow for a robotic platform.
Several authors have proposed various incremental extensions of the classic SVM algorithm (we refer the reader to section
\ref{prev-work} for a review on the subject), where typically one approximates the exact solution in order to
slow down the support vector's growth.

In this paper we propose an improvement to SVMs that we call Online
Independent Support Vector Machines (OISVMs). OISVMs incrementally
select ``basis vectors'' that are used to build the solution of the
SVM training problem, based upon \emph{linear independence in the
feature space}: vectors which are linearly dependent on already stored
ones are rejected, and a smart, incremental minimization algorithm is
employed to find the new minimum of the cost function. As a matter of
fact, our experiments indicate that the number of basis vectors of
OISVMs (the ``size'' of a machine) does not grow linearly with the
training set, as it was the case in \cite{Steinwart03}, but reaches a
limit and then stops growing. This result is theoretically confirmed,
e.g., in \cite{engel2004}, even in the case the feature space is
infinite-dimensional.

Since both the training and testing time of a SVM depend on the size
of the machine, \textbf{FRA: be more precise here: how exactly does it
depend?}, such an approach is actually what is needed to tackle the
problem of place recognition in mobile robotics. To support this
claim, we show an extensive set of experimental results obtained by
comparing SVMs and OISVMs 
on a real-world place recognition problem in an indoor
environment. Data images are acquired continuously by two robot
platforms under different weather conditions and across a time span of
several months. Our results show that our method achieves a speed-up of
$3$ times with respect to the time required by an approximated
incremental method, while retaining essentially the same accuracy.

The paper is structured as follows: after a review of the relevant
literature, Section \ref{sec:bg} gives an overview of background
mathematics proper to SVMs; in Section \ref{sec:opt} OISVMs are
described. Section \ref{sec:exp} shows the experimental results and
lastly, in Section \ref{sec:concl}, conclusions are drawn and future
work is outlined.
