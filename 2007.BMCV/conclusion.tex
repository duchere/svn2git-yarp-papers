In this paper we have shown a promising improvement to Support Vector
Machines, that we call Online Independent Support Vector Machines
(OISVM). OISVMs can effectively solve the problem of place recognition
by a mobile robot, at least in the experiment we have shown. OISVMs
were tested on the IDOL2 image database, which consists of image
sequences acquired by robot platforms under different weather
conditions and across a time span of several months; that is, in
various lighting conditions and object placements, and acquired by two
different robot platforms. OISVMs avoid using in the solution those
support vectors which are linearly dependent of previous ones in the
feature space. The optimization problem is solved via an incremental
algorithm which benefits of the small number of the basis vectors.

As far as we know, this method is radically different from all
analogous procedures presented so far in literature (e.g.,
\cite{DownsGM01,nguyen2005,LeeM01,schoel06,KeerthiCDC06}) since it
is \emph{not} an after-training simplification and it assumes
\emph{no knowledge whatsoever} of the full training set beforehand.

Our experimental results show that $(i)$ in the case of
finite-dimensional kernels, OISVMs attain the theoretical limit of
linearly independent support vectors allowed by the feature space,
without losing any accuracy with respect to ordinary SVMs; $(ii)$ in
the case of infinite-dimensional kernels, they dramatically reduce the
number of support vectors at the price of a negligible degradation in
the accuracy. In fact, we get as few as one third of the SVs required
by the fixed-partition method, while retaining essentially the same
accuracy.

Since the training and testing time... \textbf{FRA: nuovamente....}. A
careful study of the relationship between $\eta$ and the degradation
in performance is being carried on; in fact, according to
\cite{engel2004}, imposing a value of $\eta$ strictly larger than zero
will eventually result in a \emph{finite} number of basis vectors,
\emph{even in the case the feature space is
infinite-dimensional}. Finding a precise relationship between $\eta$
and this number would allow us to precisely dimension the machine
depending on the required precision.
