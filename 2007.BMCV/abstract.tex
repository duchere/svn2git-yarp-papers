In the framework of indoor mobile robotics, \emph{place recognition}
is a challenging task, where it is crucial that self-localization be
enforced precisely, notwithstanding the changing conditions of
illumination, objects being shifted around and/or people affecting the
appearance of the scene. In this scenario online learning seems the
main way out, thanks to the possibility of adapting to changes in a
smart and flexible way. Nevertheless, standard machine learning
approaches usually suffer when confronted with massive amounts of data
and when asked to work online. Online learning requires a high
training and testing speed, all the more in place recognition, where a
continuous flow of data comes from one or more cameras. In this paper
we follow the Support Vector Machines-based approach of
Pronobis et al. \cite{pronobis:iros06}, proposing an improvement that
we call Online Independent Support Vector Machines. This technique
exploits linear independence in the image feature space to
incrementally keep the size of the learning machine remarkably small
while retaining the accuracy of a standard machine. Since the training
and testing time crucially depend on the size of the machine, this
solves the above stated problems. Extensive experiments prove the
effectiveness of our approach.
