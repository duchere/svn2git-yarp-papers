In the framework of indoor mobile robotics, \emph{place recognition}
is a  challenging task,
where it is crucial that self-localisation is enforced precisely,
notwithstanding the changing conditions of illumination, objects being
shifted around and/or people affecting
the appearance of the scene.
In this scenario
continuous learning seems the main way out, thanks to the possibility of
adapting to changes in a smart and flexible way. Nevertheless,
standard machine learning approaches usually suffer when confronted
with massive amounts of data and when asked to work on-line.
%, that is,
%when the flow of data is continuous and not guaranteed to ever
%cease. 
Continuous learning requires a high training speed, all the more
in place recognition, where a continuous flow of data comes from one
or more cameras.
In this paper we follow the SVM-based approach proposed by Pronobis et al \cite{pronobis:iros06},
and we propose a new approach  that we call
Online Independent Support Vector Machines. This technique exploits
linear independence in the image feature space to incrementally keep
the size of the learning machine remarkably small while retaining the
accuracy of a standard machine. Since the training and testing time
crucially depend on the size of the machine, this solves the above
stated problem. 
Extensive experiments prove the effectivenes of our approach.

%This claim is supported by  experimental results
%achieved on the IDOL2 database, 
%a recently introduced collection that
%contains $24$ image sequences 
%of a five rooms environment, acquired using a perspective camera across a time span of several months.
