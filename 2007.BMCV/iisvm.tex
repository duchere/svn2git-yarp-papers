Let the \emph{kernel matrix} $K$ be defined such that $K_{ij} =
K(\xx_i,\xx_j)$, with $i,j=1,\ldots,l$. The possibility to obtain a more
compact representation of $f(\xx)$ follows from the fact that the
solution to a SVM problem (that is, the $\alpha_i$s) is not unique if
$K$ does not have full rank \cite{Burges98}, which is equivalent to
some of the SVs being linearly dependent on some others \emph{in the
feature space} (this is the core of Downs et al.'s \cite{DownsGM01}
original idea).
Applying the idea of Downs et al., or any other post-training
method to reduce the number of SVs, in an online setting means simplifying the
solution each time a new sample is acquired, that is obviously infeasible.
We need a way to use independent SVs only, that is to
decouple the concept of ``basis'' vectors, used to build
the classification function (\ref{eqn:sol}), from the samples
used to evaluate the errors $\xi_i$ in (\ref{eqn:svm_primal}).
If the selected basis vectors span the same subspace as
the whole sample set, the solution found will be equivalent
--- that is, we will not lose any precision.

We hereby propose, after having received a new training sample, to
incrementally add it to the basis if it is linearly independent in the
feature space from those already present in the basis itself. The
solution found is \emph{the same} as in the classical SVM formulation;
therefore, no approximation whatsoever is involved, unless one gives
it up in order to obtain even fewer support vectors (see Section
\ref{sec:exp} for a deeper discussion on this point).

Denoting the indexes of the vectors in the current basis, after $l$
training samples, by $\b$, and the new sample under judgment by
$\xx_{l+1}$, the algorithm can then be summed up as follows:

\begin{enumerate}

  \item check whether $\xx_{l+1}$ is linearly independent from the
        basis in the feature space; if it is, add it to $\b$;
        otherwise, leave $\b$ unchanged.

  \item incrementally re-train the machine.

\end{enumerate}

In the following, the notations $A_{IJ}$ and $\mathbf{v}_I$, where $A$
is a matrix, $\mathbf{v}$ is a vector and $I,J \subset \NN$ denote in
turn the sub-matrix and the sub-vector obtained from $A$ and
$\mathbf{v}$ by taking the indexes in $I$ and $J$.

\subsection{Linear independence}

In general, checking whether a matrix has full rank is done via some
decomposition, or by looking at the eigenvalues of the matrix; but
here we want to check whether a \emph{single} vector is linearly
independent from a matrix which is already known to be
full-rank. Inspired by the definition of linear independence, we check
how well the vector can be approximated by a linear combination of the
vectors in the set \cite{EngelMM02sparse}. Let $d_j \in \RR$; then let

\begin{equation} \label{eqn:ald1}
  \Delta = \min_\dd \left|\left|\sum_{j \in \b} d_j \phi(\xx_j) - \phi(\xx_{l+1}) \right|\right|^2
\end{equation}

If $\Delta > 0$ then $\xx_{l+1}$ is linearly independent with respect
to the basis, and $\left\{l+1\right\}$ is added to $\b$. In practice, we check
whether $\Delta \leq \eta$ where $\eta > 0$ is a tolerance factor, and
expect that larger values of $\eta$ lead to worse accuracy, but also
to smaller bases. As a matter of fact, if $\eta$ is set at machine
precision, OISVMs retain the exact accuracy of SVMs. Notice also that
if the feature space has finite dimension $n$, then no more than $n$
linearly independent vectors can be found, and $\b$ will never contain
more than $n$ vectors.

Expanding equation (\ref{eqn:ald1}) we get

\begin{equation} \label{eqn:ald2}
  \Delta = \min_{\dd} \left( \sum_{i,j \in \b} d_j d_i \phi(\xx_j) \cdot \phi(\xx_i) 
    - 2\sum_{j \in \b} d_j \phi(\xx_j) \cdot \phi(\xx_{l+1})
    + \phi(\xx_{l+1}) \cdot \phi(\xx_{l+1}) \right)
\end{equation}
\noindent that is, applying the kernel trick,

\begin{equation} \label{eqn:ald3}
  \Delta = \min_{\dd} \left(
      \dd^T K_{\b\b}\dd
    - 2 \dd^T \mathbf{k}
    + K(\xx_{l+1},\xx_{l+1})
  \right)
\end{equation}

\noindent where $k_i = K(\xx_i,\xx_{l+1})$ with $i \in \b$. Solving
(\ref{eqn:ald3}), that is, applying the extremum conditions with
respect to $\dd$, we obtain

\begin{equation} \label{eqn:ald4}
  \mdd = K_{\b\b}^{-1} \mathbf{k} \\
\end{equation}

\noindent and, by replacing (\ref{eqn:ald4}) in (\ref{eqn:ald3}) once,

\begin{equation} \label{eqn:ald5}
  \Delta = K(\xx_{l+1},\xx_{l+1}) - \mathbf{k}^T \mdd
\end{equation}

Note that $K_{\b\b}$ can be safely inverted since, by incremental
construction, it is full-rank. An efficient way to do it, exploiting
the incremental nature of the approach, is that of updating it
recursively: after the addition of a new sample, the new
$K_{\b\b}^{-1}$ then becomes

\begin{equation} \label{eqn:inv_upd}
  \left[\begin{array}{cccc}
       &               &   & 0 \\
       & K_{\b\b}^{-1} &   & \vdots \\
       &               &   & 0 \\
     0 &       \cdots  & 0 & 0
  \end{array}\right]
  +
  \frac{1}{\Delta}
  \left[\begin{array}{c}
    \mdd \\
    -1
  \end{array}\right]
  \left[\begin{array}{cc}
    \mdd^T & -1
  \end{array}\right]
\end{equation}

\noindent where $\mdd$ and $\Delta$ are already evaluated during the
test (this method matches the one used in Cauwenberghs and Poggio's
incremental algorithm \cite{CauwenberghsP00}). Thanks to this
incremental evaluation, the time complexity of the linear independence
check is $O(|\b|^2)$, as one can easily see from Equation
(\ref{eqn:ald4}).

With this method we are approximating the original kernel matrix $K$
with another matrix $\widehat{K}$ \cite{BachJordan2005};
the quality of the approximation
depends on $\eta$. In fact it is possible to show that
$trace(K-\widehat{K}) \leq \eta |\b| \leq \eta l$, where $l$ is the number
of samples acquired \cite{engel2004}. If we consider a normalized kernel,
that is a kernel for which $K(x,x)$ is always equal to $1$, we can write
$trace(K-\widehat{K})/trace(K) \leq \eta$.
On the other hand a bigger $\eta$ means of course a smaller number
of SVs, hence it controls the trade-off between accuracy and
speed of the OISVM.


\subsection{Training the machine}

The training method largely follows Keerthi et
al. \cite{KeerthiDC05,KeerthiCDC06}, that we have adapted for online
training. The algorithm directly minimizes problem
(\ref{eqn:svm_primal}) as opposed to the standard way of minimizing
its dual Lagrangian form, allowing to select explicitly the basis
vectors to use. We set $p=2$ in (\ref{eqn:svm_primal}) and transform
it to an unconstrained problem.  Let $\d \subset \{1,\ldots,l\}$; then
the unconstrained problem is

\begin{equation} \label{eqn:primal}
  \min_{\bb} \left( 
      \frac{1}{2} \bb^T K_{\d\d} \bb
    + \frac{1}{2} C \sum_{i=1}^l max \left(0,1-y_i K_{i\d} \bb \right)^2
  \right)
\end{equation}

\noindent where $\bb$ is the vector of the Lagrangian coefficients involved
in $f(\xx)$, analogously to the $\alpha_i$s in the original
formulation. If we set $\d = \b$, then the solution to the problem is
unique since $K_{\b\b}$ is full rank by construction. Newton's method
as modified by Keerthi et al. \cite{KeerthiDC05,KeerthiCDC06} can then
be used to solve (\ref{eqn:primal}) after each new sample. When the
new sample $\xx_{l+1}$ is received the method goes as follows:

\begin{enumerate}

   \item let $\mathcal{I} = \{ i: 1-y_i o_i<0 \}$ where $o_i =
     K_{i\b} \bb$ and $\bb$ is the vector of optimal coefficients
     with $l$ training samples; if $\mathcal{I}$ has not changed, stop.

   \item otherwise, let the new $\bb$ be $\bb - \gamma
     \mathbf{P}^{-1}\mathbf{g}$, where $\mathbf{P} = K_{\b\b} + C
     K_{\b\mathcal{I}} K_{\b\mathcal{I}}^T$ and $\mathbf{g} = K_{\b\b}
     \bb - C K_{\b\mathcal{I}} \left(
     \mathbf{y}_{\mathcal{I}}-\mathbf{o}_{\mathcal{I}}\right)$.

   \item go back to Step 1.

\end{enumerate}

In Step $2$ above, $\gamma$ is set to one. In order to speed up the
algorithm, we maintain an updated Cholesky decomposition of
$\mathbf{P}$. It turns out that the algorithm converges in very few
iterations, usually $0$ to $2$; the time complexity of the re-training
step is $O(|\b|l)$, as well as its space complexity; hence, keeping
$\b$ small will speed up the training time as well as the testing
time.
