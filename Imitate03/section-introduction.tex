
%%\citep{AB86} \citet{AD94}
%%\begin{figure}[!ht] \begin{center}

\section{Introduction}

To imitate an action, it must first be perceived correctly.  
%
The actor must be located and identified, and then tracked throughout
the action, along with any objects acted upon.
%
Perception ideally captures the essentials of an
action and discards incidental, irrelevant details.  
This context-dependent process is key to imitation, and
we believe that any theory of imitation must be first and foremost
a theory of perception.
%
[Okay, I'm just spouting nonsense here so far].

We show how a robot can acquire the appropriate percepts for imitation
through a shared activity, one that can be done either by the robot or
a human.  This activity is simply striking objects and watching how
they move.
