\subsection{View based vs. feature based}

There are two opposite views of object recognition in the brain namely: view-based and the so-called structural description. The first posits that object recognition is carried out by the brain by analyzing and blending together multiple views of the same object. The latter, instead, presuppose an internal representation by which the visual data is remapped into a set of geometric primitives (e.g. geons and the like).

To exemplify the latter, recognition by components (RBC) is briefly considered, where object recognition is built through the mapping of certain visual features (e.g. an overcomplete set) into the identification of certain volumetric visuo-geometric primitives (the above mentioned geons) and eventually recognition is carried out by an invariant process of identification of these primitives.

In the view based approach instead various views are thought to be typically acquired by the brain and used ``almost raw'' to build the basis (also overcomplete) of a space. Recognition in this case is thought to be performed by mapping the actual visual input into this space. Each point in this vector space thus represents the object as seen from a particular view point (sort of).

The question of whether any of the two views is correct is still under debate. There are results supporting both models both in human and monkey.

Another part of our understanding of object recognition in the brain comes from considering feedforward and/or feedback circuits: there are, in fact, two possibilities here, the first considers only a fast feedforward loop which is believed to end up in activating high-level neurons that somewhat code for the recognition of a specific object. Certain results on the timing of visual recognition favour a similar hypothesis. There is an alternative based on the inclusion of feedback signals (which are known to exist in the brain). This theory is known as ``analysis by synthesis'' and basically says that several options are internally enacted by the brain and the one whose predictions are is in agreement most with the perceived input is selected as the most likely object recognition. Prediction error is used both to correct the current recognition and to tune the internal models by learning. New results support also this possibility.

This is for what traditional object recognition is concerned. More recently neurophysiology has found a plethora of new and puzzling results (not so puzzling a posteriori). The pre-motor cortex (in the frontal lobe) seems to respond to the sight of objects. The more traditional view of the pre-motor cortex does not leave room for a visual (or sensorial in general) response: motor was considered as purely motor. The reality is somewhat different. Neurons in the pre-motor cortex respond to the fixation of objects and simultaneously they are truly motoric since they also respond to a grasping action directed at the same object. The two representations - motoric and visual - coexist in the same brain areas, even more so, in the same population of neurons.

Finally, similar responses have been found in the parietal cortex. This forms a conspicuous bi-directional connection with the pre-motor cortex so that it is possible to speak of the fronto-parietal system/circuit. Parietal neurons have been found that respond to geometric global object features (e.g. their orientation in 3D) which seem in fact well tuned for the control of action. But clearly the fronto-parietal circuitry is active also when an extant movement does not become an actual one. The natural question to be posed is then what is the purpose of this activation: potential motor action or true object recognition?

Links to imitation, learning by imitation, and speech make the study of the fronto-parietal system extremely interesting both for neuroscience and robotics.

\subsection{Temporal lobe contribution}

The traditional circuitry believed to activate for object recognition is linked to the visual cortex (primary V1, extrastriate areas V2, V4) and then to the inferotemporal cortex (IT). The general properties of this pathway see an ever increasing complexity of neural responses with an ever increasing size of the receptive fields: that is, while the specialization of neurons increase as considering the connections from V1, V2, V4, and IT, their spatial resolution decreases. It is like spatial invariance increases as the neural response specialize to recognition of certain objects.

Eventually, there are results showing the extreme specialization to (important) objects like faces (cite) and hands (Perrett et al.).

The study of this pathway leading from the visual cortex to the IT lobule can be traced back to the seminal work of Hubel and Wiesel (cite), and subsequently to others (e.g. Perrett). They show that the hierarchy of the so-called ventral pathway for object recognition build receptive fields (RF) of increasing robustness to various variations in the visual stimulus: e.g. scale, orientation. From the RF of V1 cells responding to oriented bars to the complexity of face and/or hand cells in IT (cite Perrett again). Other studies have shown (Perrett 93, Booth 98, Logothetis 95, Hietanen 92) that neurons have a preferred object view ``direction'' and show sensitivity to illumination.

[This section will include a more detailed description of certain specific results]

\subsection{Parietal object recognition}

Sakata et al. 3D object description.

\subsection{Objects and action}

Ungerleider and Mishkin first proposed the theory that visual processing in the brain splits in two specializing for localizing (where, dorsal) or recognizing (what, ventral) the object. Milner and Goodale subsequently linked this theory to vision for action (the dorsal pathway) and vision for perception - only - (the ventral pathway). This is a useful distinction which is dramatically shown in the case of stroke patients. For example, patient XY suffered a parietal lesion and was acting poorly when asked to grasp generic objects (e.g. cylinders), meaning that the "action pathway" was interrupted and proper judgement of the size/orientation of the object (for preshaping the hand) was impaired. Grasping of familiar objects was not affected since the object perception subsumes the experience of size and shape (but see the case of illusions). Agnosia was manifest in another patient YX with a lesioned temporal lobe who could perfectly grasp any object (as measured through preshaping) although she cannot recognize the identity of the grasped object nor judge certain aspects of shape.

In addition, the dorsal stream seems to be working almost oblivious of consciousness, while the ventral one is pretty much conscious. That is, the action of grasping an object (and describing it in these terms) does not require conscious control (once initiated perhaps) while recognition with a semantic value (e.g. judging the identity of the objects) requires consciousness. Other differences include the purported shape of the circuit of object recognition. It is well know that feedback (bi-directional) connections are present in the fronto-parietal circuit, while it seems that the reaction times (150ms) of object recognition by IT do not leave much space for a role of feedback.

Armed with this theory we can now interpret object recognition in the brain as a multi-system recognizer. Objects seem to be represented in the brain by multiple processing procedures depending on the specific use of the information being processed (e.g. action vs. identification).

One particular and interesting class of neurons seem to code 

There is a link between attention and feedback in the fronto-parietal circuit.

Also mention the attention view (Craighero et al.).


