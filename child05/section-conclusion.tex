
%% Humanoid robots are improving mechanically by leaps and bounds.
%% Stable bipedal walking (at least on flat surfaces) is now a
%% commonplace, with running and jumping on the way.  There are
%% many, many problems remaining, but it is accurate to say that
%% there has been real progress mechanically.

%% But there are some danger signs.  In robot demonstrations today,
%% you're much more likely to see synchronized dancing, gesturing, and
%% talking than any physical interaction with objects.  Robot hands and
%% arms are currently rather limited mechanically, but more importantly
%% the {\em perceptual} component of manipulation is very hard, unless we
%% constrain the environment greatly.  And while in locomotion
%% constraining the floor to be flat is not so limiting, in manipulation
%% the corresponding constraint is for the object to be of a particular
%% shape, orientation, location, etc., which is much more limiting.

In the field of humanoid robotics, researchers have a special respect
and admiration for the abilities of infants.  They watch their newborn
children with particular interest, and their spouses have to
constantly be alert for the tell-tale signs of them running an ad-hoc
experiment.  It can be depressing to compare the outcome of a 
five-year, multi-million-euro/dollar/yen project with what an
infant can do after four months.  Infants are so clearly doing what
we want to robots to do; is there any way to learn from research
on infant development?
Conversely, can infant development research be illuminated by
the struggles faced by robotics, perhaps pointing out fundamental
perceptual difficulties taken for granted?
%
%
Is there a way to create a model of development which applies both to infants
and robots?
%
It seems possible that similar sensorial constraints and opportunities
will mold both the unfolding of an infant's sensitivities to
different cues, and the organization of the set of algorithms used by
robots to achieve sophisticated perception.
%
So, at least at the level of identifying ``key events'' and mutually
reinforcing cues, a shared model is possible.
%
%For infants, this would be a theory of development, another way to
%express current theories.
%
%For robots, this would be an engineering tool, a guide to
%construction.
%
%On the infant side, empirical results would change, correct, and expand
%the model.  On the robot side, construction effort would do the same.
%
%Of course, the results could diverge, but at least they are 
%comparable, and those interested in truly humanoid robots would
%work to stay synchronized.
%
Of course, there is a lot that would not fit in the model, and this is
as it should be.  It would be solely concerned with the class of 
functional, information-driven constraints.
%
We have not in this paper developed such a model; that would be
premature.  We have at identified some points of connection that could
grow into much more.  We hope the paper will serve as one more link in
the growing contact between the fields.

%% loose threads
%\nocite{kirkham02visual}
%\nocite{johnson05building}
%\nocite{diamond05not}



%% Pending: Something about active vision?

%\nocite{bajcsy88active,aloimonos87active,ballard91animate}
%\nocite{whaite97autonomous} 
%\nocite{barnard02comparison}

