
A central question:

\begin{quote}

How could perception change with experience?

\end{quote}


\noindent General interests:

\begin{itemize}

\item Experiments that demonstrate ``development'' based on experience,
   particularly single or small numbers of episodes

\item Behaviors of the child/robot that have been demonstrated/argued
   to help generate good experience or otherwise help with development.

\end{itemize}

The world around us is made of objects, some of which can more or less
move independently.  As adults, we can judge which parts of the world
are likely to move as a group.  This is computationally a difficult
judgement to make, since regions grouped by easily-defined visual
features do not reliably correspond to physical groups.  Robots and
infants could attempt to learn to make such judgements based on
experience.  There is evidence of this in infants, and initial
attempts with robots.


Needham's experiments. 
\cite{needham01object,needham97object}
Looking time suggests that experience with objects moving separately
or together can affect future perception in the right way.
(also, the ``sticky hands'' work).


Johnson et al re segregation in the presence of partial occlusion.

For young infants, common motion is an important cue for 
perception of object unity.  Others: alignment, good form,
depth cues.  Spatial separation.

Infants also use information about specific objects or
classes of objects to guide their judgement.  

Can use information from just one experience.

Reorganize based on: judgements based on fairly general
principles, versus judgements based on specific object
knowledge.

In robotics, work related to object segregation is quite
primitive.  It is strongly influenced by the field
of computer vision, where ``object segmentation'' is a classic,
much studied problem.  See FOO for a fuller review than we give
here.  Representative work in computer vision,
see eg Berkeley (Malik and co).

How is the problem formalized in vision?  We want a function
which maps a matrix of pixels to a matrix of labels, where
any pair of locations should have the same label if and
only if their pixels are from the same object.
This is not well-defined in general; for example, in some
circumstances a person and everything they wear should be
considered a single object, in other cases they to clothes
should be separated; every object is composed of smaller
objects, etc.

What is omitted from this formalization?

What other problems are addressed in computer vision that
are relevant?



