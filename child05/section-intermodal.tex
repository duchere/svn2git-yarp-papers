
Events in the world have complicated effects, and can often have
a detectable impact on many of our senses.  Determining which
components of what we sense are due to the same event can be a 
difficult judgement to make.  It is a useful exercise, though.
(List reasons why)


\cite{lewkowicz00development}
\cite{lewkowicz80crossmodal}
\cite{lewkowicz04learning}
\cite{bahrick04development}
\cite{hernandez01development}
\cite{bahrick03development}
\cite{bahrick00intersensory}
\cite{gibson86ecological}
\cite{prince05synching}

In robotics: and in speech recognition / computer vision, there's
a lot of interest in matching lip movement with speech sounds - 
both to identify which of a set of people is speaking, and to
improve recognition performance by adding extra features.

Lorenzo suggests: tactile discrimination of objects, babybot work,
although not cross-modal.

