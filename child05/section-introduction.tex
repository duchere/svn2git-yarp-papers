
Infants and robots both face the challenge of perceiving and acting on
the world around them.  
%
%For infants, we evaluate how that challenge is
%met through analysis; for robots, we engage in synthesis.  
%
%
We expect
that there will be commonality between how infants and successful
robots operate at the information-processing level, because natural
environments are of mixed, inconstant observability -- there are
properties of the environment that can be perceived under some
circumstances and not under others.  This network of opportunities and
frustrations should place limits on information processing that apply
both to humans and robots with human-like sensors.

We focus on early perceptual development in infants.  The
perceptual judgements infants make change over time, showing
a changing sensitivity to various cues. This progression
is presumably at least partially due to knowledge gained
from experience.  In epigenetic robotics, we hope to 
create robots that undergo perceptual development.
%
We create a simple model of development which applies both to infants
and robots.  
%
For infants, it is a theory of development, another way to
express current theories.
%
For robots, it is an engineering tool, a guide to
construction.
%
On the infant side, empirical results will change, correct, and expand
the model.  On the robot side, construction effort will do the same.
%
%Of course, the results could diverge, but at least they are 
%comparable, and those interested in truly humanoid robots would
%work to stay synchronized.
%
Of course, there is a lot that is not in the model, and this is
as it should be.  It is solely concerned with the class of 
functional, information-driven constraints.

The topics we focus on are object segregation,
intermodal integration, and embodiment.


We identify opportunities that can be exploited by both robots and
infants to perceive properties of their environment that cannot be
directly perceived in other circumstances.  We review what is known of
how robots and infants can exploit such opportunities to learn to make
reasonable inferences 
%
%%of hidden object properties 
%
about object properties that are not directly given in the display
%
through correlations
with observable properties, grounding those inferences in prior
experience.


%% \begin{itemize}

%% \item Object segregation --
%%   Infant research: Amy Needham.
%%   Robotics work: Paul Fitzpatrick and Giorgio Metta. ....

%% \item Intermodal integration --
%%   Infant resesarch: Lorraine Bahrick and Robert Lickliter.
%%   Robotics work: Artur Arsenio and Paul Fitzpatrick. ...

%% \item Embodiment -- 
%%   Setting up the right behaviors, to create opportunites
%%   for rich experience, which in turn can support development.

%% %  Infant resesarch: Scott Johnson.
%% %  Robotics work: Yi Chen and Juyang Weng. ....

%% \end{itemize}

\subsection{The problem}

Humanoid robots are improving mechanically by leaps and bounds.
Stable bipedal walking (at least on flat surfaces) is now a
commonplace, with running and jumping on the way.  There are
many, many problems remaining, but it is accurate to say that
there has been real progress mechanically.

But there are some danger signs.  In robot demonstrations today,
you're much more likely to see synchronized dancing, gesturing, and
talking than any physical interaction with objects.  Robot hands and
arms are currently rather limited mechanically, but more importantly
the {\em perceptual} component of manipulation is very hard, unless we
constrain the environment greatly.  And while in locomotion
constraining the floor to be flat is not so limiting, in manipulation
the corresponding constraint is for the object to be of a particular
shape, orientation, location, etc., which is much more limiting.

In the field of humanoid robotics, researchers have a special respect
and admiration for the abilities of infants.  They watch their newborn
children with particular interest, and their spouses have to
constantly be alert for the tell-tale signs of them running an ad-hoc
experiment.  It can be depressing to compare the outcome of a 
five-year, multi-million-euro/dollar/yen project with what an
infant can do after four months.




