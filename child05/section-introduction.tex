
Infants and robots both face the challenge of perceiving and acting on
the world around them.  For infants, we evaluate how that challenge is
met through analysis; for robots, we engage in synthesis.  
%
{\bf Paul:  I'm not sure what you mean by analysis and synthesis}
%
We expect
that there will be commonality between how infants and successful
robots operate at the information-processing level, because natural
environments are of mixed, inconstant observability -- there are
properties of the environment that can be perceived under some
circumstances and not under others.  This network of opportunities and
frustrations should place limits on information processing that apply
both to humans and robots with human-like sensors.

We identify opportunities that can be exploited by both robots and
infants to perceive properties of their environment that cannot be
directly perceived in other circumstances.  We review what is known of
how robots and infants can exploit such opportunities to learn to make
reasonable inferences 
%
%%of hidden object properties 
%
about object properties that are not directly given in the display
%
through correlations
with observable properties, grounding those inferences in prior
experience.

Topics:

\begin{itemize}

\item Object segregation --
  Infant research: Amy Needham.
  Robotics work: Paul Fitzpatrick and Giorgio Metta. ....

\item Intermodal integration --
  Infant resesarch: Lorraine Bahrick and Robert Lickliter.
  Robotics work: Artur Arsenio and Paul Fitzpatrick. ...

\item Embodiment -- 
  Setting up the right behaviors, to create opportunites
  for rich experience, which in turn can support 

%  Infant resesarch: Scott Johnson.
%  Robotics work: Yi Chen and Juyang Weng. ....


\end{itemize}

