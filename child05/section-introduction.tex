
%Robots can build cars and drive on mars, but no robot can do what
%a human infant at age four months can do.

%If an adult human were presented with...

Imagine if your body's sensory experience were presented to you as
column after column of numbers.  One number might represent the amount
of light hitting a particular photoreceptor, another might be related
to the pressure on a tiny patch of skin.  Imagine further if you could
only control your body by putting numbers in a spreadsheet, with
different numbers controlling different muscles and organs in
different ways.

This is how a robot experiences the world.  It is also a (crude) {\em
model} of how humans experience the world.  Of course, our sensing and
actuation are not encoded as numbers in the same sense, but aspects of
the world and our bodies are transformed to and from internal signals that,
in themselves, bear no trace of the signals' origin.  For 
example, a neuron firing
selectively to a red stimulus is not itself necessarily red.  The
feasibility of telepresence and virtual reality shows that this kind of
model is not completely wrong-headed; if we place digital transducers
between a human and the world, they can still function well.

Understanding how to build robots requires understanding in a very
concrete and detailed way how it is possible to sense and respond to
the world.  Does this match the concerns of psychology?  For work
concerned with modeling phenomena that exist at a high level of
abstraction, or deeply rooted in culture, history, and biology, the
answer is perhaps ``no''.  But for immediate perception of the
environment, the answer must be at least partially ``yes'', because the
problems of perception are so great that intrinsic physical
%
%non-cultural, non-biological, non-historical 
%
constraints must play a
role.
%
%
%
%
%Infants and robots both face the challenge of perceiving and acting on
%the world around them.
%
%
%
%being subject to some fairly crippling
%constraints.
%
%For infants, we evaluate how that challenge is
%met through analysis; for robots, we engage in synthesis.  
%
%In engineering, that challenge can be met only in specific, narrow
%cases; in an open, unstructured environment, robots are helpless.
%
%
We expect
that there will be commonality between how infants and successful
robots operate at the information-processing level, because natural
environments are of mixed, inconstant observability -- there are
properties of the environment that can be perceived easily under some
circumstances and with great difficulty (or not at all) under others.  
This network of opportunities and
frustrations should place limits on information processing that apply
both to infants and robots with human-like sensors.

In this paper, we focus on early perceptual development in infants.  The
perceptual judgements infants make change over time, showing
an evolving sensitivity to various cues. This progression
may be at least partially due to knowledge gained
from experience.
%
%is presumably at least partially due to knowledge gained
%from experience.  In epigenetic robotics, we hope to 
%create robots that undergo perceptual development.
%
%
%
We identify opportunities that can be exploited by both robots and
infants to perceive properties of their environment that cannot be
directly perceived in other circumstances.  
%
%
%%%We review some of what is known of how robots and infants can exploit
%%%such opportunities to learn to make reasonable inferences about object
%%%properties that are not directly given in the display through
%%%correlations with observable properties, grounding those inferences in
%%%prior experience.
%
We review some of what is known of how robots and infants can exploit
such opportunities to learn how object properties not directly given
in the display correlate with observable properties.
%
%
%
The topics we focus on are object segregation,
intermodal integration, and embodiment.


%% \begin{itemize}

%% \item Object segregation --
%%   Infant research: Amy Needham.
%%   Robotics work: Paul Fitzpatrick and Giorgio Metta. ....

%% \item Intermodal integration --
%%   Infant resesarch: Lorraine Bahrick and Robert Lickliter.
%%   Robotics work: Artur Arsenio and Paul Fitzpatrick. ...

%% \item Embodiment -- 
%%   Setting up the right behaviors, to create opportunites
%%   for rich experience, which in turn can support development.

%% %  Infant resesarch: Scott Johnson.
%% %  Robotics work: Yi Chen and Juyang Weng. ....

%% \end{itemize}

%Be careful about development!  If avoidable, don't get into what parts are
%structural and what are induced~-- no room to fight those battles.

