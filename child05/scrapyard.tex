
%
Vision in robots has tended to be the poor relation, borrowing
second-hand algorithms from computer vision that don't really
fit.
%



First decision: what are the boundaries and objects with which
we are concerned?  For example, clothing may be an object in
one case and just part of a person's body in another context.
Often the default assumption is some kind of ``natural''
level of object representation (CITE Jendriks).

Second decision: what is the intended relationship between the groups
formed and the true groups formed by the object boundaries?  A common
goal is for each recovered group to be a subset (or equal) to
true groups.  The assumption is that finding true boundaries may
require knowledge, and to do the best that one can in a bottom-up
manner, and to postpone final decisions.  Could go the opposite way,
and aim for groups that are bigger than objects, but this is 
inconsistent (all groups can't be like that) and non-modular 
(requires thinking in units smaller than the groups when considering 
splitting them).

In this paper, we
stick with the term from psychology.  In computer vision, object
segregation turns out to be a key, but very difficult task, and most
general-purpose work instead focuses on the (still difficult) task of
``image segmentation'': grouping regions of similar appearance that
may correspond to either an object or part of an object.
%
More on this later.
%


We address the perception of objects indirectly, via
the perception of object boundaries.  We review experimental scenarios
where object boundaries are not evident, and the judgements that
infants make, which differ according to age and experience.  We review
related work in robotics.  



\begin{itemize}

\item Grouping elements of a single image according to some
criterion.  Each group may correspond to a single object, parts of
several objects, or part of a single object.

\item Grouping elements of a single image according to some
criterion.  Each group is intended to correspond to an object or
object part.  It is desired that groups not cross object boundaries.

\item Grouping elements of a single image according to some
criterion, where that criterion is intended to give a 
one-to-one correspondence between visible objects and
image groups.

\item Grouping may also be done with simultaneous images from
multiple cameras.

\item Grouping may also be done with video sequences from one or 
several cameras.

\end{itemize}







\section{more and more}

For stereo, the labels would correspond to disparity,
and $E_{data}$ can check that with the given disparity
the left and right image match well locally.


The problem of {\em object segmentation} is formalized as
a labelling problem.

decomposing an image

 into collections of pixels, where each collection
ideally corresponds to an object.


To make this decomposition, image
features such as color, brightness, motion, stereoscopic cues, and
texture can be used; the basic heuristic is that by any pair of
measurements made within an object should generally be more similar
numerically than a pair of measurements made on regions belonging to
different objects.  Another family of work concentrates on particular
objects rather than the general case: finding faces, or cars, for
example.

Give intro to energy formalization, form used
for min-cut approaches in stereo, color, etc?

Active segmentation.

Training for segmentation and recognition.


\subsection{Surplus - wednesday}


Adults, beware -- there are many judgements you can make without a
moment's thought that infants of a certain age cannot.

The kinds of experiences a na\"{i}ve observer might find
useful in this regard are many and have not been well characterized.



\subsection{Tuesday}

\begin{quote}

In order for infants' prior experiences with objects
to facilitate their parsing of a novel visual scene,
infants must be able to do several things.  First, they
must remember their prior experiences.  Second, they 
must realize that what they learned previously could
apply to this new situation.  Third, they must then
use the knowledge learned from previous encounters
to draw conclusions about the novel visual scene.
(Dueker, Needham...)

\end{quote}


\begin{quote}

The ultimate argument for perception of events as external distal
happenings in the world is appropriate, adaptive response to them
in the face of changing context. (E. Gibson)

\end{quote}


Object segregation is a step


The world around us is made of objects, some of which can more or less
move independently. For robots and humans, the ability to judge which
parts of the world are likely to move as a group is a key perceptual
ability.


As adults, we can judge which parts of the world
are likely to move as a group. This is computationally a difficult
judgement to make, since regions grouped by easily-defined visual
features do not reliably correspond to physical groups. Robots and
infants could attempt to learn to make such judgements based on
experience. There is evidence of this in infants, and initial attempts
with robots.



\begin{itemize}

\item what are some formalizations?

\item what are the issues in child development?

\item what is the role of motion in young infants?

\item how can we exploit motion in robots?

\end{itemize}



Issues:

\begin{itemize}

\item Measures segregation abilities at given ages.

\item Experiences that show separate movement of particular object.

\item Successful exploitation of such experiences.

\item Degree of generalization to other objects and situations.

\item Role of behavior.

\end{itemize}

Computational:

\begin{itemize}

\item object complexity

\item scene/presentation complexity

\end{itemize}




\subsection{Scrap}


How do infants, and
how should robots, perceive this world?  

In robotics, it is well understood that if we attempt to formalize a
model of the world as a complete, unique, disjoint set of units, we
run into trouble. 

 Progress has been made by focussing on 
partial



\subsection{Box and tube}



Gestalt principles.

Time course.

Role of experience.

Role of behavior.

Proximity, closure,similarity, good continuation, common fate.

proto-surfaces /superpixels

\subsection{Importance}

The importance of object segregation in robotics is 
that it is impossible for a robot to do anything 
useful without it.  

What is available when sitting back and watching; 
what is available when acting.  See Arsenio.


\subsection{Progression}

Early, infants may ignore object (surface?) features for grouping.
Later, infants may assign them high weight. (footnote in
Needham 2001).


\subsection{SCRAPYARD}

For young infants, common motion is an important cue for 
perception of object unity.  Others: alignment, good form,
depth cues.  Spatial separation.

Can use information from just one experience.





\subsection{What is the problem}

What is so hard about segregation in the first place?

A basic limitation of many segmentation algorithms is
that they are designed with a shrink-wrapped mentality.





Recognition without Correspondence using MultidimensionalReceptive Field Histograms
Full text 	Full text available on the Publisher sitePublisher Site
Source 	International Journal of Computer Vision archive
Volume 36 ,  Issue 1  (January 2000) table of contents
Pages: 31 - 50  
Year of Publication: 2000
ISSN:0920-5691 

 What goes up may come down: perceptual process and knowledge 
Cognitive Science 27 (2003) 923\u2013935

***

Object segregation means relating parts of a scene at a particular
instant in a particular sense.  Intermodal integration means relating
parts across the different senses.  Object permanence
involves extending these relations across time.

****

Suppose we lived in a very simple world, composed of discrete
objects that never break apart or merge or engage in any such 
complication.  Then, in theory, a unigue ID could be assigned
to every object in the world.

Suppose every object has a slightly different color, texture, or
sound, and this color/texture/sound can be perfectly sensed.

Then visual segregation is trivial.  Permanence is also trivial.
Integration is still non-trivial.

====================================================================
====================================================================
====================================================================
====================================================================
====================================================================
====================================================================
====================================================================
====================================================================

features used to make a judgement.

features used for generalizaton.

interesting experiments:
 - rod-behind-bar experiments
 - connected-region expectations
 - change of expectations, level of generalization


