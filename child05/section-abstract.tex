
Robots and humans receive partial, fragmentory hints about the world's
state through their respective sensors.  In this paper, we focus on
three fundamental problems in perception that have attracted the
attention of researchers in both robotics and infant development:
object segregation, intermodal integration, and object permanence.
For infants, we review what physical experiences are known to aid
infants in the development of their ability to address these problems.
And similarly, we review what physical experiences that have been
successfully exploited by robots for analogous perceptual development,
and how.


