
Robots and humans receive partial, fragmentory hints about the world's
state through their respective sensors.  In this paper, we focus on
some fundamental problems in perception that have attracted the
attention of researchers in both robotics and infant development:
object segregation, intermodal integration, and the role of
embodiment.  For infants, we review what physical experiences are
known to aid infants in the development of their ability to address
these problems.  And similarly, we review what physical experiences
have been successfully exploited by robots for analogous perceptual
development, and how.  We concentrate on identifying points of contact
between the two fields, and also important questions identified in one
field and not yet addressed in the other.



%We have tried to give a self-contained description 
%for both psychologists and roboticists.  We ask patience
%from both groups for when we've spent time on the obvious (to each
%group individually).

%{\em Warning: this paper is a draft, and is very rough around the edges.
%It includes occasional notes by the authors to themselves, discussing
%missing material.
%}
