
Robots and humans receive partial, fragmentary hints about the world's
state through their respective sensors.  In this paper, we focus on
some fundamental problems in perception that have attracted the
attention of researchers in both robotics and infant development:
object segregation, intermodal integration, and the role of
embodiment.  
%
%For infants, we review what physical experiences are
%known to aid infants in the development of their ability to address
%these problems.  And similarly, we review what physical experiences
%have been successfully exploited by robots for analogous perceptual
%development, and how.  
%
We concentrate on identifying points of contact
between the two fields, and also important questions identified in one
field and not yet addressed in the other.
%
For object segregation, both fields have 
examined the idea of 
using ``key events'' where perception is in some way simplified
and the infant or robot  acquires
knowledge that can be exploited at other times.
%
We examine this parallel research in some detail. We propose that
the identification
of the key events themselves constitutes a point of contact between
the fields.  And although the specific algorithms used in robots are
not easy to relate to infant development, the overall ``algorithmic
skeleton'' formed by the set of algorithms needed to identify and
exploit key events may in fact form a basis for mutual dialogue.

%This requires a process of generalization, which has been examined
%more carefully at an empirical level for infant development than
%for robotics~-- yet, if a connection can be made, there are
%interesting theoretical ideas about the problem of generalization
%in machine learning.
%
%Intermodal integration offers a whole family of ``key events'', with
%robotics again far behind infant research but showing some promise to
%shed light on technical issues involved with exploiting amodal
%cues.
%



%We have tried to give a self-contained description 
%for both psychologists and roboticists.  We ask patience
%from both groups for when we've spent time on the obvious (to each
%group individually).

%{\em Warning: this paper is a draft, and is very rough around the edges.
%It includes occasional notes by the authors to themselves, discussing
%missing material.
%}
