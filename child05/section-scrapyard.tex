

\subsection{Importance of motion?}

We are talking about the {\em development} of perception from an
immature form.  We assume that this development relies on acquiring
some information from the outside world (this isn't necessarily so).
How is this information acquired?  Sensory input is not 
uniformly difficult to interpret -- there are situations in
which it becomes simpler.  One particularly striking case is
the presence of motion.  Coherent group motion of part of the 
scene can be relatively easy to detect, and is a good cue for
the existence and extent of a corresponding object.  
So motion is one good place to start.
%
Moving objects are salient to infants and seem to play an
important role in perceptual learning (BACK THIS UP).
%
In robotics motion has frequently been used in all sorts of
ways.
%
Mention the technical status of motion detection versus
detection of other features (e.g. ``material'').
%
Of course this is by no means the end of the story, there 
are many other features... depth cues etc.

Egomotion is not very useful (relatively speaking).  Movemement of
others or caused by others is useful.  Motion caused by the robot
itself is particularly useful; this could be true of infants but is
moot given the limited motor control available initially.

\subsection{Preamble}

What is perception for?  One classic answer from robotics is that the
goal of perception is to recover, as faithfully as possible, the state
of the world.  That is, some model is made of the world, with some
number of free parameters, and the goal of perception is to find
values for those parameters to bring the model into as close an
alignment as possible with the world.

This view is my no means unquestioned; to achieve a particular
task, the most useful model to estimate may vary.  It may be
trivial, or the robot's behavior may be easier to produce or
describe in alternate ways that don't use the language of
model estimation.

In this paper, we will assume that the robot is engaged
in manipulation tasks (as opposed to, for example, navigation).

{\bf Behavioral view}: strategies that make it likely for the robot
to be looking somewhere useful (hand/eye coordination).
{\bf Model view}: ability to demonstrate flexible knowledge of presence of 
objects and some of their properties.


\subsection{Formalities}

Should provide the clearest possible definition of terms,
as a reference.  Human and robot uses of terms could deviate
from this definition if the deviation is also clearly described.

Object segregation, intermodal integration, object permanence.

Define a cell as the smallest sensing unit, for whichever sensors are
under consideration (e.g. pixels for images).  Assume there are a
fixed number of cells $c_1...c_{N}$.

There is also an unknown set of causes $s_1...s_{M}$, which for a
bounded period of consideration we will consider fixed.  Each cell
$c_i$ provides varying sensor readings $c_i(t)$.  We would like to
generate assignments $a_j(t)$ which, for each cause, lists all the
cells that can reasonably be attributed to that cause.  Then we choose
our causes to be maximally useful in describing the environment.

It is a little difficult to say exactly what judgement should be
made in complicated cases.  Might be better to give examples,
and underspecify.

Specification for object segregation:

$I(t) = [c_1.val(t) ... c_N.val(t)]$

Goal of segregation:
  Find a set $S_i$ = e.g. ${ c_2,c_4,c_5,c_6,... }$ that lists cells that
  belong together at a particular time.
  Find a set of such sets - Z.

Goal of intermodal integration:
  Bring together such sets across modal boundaries.

Goal of object permanence:
  Bring together such sets across time and disappearances.


***********

stuff to be integrated [from Lorenzo]:


The paper focuses on objects. Can we say something about the role of the
body? Achieving eye-hand coordination proved to be extremely useful on the
robots we have worked on. Learning to act is necessary to perform active
exploration of objects (poking/pushing/tapping/grasping). During action the
ability to identify the body helps the robot to focus the attention on the
area of the visual space where "things are happening" (i.e. on the hand upon
contact with the object).


***********

Big potential difference between infants and humans: the role
of manipulation in shaping early perception.  Infants can't
act that much to begin with.



