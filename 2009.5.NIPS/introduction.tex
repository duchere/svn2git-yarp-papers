\begin{itemize}

\item learning by imitation great capability of cognitive systems. It permits to learn how to grasp a cup never seen before just by seeing someone else doing it.
Fundamental learning mechanism etc 

\item a widely accredited hip is that the underlying mechanism of learning by imitation is the existence of a sensor motor map that links
the visual perception of an object to the motir position of the hand when it manipulates it. Mirror neurons blabla 

\item another consequence of the existence of a sensor-motor map is that when we learn an object by seeing and manipulating it, we are then able
to recognize it with a higher degree of accuracy and robustness than if we would have learned it on visual data only. 

\item Enabling a robot to display similar abilities is one of the holy grails of research in artificial cognitive systems. 

\item In this paper we present a mirror neurons inspired algorithm for building perception action maps between visual, passive perception and motor, active
perception. During learning, the algorithm takes as input visual and sensomotor data and (a) it builds a mapping between the two modalities (b) it builds 
a classifier on both modalities. After training, wehn presented with a visual input, the system is able to perform grasp priming (= is able to predict
which are the possible way to grasp the seen object; this information could be used to pre-activate a robot hand) and enhanced visual recognition
(= is able to recognize objects with a higher degree of accuracy and robustness compared to a model learned only on visual features, even if
the sensor modality is not perceived by the agent). Experiments show that.....


\item in the rest of the paper...


\end{itemize}

%
%
%
