The model adaptation method presented in this paper stems from a
problem in adaptive hand prosthetics, namely: is it possible to help a
patient to learn to use a dexterous hand prosthesis 
%in a quicker and
%better way, 
by exploiting the common features found in models trained upon other
patients?
The answer, at least as far as healthy subjects are concerned, is yes:
we have hereby presented a novel method for model adaptation in
machine learning, using Least-Squares SVMs; the idea is to build a SVM
solution which is \emph{close} to one of a set of pre-stored
models. The choice of which model to use among the pre-trained ones,
as well as the parameter $\beta$, determining the degree of closeness
to start the training from, are completely automatic, as we use an
estimation of the generalization error.

We tested our method on a database built with EMG and force data from
$10$ healthy subjects, trying to improve the training times and
asymptotic performance of one subject by pre-training on other
subjects. The outcome of the experiment is positive: our method gains
consistently both in the classification and regression tasks in the
best and average cases, and it resorts to the non-adaptive performance
in the worst.

Therefore, it is apparent that a large amount of knowledge stored in
LS-SVM models is common to all subjects, which is obviously due to the
analogies among the tasks performed by the subjects, as well as to the
anatomical similarities among the arms and the careful positioning of
the electrodes on the subjects' forearms. A further interesting point
is that, almost uniformly, models obtained by adaptation from a
pre-trained model obtain a \emph{better} performance than those
trained from scratch. This result is somehow surprising, although very
encouraging, and subject of future research.

Notice that we present no results on a real prosthetic/robotic hand so far --- this is subject of immediate-future research. We successfully applied a similar system to the DLR-II mechanical hand (see \cite{2008.ICRA,2008.BioCyb}), and since the accuracy of the system presented here is analogous to that of the one therein, there is no reason why the results presented here should not apply as well in the practical case. One interesting possibility is that of using this system to speed up the adaptation of an already existing dexterous hand prostheses, such as, e.g., Touch Bionics's i-LIMB \cite{ilimb} prosthetic hand, as already mentioned in the introduction.

Lastly, let us consider the fact that, most likely, the overall
performance of the method will increase when more subjects are
available, since this would mean a larger probability of finding a
matching pre-trained model. In a clinical setting, this means that
after an experimental phase, adaptive prostheses employing this method
could actually be built. It remains, of course, to discover whether
this idea can be transferred to amputees: amputations are, obviously,
non-controlled, traumatic events (except in some cases), and therefore
stumps exhibit much more variability than healthy forearms. This is
the subject of ongoing as well as future research.
