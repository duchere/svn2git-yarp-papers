The current state-of-the-art in hand prosthetics, as far as control is
concerned, is far from optimal. The main control interface is
represented by surface electromyography (sEMG): the activation
potentials of the remnants of large muscles of the stump (usually, the
wrist flexor and extensor) are used in a quite non-natural way to
control one or, at best, two degrees-of-freedom. This has two
drawbacks: first, the dexterity of the prosthesis is limited, leading
to poor interaction with the environment; second, the patient
undergoes a long training time.

So, as more dexterous hand prostheses are put on the market, the need
for a finer and more natural control arises. Machine learning
techniques such as, e.g., Support Vector Machines (SVM) can be
employed to this end; and, in this framework, a desired requirement is
that of providing a pre-trained model to the patient, so that a
quicker and better interaction can be obtained.

To this end we propose \emph{model adaptation} for least-squares SVMs,
a technique in which \textbf{FRANCESCO...}. We test the effectiveness
of the approach on a large database of EMG signals gathered from human
subjects and show that training times are dramatically reduced when
pre-trained models are used, as opposed to starting from scratch.
