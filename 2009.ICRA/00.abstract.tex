The state-of-the-art of control in hand prosthetics is far from
optimal. The main control interface is represented by surface
electromyography (EMG): the activation potentials of the remnants of
large muscles of the stump are used in a non-natural way to control
one or, at best, two degrees-of-freedom. This has two drawbacks:
first, the dexterity of the prosthesis is limited, leading to poor
interaction with the environment; second, the patient undergoes a long
training time.

As more dexterous hand prostheses are put on the market, the need
for a finer and more natural control arises. Machine learning can be
employed to this end. A desired feature is that of providing a
pre-trained model to the patient, so that a quicker and better
interaction can be obtained.

To this end we propose \emph{model adaptation} for least-squares SVMs,
a technique by which \textbf{FRANCESCO...}. We test the effectiveness
of the approach on a database of EMG signals gathered from human
subjects.
We show that, when pre-trained models are used, the training time is reduced and the overall performance is increased,
compared to what would be achieved by starting from scratch.
% and show that not only the training time can be reduced, but
%the overall performance can be increased, when pre-trained models are
%used, as opposed to starting from scratch.
