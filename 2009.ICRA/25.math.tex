\subsection{Background}

Assume $\xx_i \in \RR^m$ is an input vector and $y_i \in \RR$ is its associated output.
Given a set $\{\xx_i,y_i\}_{i=1}^l$ of samples drawn from an unknown probability
distribution, we want to find a function $f(\xx)$ such it
best determines the corresponding associated $y$ of any future sample $\xx$
drawn from the same distribution.
This is a general framework that includes regression and classification problems.
This problem can be solved in various way, we will use kernel methods and in
particular Least-Square Support Vector Machines (LS-SVM) \cite{Cristianini00}.
In LS-SVM the function $f(\xx)$ is built as a linear model
$\ww \cdot \phi(\xx) + b$, where $\phi(\xx)$ is a non-linear function that maps
the data in a fixed high dimensional \emph{feature} space.
However, rather than specifying the feature space directly,
it can be implied by a kernel function $K(\xx,\xx')$, giving the
inner product between the images of vectors in the feature
space, i.e. $K(\xx,\xx')=\phi(\xx) \cdot \phi(\xx')$.
A common kernel function is the isotropic Gaussian kernel
\begin{equation}
	K(\xx,\xx')=\exp(-\gamma ||x-x'||^2)
	\label{eq:rbf}
\end{equation}
that will be used in all our experiments.

The parameters of the linear model, $\ww$ and $b$, are found minimizing a
regularized least-squares loss function \cite{Cristianini00}.
This approach is similar to the well-known formulation of Support Vector
Machines (SVMs). The difference is that the loss function is the square loss and it
does not induce a sparse solution. On the other hand it is possible to write
the leave-one-out error in closed form \cite{Rifkin07}. This is known to be
approximately an unbiased estimator of the classifier generalization error
\cite{LuntzB69}. This property is useful to find the best parameters for the
learning (e.g. $\gamma$ in (\ref{eq:rbf})) and it will be used in our
adaptation method. Note that the same formulation is used to solve both
regression and classification problems.

\subsection{Model Adaptation}

We assume that the system has a number of previous models trained off-line on a
number of different persons. The system then starts acquiring new data, coming
from the person wearing the prosthetic hand. Given the differences in the
electrodes placement and between the different subjects, the new data will
belong to a distribution that is different from the ones of the other stored
subject. Still we expect that the distributions will be \emph{close}. Thus
it could be possible to use one of the previous model as a
\emph{starting point} for the training on the new data.
Our desire is that using the previous model the learning should be faster
than using the new data alone. Moreover the adaptation should not use the
auxiliary data for training.

We generalize the framework for adaptation proposed in \cite{YangYH07} for
SVM. They propose to sligtly change the regularization term of the SVM, to
induce the solution to be close to the one of the previous model. Their 
optimization problem is the following
\begin{align}
\min_{\ww,b} \frac{1}{2} \|\ww- \ww'\|^2 + \frac{C}{2} \sum_{i=1}^N \xi^2 \\
\mbox{subject to} \;\; \xi_i \geq 0,\;\; y_i w \cdot \phi(\xx_i) + b \geq 1-\xi_i
\label{eq:opt_prob_orig}
\end{align}
\noindent where $\ww'$ is the previous model.
In our opinion the above formulation has the disavantage of giving a fixed
weight to the previous model. Hence we introduce a scaling factor for the previous
model. Moreover we change the loss into the standard square loss. This gives
us the possibility to calculate the leave-one-out error with a closed formula, and
in this way to automatically tune this additional parameter.
So we obtain the following optimization problem
\begin{align}
\min_{\ww,b} \frac{1}{2} \|\ww- \beta \ww'\|^2 + \frac{C}{2} \sum_{i=1}^N \xi^2 \\
\mbox{subject to} \;\; y_i = w \cdot \phi(\xx_i) + b + \xi_i
\label{eq:opt_prob}
\end{align}

It is easy to show that the optimal solution is the form
\begin{equation}
\ww = \beta \ww' + \sum_{i=1}^N \alpha_i \phi(\xx_i), \; \alpha_i \in \RR
\end{equation}
\noindent hence the final solution is composed by the sum of the previous model
scaled by the parameter $\beta$, and a new model given by the new data points.
Note that when $\beta$ is $0$ we recover the original LS-SVM formulation, that is without any
adaptation to the previous data.
As said before, with LS-SVM it is possible to write the leave-one-out error
in a closed form. It turns out that it is possible to do the same if with the
modified formulation (\ref{eq:opt_prob}). Hence it is possible to found the
parameter $\beta$ in order to minimize the leave-one-out error. In particular
for regression there is a closed formula for the optimal $\beta$. We leave the
mathematical details for a longer version of this paper.
Hence with a negligible additional computational cost we can choose the optimal
value for the parameter $\beta$ and at the same time we can choose the best
previous model for the adaptation.

Note that the previous model $\ww'$ can be obtained by any training algorithm,
as far as it can be expressed as a weighted sum of kernel functions.