This section describes our mathematical framework. We first introduce
the basic notation (Section \ref{sec:back}), then we present our algorithm
for online model adaptation (Section \ref{sec:adapt}).

\subsection{Background}
\label{sec:back}

Assume $\xx_i \in \RR^m$ is an input vector and $y_i \in \RR$ is its
associated output.  Given a set $\{\xx_i,y_i\}_{i=1}^l$ of samples
drawn from an unknown probability distribution, we want to find a
function $f(\xx)$ such that it determines best
%determines 
the corresponding 
%associated 
$y$ for any future sample $\xx$.
%drawn from the same distribution.
This is a general framework that includes both regression and
classification.  The problem can be solved in various
ways. Here we will use kernel methods and in particular Least-Square
Support Vector Machines (LS-SVM) \cite{Cristianini00}. In LS-SVM the
function $f(\xx)$ is built as a linear model $\ww \cdot \phi(\xx) +
b$, where $\phi(\cdot)$ is a non-linear function mapping input samples
to a high-dimensional (possibly infinite-dimensional) Hilbert space
called \emph{feature space}. Rather than being directly specified, the
feature space is usually induced by a \emph{kernel function}
$K(\xx,\xx')$ which evaluates the inner product of two samples in the
feature space itself,
%between the images of vectors in the feature space,
i.e. $K(\xx,\xx')=\phi(\xx) \cdot \phi(\xx')$. A common kernel
function is the Gaussian kernel
\begin{equation}
  K(\xx,\xx')=\exp(-\gamma ||x-x'||^2)
  \label{eq:rbf}
\end{equation}
\noindent that will be used in all our experiments.

The parameters of the linear model, $\ww$ and $b$, are found by
minimizing a regularized least-squares loss function
\cite{Cristianini00}. This approach is similar to the well-known
formulation of Support Vector Machines (SVMs), the difference being
that the loss function is the square loss.
%and it
While this does not induce a sparse solution,
% On the other hand 
it makes it possible to write the leave-one-out error in closed form and with a negligible additional computational cost \cite{Rifkin07}. This is known to be approximately an unbiased
estimator of the classifier generalization error \cite{LuntzB69}. This
property is useful to find the best parameters for learning
(e.g. $\gamma$ in (\ref{eq:rbf})) and it will be used in our
adaptation method. Note that we use the same formulation to solve both
regression and classification problems.

\subsection{Model Adaptation}
\label{sec:adapt}

Let us assume we have $N$ pre-trained models stored in memory, trained
off-line on data acquired on $N$ different subjects. When the
prosthetic hand starts to be used by subject $N+1$, the system begins
to acquire new data. Given the differences among the subjects' arms
and as well in the placement of the electrodes, these new data will
belong to a new probability distribution, different from the $N$
previously modelled and stored. Still, as all subjects perform the
same grasp types, it is reasonable to expect that the new distribution
will be \emph{close} to at least one of those already modelled.
%We assume that the system has a number of previous models trained off-line on a
%number of different persons. The system then starts acquiring new data, coming
%from the person wearing the prosthetic hand. Given the differences in the
%electrodes placement and between the different subjects, the new data will
%belong to a distribution that is different from the ones of the other stored
%subject. Still we expect that the new distribution will be \emph{close} to at
%least one of the stored ones. 
Thus it should be possible to use one of the pre-trained model as a
\emph{starting point} for the training using the new data.
We expect that, by doing so,  learning should be faster
than using the new data alone. 
%Moreover the adaptation should not use the
%auxiliary data for training.
%FIXME FRA: frase ambigua, io la toglierei
To solve this problem we generalize the framework for adaptation proposed in
\cite{YangYH07} for SVM. 
%They propose 
The basic idea is to slightly change the regularization term of the
SVM, so that the solution will be close to the one of the pre-trained
model.  The optimization problem is \cite{YangYH07} %the following
\begin{align}
  \min_{\ww,b} \frac{1}{2} \|\ww- \ww'\|^2 + C \sum_{i=1}^l \xi^2 \nonumber \\
  \mbox{subject to} \;\;
  \xi_i \geq 0,\;\; y_i \ww \cdot \phi(\xx_i) + b \geq 1-\xi_i~,
  \label{eq:opt_prob_orig}
\end{align}
\noindent where $\ww'$ is one pre-trained model.
%In our opinion the above formulation has the disadvantage of giving a fixed
%weight to the previous model. 
This formulation gives a fixed weight to $\ww'$. This can be a
disadvantage in our scenario: for instance,
in the case none of the stored models is useful for the task at hand,
imposing the closeness could harm the performance of the new model.
To solve this problem, here we introduce a scaling factor for
the pre-trained model. In this way we can control the degree to which
the new model is close to the previous one.
We also change the loss into the standard square loss. This
gives us the possibility to calculate the leave-one-out error with a
closed formula, therefore
%and
%in this way 
to automatically tune this additional parameter. So we obtain the
following optimization problem
\begin{align} 
  \min_{\ww,b} \frac{1}{2} \|\ww- \beta \ww'\|^2 + \frac{C}{2} \sum_{i=1}^l \xi^2 \nonumber \\
  \mbox{subject to} \;\; y_i = \ww \cdot \phi(\xx_i) + b + \xi_i~.
  \label{eq:opt_prob}
\end{align}
It is easy to show that the optimal solution is %the form
\begin{equation}
  \ww = \beta \ww' + \sum_{i=1}^l \alpha_i \phi(\xx_i), \; \alpha_i \in \RR~,
\end{equation}
\noindent hence the final solution is given by the sum of the pre-trained model
scaled by the parameter $\beta$, and a new model given by the new data
points.  Note that when $\beta$ is $0$ we recover the original LS-SVM
formulation, that is without any adaptation to the previous data.  As
already mentioned, LS-SVM makes it possible to write the leave-one-out
error in a closed form. It turns out that it is possible to do the
same with the modified formulation (\ref{eq:opt_prob}). Hence it is
possible to set the parameter $\beta$ optimally, so to minimize the
leave-one-out error,
%. In particular
%for regression there is a closed formula for the optimal $\beta$. We leave the
%mathematical details for a longer version of this paper.
%Hence with a negligible additional computational cost we can choose the optimal
%value for the parameter $\beta$ and 
while at the same time we can choose the best pre-trained model for
adaptation.

As a last remark, we underline that the pre-trained model $\ww'$ can be
obtained by any training algorithm, as far as it can be expressed as a
weighted sum of kernel functions.  The framework is therefore very
general.
