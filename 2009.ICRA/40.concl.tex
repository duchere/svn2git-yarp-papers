The model adaptation method presented in this paper stems from a
problem in adaptive hand prosthetics, namely: is it possible to help a
patient to learn to use a dexterous hand prosthesis 
%in a quicker and
%better way, 
by exploiting the common features found in models trained
upon other patients? The answer, at least as far as healthy subjects
are concerned, is yes. We hereby presented a novel method for
model adaptation in machine learning, using Least-Squares SVMs. The
idea is to 
%build the contrain 
the SVM solution to be close to the stored model
of another subject. The degree of closeness and the choice of which model
to take among the pre-trained ones are completely automatic, as it uses an estimation
of the generalization error.

\textbf{FRANCESCO: tutta questa parte qui sopra e` molto poco chiara. occhio all'inglese.}

We tested our method on a database built with EMG and force data from
$10$ healthy subjects, trying to improve the training times and
asyntotic performance of one subject by pre-training on other
subjects. The outcome of the experiment is positive: it is apparent
that a large amount of knowledge stored in LS-SVM models is common to
all subjects, which is probably due to the anatomical analogies among
the arms. A careful positioning of the electrodes on the subjects'
forearms has the beneficial effect of damping the noise introduced by
anatomical differences. A further interesting point is that, almost
uniformly, models obtained by adaptation from a pre-trained model
obtain a \emph{better} performance than those trained from
scratch. This result is somehow surprising, although very encouraging.

It remains to discover whether this idea can be transferred to
amputees: amputations are, obviously, non-controlled, traumatic events
(except in some cases), and therefore stumps exhibit much more
variability than healthy forearms. This is the subject of future
research.
