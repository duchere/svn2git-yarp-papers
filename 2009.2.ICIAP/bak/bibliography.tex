\begin{thebibliography}{4}

\bibitem{rizz} Rizzolatti, G., Craighero, L.: The Mirror-Neuron System. Annual Review of Neuroscience. 27:169-92 (2004)

\bibitem{harris} Harris, C., Stephens, M.: A Combined Corner and Edge Detector. Proceedings of The Fourth Alvey Vision Conference. pp. 147-151 (1988)

\bibitem{schmid} Mikolajczyk, K., Schmid, C.:Scale and Affine Invariant Interest Point Detectors. In IJC. V 60(1):63-86 (2004)

\bibitem{lowe} Lowe, D. G.: Distinctive Image Features from Scale-Invariant Keypoints. International Journal of Computer Vision. 60 (2): 91–110 (2004) 

%\bibitem{lindeberg} Lindeberg, T.: Feature Detection with Automatic Scale Selection. International Journal of Computer Vision. 30 (2): 79–116 (1998)

%\bibitem{perona} Moreels, P., Perona, P.: Evaluation of Features Detectors and Descriptors Based on 3D Objects. In ICCV. (2005)

\bibitem{schmid2} Mikolajczyk, K., Schmid, C.:A Performance Evaluation of Local Descriptors. Trans on PAMI. 27(10) (2005)

%\bibitem{leibe} Leibe, B., Mikolajczyk, K., Schiele, B.: Efficient Clustering and Matching for Object Class Recognition. In BMVC. (2006)

\bibitem{csurka} Csurka, G., Dance, C.R., Fan, L., Bray, C.: Visual Categorization with Bag of Keypoints. In ECCV. (2004)

\bibitem{ferrari} Ferrari, V., Tuytelaars, T., Van Gool, L.: Simultaneous Object Recognition and Segmentation from Single or Multiple Model Views. IJVC. 67(2) (2006)

\bibitem{LoGerfo08Spectral} Lo Gerfo, L., Rosasco, L., Odone, F., De Vito, E., Verri, A.:
  Spectral Algorithms for Supervised Learning. Neural Computation. 20(7) (2008)

 \bibitem{Yao07Early}
   Yao, Y., Rosasco, L., Caponnetto, A.: On Early Stopping in Gradient Descent Learning.
   Constructive Approximation. 26(2) (2007)

\bibitem{MicchPon05Onlearning} Micchelli, C.~A., Pontil, M.: On learning vector-valued functions.
  Neural Computation. 17 (2005)

\bibitem{dev04representer} De Vito, E., Rosasco, L., Caponnetto, A., Piana, M.,	Verri, A.:Some Properties of Regularized Kernel Methods. Journal of Machine Learning Research. 5 (2004)

\bibitem{preprint} Baldassarre, L., Barla, A.,  Rosasco, L., Verri, A.:
Learning vector valued functions with spectral regularization. (preprint)

% \bibitem{jung} Hye-Won, J., Yong-Ho, S., Ryoo, M.S., Yang, H.S: Affective Communication System with Multimodality for a Humanoid Robot, AMI.
% 4th IEEE/RAS International Conference on Humanoid Robots. 2 (2004)

% \bibitem{kruger} Kr$\ddot{u}$ger, N., Felsberg, M., W$\ddot{o}$rg$\ddot{o}$tter, F.: Processing Multi-modal Primitives from Image Sequences. 4th ICSE Symposium on Engineering of Intelligent Systems. (2004) 

% \bibitem{yang} Yang, G., Lin, Y., Bhattacharya, P.: Multimodality Inferring of Human Cognitive States Based on Integration of Neuro-Fuzzy Network and Information Fusion Techniques. EURASIP Journal on Advances in Signal Processing. 8(2008)

\bibitem{gallese} Gallese, V., Fadiga, L., Fogassi, L., Rizzolatti, G.: Action Recognition in the Premotor Cortex. Brain. 119, 593–609 (1996)

% \bibitem{grans} Granström, B. House, D., Karlsson, I.: Multimodality in Language and Speech Systems. Text, Speech and Language Technology. 19(2002)

\bibitem{metta} Metta, G., Sandini, G., Natale, L., Craighero, L., Fadiga, L.: Understanding  Mirror Neurons: A Bio-Robotic Approach. Interaction Studies. 7, 197–232, (2006)

% \bibitem{caputo} Luo, J., Pronobis, A., Caputo, B.: SVM-based Transfer of Visual Knowledge Across Robotic Platforms. 5th International Conference on Computer Vision Systems. (2007)

% \bibitem{thrun} Thrun, S., Mitchell, T.: Lifelong Robot Learning. Robotics and Autonomoues Systems. 15 (1995)

% \bibitem{malak} Malak, R.J., Khosla, P.K.: A Framework for the Adaptive Trasfer of Robot Skill Knowledge Using Reinforcement Laerning Agents. Proc of ICRA. (2001)

% \bibitem{barto} Konidaris, G., Barto, A.G.: Autonomous Shaping: Knowledge Transfer in Reinforcement Learning. Proc. of ICML. (2006) 

\bibitem{wong} Hartigan, J. A., Wong, M. A.: A K-Means Clustering Algorithm". Applied Statistics. 28(1) (1979) 

\bibitem{cutkosky} Cutkosky, M.: On grasp choice, grasp models and the design of hands for manufacturing tasks. IEEE Transactions on Robotics and
Automation. (1989)

\bibitem{2007.AR} Castellini, C., Orabona, F., Metta, G., Sandini, G.: Internal Models of Reaching and Grasping. Advanced Robotics. 21(13) (2007)

%\bibitem{papcun} Papcun, G., Hochberg, J., Thomas, T. R., Laroche, F., Zacks, J., Levy, S.: Inferring articulation and recognizing gestures from acoustics with a neural network trained on x-ray microbeam data. J Acoust Soc Am. 92(2) (1992)

%\bibitem{richmond} Richmond, K., King, S., Taylor, P.: Modelling the uncertainty in recovering articulation from acoustics. Computer Speech and Language. 17 (2003)

\bibitem{bulmann02boosting} Buhlmann, P.: Boosting for High-Dimensional Linear Models". Annals of Statistics. 34(2), 559–58 (2006)

\bibitem{micchelli04kernels} Micchelli, C. A., Pontil, M.: Kernels for Multi-task Learning. NIPS (2004)

\bibitem{rifkin03rlsc} Rifkin, R., Yeo, G., Poggio, T.: Regularized Least-Squares Classification. Advances in Learning Theory: Methods, Models and Applications. (2003) 

\bibitem{pontil08transferlearning} Argyriou, A., Maurer, A., Pontil, M.: An Algorithm for Transfer Learning in a Heterogeneous Environment. ECML/PKDD. (1) 71-85 (2008) 

\bibitem{jacob08clusteredmtl} Jacob, L., Bach, F., Vert, J.P.: Clustered Multi-Task Learning: a Convex Formulation. NIPS. (2008) 

\end{thebibliography}