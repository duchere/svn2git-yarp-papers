\section{Discussion and future work}
In this paper we proposed a general architecture for learning multi-modal patterns of data. 
The underlying assumption is that the system we want to model has several perceptual channels available, 
but among them some might be inactive. 
We adopted a regression-based approach to build a behavioral model of the system 
that can be exploited to amend such inactivity.
As a validation attempt, we presented an application for grasp prediction by means of vector 
valued regression: the experimental phase produced very promising results that encourage 
us to further investigate this framework. 
Even though the regression problem is inherently vector-valued, we  restricted 
our analysis to the simple scalar-valued case. 
A preliminary analysis on the covariance matrix of the sensors measures 
shows some correlation among the sensors, both positive and negative, 
pointing at the usefulness of a full-fledged vector-valued approach. 
Recently, much work has been devoted on how to best exploit the similarity among the components 
and learn all of them simultaneously. The main idea behind most of the literature is to use prior 
knowledge on the components relatedness to design a particular penalization term or a proper 
matrix-valued kernel \cite{micchelli04kernels}. 
In absence of prior knowledge, one approach is to design an heuristic to evaluate the similarity 
among the components from the available data, e.g. by computing the sample covariance of the 
sensor measures. 
Our current research is focused on how to translate this information into a viable matrix-valued kernel. 
Alternatively one can learn the vector structure directly in the training phase 
\cite{pontil08transferlearning,jacob08clusteredmtl}.
Future work will be partially devoted to enrich the dataset with a higher number of subjects: 
this will be the basis for an extensive investigation about the potentialities of our proposal, allowing 
us to perform a more reliable statistical analysis. 
We aim also to test the adaptability with respect to different settings: as mentioned in 
Sec. \ref{sec::framework} we assumed a one-to-one mapping between objects and grasps classes; 
however, this is quite far away from real world situations, where the mapping should be instead 
one-to-many or many-to-many. Extensions to these cases will be designed and explored.
% (Fig.\ref{fig:cov})
%High differences between the two matrices and still unclear which one best represents the problem at hand.\\
%Both positive and negative strong correlations are present among the sensors, not obvious how to exploit negative correlation.

%Most of this reasearch is focused on multi-task learning, where the inputs may differ between the problems. Our setting corresponds to a natural vector-valued regression problem, where the inputs are the same for each component $f^\ell$ of the estimator.\\
