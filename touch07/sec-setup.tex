\begin{figure*}[htbp]
\centerline{
\includegraphics[height=\textwidth, angle=270 ]{./figures/ObHardware.eps}
} \caption[ObHardware]{The robot Obrero. (B) The robot has a 2DOF
head a 6DOF arm. (C) The hand has two fingers and a thumb. The
thumb and the middle finger can rotate 90\deg as shown in B. Each
finger has two phalanges with series elastic actuators. The two
phalanges are coupled and driven by one motor but they can
decouple. The phalanges and the palm are covered by tactile
sensors. There are 16 in the palm and 4 in each phalange. (A) The
tactile sensor is inspired in the ridges of the human skin. In A1
we can see a cut of the sensor that is made of silicon rubber. In
A2 and A3 we can see the deformation that a an applied force
 will cause. The position of the sensor's center is used as an estimation of the force applied.}
 
\label{fig:ObHardware}
\end{figure*}

\section{Experimental Setup}
%
The setup used in these experiment is called Obrero \ref{ObHardware}.
Obrero is an upper
torso humanoid with a head, an arm and a hand. The head is simple
two degrees of freedom (DOF) pan-tilt structure, on top of which
there is a video camera. The arm has 7 DOFs, 4 in the shulder and
three at the wrist. The hand has three fingers, that we will call
thumb, index and middle. Both the hand and the arm are equipped
with series elastic actuators, which provide torque sensing and
intrinsic elasticity to the joints. Tactile feedback is provided
to the robot by 40 sensors, 8 on each fingers and 16 on the palm
(see \ref{ObHardware}).

Might need to add citations to DOMO.
Add citation to SEA.
Add explanation/ref to finger's SEAs (they are on the figure).
