\section{Introduction}
Put here the introduction.

%
Current robot are not nearly as dexterous as humans when using
their hands to manipulate objects in the environment. Robots only
work well in very controlled environments such as car factories.
To illustrate this we can consider the task of picking up an
object from a surface, moving it and placing it in a different
surface. To do this task, the details of the object to be picked
up will be determined. The coordinates of the initial and the
final position will be determined with high precision. A feeder
will position the object in a known place with some sort of
fixture. Once the environment has been modelled, the robot's
gripper is positioned in the exact position where the object will
be. The gripper is then closed with a predetermined force, enough
to hold this specific object. The arm is them moved to the final
position where the gripper opens and the object is released. If
the object and the final surface are at the right position the
whole sequence works well.
%
However, a minimal change on the position of the parts will make
it to fail. The reason is that the robot simply repeats a sequence
without any sensing and consequently cannot adapt to changes.
%
This approach to manipulation is not very well fit to deal with
unstructured environments such as a house or an office.

Instead, we propose an approach based in dense tactile sensing. We
consider that obtaining information about the physical interaction
between the gripper and the object is more relevant to the task of
manipulation than precisely position the arm using a pre-built
model as a reference.

We use the task of picking up an object as example to illustrate
this point. We show that the robot OBRERO is able to effectively
perform this task by using tactile feedback. The feedback allows
the robot to position its hand around an unknown object by feeling
the contact, to determine the force exerted by the fingers to
avoid slippage, and to detect the contact of the object with the
surface to safely release it (or insertion).
%
This approach shows many parallel with what is known about human
manipulation. That is (to be continued)
