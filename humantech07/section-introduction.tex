
\section{Outline}

The sad fate of most robot software

Modularity in robotics

YARP: Yet Another Robot Platform

Excising communication ``plumbing'' from code

Excising device dependencies from code

Conclusions


\section{Call for publication}

As a research community, we both read and produce papers, building on
each others' work.

We also both acquire and produce software

Our software tends to die with our projects

Sad!  Software collaboration speeds things up

Research groups that all use a specific robot (Khepera, Pioneer, AIBO,
...) often form a natural software community

But each alone is a small subset of robotics

Groups developing new robots face obstacles

Differences in sensors, actuators, bodies...

Differences in processors, operating systems, libraries, frameworks,
languages, compilers...

Big barriers to software collaboration


\section{Modularity}

Constant hardware flux

Parts change rapidly

Interfaces change slowly

Lots of software grew and evolved alongside the changing hardware

Parts change rapidly

Interfaces change slowly

``Modularity'' is rewarded


\subsection{Broom}

The way parts interact can last longer than the parts themselves

E.g. an eternal broom

replace broom head

replace broom handle


\subsection{Theseus}

Long-lived software is like the Ship of Theseus

The mast gets replaced

The planks get replaced

Over time, everything may get replaced

In philosophy, this is a ``paradox of identity''

For us, it's just our job


\subsection{dark path}

The opposite of a modular system is a coupled one.

In a ``coupled'' system, changes in one part trigger changes in another.

Coupling leads to complexity

Complexity leads to confusion

Confusion leads to suffering

This is the path to the Dark Side

\subsection{modular robots}

Robot software is notoriously hardware-specific and task-specific

Both hardware and target tasks change quickly, even within the
lifetime of one project

Our humanoid robots are far more complex than one person can build and
maintain, both in terms of hardware and software

They need to be modular


\subsection{YARP}

YARP is an open-source software library  for humanoid robotics

History

An MIT / LIRA-Lab collaboration

Born on Kismet, grew on COG

With a major overhaul, now used by RobotCub consortium

Exists as an independent open source project

C++ source code


\subsection{Things we use}

CMake slide.

SWIG slide.


\subsection{What is YARP for}


Factor out details of data flow between programs from program source code

Data flow is very specific to robot platform, experimental setup,
network layout, communication protocol, etc.

Useful to keep ``algorithm'' and ``plumbing'' separate

Factor out details of devices used by programs from program source code

The devices can then be replaced over time by comparable alternatives;
code can be used in other systems


\section{Literature}

The literature of a research community both expresses its ideas, and
aids in their evolution

Published ideas are read, evaluated, and built upon

Useful advances get published

Publication of software can speed progress

Facilitates evaluating and comparing approaches

Brings new research topics into reach

Publish or perish!


