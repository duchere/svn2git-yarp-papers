
\section{Introduction}


Software development for humanoid robots is somewhat of a back-water.
Although it is a topic that excites the public imagination, the
actual resources devoted worldwide are quite low.  The Japanese
government and car manufacturers have invested in some high-profile
projects




Humanoid robotics is our passion, but in the grand scheme
of world-wide software development 




Big companies built on OSS (e.g. Google)


\ \\
\newpage


\section{Summary}

\begin{itemize}

\item Driver access in C/C++ with minimal bureaucracy -- no
entanglement with communication/middleware.  When operating
on a single machine, if the user wishes, operation without
transport or marshalling/demarshalling should be possible, ideally
direct C/C++ calls.

\item Ideally, drivers would be expressed in a form that
can be ``ripped out'' of any library or framework in use
with minimal hassle.

\item Similarly, a communication/middleware mechanism that is 
not tied up with the notion of drivers.

\item A communication/middleware mechanism that is transport
neutral, only depending on broad outline of services.



\end{itemize}


\section{Cites}

\cite{kramer2007development}

\cite{gerkey03player}

\cite{natale05developmental}

\cite{nesnas2006claraty}

\cite{vaughan2006really}

\cite{vonkrogh2006promise}


The case of embedded Linux \cite{henkel2006selective}.

Missing infrastructure.
Need a lot of software.
Ideally (for researchers starting out) should be commoditized.

\section{Hardware diversity}

Research groups that all use a specific robot (Khepera, Pioneer, AIBO,
...) often form a natural software community.  But each alone is 
a small subset of robotics.

Groups developing new robots face obstacles.  There are big barriers
to software collaboration: differences in sensors, actuators, and
bodies; differences in processors, operating systems, libraries,
frameworks, languages, compilers.


\section{Device support}

Summarize argument that, long term, free software approach 
is better for supporting devices especially with smaller user
bases, obsolescence.

\section{Being different}


Hardware defines the environment within which software operates.  Just
like the niches of natural species, there are limits to the
environments in which particular software can operate, although some
are far more specialized than others.  Robots with unusual 
hardware can find themselves initially bereft of software,
until adaptation occurs.  Therefore robots that take advantage
of novel hardware may require significant software effort.

We examine the plight of small research groups, either academic or
industrial, who wish to develop novel robots (as opposed to 
build applications on existing robots).  We want to maximize the 
reach of such research groups.

Any given piece of software can operate in a certain set of 
environments.  Every new robot is a new environment, with some
overlap with existing ones.  Some software will run there,
some will not.

The more different a robot is from a standard PC, the ...

In terms of software, robotic platforms can be quite ``genetically
isolated'' from the mainstream population of PCs.  

Computer science and PCs operate in a world that has been at least
partially commodified; research groups in CS do not have to 
reinvent the operating system for every project they undertake.




\section{Outline}

The sad fate of most robot software

Modularity in robotics

YARP: Yet Another Robot Platform

Excising communication ``plumbing'' from code

Excising device dependencies from code

Conclusions


\section{Sad fate}

Many robot projects are ``black holes'', in terms of software.  A lot
of software gets sucked in, but very little comes out.  Once a piece
of software has been adapted to a particular robot, it takes a lot
of work to extricate it again and apply it to another.

Obviously the answer to this problem is modularity.  So there are 
now many architectures/frameworks/... for modular robot systems.
The prime concern for any such system should be that it is not
a ``black hole'' -- that once a piece of software has been adapted
to a particular framework, it takes a lot of work to extricate it
again and apply it to another.  That would be a bit self-defeating.

We study YARP from this perspective.  How sticky is resultant user
code to the robot and to the framework itself?

\section{Modularity}

The way parts interact can last longer than the parts themselves.
Long-lived software is like the Ship of Theseus -- the mast gets replaced,
the planks get replaced, over time everything may get replaced (``paradox
of identity'').


\section{Free and Open Source}

Useful, more malleable.

Has the pragmatic benefit that a user of the software can
modify and integrate it to their hearts content without the 
pain of dealing with opaque binaries.

Has the revolutionary benefit that the user is not trapped in the role
of being a ``consumer'' of software, but can also be a publisher of
the changes, additions, and integrative work they do in an effective
form.  This is achieved by explicitly granting far more rights to
users than they have under the law of most countries, contrasting with
agreeably with the formerly more common practice of attempting to
minimize user rights.  These rights are typically granted
conditionally; a user may only make use of these extended rights if
(for example) distributed code is always available in its most useful
original (source) form, with compatible freedoms attached to it.  This
condition seeks to balance freedoms of individuals versus benefit to
the group.  The existence of code in usable form with freedoms 
attached can benefit many people;

The freedom to distribute code in obscure (compiled) forms



Split between people who emphasize pragmatic concerns and those
who emphasize freedom.  Just cite the issue, no need to revisit
it here.


\section{CMake}

Open-source, deals well with various IDEs and command-line development.

Not as familiar as autoconf/automake/... etc.

Has the excellent property of being simpler than making Makefiles
or configuring a project, when external libraries are involved.

The big downside is that the language is unfamiliar and a bit ugly.
It is simple and well-documented, but quirky.  An alternative with
some similar properties, scons, uses python instead.  The ant system
uses with java also seems cleaner.  However, it gets the job
done, and has the huge advantage of not being dependent on an
external language being installed.

CMake is free and open-source, with a healthy community of 
developers.



\section{Call for publication}

As a research community, we both read and produce papers, building on
each others' work.

We also both acquire and produce software

Our software tends to die with our projects

Sad!  Software collaboration speeds things up

Research groups that all use a specific robot (Khepera, Pioneer, AIBO,
...) often form a natural software community

But each alone is a small subset of robotics

Groups developing new robots face obstacles

Differences in sensors, actuators, bodies...

Differences in processors, operating systems, libraries, frameworks,
languages, compilers...

Big barriers to software collaboration


\section{Modularity}

Constant hardware flux

Parts change rapidly

Interfaces change slowly

Lots of software grew and evolved alongside the changing hardware

Parts change rapidly

Interfaces change slowly

``Modularity'' is rewarded


\subsection{Broom}

The way parts interact can last longer than the parts themselves

E.g. an eternal broom

replace broom head

replace broom handle


\subsection{Theseus}

Long-lived software is like the Ship of Theseus

The mast gets replaced

The planks get replaced

Over time, everything may get replaced

In philosophy, this is a ``paradox of identity''

For us, it's just our job


\subsection{dark path}

The opposite of a modular system is a coupled one.

In a ``coupled'' system, changes in one part trigger changes in another.

Coupling leads to complexity

Complexity leads to confusion

Confusion leads to suffering

This is the path to the Dark Side

\subsection{modular robots}

Robot software is notoriously hardware-specific and task-specific

Both hardware and target tasks change quickly, even within the
lifetime of one project

Our humanoid robots are far more complex than one person can build and
maintain, both in terms of hardware and software

They need to be modular


\subsection{YARP}

YARP is an open-source software library  for humanoid robotics

History

An MIT / LIRA-Lab collaboration

Born on Kismet, grew on COG

With a major overhaul, now used by RobotCub consortium

Exists as an independent open source project

C++ source code


\subsection{Things we use}

CMake slide.

SWIG slide.


\subsection{What is YARP for}


Factor out details of data flow between programs from program source code

Data flow is very specific to robot platform, experimental setup,
network layout, communication protocol, etc.

Useful to keep ``algorithm'' and ``plumbing'' separate

Factor out details of devices used by programs from program source code

The devices can then be replaced over time by comparable alternatives;
code can be used in other systems


\section{Literature}

The literature of a research community both expresses its ideas, and
aids in their evolution

Published ideas are read, evaluated, and built upon

Useful advances get published

Publication of software can speed progress

Facilitates evaluating and comparing approaches

Brings new research topics into reach

Publish or perish!


\section{PC analogy}

See Bill Gates article in Scientific American, January 2007

Difference: now we have the network.  Go the player route, rather than
the single IDE.  Transform from hardware to open protocols as first
step.  Then whole ecology of computation is available.

Robots aren't that special.  webcam/microphone/games...

The resources available to us are generally lower.  Smaller communities.
Less software expertise.  This could chage.


