

\section{YARP Basics}


The sad fate of most robot software

Modularity in robotics

YARP: Yet Another Robot Platform

Excising communication ``plumbing'' from code

Excising device dependencies from code



YARP is an open-source software library  for humanoid robotics

History

An MIT / LIRA-Lab collaboration

Born on Kismet, grew on COG

With a major overhaul, now used by RobotCub consortium

Exists as an independent open source project

C++ source code



\subsection{What is YARP for}


Factor out details of data flow between programs from program source code

Data flow is very specific to robot platform, experimental setup,
network layout, communication protocol, etc.

Useful to keep ``algorithm'' and ``plumbing'' separate

Factor out details of devices used by programs from program source code

The devices can then be replaced over time by comparable alternatives;
code can be used in other systems




\subsection{Two views of a robot}

From a software perspective, it is natural to consider a robot
as a bundle of devices (sensors and actuators) that can be
queried and controlled via whatever interfaces they provide.
In YARP, we call this the ``device view'' of the robot.

Interfaces to devices can be very idiosyncratic.  Some
classes of devices (such as certain classes of camera) have become
commoditized, and have relatively standard interfaces, leaving the
user with considerable freedom in how those interfaces are used in
software.  Other devices may be bundled with vendor-supplied software,
without much freedom for the user to avoid dependencies or work around
flaws.  This is understandable behavior on the part of a company,
since decoupling their hardware and software could lead to more rapid
competition and eventually commoditization, requiring rapid adjustment
of business model in order to maintain competitiveness.  However,
it can leave users of hardware in a bit of a bind.


YARP offers two almost equivalent views of a robot.


A set of ports to which you can connect and get data or send commands
(port view).

A set of devices which you can control or query according to a choice
of interfaces (device view).

The local device view: you are responsible for configuring and starting devices

The remote device view: configuration and starting-up/shutting-down is
packaged with the robot

