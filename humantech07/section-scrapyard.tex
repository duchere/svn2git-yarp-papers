
%% put text you don't know what to do with here

\section{SCRAPYARD - to be deleted probably}


\subsection*{YARP Network Scraps}

Not localization friendly, English bias at a quite low level.
Probably localization would not be practical at the network
protocol level.  Also approach to character encoding is
primitive and ad-hoc; no guarantee that encoded data
will survive in text-mode.



YARP is a free and open source project.  Since its source code is
released under a free and open license, useful parts of it can be used
by other systems, and its operation can be studied.

YARP has a large quantity of documentation (although we always need
more).  The communication protocol it uses is documented, and can be
interfaced with without using the YARP code-base.

Beyond just documenting the communication protocol, particular attention
has been devoted to make sure that that reading and writing data to a
YARP port can be done with very little effort.  YARP ports will 
accept and make connections of any of several different forms;
for a program build without the YARP code-base, it suffices
to implement just one of those connection types in order to
get basic connectivity.  If bandwidth requirements are not
excessive, the very simplest connection type can be implemented:
a very basic text-mode protocol.


