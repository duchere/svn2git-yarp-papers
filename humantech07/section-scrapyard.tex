
%% put text you don't know what to do with here

\section{SCRAPYARD - to be deleted probably}


\subsection*{YARP Network Scraps}

Not localization friendly, English bias at a quite low level.
Probably localization would not be practical at the network
protocol level.  Also approach to character encoding is
primitive and ad-hoc; no guarantee that encoded data
will survive in text-mode.



YARP is a free and open source project.  Since its source code is
released under a free and open license, useful parts of it can be used
by other systems, and its operation can be studied.

YARP has a large quantity of documentation (although we always need
more).  The communication protocol it uses is documented, and can be
interfaced with without using the YARP code-base.

Beyond just documenting the communication protocol, particular attention
has been devoted to make sure that that reading and writing data to a
YARP port can be done with very little effort.  YARP ports will 
accept and make connections of any of several different forms;
for a program build without the YARP code-base, it suffices
to implement just one of those connection types in order to
get basic connectivity.  If bandwidth requirements are not
excessive, the very simplest connection type can be implemented:
a very basic text-mode protocol.


\subsection*{Software publishing}

RANT FOLLOWS

The literature of a research community both expresses its ideas, and
aids in their evolution

Published ideas are read, evaluated, and built upon

Useful advances get published

Publication of software can speed progress

Facilitates evaluating and comparing approaches

Brings new research topics into reach

Publish or perish!

REPEATED RANT FOLLOWS

As a research community, we both read and produce papers, building on
each others' work.

We also both acquire and produce software

Our software tends to die with our projects

Sad!  Software collaboration speeds things up

Research groups that all use a specific robot (Khepera, Pioneer, AIBO,
...) often form a natural software community

But each alone is a small subset of robotics

Groups developing new robots face obstacles

Differences in sensors, actuators, bodies...

Differences in processors, operating systems, libraries, frameworks,
languages, compilers...

Big barriers to software collaboration




\subsection*{Modularity}

Constant hardware flux

Parts change rapidly

Interfaces change slowly

Lots of software grew and evolved alongside the changing hardware

Parts change rapidly

Interfaces change slowly

``Modularity'' is rewarded


The way parts interact can last longer than the parts themselves

E.g. an eternal broom

replace broom head

replace broom handle


Long-lived software is like the Ship of Theseus

The mast gets replaced

The planks get replaced

Over time, everything may get replaced

In philosophy, this is a ``paradox of identity''

For us, it's just our job


The opposite of a modular system is a coupled one.

In a ``coupled'' system, changes in one part trigger changes in another.

Coupling leads to complexity

Complexity leads to confusion

Confusion leads to suffering

This is the path to the Dark Side

Robot software is notoriously hardware-specific and task-specific

Both hardware and target tasks change quickly, even within the
lifetime of one project

Our humanoid robots are far more complex than one person can build and
maintain, both in terms of hardware and software

They need to be modular

