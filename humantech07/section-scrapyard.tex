
%% put text you don't know what to do with here

\section{SCRAPYARD - to be deleted probably}


\subsection*{YARP Network Scraps}

Not localization friendly, English bias at a quite low level.
Probably localization would not be practical at the network
protocol level.  Also approach to character encoding is
primitive and ad-hoc; no guarantee that encoded data
will survive in text-mode.



YARP is a free and open source project.  Since its source code is
released under a free and open license, useful parts of it can be used
by other systems, and its operation can be studied.

YARP has a large quantity of documentation (although we always need
more).  The communication protocol it uses is documented, and can be
interfaced with without using the YARP code-base.

Beyond just documenting the communication protocol, particular attention
has been devoted to make sure that that reading and writing data to a
YARP port can be done with very little effort.  YARP ports will 
accept and make connections of any of several different forms;
for a program build without the YARP code-base, it suffices
to implement just one of those connection types in order to
get basic connectivity.  If bandwidth requirements are not
excessive, the very simplest connection type can be implemented:
a very basic text-mode protocol.


\subsection*{Software publishing}

RANT FOLLOWS

The literature of a research community both expresses its ideas, and
aids in their evolution

Published ideas are read, evaluated, and built upon

Useful advances get published

Publication of software can speed progress

Facilitates evaluating and comparing approaches

Brings new research topics into reach

Publish or perish!

REPEATED RANT FOLLOWS

As a research community, we both read and produce papers, building on
each others' work.

We also both acquire and produce software

Our software tends to die with our projects

Sad!  Software collaboration speeds things up

Research groups that all use a specific robot (Khepera, Pioneer, AIBO,
...) often form a natural software community

But each alone is a small subset of robotics

Groups developing new robots face obstacles

Differences in sensors, actuators, bodies...

Differences in processors, operating systems, libraries, frameworks,
languages, compilers...

Big barriers to software collaboration




\subsection*{Modularity}

Constant hardware flux

Parts change rapidly

Interfaces change slowly

Lots of software grew and evolved alongside the changing hardware

Parts change rapidly

Interfaces change slowly

``Modularity'' is rewarded


The way parts interact can last longer than the parts themselves

E.g. an eternal broom

replace broom head

replace broom handle


Long-lived software is like the Ship of Theseus

The mast gets replaced

The planks get replaced

Over time, everything may get replaced

In philosophy, this is a ``paradox of identity''

For us, it's just our job


The opposite of a modular system is a coupled one.

In a ``coupled'' system, changes in one part trigger changes in another.

Coupling leads to complexity

Complexity leads to confusion

Confusion leads to suffering

This is the path to the Dark Side

Robot software is notoriously hardware-specific and task-specific

Both hardware and target tasks change quickly, even within the
lifetime of one project

Our humanoid robots are far more complex than one person can build and
maintain, both in terms of hardware and software

They need to be modular



\begin{itemize}

\item Summarize argument that, long term, free software approach 
is better for supporting devices especially with smaller user
bases, obsolescence.


\item Driver access in C/C++ with minimal bureaucracy -- no
entanglement with communication/middleware.  When operating
on a single machine, if the user wishes, operation without
transport or marshalling/demarshalling should be possible, ideally
direct C/C++ calls.

\item Ideally, drivers would be expressed in a form that
can be ``ripped out'' of any library or framework in use
with minimal hassle.

\item Similarly, a communication/middleware mechanism that is 
not tied up with the notion of drivers.

\item A communication/middleware mechanism that is transport
neutral, only depending on broad outline of services.


\end{itemize}




\subsection*{Separation of concerns}

There are three separate concerns related to devices in YARP:

Implementing specific drivers for particular devices 

Defining interfaces for device families 

Implementing network wrappers for interfaces

Basic idea: if you view your devices through well thought out
interfaces, the impact of device change can be minimized.






\subsection*{Sticky devices}

Consider the following scenarios:

\begin{itemize}

\item A user acquires a piece of hardware they want to use.  The
hardware is very new.  It is bundled with a binary library and a
skimpy example program that is the only way currently known to access
that hardware.  Use of the library is quite restricted; it may be
specific to a particular operating system version or even development
environment.

\item A user acquires many such devices, and wants to use them
all at the same time.

\end{itemize}

Let us call devices which can only be accessed using vendor supplied
material ``sticky devices'' because they tend to make the particular
set of assumptions made by the vendor stick to the user's code.
Dealing with several sticky devices at once becomes a nightmare.
Worse still, any and all of those assumptions could change on the
next release of the hardware~-- this is particularly likely if the 
hardware is very new.

A logical step in such a situation is to wrap the functionality
supplied by the vendor in a facade, so that at least source code
dependencies can be reduced.  But at a more basic level,
compilation and linking of application that accesses two or more
sticky devices can be very involved.  An option that YARP makes
available is for wrappers around the devices to be made individually,
compiled and built separately, and then used across the network.
When performance requirements permit this solution it is an
effective mechanism for quarantining the sticky devices.

We split the process of creating an interface to a device into two
parts: the ``thin wrapper'', and the ``network wrapper''.  

Writing the thin wrapper means creating a C++ API to the device
that follows a simple regular structure defined by YARP
(examples for interfacing with devices are usually written in C/C++).
If there are similarities between this device and others,
those similarities can be captures at this point by 
shared interfaces with different implementations.
Once the thin wrapper is written, user code that accesses
the device through that wrapper is less affected by its ``stickiness''
because:

\begin{itemize}

\item To the extent that user code uses interfaces shared by other
devices, another device can be substituted later without change to
that part.  This could also include future versions of the same family
of hardware.

\item The network wrapper, when written, will allow the thin wrapper
interface to be used via the network.  This decouples the compilation,
build environment, libraries, operating system, and language
dependencies of the device software and the user software.

\end{itemize}

Wrappers are written for families of interfaces, and so in as much
as one device is similar to existing devices, the network 
wrapper may come for free.

Why is it important to make devices easy to write?

\begin{itemize}

\item New hardware appears all the time, so the odds of a particular
user being one of the first people interested in a particular device are
non-negligible.  We are particularly interested in supporting humanoid
robotics, where today's hardware is very lacking and we are guaranteed
to be dealing with novel hardware for the forseeable future.

\item Vendors may bundle hardware and software, and not permit
redistribution of their software.  In practice, this can mean that
work done to wrap a device by one group may not end up in a public
repository, and so not reach other groups.  Licensing issues can push
the amount of effort required above what a group is willing to
casually contribute.  So some reinventing of the wheel is inevitable.
This seems to be particularly the case on the Microsoft Windows
platform, where the pool of free and open software available to build
on is smaller and much more poorly packaged.

\item There are several software architectures in existence for
various overlapping robotics communities.  If someone has adapted
a device to one of these architectures, it is still not trivial
to use it in another.  YARP is trying to be a good citizen by
reducing how much the core device code depends on YARP itself,
making it in principle easier to reuse.


\end{itemize}



\subsection*{The many-architecture problem}

YARP is but one software architecture of many.  For example, in the
world of mobile robotics, the Player/Stage project is widely used.

We should be careful of assuming that the interface the end-user will
want to the device is a YARP interface.

The thin wrapper is a first step that attempts to reshape a
device from a vendor into something a little bit more organized.

The trajectory imagined for device creation:

\begin{itemize}

\item User works with whatever code they can get -- vendor
supplied, from another project, etc.

\item User wraps a simple class around the functionality,
meeting YARP's simple specification for devices.  At this
point, the reward they get is that device configuration
can be easily made external to the code.  This is often
useful when starting up a device for different purposes
(normal, testing, different experimental parameters etc).

\end{itemize}


\subsection*{Driver}

We use the term ``Driver'' to refer to software interface the end-user
has to a hardware device.  The software components involved in
implementing the driver can be considered to be part of the device if
we wish.  We can consider the process of taking a ``sticky device''
and placing in a wrapper as a process of transforming the device such
that drivers are easier to build.


Orphan text:

The goal of YARP is to support humanoid robotics, in which a broad 
variety of hardware is often employed. In YARP we try to facilitate code 
exchange between researchers, especially when this speeds up the time 
it takes to develop a platform and use it for research. In this sense 
device-level software development is a very time consuming and tedious 
task, so the possibility to share code in this context is extremely 
profitable. YARP devices consists in a set of \emph{wrapper classes}, 
which usually link vendor's libraries.... 
