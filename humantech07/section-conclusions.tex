
\section{Conclusions}

%%% this paper is now cited elsewhere
%A survey of robot development environments
%\cite{kramer2007development} (p.s. rules YARP out, claim it has
%no coherent program, think they were seeing YARP1).

% \cite{gerkey03player} % redundant to newer paper

% Not sure why this is here \cite{natale05developmental}

Nesnas \cite{nesnas2006claraty} - ``coping with hardware and software 
heterogeneity''.  It is useful for the list of problems it
works through (solutions are not that exciting).
%
Mentions danger of overgeneralizing interfaces, argues for 
``multi-level abstraction models, object-oriented methodologies
and design patterns''.  A bit vacuous.

%%% this paper is now cited elsewhere
%%Vaughan \cite{vaughan2006reusable} - this is a good player/stage
%%paper.


Our previous comments on YARP1 \cite{metta2006yarp} which
describe its history and some usage information.

Gates thinks robotics could take off, Microsoft has
released a robotics platform \cite{gates2007robot}.

Missing infrastructure.
Need a lot of software.
Ideally (for researchers starting out) should be commoditized.
Free software is doing that.
%
Consider the effect of the KLT tracking code; by no means
the best in the world, but was available and still used
today for that reason.


%comparison with PC:
%Difference: now we have the network.  Go the player route, rather than
%the single IDE.  Transform from hardware to open protocols as first
%step.  Then whole ecology of computation is available.

Robots aren't that special.  webcam/microphone/games...

%The resources available to us are generally lower.  Smaller communities.
%Less software expertise.  This could chage.







