
\section{Conclusions}

A bunch of stuff we should talk about...

A survey of robot development environments
\cite{kramer2007development} (p.s. rules YARP out, claim it has
no coherent program, think they were seeing YARP1).

% \cite{gerkey03player} % redundant to newer paper

% Not sure why this is here \cite{natale05developmental}

Nesnas \cite{nesnas2006claraty} - ``coping with hardware and software 
heterogeneity''.  It is useful for the list of problems it
works through (solutions are not that exciting).
%
Mentions danger of overgeneralizing interfaces, argues for 
``multi-level abstraction models, object-oriented methodologies
and design patterns''.  A bit vacuous.

Vaughan \cite{vaughan2006reusable} - this is a good player/stage
paper.

von Krogh \cite{vonkrogh2006promise} - looking at open source
software from a (management) research perspective.

Our previous comments on YARP1 \cite{metta2006yarp} which
describe its history and some usage information.

The case of embedded Linux \cite{henkel2006selective} --
interesting overlaps with robotics.

See Bill Gates article in Scientific American, January 2007

Missing infrastructure.
Need a lot of software.
Ideally (for researchers starting out) should be commoditized.


comparison with PC:

Difference: now we have the network.  Go the player route, rather than
the single IDE.  Transform from hardware to open protocols as first
step.  Then whole ecology of computation is available.

Robots aren't that special.  webcam/microphone/games...

The resources available to us are generally lower.  Smaller communities.
Less software expertise.  This could chage.







\subsection{YARP Network Scraps}

Not localization friendly, English bias at a quite low level.
Probably localization would not be practical at the network
protocol level.  Also approach to character encoding is
primitive and ad-hoc; no guarantee that encoded data
will survive in text-mode.



YARP is a free and open source project.  Since its source code is
released under a free and open license, useful parts of it can be used
by other systems, and its operation can be studied.

YARP has a large quantity of documentation (although we always need
more).  The communication protocol it uses is documented, and can be
interfaced with without using the YARP code-base.

Beyond just documenting the communication protocol, particular attention
has been devoted to make sure that that reading and writing data to a
YARP port can be done with very little effort.  YARP ports will 
accept and make connections of any of several different forms;
for a program build without the YARP code-base, it suffices
to implement just one of those connection types in order to
get basic connectivity.  If bandwidth requirements are not
excessive, the very simplest connection type can be implemented:
a very basic text-mode protocol.

