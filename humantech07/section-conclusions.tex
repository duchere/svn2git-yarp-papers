
\section{Conclusions}

In recent years we have seen the beginning of many new and 
ambitious robotic projects 
\cite{02-c59-Hirai-1998,02-c59-Kaneko-2004,adams00humanoid,02-c59-Sakagami-2002}. 
%
However research to provide intelligence 
to these complicated robots is advancing at a snail's pace.
%
Accumulating knowledge in the form of working demonstrable systems is 
plagued by the difficulty of forming teams, on agreeing on standards, and 
in general by the lack of a critical mass in any existing laboratory no 
matter the size or funding. 

%% ok this looks too much like another introduction, might need to 
%% be shortened but perhaps could help to make a concluding point
%% if linked to the story of the PC and the need for standards...
%%
The problem of artificial intelligence seems to be more baffling, and 
arguably more difficult, than what originally thought.
% \ref{} and the year 2006 
%celebrated the $50^{th}$ anniversary of the workshop at Dartmouth College
%which is unanimously considered the birth of artificial intelligence at 
%least in some modern sense \ref{}. %Turing had previously considered the problem%
A generation of researchers did not manage to make the progress that was
certainly hoped for. Significant progress can certainly be made either 
because of a breakthrough in our understanding of the problems or through 
a slower accumulation of knowledge. Or it can be due to a combination of 
these two elements. The recent paradigmatic shift toward 
embodiment seems to require the realization of appropriate robotic 
hardware \cite{YARB}. 
This is to say that alternative means have to be found to build a community
that can accumulate knowledge and make effective progress. The niche of
humanoid robotics and artificial intelligence should be enlarged.

In this respect, the parallel with the commercial PC is easily made. 
The success of the PC was determined, among other factors, by the definition 
of hardware standards that everybody could understand, copy, and reimplement. 
From time to time new standards were required (e.g. the ISA bus slowly left
space to PCI slots) but the system flourished. Under the hood, the PC is a 
few orders of magnitude faster and of larger
storage capacity. On the software side, the benefit of a common architecture, 
allowed creating operating systems and application software consisting of 
several millions of lines of code. Without a standard hardware things
might have been more difficult.
%
A PC of today is the modern 
version of the Ship of Theseus\footnote{The Ship of 
Theseus -- the mast gets replaced,
the planks get replaced, over time everything may get replaced,
but it is still in some important sense the same ship (``paradox
of identity'')}, everything changed but the PC is still considered
a PC. 
%

Is robotics really facing the same challenges as the computer industry
three decades ago~\cite{gates2007robot}? 
%
It is clearly difficult to foresee the future of humanoid robotics. However
a few dedicated software platforms are appearing as either 
commercial~\cite{microsoft}
or academic~\cite{vaughan2006reusable} products (see also 
\cite{kramer2007development} for a survey). It is
easier %though% 
to imagine a scenario where common standards both in 
software and hardware will find the fertile soil to flourish when 
isolated breakthroughs will happen.

%%% this paper is now cited elsewhere
%A survey of robot development environments
%\cite{kramer2007development} (p.s. rules YARP out, claim it has
%no coherent program, think they were seeing YARP1).

% \cite{gerkey03player} % redundant to newer paper

% Not sure why this is here \cite{natale05developmental}

Nesnas \cite{nesnas2006claraty} - ``coping with hardware and software 
heterogeneity''.  It is useful for the list of problems it
works through (solutions are not that exciting).
%
Mentions danger of overgeneralizing interfaces, argues for 
``multi-level abstraction models, object-oriented methodologies
and design patterns''.  A bit vacuous.

%%% this paper is now cited elsewhere
%%Vaughan \cite{vaughan2006reusable} - this is a good player/stage
%%paper.


Our previous comments on YARP1 \cite{metta2006yarp} which
describe its history and some usage information.

Gates thinks robotics could take off, Microsoft has
released a robotics platform \cite{gates2007robot}.

Missing infrastructure.
Need a lot of software.
Ideally (for researchers starting out) should be commoditized.
Free software is doing that.
%
Consider the effect of the KLT tracking code; by no means
the best in the world, but was available and still used
today for that reason.


%comparison with PC:
%Difference: now we have the network.  Go the player route, rather than
%the single IDE.  Transform from hardware to open protocols as first
%step.  Then whole ecology of computation is available.

Robots aren't that special.  webcam/microphone/games...

%The resources available to us are generally lower.  Smaller communities.
%Less software expertise.  This could chage.
