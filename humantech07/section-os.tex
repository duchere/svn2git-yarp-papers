

\section{Two views of a robot}

From a software perspective, it is natural to consider a robot
as a bundle of devices (sensors and actuators) that can be
queried and controlled via whatever interfaces they provide.
In YARP, we call this the ``device view'' of the robot.

Interfaces to devices can be very idiosyncratic.  Some
classes of devices (such as certain classes of camera) have become
commoditized, and have relatively standard interfaces, leaving the
user with considerable freedom in how those interfaces are used in
software.  Other devices may be bundled with vendor-supplied software,
without much freedom for the user to avoid dependencies or work around
flaws.  This is understandable behavior on the part of a company,
since decoupling their hardware and software could lead to more rapid
competition and eventually commoditization, requiring rapid adjustment
of business model in order to maintain competitiveness.  However,
it can leave users of hardware in a bit of a bind.


\subsection{Sticky devices}

Consider the following scenarios:

\begin{itemize}

\item A user acquires a piece of hardware they want to use.  The
hardware is very new.  It is bundled with a binary library and a
skimpy example program that is the only way currently known to access
that hardware.  Use of the library is quite restricted; it may be
specific to a particular operating system version or even development
environment.

\item A user acquires many such devices, and wants to use them
all at the same time.

\end{itemize}

Let us call devices which can only be accessed using vendor supplied
material ``sticky devices'' because they tend to make the particular
set of assumptions made by the vendor stick to the user's code.
Dealing with several sticky devices at once becomes a nightmare.
Worse still, any and all of those assumptions could change on the
next release of the hardware~-- this is particularly likely if the 
hardware is very new.

A logical step in such a situation is to wrap the functionality
supplied by the vendor in a facade, so that at least source code
dependencies can be reduced.  But at a more basic level,
compilation and linking of application that accesses two or more
sticky devices can be very involved.  An option that YARP makes
available is for wrappers around the devices to be made individually,
compiled and built separately, and then used across the network.
When performance requirements permit this solution it is an
effective mechanism for quarantining the sticky devices.

We split the process of creating an interface to a device into two
parts: the ``thin wrapper'', and the ``network wrapper''.  

Writing the thin wrapper means creating a C++ API to the device
that follows a simple regular structure defined by YARP
(examples for interfacing with devices are usually written in C/C++).
If there are similarities between this device and others,
those similarities can be captures at this point by 
shared interfaces with different implementations.
Once the thin wrapper is written, user code that accesses
the device through that wrapper is less affected by its ``stickiness''
because:

\begin{itemize}

\item To the extent that user code uses interfaces shared by other
devices, another device can be substituted later without change to
that part.  This could also include future versions of the same family
of hardware.

\item The network wrapper, when written, will allow the thin wrapper
interface to be used via the network.  This decouples the compilation,
build environment, libraries, operating system, and language
dependencies of the device software and the user software.

\end{itemize}

Wrappers are written for families of interfaces, and so in as much
as one device is similar to existing devices, the network 
wrapper may come for free.

Why is it important to make devices easy to write?

\begin{itemize}

\item New hardware appears all the time, so the odds of a particular
user being one of the first people interested in a particular device are
non-negligible.  We are particularly interested in supporting humanoid
robotics, where today's hardware is very lacking and we are guaranteed
to be dealing with novel hardware for the forseeable future.

\item Vendors may bundle hardware and software, and not permit
redistribution of their software.  In practice, this can mean that
work done to wrap a device by one group may not end up in a public
repository, and so not reach other groups.  Licensing issues can push
the amount of effort required above what a group is willing to
casually contribute.  So some reinventing of the wheel is inevitable.
This seems to be particularly the case on the Microsoft Windows
platform, where the pool of free and open software available to build
on is smaller and much more poorly packaged.

\item There are several software architectures in existence for
various overlapping robotics communities.  If someone has adapted
a device to one of these architectures, it is still not trivial
to use it in another.  YARP is trying to be a good citizen by
reducing how much the core device code depends on YARP itself,
making it in principle easier to reuse.


\end{itemize}



\subsection{The many-architecture problem}

YARP is but one software architecture of many.  For example, in the
world of mobile robotics, the Player/Stage project is widely used.

We should be careful of assuming that the interface the end-user will
want to the device is a YARP interface.

The thin wrapper is a first step that attempts to reshape a
device from a vendor into something a little bit more organized.

The trajectory imagined for device creation:

\begin{itemize}

\item User works with whatever code they can get -- vendor
supplied, from another project, etc.

\item User wraps a simple class around the functionality,
meeting YARP's simple specification for devices.  At this
point, the reward they get is that device configuration
can be easily made external to the code.  This is often
useful when starting up a device for different purposes
(normal, testing, different experimental parameters etc).

\end{itemize}


\subsection{Driver}

We use the term ``Driver'' to refer to software interface the end-user
has to a hardware device.  The software components involved in
implementing the driver can be considered to be part of the device if
we wish.  We can consider the process of taking a ``sticky device''
and placing in a wrapper as a process of transforming the device such
that drivers are easier to build.


\subsection{More}


YARP offers two almost equivalent views of a robot.


A set of ports to which you can connect and get data or send commands
(port view).

A set of devices which you can control or query according to a choice
of interfaces (device view).

The local device view: you are responsible for configuring and starting devices

The remote device view: configuration and starting-up/shutting-down is
packaged with the robot



\section{Ports}

Data source knows nothing about identity of modules that monitor it.
(picture).

We follow the Observer design pattern. 

Special ``Port'' objects deliver data to:

Any number of observers (other ``Port''s) ...

... in any number of processes ...

... distributed across any number of computers/OSes ...

using any of several underlying communication protocols with different
technical advantages, streaming or RPC

This is called the YARP Network

Connections can use different protocols

Ports belong to processes

Processes can be on different machines/OS

(picture)

\section{Network example}

RobotCub picture, with gigabit network.

\section{Why?}

We've separated out most of the plumbing

We get to change it dynamically (handy)

More importantly, we have better modularity

Programs can be moved around as load and OS/device/library
 dependencies dictate

Fundamental protocol for communication can be changed without 
affecting programs

Better chance that your code can be used by others (even just within 
your group)


\section{YARP Devices}

There are three separate concerns related to devices in YARP:

Implementing specific drivers for particular devices 

Defining interfaces for device families 

Implementing network wrappers for interfaces

Basic idea: if you view your devices through well thought out
interfaces, the impact of device change can be minimized.


\subsection{A light touch}

New devices come out all the time -- needs to be easy to connect them
to existing code

YARP needs a minimal ``wrapper'' class to match vendor-supplied
library with relevant interfaces that capture common capabilities

YARP encourages separating configuration from source code -- separating
the ``plumbing''

Devices and communications remain distinct concerns

(pictures)


\subsection{Why?}

Allows collaboration between groups whose robots have different devices

Makes device changes less painful

Devices and communications are orthogonal features

Can switch from remote use of device to local use and vice versa without pain

Local use can be very efficient, just an extra virtual method call


\subsection{Interoperability}

YARP is a free and open source project.  Since its source code is
released under a free and open license, useful parts of it can be used
by other systems, and its operation can be studied.

YARP has a large quantity of documentation (although we always need
more).  The communication protocol it uses is documented, and can be
interfaced with without using the YARP code-base.

Beyond just documenting the communication protocol, particular attention
has been devoted to make sure that that reading and writing data to a
YARP port can be done with very little effort.  YARP ports will 
accept and make connections of any of several different forms;
for a program build without the YARP code-base, it suffices
to implement just one of those connection types in order to
get basic connectivity.  If bandwidth requirements are not
excessive, the very simplest connection type can be implemented:
a very basic text-mode protocol.

For example, suppose there is a YARP system running, with a 
port called ``/motor'' which will accept commands to move a
motor.  For concreteness, let's imagine we have started the
following standard YARP programs (on the same or different 
machines):

\begin{verbatim}
yarp server
yarpdev --device test_motor --axes 2
\end{verbatim}




