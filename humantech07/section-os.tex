

\section{Two views of a robot}

From a software perspective, it is natural to consider a robot
as a bundle of devices (sensors and actuators) that can be
queried and controlled via whatever interfaces they provide.
In YARP, we call this the ``device view'' of the robot.

Interfaces to devices can be very idiosyncratic.  Some
classes of devices (such as certain classes of camera) have become
commoditized, and have relatively standard interfaces, leaving the
user with considerable freedom in how those interfaces are used in
software.  Other devices may be bundled with vendor-supplied software,
without much freedom for the user to avoid dependencies or work around
flaws.  This is understandable behavior on the part of a company,
since decoupling their hardware and software could lead to more rapid
competition and eventually commoditization, requiring rapid adjustment
of business model in order to maintain competitiveness.  However,
it can leave users of hardware in a bit of a bind.


\subsection{Sticky devices}

Consider the following scenarios:

\begin{itemize}

\item A user acquires a piece of hardware they want to use.  The
hardware is very new.  It is bundled with a binary library and a
skimpy example program that is the only way currently known to access
that hardware.  Use of the library is quite restricted; it may be
specific to a particular operating system version or even development
environment.

\item A user acquires many such devices, and wants to use them
all at the same time.

\end{itemize}

Let us call devices which can only be accessed using vendor supplied
material ``sticky devices'' because they tend to make the particular
set of assumptions made by the vendor stick to the user's code.
Dealing with several sticky devices at once becomes a nightmare.
Worse still, any and all of those assumptions could change on the
next release of the hardware~-- this is particularly likely if the 
hardware is very new.

A logical step in such a situation is to wrap the functionality
supplied by the vendor in a facade, so that at least source code
dependencies can be reduced.  But at a more basic level,
compilation and linking of application that accesses two or more
sticky devices can be very involved.  An option that YARP makes
available is for wrappers around the devices to be made individually,
compiled and built separately, and then used across the network.
When performance requirements permit this solultion it is an
effective mechanism for quarantining the sticky devices.






\subsection{More}


YARP offers two almost equivalent views of a robot.


A set of ports to which you can connect and get data or send commands
(port view).

A set of devices which you can control or query according to a choice
of interfaces (device view).

The local device view: you are responsible for configuring and starting devices

The remote device view: configuration and starting-up/shutting-down is
packaged with the robot



\section{Ports}

Data source knows nothing about identity of modules that monitor it.
(picture).

We follow the Observer design pattern. 

Special ``Port'' objects deliver data to:

Any number of observers (other ``Port''s) ...

... in any number of processes ...

... distributed across any number of computers/OSes ...

using any of several underlying communication protocols with different
technical advantages, streaming or RPC

This is called the YARP Network

Connections can use different protocols

Ports belong to processes

Processes can be on different machines/OS

(picture)

\section{Network example}

RobotCub picture, with gigabit network.

\section{Why?}

We've separated out most of the plumbing

We get to change it dynamically (handy)

More importantly, we have better modularity

Programs can be moved around as load and OS/device/library
 dependencies dictate

Fundamental protocol for communication can be changed without 
affecting programs

Better chance that your code can be used by others (even just within 
your group)


\section{YARP Devices}

There are three separate concerns related to devices in YARP:

Implementing specific drivers for particular devices 

Defining interfaces for device families 

Implementing network wrappers for interfaces

Basic idea: if you view your devices through well thought out
interfaces, the impact of device change can be minimized.


\subsection{A light touch}

New devices come out all the time -- needs to be easy to connect them
to existing code

YARP needs a minimal ``wrapper'' class to match vendor-supplied
library with relevant interfaces that capture common capabilities

YARP encourages separating configuration from source code -- separating
the ``plumbing''

Devices and communications remain distinct concerns

(pictures)


\subsection{Why?}

Allows collaboration between groups whose robots have different devices

Makes device changes less painful

Devices and communications are orthogonal features

Can switch from remote use of device to local use and vice versa without pain

Local use can be very efficient, just an extra virtual method call

