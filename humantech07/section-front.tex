\begin{frontmatter}

% Title, authors and addresses

% use the thanksref command within \title, \author or \address for footnotes;
% use the corauthref command within \author for corresponding author footnotes;
% use the ead command for the email address,
% and the form \ead[url] for the home page:
% \title{Title\thanksref{label1}}
% \thanks[label1]{}
% \author{Name\corauthref{cor1}\thanksref{label2}}
% \ead{email address}
% \ead[url]{home page}
% \thanks[label2]{}
% \corauth[cor1]{}
% \address{Address\thanksref{label3}}
% \thanks[label3]{}

\title{Towards Long-Lived Robot Software}

% use optional labels to link authors explicitly to addresses:
% \author[label1,label2]{}
% \address[label1]{}
% \address[label2]{}

\author[at_iit]{Paul Fitzpatrick}
\author[at_iit,at_dist]{Giorgio Metta}
\author[at_iit]{Lorenzo Natale}

\address[at_iit]{IIT Address}
\address[at_dist]{DIST Address}

\begin{abstract}
% Text of abstract

Robots can be very unusual software platforms to develop on, perhaps
with novel devices or uncommon processors, frequently assembled by a
team stronger in mechanical design than computer science.  In fields
such as mobile robotics some standardization can be seen, but in
humanoids idiosyncracy is the rule.  Can we avoid being a backwater for
software development?  We draw lessons from the free software community
and UNIX. (list)


%Goes here.  Key point: we're a long way from standards that
%make novel robots look the same.




\end{abstract}

\begin{keyword}
% keywords here, in the form: keyword \sep keyword

keyword1 \sep keyword2 \sep keyword3

% PACS codes here, in the form: \PACS code \sep code

\end{keyword}

\end{frontmatter}


