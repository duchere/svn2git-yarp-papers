

%%\tableofcontents
%%\newpage


\begin{frontmatter}

% Title, authors and addresses

% use the thanksref command within \title, \author or \address for footnotes;
% use the corauthref command within \author for corresponding author footnotes;
% use the ead command for the email address,
% and the form \ead[url] for the home page:
% \title{Title\thanksref{label1}}
% \thanks[label1]{}
% \author{Name\corauthref{cor1}\thanksref{label2}}
% \ead{email address}
% \ead[url]{home page}
% \thanks[label2]{}
% \corauth[cor1]{}
% \address{Address\thanksref{label3}}
% \thanks[label3]{}


%% Giorgio wants the title to include hardware somehow
%%\title{Towards Long-Lived Robot Software}

\title{Towards Long-Lived Robot Genes}


% use optional labels to link authors explicitly to addresses:
% \author[label1,label2]{}
% \address[label1]{}
% \address[label2]{}

\author[at_iit]{Paul Fitzpatrick}
\author[at_iit,at_dist]{Giorgio Metta}
\author[at_iit]{Lorenzo Natale}

\address[at_iit]{
Italian Institute of Technology \\
Via Morego, 30 \\ 16163 Genova, Italy
}
\address[at_dist]{
LIRA-Lab, University of Genoa \\
Viale F. Causa, 13 \\
16145 Genova, Italy
}

\begin{abstract}
% Text of abstract

Robot projects are often evolutionary dead ends, with the software 
and hardware they produce disappearing without trace afterwards.
%%
%%This is true for all sorts of projects, but it is
%%particularly true in robotics 
%%
%Common causes include dependencies
%on uncommon or obsolete devices or libraries, and
%dispersion of an already small group of users.
%
In humanoid robotics, a small field with an avid appetite
for novel devices, we experience a great deal of ``churn'' of 
this nature.
%%
%Overcoming this would be a important technological and social
%advance for humanoid development.
%
In this paper, we explore how best to make our projects
%
% connect our software with the
%mainstream, so that it can be more 
%
stable and long-lasting, without
compromising our ability to constantly change our sensors, actuators,
processors, and networks.  We also look at how to encourage
the propagation and evolution
of hardware designs, so that we can start to
build up a ``gene-pool'' of material to draw upon for new projects.

We advance on two fronts, software and hardware.
%
%On the software front, we focus on modularity.
%We are concerned not only with how to 
%minimize the dependencies between modules, but also 
%with the more subtle problem of how to minimize
%the dependencies between modules and the middleware 
%that lies between them.
%
For some time,
we have been developing and using 
the YARP robot software architecture~\cite{metta2006yarp},
which helps organize communication between sensors, processors,
and actuators so that loose coupling is encouraged, making 
gradual system evolution much easier.  YARP includes a model of 
communication that is transport-neutral, so that data flow
is decoupled from the details of the underlying networks and protocols
in use.
%
%allowing several to be used simultaneously, key to
%smooth evolution).
%
%It encourages loose coupling with devices (sensors, actuators, etc.) 
%and can make changes in devices less
%disruptive.
%
Importantly for the long term, YARP is designed to play well with
other architectures.  
%
%It is not meant to take over your system.  
%
Device drivers
written for YARP can be ripped out and used without any
``middleware.''  On the network, basic interoperation is possible
with a few lines of code in any language with a socket library, 
and 
%
% by a 
%simple plain-text representation across TCP/IP sockets.
%
maximally efficient interoperation
%
% with YARP for demanding scenarios 
%
can be achieved by following documented protocols.
%
These features are not normally the first things that end-users
look for when starting a project, but they are crucial for longevity.



% may not know or care about interoperability,


%their is never automatically benefits by being easier 
%to access by a wider audience.
%
%
%Maintaining this ability to change transport choices depending on
%the relative importance of simplicity and efficiency is a key part
%of YARP's design.

%But when we look at the broader world of robotics, we see many
%architectures, with different priorities and concerns.  

%But developing a flexible architecture is 
%
%
%Historically, the answer to this problem has been standardization
%for protocols and APIs
%
%So we adopt a conscious strategy of making YARP 
%
%We therefore restrain YARP to being a library, and avoid any
%
%
%We take advantage of the distributed, multi-transport nature of
%YARP to make reading and writing data to YARP modules extremely
%simple

We emphasize the strategic utility of the Free Software
social contract \cite{perens1999open} to
software development for small communities with idiosyncratic
requirements.
%
%We enumerate the advantages participation in the
%Free Software community offers to a research field
%that could otherwise be isolated.
%
%
We also work to expand our community by releasing the design of our
ICub humanoid~\cite{tsagarakis2007icub} under a free and open license,
and funding development using this platform.


%this �community can give
%a considerable cost reduction, since mutual �granting of free and
%open-source licenses leads to less duplication �of engineering effort.
%
%We describe YARP.  YARP offers solutions to basic 
%Goes here.  Key point: we're a long way from standards that
%make novel robots look the same.
%



\end{abstract}

\begin{keyword}
% keywords here, in the form: keyword \sep keyword

humanoid robotics \sep free software \sep device drivers \sep ICub humanoid \sep YARP

% PACS codes here, in the form: \PACS code \sep code

\end{keyword}

\end{frontmatter}


