

%%\tableofcontents
%%\newpage


\begin{frontmatter}

% Title, authors and addresses

% use the thanksref command within \title, \author or \address for footnotes;
% use the corauthref command within \author for corresponding author footnotes;
% use the ead command for the email address,
% and the form \ead[url] for the home page:
% \title{Title\thanksref{label1}}
% \thanks[label1]{}
% \author{Name\corauthref{cor1}\thanksref{label2}}
% \ead{email address}
% \ead[url]{home page}
% \thanks[label2]{}
% \corauth[cor1]{}
% \address{Address\thanksref{label3}}
% \thanks[label3]{}


%% Giorgio wants the title to include hardware somehow
%%\title{Towards Long-Lived Robot Software}

\title{Towards Long-Lived Robot Genes}


% use optional labels to link authors explicitly to addresses:
% \author[label1,label2]{}
% \address[label1]{}
% \address[label2]{}

\author[at_iit]{Paul Fitzpatrick}
\author[at_iit,at_dist]{Giorgio Metta}
\author[at_iit]{Lorenzo Natale}

\address[at_iit]{
Italian Institute of Technology \\
Via Morego, 30 \\ 16163 Genova, Italy
}
\address[at_dist]{
LIRA-Lab, University of Genoa \\
Viale F. Causa, 13 \\
16145 Genova, Italy
}

\begin{abstract}
% Text of abstract

Robot projects are often dead ends, with the software 
and hardware they produce disappearing without trace afterwards.
%%
%%This is true for all sorts of projects, but it is
%%particularly true in robotics 
%%
Common causes include dependencies
on uncommon or obsolete devices or libraries, and
dispersion of an already small group of users.
Humanoid robotics is particularly
prone to all these causes of ``churn''.
%%
Overcoming this wiould be a key technological and social
advance for humanoid development.
%
In this paper, we explore how best to connect our software with the
mainstream, so that it can be more stable and long-lasting, without
compromising our ability to constantly change our sensors, actuators,
processors, and networks.  We also look at how to enable propagation
of our hardware designs to interested parties, so that we start to
build up a ``gene-pool'' of tried and tested designs.


Building on robot software architecture YARP~\cite{metta2006yarp},
we focus on how to organize communication between sensors, processors,
and actuators so that loose coupling is encouraged, making 
gradual system evolution much easier.  We develop a model of 
communication that is transport-neutral, so that data flow
is decoupled from the details of the underlying networks and protocols
in use (allowing several to be used simultaneously, a key to
smooth evolution).
%
We develop a methodology for interfacing with devices (sensors,
actuators, etc.) that again encourages loose coupling and can make
changes in devices less disruptive.
%
We try to avoid becoming ``middleware''; we are concerned with the
problem of incompatible architectures and frameworks, and discuss how
we work around this..
%

We emphasize the strategic utility of the Free Software approach to
software development for small communities with idiosyncratic
requirements (which, when you zoom in, includes everyone).
%
%We enumerate the advantages participation in the
%Free Software community offers to a research field
%that could otherwise be isolated.
%
%
We also work to expand our community by releasing the design of our
iCub humanoid~\cite{tsagarakis2007icub} under a free and open license,
and funding development using this platform.


%this �community can give
%a considerable cost reduction, since mutual �granting of free and
%open-source licenses leads to less duplication �of engineering effort.
%
%We describe YARP.  YARP offers solutions to basic 
%Goes here.  Key point: we're a long way from standards that
%make novel robots look the same.
%



\end{abstract}

\begin{keyword}
% keywords here, in the form: keyword \sep keyword

humanoid robotics \sep free software \sep device drivers \sep iCub humanoid \sep YARP

% PACS codes here, in the form: \PACS code \sep code

\end{keyword}

\end{frontmatter}


