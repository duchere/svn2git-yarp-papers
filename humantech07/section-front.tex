\begin{frontmatter}

% Title, authors and addresses

% use the thanksref command within \title, \author or \address for footnotes;
% use the corauthref command within \author for corresponding author footnotes;
% use the ead command for the email address,
% and the form \ead[url] for the home page:
% \title{Title\thanksref{label1}}
% \thanks[label1]{}
% \author{Name\corauthref{cor1}\thanksref{label2}}
% \ead{email address}
% \ead[url]{home page}
% \thanks[label2]{}
% \corauth[cor1]{}
% \address{Address\thanksref{label3}}
% \thanks[label3]{}

\title{Towards Long-Lived Robot Software}

% use optional labels to link authors explicitly to addresses:
% \author[label1,label2]{}
% \address[label1]{}
% \address[label2]{}

\author[at_iit]{Paul Fitzpatrick}
\author[at_iit,at_dist]{Giorgio Metta}
\author[at_iit]{Lorenzo Natale}

\address[at_iit]{
Italian Institute of Technology \\
Via Morego, 30 \\ 16163 Genova, Italy
}
\address[at_dist]{
LIRA-Lab, University of Genoa \\
Viale F. Causa, 13 \\
16145 Genova, Italy
}

\begin{abstract}
% Text of abstract

Software is unlikely to survive for very long if it is
tied to uncommon hardware or a small group of users.
Humanoid robotics is currently in exactly this condition,
leading to a lot of software ``churn''.
%
In this paper, we explore how best to connect our software with the
mainstream, so that it can be more stable and long-lasting,
without compromising our ability to constantly change our
sensors, actuators, processors, and networks.

We focus on how to organize communication between sensors, processors,
and actuators such that they remain decoupled and replaceable.
We emphasize the importance of the Free Software approach to software
distribution and licensing, and the benefit it brings to small
communities with idiosyncratic requirements.
%
%We enumerate the advantages participation in the
%Free Software community offers to a research field
%that could otherwise be isolated.
%
We discuss the importance and difficulty of creating 
repositories of drivers.


Of course, it would also be better if humanoid robot software had
larger number of users with similar hardware, and we're doing our part
to encourage this by releasing the design of our iCub humanoid under a
free and open license, and funding development using this platform.
Among other benefits to individual projects, 


%this �community can give
%a considerable cost reduction, since mutual �granting of free and
%open-source licenses leads to less duplication �of engineering effort.
%
%We describe YARP.  YARP offers solutions to basic 
%Goes here.  Key point: we're a long way from standards that
%make novel robots look the same.
%



\end{abstract}

\begin{keyword}
% keywords here, in the form: keyword \sep keyword

humanoid robotics \sep free software \sep device drivers \sep iCub humanoid \sep YARP

% PACS codes here, in the form: \PACS code \sep code

\end{keyword}

\end{frontmatter}


