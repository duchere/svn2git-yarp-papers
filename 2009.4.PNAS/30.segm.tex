\section{Speech segmentation using motor invariants}
\label{sec:segm}

The canonical approach in ASR consists of uniformly segmenting the
audio signal with a fixed-length Hamming window, usually 20ms. long.
The resulting audio bit is then analysed in the frequency or cepstral
domain and the resulting coefficients are used as features for a
classification system, to associate them to phonetic labels. This
approach has several disadvantages, mainly that of neglecting the
qualitative overall characteristics of the phoneme being uttered.
Depending on the speed of the speech, a consonant can have different
lengths and, by using the above approach, global information about it
is lost (see \cite{}, where this approach is dubbed "beads-on-a-string").

On the other hand, we concentrate on phonemes as audio objects produced
by phonetic gestures; therefore, a phoneme in the audio domain must match
the length of the phonetic gesture. The above described approach is
therefore deemed unsuitable to use motor information.

So, for each consonant under examination, we first identify physiological
motor invariants associated with it



