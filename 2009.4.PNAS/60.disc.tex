\section{Discussion}
\label{sec:disc}

The experiments presented in this paper were inspired by the intuition that the
proficiency of humans in speech recognition is grounded in the interaction
between production and understanding of speech in the human brain. Alvin
Liberman's motor theory of speech perception, although controversial and
recently reviewed and revised \cite{liberman1,liberman2,galant,massaro},
provides a theoretical framework to this intuition, which recent neurological
evidence \cite{dausilio} supports even further.

Assuming this idea as a working hypothesis, we have first tried to identify
a new, biologically inspired, way to segment the audio signal. It is nowadays
widely accepted that this task is impossible by just looking at a standard
speech signal, since the same phoneme has different audio characteristics
depending on the speaker, the voice pitch / speed / accent, and due to the
phenomenon of coarticulation. But physiology tells us that there actually \emph{is} an invariant
over all possible /b/s, for instance, and that is the act that produced it,
that is, a labial plosion. Therefore we propose to segment the audio signal
using the articulators' trajectories to detect
the beginning and end of phonemes. The choice of the articulators and the
conditions on the trajectories are established according to basic rules of
the phonetic production; for example, a /b/ will be identifed from
the beginning to the end of an associated labial plosion. This is possible,
in our case, thanks to a database of synchronised speech / articulographic
data gathered from several human subjects.

With respect to the traditional speech segmentation, done via uniform time
windows of about $20$ milliseconds, this approach focusses on the \emph{act}
that produced the sound. If it is true that speech acts are invariant across
subjects and words, then it should be possible to extract directly from the motor data
fetaures that are more robust to changes in noise, speakers,
coarticulation, etc., than standard
audio features. We use to this end the coeffcients of a cubic fit of the
motor trajectories, so to obtain a qualitative representation of the speech
act.

%
% Giorgio: the cubic fit is not mentioned earlier in the paper?
%

It is clearly shown by our results (Figures \ref{fig:class1_perf}
and \ref{fig:class2_perf}) that these features perform
better than a set of $20$ cepstral coefficients evaluated across each utterance,
while discriminating /b/, /p/, /d/ or /t/ in the database. This gap is even
more evident if examined from a per-speaker or per-coarticulation perspective.
In these tasks, the audio features show a dramatically decreasing discrimination
ability, which is not the case with the motor features. In short, motor features
are ``more invariant'' than audio features across speakers and coarticulations.
This agrees with the motor theory of speech perception.

Now, speech acts are not available in real life --- according to the mirror
paradigm, human beings reconstruct them from the audio signal, after a long
training which begins in infanthood and probably never ends. The AMM is a
representative of this mechanism. Its numerical evaluation
reveals that the normalised root-mean-square-error obtained is rather
uniform across the reconstructed signals, and is at worst $15.55\%$. These
results are coherent with those presented in \cite{papcun,richmond2002,richmond2007}.
But what is more interesting is that the AMM reconstructs labial trajectories
better than dental trajectories when presented with labials and viceversa,
and dually for tongue/dental trajectories and dentals, giving rise to a double
dissociation. A simple explanation of this fact is that human beings
(upon whose data the AMM is trained) identify phonetic actuators which produced
the sound they hear, and neglect the others.

The AMM is used to reconstruct the speech acts from the audio signal; these
reconstructed motor trajectories are then used as input to the phoneme
discrimination classifiers used in the previous experiments. The results
(consider Figures \ref{fig:class1_perf} and \ref{fig:class2_perf} again)
obviously show that they attain lower performances than the features
extracted from the real motor signals, but actually, comparable
to the audio features; and that if we join the audio classification model
to the one obtained for these features (with a trivial probabilistic calculation)
we again obtain a high performance, superior to that of audio features alone
and comaprable to that of real motor features.

AMM-reconstructed audio features also help in the presence of noise. This time
we evaluate the performance of a $200$ milliseconds wide Mel cesptrogram (a total
of $351$ numbers, each one coding \emph{one} phoneme) as noise is added to the
speech signal; the performance degrades steadily. But again, if we join this
model with one obtained by running the AMM on the noisy speech, we uniformly
and significantly increase the performance (Figure \ref{fig:class3_perf}). The
gap seems to widen as more and more noise is added, although this claim would
require a deeper analysis.

Of course, AMM-reconstructed motor features do not carry any more information
(in the sense defined by Shannon) than what is in the speech signal; therefore,
from the point of view of machine learning / speech recognition, one can see
this technique as ``just'' a better way of extracting invariant features from
speech.
