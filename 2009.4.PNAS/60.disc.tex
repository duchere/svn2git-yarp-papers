\section{Discussion}
\label{sec:disc}

\subsection{Do motor features help?}

The experimental results presented in the previous Section clearly prove that,
at least in the case here examined, and with the techniques here exploited,
the answer to the question posed in the Introduction, ``can the use of MI-based
features improve a phoneme classifier's performance?'' is positive.

Overall, $16$ features extracted from motor invariants detected with an articulograph
(what we have called real motor features) exhibit dramatically better error rates
than $390$ state-of-the-art audio features in an automated discrimination task between
two bilabial and two dental consonants. Since the discrimination is performed using an
absolutely standard classifier (and, according to the literature, a good one), that is,
a Support Vector Machine with Gaussian kernel whose hyperparameters are found via
grid-search, this result should have a somehow more general validity than what is shown
in this paper.

The performance gap is apparent and statistically significant in all our experiments.
It increases as the training sets are restricted, for example
when per-subject (i.e., training on some subjects, testing on the others) or
per-coarticulation (i.e., training on some coarticulating phonemes and testing on
others) tests are conducted. This clearly indicates that MI-based features are
somehow ``more invariant'' than audio-based ones across subjects and
coarticulation --- a quantitative confirmation of a basic
intuition, almost common-sensical: to produce, e.g., a bilabial, the act of
the labial plosion is common to all human beings and is not affected by
coarticulation. This is one more hint at the fact that the use of motor features
could be a great leap forward in ASR.

Now obviously, knowing that motor information is useful to improve ASR is just half
of the issue, since the problem of \emph{gathering it} during speech recognition is
still unexplored --- one cannot expect the standard user of an ASR system to wear
an articulograph while, e.g., dictating. Here the MTS and the theory of mirror neurons
inspire us to build an AMM, that is, to try and reconstruct the distal speech acts from
the audio signal alone. All in all, not even humans have access to the distal speaker's
motor data, and recent studies, among which D'Ausilio et al.'s \cite{dausilio}, indicate
that they might be reconstructing it while hearing the sound of speech; and that this
mechanism is activated mainly in hostile conditions (e.g., in the presence of noise).

Our Audio-Motor-Map, this one too built using a standard machine learning
method (namely, a feed-forward neural network), is able to reconstruct the MIs
to such a degree of precision that the same $16$ motor features, extracted from these
reconstructed trajectories, exhibit comparable or better error rates than  
those found with the audio features when the training sets are restricted (Experiments 1 and 2);
 and they boost a largely and significantly \emph{better} performance then the audio ones, as
noise is added to the audio signal (Experiment 3). This latter result seems to be
somehow in agreement with what D'Ausilio et al. have found using TMS on humans.

Note that in the most critical cases (i.e., when the training data sets are extremely restricted) of Experiments 1 and 2
the reconstructed motor features outperform the audio features. These results and the results of Experiment 3
suggest that when the difficulty of the classification task increases (because of a reduction of speech 
variability in the training data and/or of an increased variability in the testing data) the reconstructed 
motor features become more and more useful for the task.

Lastly, when audio and reconstructed motor features are joint using a simple probabilistic schema, the error rates
are sometimes significantly better than when the feature sets are used independently.
When one set of features is obviously far worse than the other, such a joint model performs in-between (e.g., consider
Experiment 3 when noise is higher than $50\%$); a more interesting case is that
found in Experiment 2, CV schemas \spkb\ and \spkc, where no clear advantage is seen
when using either the audio or the reconstructed motor features alone, while
the joint models perform significantly better. This means that the MI-based models are
correctly classifying with high probability some consonants that the audio-based models
moderately misclassify; and vice-versa. Sometimes the audio features help, sometimes the
MI-based features do.

This indicates that motor features, even when the audio signal is the only source of
information available (a realistic scenario) can improve the discrimination of phonemes.

\subsection{Further remarks}

The experiments presented in this paper are inspired by the intuition that the
proficiency of humans in speech recognition is grounded in the interaction
between production and understanding of speech in the human brain. Alvin
Liberman's motor theory of speech perception, although controversial and
recently reviewed and revised \cite{liberman1,liberman2,galant,massaro},
provides a theoretical framework to this intuition, which recent neurological
evidence \cite{dausilio} supports even further; but we do not feel we are in
the position of claiming that our results support the MTS. Clearly, more
experiments are required, with larger data sets, e.g., more words, more subjects
and more sensors.

In this work, also a novel way of segmenting the speech signal is introduced.
The traditional equal-length segmentation, carried out using acoustic properties
only, has strong limitations mainly due to intra-speaker speech variability and
to coarticulation. Here we propose to segment the audio signal using the articulators'
trajectories to detect the beginning and end of phonemes. The choice of the articulators
and the conditions on the trajectories are established according to basic rules of
the phonetic production; for example, /b/s are identifed using the beginning and end of
a bilabial plosion. With respect to the traditional speech segmentation, this approach
focuses on the \emph{act} that produced the sound. To capture this act, we use the
coefficients of a cubic fit of the motor trajectories, so to obtain a qualitative
representation of it.

About the AMM: from an information-theoretical point of view, AMM-reconstructed
motor features do not carry more information than what already is in the speech signal.
The AMM is a function, so one could see this technique as
``just a better way of extracting ASR features from speech''.
The main advantage in using it is that it is highly bio-inspired,
having been trained to associated human speech data to motor data. The double dissociation
observed in Figure \ref{fig:DD} reflects the rather predictable phenomenon
that consonant-critical articulators exhibit less variance than non-critical ones (e.g.,
when a /b/ uttered the labial trajectory is highly constrained, as opposed to the tongue-dental
trajectory). This results in a better prediction of bilabial (dental) trajectories when the
AMM is presented with a bilabial (dental) consonant.

Lastly, notice that in Experiment 2 the AMM-reconstructed motor features perform, in general,
as well as the audio features, while the real motor features are by far better. So, at first
sight, one could be tempted to think that a better reconstruction should achieve better error
rates, getting close to those of the real motor features;
but on the other hand, the AMM uses the speech signal too, so it is not clear
whether a much better reconstruction can be achieved in our case at all. A larger training
database and more extensive experiments could shed light on this still open point.
