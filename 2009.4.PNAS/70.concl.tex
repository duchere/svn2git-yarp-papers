\section{Conclusions}
\label{sec:concl}

Although applied to a simple problem like phoneme discrimination, and on a
limited data set, the ideas presented in this paper seem promising and foster
the use of acoustic motor invariants in speech recognition. Also,
the results we present support some of the claims of the motor theory
of speech perception; namely, that speech acts are the real invariant
quantities over speech; and that the phenomenon of coarticulation cannot
be avoided if not resorting to motor-based features.




