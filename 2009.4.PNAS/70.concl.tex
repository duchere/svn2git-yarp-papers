\section{Conclusions}
\label{sec:concl}

Although applied to a simple problem like phoneme discrimination, and on a
limited data set, the work presented in this paper adds evidence of the utility 
of motor information in automatic speech recognition. Also,
the results we present support some of the claims of the motor theory
of speech perception; namely, that speech acts are the real invariant
quantities over speech; and that the phenomenon of coarticulation cannot
be avoided if not resorting to motor-based features.













 Our results show that
this is the case,
especially when the classifier is trained on incomplete data sets such as 
per-speaker (e.g., training on speakers $1,2,3$ and testing on $4$) and
per-coarticulation (e.g., training on /ba/, /be/, /bi/, /bo/ and testing on /bu/); or when noise is added,
in which case motor features significantly help classification, even when added to a
state-of-the-art set of audio features about $20$ times larger than that extracted
from the MIs.