\section{Phoneme discrimination}
\label{sec:class}

\subsection{Experiment 1}
\label{subsec:exp1}

In the first experiment the performance of a standard classifier is evaluated
according to the \overall\ CV schema using four different sets of features as
input. Figure \ref{fig:class1_perf} shows the results. The balanced error rate
is shown as a comparative measure of performance. This error rate is defined in
our case as the average of the ratios of correctly guessed bilabials and dentals.
With respect to the more popular error rate, i.e., the overall ratio of correctly
guessed labels, it has the advantage of favouring models which can correctly guess
\emph{both} the bilabials and the dentals.

``Audio'' is a set of $390$ cepstral coefficients evaluated as follows: for each
sequence considered, $10$ time slices of the audio signal between $20$Hz and $2$KHz
are considered, equally spaced over the sequence itself,
each one $25$ milliseconds long. In case the sequence
is shorter than $250$ milliseconds, the slices are allowed to overlap. For each slice,
$13$ Mel cepstral coefficients, plus their first- and second-order derivatives are
evaluated. This makes it for $39$ coefficients for each slice. This is a standard set
of features according to state-of-the-art literature, when the segment
length is fixed a priori. In this case the segment length is not fixed so the
cepstra are evaluated over the whole window.

``Real motor'' is a set of $16$ coefficients evaluated as follows: for each
signal considered (\vlio, \alio, \vttu\ and \attu), a least-squares piecewise
Hermite cubic interpolation is generated over the sequence. This results in $4$ real
numbers per signal (constant, I-, II- and III-order coefficient of the cubic
interpolant), which qualitatively encode the shape of the considered trajectory.

``Reconstructed motor'' refers to the same procedure as above, but applied
to the AMM-reconstructed signal curves.

Lastly, ``Joint'' denotes a decision procedure obtained by averaging out the
label probabilities obtained from the best classifiers for the audio and
reconstructed motor features, and then using a threshold at $0.5$.

The classifier is a Support Vector Machine \cite{BGV92} with Gaussian kernel
and hyperparameters $C, \sigma$ found by grid-search. Samples are normalised
by subtracting the mean values and dividing by the standard deviations,
dimension-wise, in the real motor and reconstructed motor cases, while
no normalisation is applied to the audio features.

The error rates obtained are, in turn,
$5.73\% \pm 0.74\%$ (mean $\pm$ one standard error of the mean),
$0.97\% \pm 0.36\%$,
$5.36\% \pm 0.46\%$ and
$4.50\% \pm 0.58\%$. Student's two-tailed t-test shows $p<0.01$ between real motor features
and all the others, while in all other cases $p$ denotes weak statistical difference (e.g.,
$p=0.18$ between audio and joint features). Together with the error rate values, this lets
us claim that there is a marginal advantage in using joint features
over audio only, but that a large and evident advantage is found using the real motor features
over all the others.

\subsection{Experiment 2}
\label{subsec:exp2}

Experiment 2 replicates Experiment 1 using the remaining CV schemas.
Figure \ref{fig:class2_perf} shows the results. (For comparison,
the figure also shows in the first column the accuracies obtained for
Experiment $1$.)

Consider the per-speaker schemas, i.e., \spka, \spkb\ and \spkc. The real motor
features are, again, strikingly (and significantly, $p<0.01$) better than all others,
with increasing error rates of
$2.40\% \pm 1.66\%$,
$2.62\% \pm 0.26\%$ and
$7.27\% \pm 4.40\%$ for \spka, \spkb\ and \spkc\ in turn. Increasing (and larger) error
rates are found when using audio and reconstructed motor features in all schemas, with
no significant statistical difference. Significantly different performances are obtained
with the joint features in the \spkb\ and \spkc\ schemas ($p<0.01$ with error rates, in turn,
of $8\% \pm 0.36\%$ and $11.92\% \pm 0.73\%$).

In the per-coarticulation cases, the error rate is generally high (between $30\%$ and $38\%$
where chance level is $50\%$) and statistically similar ($p>0.05$) among audio, reconstructed
motor and joint features. The real motor features, again, perform dramatically better
($6.41\% \pm 1.19\%$ and $6.37\% \pm 1\%$ for \coa\ and \cob).

\subsection{Experiment 3}
\label{subsec:exp3}

Lastly, in Experiment 3 the comparison among feature sets is evaluated with the
overall CV schema (which gives the best results in Experiment 2), as white noise is added
to the audio signal. The intensity of noise is changed from $10\%$ to $150\%$ of
the standard deviation of each utterance considered; for each sequence, $10$ noisy
ones are generated, in order to obtain a larger statistical basis.
Figure \ref{fig:class3_perf} shows the results.

The real motor features, not affected by noise, are shown as comparison, and stay at
the above mentioned error rate (see Experiment 1) of $0.97\%$. The error rate of the
other three sets of features starts from values only slightly worse those of
Experiment 1 when noise is at $10\%$: namely,
$7.49\% \pm 0.26\%$, 
$6.89\% \pm 0.19\%$ and 
$5.20\% \pm 0.14\%$ for audio, reconstructed motor and joint features, in turn.
As the level of noise is increased though, the audio features' error rate
increases superlinearly until it reaches about $45\%$ when the noise is at $70\%$
level, going then asymptotically to chance level. As opposed to that, the reconstructed
motor features exhibit a much higher resilience to noise, increasing the error rate
only linearly and reaching, e.g., $19.89\% \pm 0.45\%$ when the noise is at $70\%$.
At the maximum level of noise, $150\%$, the reconstructed motor features still keep
the error rate at $32\% \pm 0.62\%$ while the audio features essentially reach chance
level. Actually, we ourselves checked how some of the phones sound when the noise is
so high, and found them rather hard to understand!

Lastly, the joint features perform best or as well as the reconstructed motor features
at low levels of noise (until $40\%$), while they then become less useful than the
reconstructed motor alone. This is obviously due to the weak performance of the audio
features.

The t-test reveals statistically different mean error rates ($p<0.01$) for all levels of
noise, except for audio and reconstructed motor when the noise
is at $10\%$ and for reconstructed motor and joint when the noise is at $30\%$
and $40\%$.
