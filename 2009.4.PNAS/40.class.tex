\section{Phoneme discrimination}
\label{sec:class}

\subsection{Experiment 1}
\label{subsec:exp1}

In the first experiment the performance of a standard classifier is evaluated
according to the \overall\ CV schema for four different sets of features.
Figure \ref{fig:class1_perf} shows the results.

``Audio'' is a set of $20$ cepstral coefficients evaluated according to the
Mel scale over a bandwidth of $20$Hz to $2$KHz. This is a standard set of features
according to state-of-the-art literature, when the segment
length is fixed a priori. In this case the segment length is not fixed so the
cepstra are evaluated over the whole window, regardless of its duration.

``Real motor'' is a set of $17$ coefficients evaluated as follows: for each
signal considered (\vlio, \alio, \vttu\ and \attu), a least-squares cubic fit
is generated over the selected segment, that is, from $t_{start}$ to $t_{end}$
as defined above. This results in $4$ real numbers
per signal, which qualitatively encode the shape of the signal considered. The
$17$th number is the average of the voicing signal over the selected segment.

``Reconstructed motor'' refers to the same procedure as above, but applied
to the AMM-reconstructed signal curves.

Lastly, ``Joint'' denotes a decision procedure obtained by multiplying the
label probabilities obtained from the best classifiers for the audio and
reconstructed motor features.

The classifier is a Support Vector Machine \cite{BGV92} with Gaussian kernel
and hyperparameters $C, \sigma$ found by grid-search. Samples are normalised
by subtracting the mean values and dividing by the standard deviations,
dimension-wise.

The accuracies obtained are, in turn,
$87.64\% \pm 1.05\%$
$92.05\% \pm 0.70\%$
$87.38\% \pm 0.83\%$
$92.65\% \pm 1.00\%$, with statistically significant difference (Student's t-test,
$p<0.01$) between audio and real motor features, audio and joint features, and
reconstructed motor and joint features.

\subsection{Experiment 2}
\label{subsec:exp2}

Experiment 1 was replicated using this time the remaining CV schemas.
Figure \ref{fig:class2_perf} shows the results. (For comparison,
the Figure also shows in the first column the accuracies obtained for
experiment $1$.)

For each and every CV schema, the statistically significant differences seen in
Figure \ref{fig:class1_perf} are present (again, tested via $p$-values);
in particular, in the per-speaker CV schemas, the audio features versus
the joint models show accuracies of
$81.92\% \pm 2.04\%$ and $87.27\% \pm 1.17\%$ (\spka),
$78.90\% \pm 0.61\%$ and $85.74\% \pm 0.44\%$ (\spkb), and
$70.29\% \pm 0.73\%$ and $76.25\% \pm 1.53\%$ (\spkc).

In the per-coarticulation results the gap is more evident; the joint models
here perform worse than the reconstructed motor features. Audio features
versus reconstructed motor features show accuracies of
$41.60\% \pm 9.72\%$ and $62\% \pm 4.86\%$ (\coa) and
$34.84\% \pm 3.64\%$ and $55\% \pm 5.5\%$ (\cob).

\subsection{Experiment 3}
\label{subsec:exp3}

Lastly, for experiment 3 we replaced the audio features with
a $351$-dimensional set of $20$ mel-cepstral
coefficient, this time obtaining by taking $19$ slices of $20$ milliseconds each,
shifted over the audio segment by the required amount of time. This set of
features was compared, again, with the real and reconstructed motor features and
a probabilistic joint model, as white noise was added to the audio signals.

The intensity of noise was changed from nil to $150\%$ of the standard deviation
of each utterance considered; for each sample, $10$ noisy ones were generated, in
order to obtain a larger statistical basis. As in experiment $1$, the \overall\ CV
schema was chosen. Figure \ref{fig:class3_perf} shows the results.
(Notice that the real motor features are not affected by noise.) Student's t-test
reveals that the cepstra-2KHz and joint features are always significantly different.
The cepstra-2KHz and joint features show accuracies ranging from
$91.96\% \pm 0.22\%$ and $93.26\% \pm 0.25\%$ for no noise, to
$69.34\% \pm 0.47\%$ and $75.69\% \pm 0.53\%$ for maximum noise.
