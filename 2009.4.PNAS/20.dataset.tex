\section{Data Set}
\label{sec:dataset}

\subsection{Subjects and Set-up}
\label{subsec:setup}

Six female subjects, Italian mothertongue, were recorded while uttering
Italian words and pseudo-words. Words were mainly stress-initial, e.g.,
/matto/, /nome/, /strada/ (mad, name, road), and were chosen in order
to have consonants both at the beginning and in the middle of
words, followed by different vowels and consonants.
The data recording setup includes a \emph{Laryngograph Microprocessor}
device (Laryngograph Ltd., London, www.laryngograph.com) which gathers speech audio
signal and the electroglottographic (EGG) signal at $16$KHz sampling
rate; and an AG500 electromagnetic articulograph (Carstens Medizinelektronik
GmbH, Germany, www.articulograph.de) that records the
3D positions of a set of sensors glued on the tongue, lips and front teeth
during speech production at a sampling rate of $200$Hz. A full description of the 
acquisition set-up and the obtained database can be found in \cite{tavella}.

The subset used in this work comprises the $77$ words in the database
which contain /b/, /p/, /d/ or /t/. This includes utterings from each of the
$6$ subjects, and consonants are found both at the beginning of the word or
in the middle.

\subsection{MI-Based Signal Segmentation}
\label{subsec:segm}

We define the length of a phoneme in terms of the MI underlying its production;
the audio signal is, therefore, segmented according to it.
A qualitative examination of the synchronised audio and motor
signals obtained from utterances of /b/, /p/, /d/ and /t/
by different speakers indicates that common patterns can
actually be found in the behaviour of the related articulators.
For instance, as is apparent from Figure \ref{fig:isdView}, 
recurring shapes of the lips opening velocity and acceleration appear
when both /ba/ and /bufalo/ are considered, even when uttered by different
speakers. The same patterns can be observed when other words containing
/b/ and /p/ are considered, both when the phoneme appears at the beginning
or inside a word, and regardless of the phoneme appearing immediately after.

These observations visually confirm the basic taxonomy of stop consonants 
as found in any linguistics textbook.
In particular, all considered consonants are plosives,
i.e., produced by a fast release of air; /b/ and /p/ are bilabials
(plosion produced using the upper and lower lips) while /d/ and /t/ are
dentals (plosion produced using the tongue tip and the upper teeth);
and, lastly, /b/ and /d/ are voiced, i.e., the vocal folds are active during
the plosion, while /p/ and /t/ are voiceless.

The following motor invariants are then defined and associated to the
consonants under examination:

\begin{enumerate}

  \item Let $s_1(t)$ and $s_2(t)$ be the signals associated
    with sensors placed on two phonetic actuators (e.g., the upper and
    lower lips), and $\delta(t) = ||s_1(t)-s_2(t)||$ be their
    Euclidean distance. Then, a plosion is defined as the interval
    between two instants $t_{start}$ and $t_{end}$ such that
    $\dot{\delta}(t_{start}) = 0 $ and $\ddot{\delta}(t_{start}) > 0$,
    and $\dot{\delta}(t_{end}) > 0 $ and $\ddot{\delta}(t_{end}) = 0$.

  \item For /b/ and /p/, the sensors on the upper and lower
    lip are considered for $s_1(t)$ and $s_2(t)$, whereas for /d/ and /t/
    those on the tongue tip and upper teeth are. In turn, the associated
    distances will be denoted as \lio\ (lips opening) and \ttu\
    (tongue tip - upper teeth distance).

  \item To discriminate /b/ and /d/ from /p/ and /t/, the average of the
    voicing signal over the duration of the phoneme is considered. The
    voicing signal is derived from the EGG signal and indicates
    whether the vocal folds are active.

\end{enumerate}

Condition $(1)$ physically defines a plosion, e.g., considering \lio, $t_{start}$
marks the onset of the act of opening the lips, while $t_{end}$ is found at the
instant of maximum opening velocity. The choice of cutting the signals at $t_{end}$
rather than, say, when the lips are still and \lio\ is maximum, is motivated by
the need to capture the plosion only, with as little as possible of the following
phoneme. By manual inspection of the audio segments so obtained, we could actually
verify that only a tiny fraction of the coarticulating phoneme could be heard
at the end of the uttering.

Then, condition $(2)$ selects an appropriate pair of articulators needed for the
phoneme under consideration, and condition $(3)$ discriminates voiced from voiceless
consonants. This schema matches the above-mentioned taxonomy. In Figure
\ref{fig:isdView} the gray zone indicates the detected interval of time using conditions
$1$ and $2$.

The segmentation is carried out semi-automatically: for each
utterance, all sequences matching conditions $(1)$---$(3)$ are displayed and the
associated speech is played, so that the experimenter can choose whether the
sequence is a correct guess or it is a false positive. In this experiment we only
monitor \lio\ and \ttu, so that false positives appear, e.g., when considering
/ts/ and /dz/. This is why, at this stage, a completely automatic segmentation
cannot be enforced. If the sequence is accepted, it is labelled
with the associated consonant, the speaker, and the
coarticulating phoneme. For example, from the word /bronzo/ (bronze) a /b/
sequence is extracted, and the letter "r" is stored as the coarticulating phoneme.
This way, from the original $77$ words and pseudowords, a total
of $1157$ audio/motor sequences are extracted, with a length of $122 \pm 41.2$
milliseconds (mean $\pm$ one standard deviation).

% qui potremmo mettere una tabella con le statistiche sul data set:
%
%In turn, 
%
%\begin{itemize}
%
%  \item $292$ /b/, $303$ /p/, $59$ /d/ and $503$ /t/;
%
%  \item $213,149,156,229,241,169$ for subjects $1,\ldots,6$;
%
%  \item $352,59,99,377,167,71,32$ for each coarticulating phoneme
%    /a/, /e/, /i/, /o/, /u/, /r/ and /s/.
%
%\end{itemize}
%
%Notice that no distinction is made whether the consonant is at the beginning
%of an utterance (e.g., /b/ in /baffo/, moustache) or in the middle of
%it (e.g., /d/ in /strada/, \emph{road}).

\subsection{Cross-Validation}
\label{subsec:cv}

As is standard practice in machine learning, the obtained dataset was divided into
splits to perform cross-validation (CV). CV \cite{stat} is usually used
to determine an estimate of the generalisation error of a learning method, at the
same time providing a ground for grid-searching the optimal parameters\footnote{These
two points should really be independent of each other, a feature provided, e.g., by
\emph{nested} CV \cite{nestedCV}; but here the size of the dataset is
prohibitively small for such an analysis.}. The dataset was therefore divided into
$10$, almost equally sized, random disjoint sets,
whose union coincides with the original dataset; every
learning method would then be trained upon $9$ of these sets and tested upon the
tenth. This ensures that no data used for testing have been used for training. Such a
training/testing set pair is called a \emph{split}. (Data are normalised using the
statistics of the training set.) After choosing a standard performance index for the machine
considered (e.g., the classification accuracy for classifiers), we would run the machine
on each split and show the mean performance, plus/minus one standard error of the mean,
that is the standard deviation divided by the square root of the number of splits.

We will call this type of CV, the \emph{overall} CV schema,
since every training and testing sets contain data tagged with every subject and
coarticulating phoneme. In our case, however, CV is used also to perform more subtle
tests, for instance to check whether a classifier trained on some subjects would
then effectively predict samples uttered by an another subject; or whether a classifier
trained upon a subset of coarticulating phonemes (e.g., on /$\alpha\beta$/ with
$\alpha \in \{b,p,d,t\}$ and $\beta \in \{a,e,i,o\}$) would then effectively predict samples
coarticulated with other phonemes (in the example above, predict samples of the form
/$\alpha\gamma$/ with $\gamma \in \{u,r,s\}$). To this end, $5$ more CV schemas were devised:

\begin{itemize}

  \item \spka\ The training sets contain samples
  	uttered by $5$ speakers while the testing set is
  	uttered by the remaining speaker; this gives us $6$ splits.

  \item \spkb\ Likewise, but training on $3$ speakers and testing on the
  	other $3$. This results in $\binom{3}{6} = 20$ splits.

  \item \spkc\ Likewise, but training on $1$ speaker and testing on the
  	other $5$, resulting in $6$ splits.

  \item \coa\ The training sets contain samples
  	with $4$ coarticulating vowels, whereas the testing sets contain samples
  	with the remaining two, plus /r/ and /s/. This gives us $5$ splits.

  \item \cob\ Likewise, but training on $3$ coarticulating vowels and
  	testing on the remaining $2$ plus /r/ and /s/. This gives us
  	$\binom{3}{5} = 10$ splits.

\end{itemize}

Notice that, the more splits, the more reliable the results are, as far as the
generalisation error is concerned; but as well, in some cases the training sets
are rather small if compared to the testing sets. This is the reason why we could
not try, e.g., a CV schema with training on utterings containing
one coarticulating vowel and testing on all the others (e.g., training on
/$\alpha$a/ and testing on /$\alpha\beta$/, where $\alpha$ is any consonant
and $\beta \in \{e,i,o,u,r,s\}$.
