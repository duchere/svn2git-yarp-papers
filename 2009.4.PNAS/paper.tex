% On the use of motor invariants for speech recognition
% Castellini, Metta, Fadiga, Tavella (?)
% submitted to PNAS

\documentclass{pnastwo}

% remove for submission
\usepackage[dvips]{graphicx}

\usepackage{amssymb,amsfonts,amsmath}

\contributor{Submitted to Proceedings of the National Academy of Sciences of the United States of America}
\url{www.pnas.org/cgi/doi/10.1073/pnas.0709640104}
\copyrightyear{2008}
\issuedate{Issue Date}
\volume{Volume}
\issuenumber{Issue Number}

\begin{document}

\title{On the use of motor invariants for speech recognition}

\author{
Claudio Castellini\affil{1}{LIRA-Lab, University of Genova, Italy},
Giorgio Metta\affil{2}{Italian Institute of Technology, Genova, Italy},
Luciano Fadiga\affil{3}{DSBTA, University of Ferrara, Italy}, \and
Michele Tavella\affil{4}{EPFL, Lausanne, Switzerland}
}

\contributor{Submitted to Proceedings of the National Academy of Sciences
of the United States of America}

\maketitle

\begin{article}

\begin{abstract}
Speech recognition is the task of enabling a machine to understand speech and write down what a human speaker is saying. Currently, the task is carried out under a number of simplifying assumptions, namely uniqueness of the speaker, uniformity of speech speed and pitch, and knowledge of an underlying vocabulary. We hereby propose the detection and use of motor invariants, that is recurring kinematic patterns of the phonetic apparatus during speech, to lift these assumptions. We demonstrate that the speech recognition capability of a standard machine learning method can be improved both qualitatively and quantitatively by the use of such motor invariants. This method paves the way to multi-speaker, pitch-, speed- and vocabulary-independent speech recognition.
\end{abstract}

\keywords{machine learning | speech recognition}

\section{Introduction}

\dropcap{I}ntroduction ....

\section{Conclusions}

Conclusions ....

%\begin{materials}
%materials and methods
%\end{materials}

\begin{acknowledgments}
This work is supported by the EU-funded project Contact (NEST 5010).
\end{acknowledgments}

%\begin{thebibliography}{}
%
%\end{thebibliography}

\end{article}

%%%%%%%%%%%%%%%%%%%%%%%%%%%%%%%%%%%%%%%%%%%%%%%%%%%%%%%%%%%%%%%%

%% Adding Figure and Table References
%% Be sure to add figures and tables after \end{article}
%% and before \end{document}

%% For figures, put the caption below the illustration.
%%
%% \begin{figure}
%% \caption{Almost Sharp Front}\label{afoto}
%% \end{figure}

%% For Tables, put caption above table
%%
%% Table caption should start with a capital letter, continue with lower case
%% and not have a period at the end
%% Using @{\vrule height ?? depth ?? width0pt} in the tabular preamble will
%% keep that much space between every line in the table.

%% \begin{table}
%% \caption{Repeat length of longer allele by age of onset class}
%% \begin{tabular}{@{\vrule height 10.5pt depth4pt  width0pt}lrcccc}
%% table text
%% \end{tabular}
%% \end{table}

%% For two column figures and tables, use the following:

%% \begin{figure*}
%% \caption{Almost Sharp Front}\label{afoto}
%% \end{figure*}

%% \begin{table*}
%% \caption{Repeat length of longer allele by age of onset class}
%% \begin{tabular}{ccc}
%% table text
%% \end{tabular}
%% \end{table*}

\end{document}
