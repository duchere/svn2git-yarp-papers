\section{Introduction/Motivation}
\label{sec:intro}

\dropcap{A}utomatic speech recognition (ASR) is the ability of a machine
to convert human speech, coded as an audio signal, into words.
Potential applications of ASR range from human-computer interfaces
to informatics for the disabled to data mining in large speech corpora.
%Despite decades of research, state-of-the-art ASR
%systems still need to be trained upon very large and heterogeneous data sets
%to account for speech variability.
%, or upon a single speaker's speech in controlled conditions.
%And nevertheless, human beings show an excellent ability
%to understand one another's speech, independently of the speaker, the
%accent, the pitch and speed, noise, etc.
While human beings show an excellent ability to understand one another's speech,
 independently of the speaker, the accent, the noise, etc., the robustness to speech varialbility
of state-of-the-art ASR systems is still an active research topic.

Recent neuroscientific evidence indicates that the brain motor areas responsible for producing labial
and dental phonemes are also involved in their perception; D'Ausilio et al. \cite{dausilio}
show that in a discrimination task of /b/,/p/,/d/ and /t/, trans-cranial magnetic
stimulation of the lips and tongue \emph{motor areas} creates a bias in favor
of the \emph{perception} of labials, and similarly, stimulation of the tongue
favors dentals. This suggests that motor information may be paramount for
understanding speech in humans.

Inspired by these findings, in this paper we investigate whether the knowledge of speech production in humans 
integrated into an automatic phoneme classifier improves the identification in the acoustic dimension 
of the specific behaviors of the /b/,/p/,/d/ and /t/ plosive consonants.
 
In ASR, approaches that combine explicit speech production knowledge and audio features
have been proposed (see \cite{king} for a review) as alternatives 
to the classic approach  in which the complex acoustic effects of speech production variability 
(e.g., due to speaking rate) and coarticulation (the phenomenon by which the phonetic realization of a phoneme is affected by its phonemic context) are directly and implicitly modeled in the acoustic domain.

%Although conclusions on the actual utility of speech production knowledge are somehow contradictory

By limiting our investigation on the utility of motor information to the much simpler (than ASR) task of four consonants classification\footnote{Note that a recognition task requires both segmentation of speech into phones and their classification.} we are able to relax working assumptions and avoid technical difficulties that have so far hampered a satisfactory integration of motor information into ASR systems. 

Additionally, in previous work it is not possible to properly identify which aspects of the recognition process benefit from motor information. For example, motor knowledge may improve the modeling (and so the identification) of coarticulation effects that are seen in the training data set, but not necessarily improve the recognition of phonemes in unseen contexts, i.e., it may not necessarily improve the generalization ability of the ASR system. On the other hand the experimental setup we have designed has the main goal of investigating whether motor information improves the generalization ability of a phoneme classifier.  

%Although the integration of speech production knowledge in an ASR system often brings some improvements, %it is commonly held that the potential of speech production knowledge is far from being exhaustively exploited.  

To this end, we have focused on the automatic version of
the problem tackled in D'Ausilio et al.'s work. For each consonant,
a corresponding typical phonetic motor invariant (MI) was
identified according to basic physiology of speech;
e.g., a fast, voiced opening (plosion) of the lips for /b/, and so on.
MIs were then used to semi-automatically segment the audio/motor data found in a
database of speech/motor trajectories recorded from $6$ subjects.

Subsequently, a simple regression method (namely, a feed-forward neural network) was employed
to build an Audio-Motor Map (AMM), which converts audio features of the isolated segment to
features of the related MI. On an abstract level, an AMM is a mathematical proxy of a mirror
structure \cite{umilta-01}, reconstructing the distal speaker's speech production act while
listening to the related piece of speech.

To test the approach, we have devised three experiments involving a 
classifier in the form of a Support Vector Machine \cite{BGV92}. We wanted to check whether
the use of MI-based features, either those recorded in the database (the ``real''
motor features) or the AMM-reconstructed ones (a more ecological scenario),
could improve the classifier's performance. Our results show that this is the case,
especially when the classifier is trained on incomplete data sets such as 
per-speaker (e.g., training on speakers $1,2,3$ and testing on $4$) and
per-coarticulation (e.g., training on /ba/, /be/, /bi/, /bo/ and testing on /bu/); or when noise is added,
in which case motor features significantly help classification, even when added to a
state-of-the-art set of audio features about $20$ times larger than that extracted
from the MIs.

\subsection{Related Work}

It is known since the Sixties \cite{liberman1} that the audio signal of speech
cannot be effectively segmented down to the level of the single phoneme,
especially as far as stop consonants such as bilabial plosives
are concerned; in particular, their representations in the audio domain are
radically different according to the phoneme which immediately follows.
It remains an open question then, how humans can
distinctly perceive a common phoneme, e.g.,/b/ in  /ba/ and /bi/, since they
apparently have access to the speaker's audio signal only.

The explanation put forward by the so-called motor theory of speech perception
(MTS, \cite{liberman2,galant}) is that, while perceiving sounds,
humans reconstruct \emph{phonetic gestures}, the physical acts that 
produce the phonemes, as they were trained since birth to associate
articulatory gestures to the sounds they heard. 

Even ignoring the motor theory of speech perception, the use of speech production knowledge is appealing in that the coupling of articulatory and audio streams allows for explicit models of the effects of speech production phenomena (e.g., coarticulation) on the acoustic domain. These effects cannot be precisely modeled (e.g., when the phoneme /a/ affects the phonetic realization of /b/ in /ba/?)  or modeled at all (e.g., what happens, in the acoustic domain, when I utter a /o/ with exaggeratedly open jaw?) when the phonemic stream is directly mapped onto the acoustic dimension as in the standard approach to ASR.  

Different solutions have been proposed to integrate speech production knowledge into an ASR system and different types of speech production information has been used, ranging from articulatory measurements (see \cite{zlokarnik,stephenson,wrench}, for example) to symbolic non-measured representations of articulatory gestures that "replicate" a (symbolic) phoneme into all its possible articulatory configurations\footnote{Articulatory configurations are configurations of the positions of the phonetic articulators.} (see \cite{richardson, livescu}, for example).
 
   
%One possible reason why ASR is so difficult is then that
%machines have in general no access to the motor representation of the
%audio signal they are supposed to understand. We hypothesize that motor 
%information might help ASR, especially when tests on different speakers and different
%coarticulations are performed: for example, when training on subject $A$ and
%testing on subject $B$, or when training on pseudo-words such as /ba/, /bi/,
%/be/ and then testing for the presence of /b/ in /bo/, /bu/ or even /br/.

Although some studies have shown increased word recognition accuracy when including speech production knowledge in ASR, it is commonly held that the potential of speech production knowledge is far from being exhaustively exploited. Limits of current approaches include: the use of the phoneme as basic unit (as opposed to articulatory configuration, for example) which appears to be too "coarse", especially in the context of spontaneous spoken speech
%where coarticulation effects are more frequent and marked
; and  the lack of a mechanism that accounts for the different importance of articulators in the realization of a given phoneme (e.g., in the generation of phoneme /b/ lips are critical, i.e., important, while tongue is non-critical).

The traditional approach in which the speech signal is segmented into phones, often referred to as "beads on a string" approach, poses problems to an accurate modeling of spontaneous speech where coarticulation phenomena such as phone deletion or assimilation (where a phone assimilates some articulatory gestures of the preceding/following phone) that "distort" the acoustic appearance of phonemes, are frequent and not always predictable and call for finer-grained basic units (see \cite{ostendorf})). To partly make-up for such limitation we propose an alternative approach where, instead of segmenting the audio stream looking at audio features only and then observing the articulatory gestures within the identified phones, we give priority to the (expectedly) more distinctive and stable motor information in that speech is segmented by searching for phone-specific patterns of the (critical) articulatory gestures.

%Concerning the necessity of a phoneme dependent distinction between critical and non-critical articulators we do not 
%Traditionally (e.g., \cite{bourl,salvi}), the audio speech signal is segmented with a
%fixed-length Hamming window, usually 20ms. long. The resulting sequence
%is then analysed in the frequency or cepstral domain and the
%resulting coefficients are used as features for a classification system.
%One negative aspect of this approach is that it
%neglects the qualitative overall characteristics of the
%phoneme being uttered: depending on the speed of the speech, a consonant
%can have different lengths and, by using the above approach, global
%information about it is lost (see \cite{ostendorf}, where this approach is
%dubbed ``beads-on-a-string''). Nevertheless, as far as we know, there is
%so far no widely accepted alternative method for speech segmentation,
%if the audio signal is the only one available. One attempt, but not based
%upon articulatory data atl all, appears in \cite{bourlard}.


During recognition, articulatory gestures have to be recovered 
from audio information as audio is the only signal available.
Reconstruction of articulatory features has been attempted since a long
time, but in most cases it is not derived from articulatory \emph{data}
gathered from human subjects. One pioneering case is that of Papcun
et al. \cite{papcun} where the AMM is carried out by a Multilayer Perceptron.
Our procedure for building the AMM is deeply inspired by this work.
%By using a Multilayer Perceptron we implicitly assume that all articulators have the same importance and that the AMM is a %one-to-one mapping.   
%Papcun et al. \cite{papcun} observed that non-critical articulators have higher variance (in terms of position) than critical %articulators. 
The Multilayer Perceptron attempts the best recovery of all articulatory gestures while more emphasis to the recovery of the gestures of the critical articulators should be given to the detriment of the non-critical articulators, which have higher variance (in terms of position, see \cite{papcun,rose}). Although we do not address this issue, the simple fact that we only consider two articulators alleviates a problem that would be otherwise far more relevant if all articulators were taken into account\footnote{The higher variance of the non-critical articulators is the main cause that makes AMM a one-to-many mapping: different articulatory configurations result in the same acoustic realization. Solutions to properly address this "ill-posed" nature of the AMM have been proposed by Richmond et al. \cite{richmond} and Toda et al. \cite{toda}. }.

%and subsequently Korin Richmond's work
%\cite{richmond2002,richmond2007} who have been able to reconstruct point-by-point
%the trajectories of articulators from the audio signal to a remarkably low
%error rate. The procedure for building the AMM is deeply inspired by their
%work.

Interestingly, the idea of using information about the mechanisms involved in the production of a human action to improve its classification/recognition (in a domain different from the production domain) has not only been applied in the context of speech recognition. For example Metta et al. \cite{metta-06} and Hinton \cite{hinton-2006} 
have shown that articulatory data can improve classification accuracy in automated hand action classification.

%TODO  
% Transferring the method to speech perception seems
% like a natural choice.
