\subsection{Patients}

Three hand amputees, patients of the INAIL Centro Protesi in Vigorso
di Budrio, Bologna, Italy, joined the experiment.

The first subject is male, aged 63, trans-radial one-third proximal,
amputated in 1963; he is a pioneer of myoelectric prostheses, having
started using them in the Sixties. The second subject is male, aged
56, trans-radial one-third distal, amputated in 1972; he also started
using myoelectric prostheses very early, actually in 1974. The third
subject is male again, aged 25, trans-carpal, amputated in 2007; he
was in the process of learning how to use a standard myoelectric
prosthesis at the time of writing.

A set of three patients is obviously not sufficient to gather
statistics about the applicability of our method, but we are lucky
enough that they present a rather wide variety of operations and
conditions. In particular, subject $1$ has about $9$cm left of his
forearm, subject $2$ has some $20$cm, and subject $3$, a trans-carpal
amputee, has the whole forearm plus some of the carpus (Figure
\ref{fig:stumps} shows the subjects' stumps). Moreover, subjects $1$
and $2$ were amputated a very long time ago, although they have been
using myodevices since then, whereas subject $3$ is freshly
operated. Lastly, subject $3$ is rather young as opposed to the other
subjects.

\begin{figure*}[!ht] \centering
  \begin{tabular}{ccc}
    \includegraphics[width=0.3\textwidth]{figs/stump_1} &
    \includegraphics[width=0.3\textwidth]{figs/stump_2} &
    \includegraphics[width=0.3\textwidth]{figs/stump_3} \\
    subject $1$ & subject $2$ & subject $3$ \\
  \end{tabular}
  \caption{the subjects' stumps. Subject $1$ has a trans-radial
    one-third proximal amputation, with a stump about $9$cm long;
    subject $2$ is trans-radial one-third distal, stump about $20$cm
    long; and subject $3$ is trans-carpal, retaining the full
    forearm.}
  \label{fig:stumps}
\end{figure*}

\subsection{Setup}

We placed on each subject's stump $5$ surface EMG electrodes without
searching for the best anatomical position, but rather in a standard
way for all, around the stump, near the elbow (for subject $1$ this
was actually the only possible choice!) and at uniform angles from one
another, in such a way to "wrap" the stump. See the ``Discussion and
Conclusions'' Section for more about this issue.

The electrodes we employed are standard commercial surface EMG
devices, namely OttoBock Myobock models \cite{ottobock}, two of the
13C7=50 type and three of the 13E125=50 type. Myobock electrodes enjoy
an excellent noise rejection ratio and pre-amplify the signal --- an
amplification gauge can be set for each electrode, and here it was set
at a mid-range value for all electrodes. Moreover, they perform a
run-time root-mean square evaluation of the signal; this results in an
exceptionally good output, which is already highly correlated with the
force exerted by the muscle(s) whose activity the electrode is
gathering.

Each subject was also given a FUTEK LMD500 Hand Gripper force sensor
\cite{futek} in order to detect the required force during the
experiment. Data were gathered via a standard digital acquisition
card, namely a National Instruments NI-DAQ 6122 USB card, connected to
the EMG electrodes and force sensor on one side, and to an entry-level
laptop on the other side. We employed National Instruments's
SignalExpress application to sample the (synchronised) data at a
sampling rate of $100$Hz.

\subsection{Experiment Design}

The patients were induced to imagine performing with their missing
hand $5$ different postures / grips: no action, pointing index, pinch
grip, tripodal grip, power grasp; subject $3$ was also asked to
stretch his hand --- a posture which most amputees deem very useful
for, e.g., slipping the hand in a pocket. The postures / grips were
performed with free force and speed by the subjects, while we would
record the EMG and force sensor activity.

Since the beginning we decided to employ a \emph{supervised learning}
strategy to build our models, as has been done in literature so
far. To this end, three ways of training the system
(\emph{modalities}) were designed and employed:

\begin{enumerate}

  \item \emph{teacher imitation.} A healthy subject (the teacher)
    would place his arm besides the patient's stump and ask him
    to imagine replicating the teacher's postures and grips. The
    subject was asked to imagine gripping with his maximum strength,
    while the teacher would grip the force sensor in order to mark the
    postures / grips.

  \item \emph{bilateral action.} The patient was asked to grip the
    force sensor with his healthy hand while imagining doing the same
    things with his missing hand.

  \item \emph{mirror-box.} Same as modality $2$, but a simple, plain
    mirror (in the case of subject $1$), or a \emph{mirror-box} (for
    subjects $2$ and $3$, see \cite{mirror-box}), was placed
    in-between the patient's arms, in order to increase the visual
    feedback.

\end{enumerate}

The idea behind the mirror-box modality is inspired by Ramachandran's
experiments on amputees of the mid-Nineties \cite{ramachandran}, where
it was noted that the illusion of seeing one's hand moving would
reinforce the visual feedback loop and ease the ghost limb pain in
monolateral hand amputees. We figured out that such a device could
actually reinforce the patient's ability to produce different
activation patterns.

Figure \ref{fig:modalities} shows various subjects performing the
required actions in the three modalities.

\begin{figure*}[!ht] \centering
  \begin{tabular}{ccc}
    \includegraphics[width=0.3\textwidth]{figs/mod1} &
    \includegraphics[width=0.3\textwidth]{figs/mod2} &
    \includegraphics[width=0.3\textwidth]{figs/mod3} \\
    teacher imitation & bilateral action & mirror-box \\
  \end{tabular}
  \caption{the three training modalities. (left to right) Subject $1$
    imitating a pinch grip; subject $3$ bilaterally performing a pinch
    grip; subject $2$ assuming the pointing index posture while
    looking in the mirror-box.}
  \label{fig:modalities}
\end{figure*}

The patients were left free, to a large extent, to exert the postures
/ grips with the amount of force and the speed they liked; in some
phases of the experiment, the teacher would command them to grip
faster or slower, or with a certain desired force. The result is that
the patients applied a wide range of gripping speeds and forces, which
helped test whether our approach would work equally well with signals
gifted with diverse frequency and amplitude components. Figure
\ref{fig:ex_signals} shows some sample force and EMG signals. On
average, each modality lasted something more than $5$ minutes and no
subjects reported fatigue or pain. At the aforementioned sampling rate
of $100$Hz, we gathered a total of about $270000$ samples.

\begin{figure*}[!ht] \centering
  \begin{tabular}{ccc}
    \includegraphics[width=0.3\textwidth]{figs/example_signal1} &
    \includegraphics[width=0.3\textwidth]{figs/example_signal2} &
    \includegraphics[width=0.3\textwidth]{figs/example_signal3} \\
  \end{tabular}
  \caption{three examples of force (black, continuous line) and EMG
    signals (coloured and dotted lines) during subjects'
    activity. (left panel) Modality $1$: around the $5000$th sample a
    switch from pointing index to  power grasp appears --- notice the
    related change in magnitude of the EMG. (center
    and right panels) Modality $3$, slow and fast power grasping ---
    notice two EMG electrodes spoiled by a non-null baseline and a
    slow drifting component, which was later on determined to be due
    to sweat.}
  \label{fig:ex_signals}
\end{figure*}

\subsection{Methods}

As already pointed out in literature, Support Vector Machines (SVM)
\cite{BGV92} are a good machine learning method to solve this problem,
so we employed them. For an explanation of how SVMs work in the
context of EMG signals, please refer to, among others,
\cite{2008.ICRA,2008.BioCyb}, whereas, for a comprehensive tutorial on
SVMs in the more general framework of classification and regression,
refer to \cite{Burges98,SmolaTut2004}. SVMs are a statistical learning
method able to build an approximated map between an input space and a
label (classification) or a real value (regression). Classification is
here used to classify the type of grasp according to the EMG signal,
whereas regression is used to understand how much force the subject is
exerting, independently from the grasp type.

The input space is chosen to be $\RR^5$, one coordinate for each EMG
electrode; the labels are five integer numbers, one for each grasp
type (six for subject $3$, who also performed the hand stretching
posture); and the real value is exactly the force value read off the
force sensor. Notice that we work in real-time, that is, our machines
associate a grasp type and a force value to an EMG value at each
instant of time. This approach enables us to detect a grasp type
almost at the onset of the grasping movement.

\subsection{Data pre-processing and preliminary analysis}

The data related to each subject and modality were chronologically
juxtaposed exactly in the order the experiments were performed; then a
label was attached to each sample, according to the type of grasp
required from the subject. Samples associated with a low force value
were given the label $0$, since they denote no activity of the
muscles, that is, a resting condition. Subject $1$ actually needed the
no-action data set to be replicated for each modality, since we did
not record its baseline each time --- a mistake which was corrected
with the second and third subject. Moreover, we could not record the
"pointing index" activity for this subject in the second modality,
which is therefore null.

\begin{figure*}[!ht] \centering
  \begin{tabular}{ccc}
    \includegraphics[width=0.3\textwidth]{figs/spectrum_force} &
    \includegraphics[width=0.3\textwidth]{figs/spectrum_electrode_1} &
    \includegraphics[width=0.3\textwidth]{figs/spectrum_electrode_3} \\
    force & electrode $1$ & electrode $3$ \\
  \end{tabular}
  \caption{frequency analysis of the force signal and two
    typical electrodes.}
  \label{fig:spectra}
\end{figure*}

Spectral analysis of the EMG signal as read from the electrodes, in
agreement with the literature, shows that its relevant bandwidth lies
below $10$-$12$Hz (see Figure \ref{fig:spectra}), so we could safely
subsample the signals at $25$Hz, that is, considering one sample in
four of the original data stream. This made the data set to be dealt
with much smaller and computationally tractable. Subsequently, we
applied a II order low-pass filter with cutoff frequency at $5$Hz in
order to remove all possible high-frequency noise. This has proved to
be a very effective way of getting a good signal in early experiments
(see \cite{2008.Neurorob}).

Principal Component Analysis (PCA) reveals that the 5 signals can be
linearly reduced to two losing, on average, only $7.7\% \pm 4.4\%$ of
the signal variance; therefore, we can visualise the samples, tagging
them according to the labels (and therefore according to the action)
and visually detecting how well the subjects can produce different EMG
patterns when they are asked to simulate different grasping
actions. Figure \ref{fig:PCA} shows the results, according to each
subject and modality.

\begin{figure*}[!ht] \centering
  \begin{tabular}{ccc}
    \includegraphics[width=0.3\textwidth]{figs/data11} &
    \includegraphics[width=0.3\textwidth]{figs/data12} &
    \includegraphics[width=0.3\textwidth]{figs/data13} \\
    \includegraphics[width=0.3\textwidth]{figs/data21} &
    \includegraphics[width=0.3\textwidth]{figs/data22} &
    \includegraphics[width=0.3\textwidth]{figs/data23} \\
    \includegraphics[width=0.3\textwidth]{figs/data31} &
    \includegraphics[width=0.3\textwidth]{figs/data32} &
    \includegraphics[width=0.3\textwidth]{figs/data33} \\
  \end{tabular}
  \caption{ PCA analysis of the subjects' data. (top to bottom)
    Subject $1$, $2$ and $3$; (left to right) modality $1$, $2$ and
    $3$. Notice that subject $1$, modality $2$ has no ``pointing
    index'' data, and that subject $3$ has the ``hand stretch'' data.}
  \label{fig:PCA}
\end{figure*}

As is apparent from the Figure, all subjects can produce remarkably
well separated and distinct signals, according to the elicited type of
grasp. In particular, notice how two very similar grasp types, i.e.,
pinch grip (thumb and index come together as to precisely grip, e.g.,
a pen) and tripodal grip (the same, but done with the middle finger,
too) appear well separated on almost each graph --- look at the black
and pink coloured samples.

Notice, as well, that the graphs have not all the same scale and that
they appear incongruent with one another, but this is due to
having positioned the electrodes irrespective of their order number on
the stumps of the patients. This is anyway ininfluent, since SVMs
are exaclty supposed to automatically find patterns and regularities
among the samples for each subject.

One last point is that PCA being so effective in reducing the
dimensionality of our data to two does not necessarily imply that we
could use two electrodes only to obtain the same results; this depends
on the PCA coefficients, which show consistently the same magnitude
(except in the case of subject $3$, where a heavy drift was observed
on two electrodes, very likely due to sweating --- this is a well-known
problem of EMG-controlled prosthetics). This means that each electrode
is required to give a uniformly-weighted contribution to the
PCA-transformed $2$-dimensional samples. Of course reducing the
required number of electrodes is a requirement, since this would make
the prosthesis cheaper. We are invesitigating the issue.
