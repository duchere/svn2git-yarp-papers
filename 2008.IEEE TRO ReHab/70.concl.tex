To sum up, this paper can be seen as the conclusion of a theoretical
piece of research focussed upon advanced myoelectric control of
mechanical hands, with a special emphasis on prostheses. In this
research we have first demonstrated, both theoretically and
practically, that EMG activity of a healthy subject in highly
controlled conditions can be used to position- and force-control a
non-prosthetic mechanical hand \cite{2008.ICRA,2008.BioCyb}. We have
then extended the work towards the multi-subect and non-controlled
case, showing that the same good results appear (see the poster
\cite{2008.GNB}; a detailed report is about to be submitted to a
journal and is confidentially available from the corrsponding
author). Here (see also the poster \cite{2008.Neurorob}) we extend, at
last, the method to amputees, showing that significantly diverse
patients can train our system just as easily, and that good chances
are that the approach can actually be miniaturised and placed on-board
a commercially available prosthesis.
