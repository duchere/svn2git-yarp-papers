\section{Discussion}
\label{sec:disc}

\subsection{Quality of the results}

In \cite{2008.BioCyb} it was shown that a dexterous mechanical hand such as
the DLR-II could be feed-forward force and position controlled in
real time by a healthy subject using surface EMG. Although this opens
interesting possibilities, e.g., in teleoperation, virtual reality
and robotic surgery, it is in the field of hand prosthetics that it has the
most immediate applications. The question remained open then, whether comparable
results could be \emph{in principle} achieved by amputees:

\begin{enumerate}

  \item is there enough fine muscular activity left in a stump?
  \item how do the age of operation, the type of
	amputation and the use of myoelectric prosthesis affect the results?
  \item does the essential uniqueness of each single stump challenge the
  	general applicability of the method?

\end{enumerate}

The results of our experiment show that, at least in the case of our subjects, the
answer to the first question is positive. In the aforementioned paper, a classification
accuracy of about $90\%$ and a normalised RMSE of about $10\%$ were sufficient to
correctly control the DLR-II hand. Comparing these figures with those presented in the
"Results" Section, we conclude that our subjects would actually be able to
do the same, even possibly a little better. 

It must be taken into account that, with respect to \cite{2008.BioCyb},
the duration of each cycle is short (a few minutes versus one and a half hour)
and that the electrodes were left in the same position for each subject. Nevertheless,
these problems have been tackled and solved in that paper, and the very same
techniques proposed therein can be employed in this case, in a practical setting.
Moreover, a number of simple strategies can be used to further improve the
performance: since most classification errors happen at the onset
of grasps, for example, the regression machine can be used to start the classification only
when the predicted force is above a threshold, when the chance of mistaking the
posture is close to zero. Buffering, i.e., considering a few samples before
switching category, would also help. The regression output
signal can further be low-pass filtered to accommodate for the high-frequency oscillation
noticed in Figure \ref{fig:examples}. All this is, of course, subject to experiments
once a practical implementation is realised.

As far as the second question is concerned, of course more subjects and more cycles per
subject are needed for a full answer; but if we restrict to our experiment, we see that
our subjects obtain good results regardless of the age of operation (decades for Subjects
$1$ and $2$, one year for Subject $3$) and type of amputation (Subject $1$ has a very
short stump, Subject $2$ has a longer one and Subject $3$ retains his full forearm).
Lastly, all our Subjects have been using myoelectric prostheses, (Subject $3$
was actually being trained at the time of the experiment); this probably gives them
a little advantage, but not that much, since commercial myoelectric prostheses, so far,
mainly employ the wrist flexor and extensor; so we would expect a stronger fitness of
these muscles, but not necessarily of all the others in the forearm. A similar consideration
appears in \cite{sebelius}, actually the only work we can compare with.

Question $3$ seems to have a negative answer: the same uniform electrode positioning strategy
worked fine for all Subjects, notwithstanding their diversity.
If confirmed, this finding would definitely simplify the clinical training procedures as well
as the manufacturing of prosthetic sockets, to the benefit of the patients.

\subsection{Practical issues}

Regarding the practical issue of manufacturing such a system,
one wonders how easy it will be to miniaturise the optimal
models obtained for each subject, both for classification and regression;
we have conducted a preliminary study which indicates that it should be possible
with a reasonable effort using standard micro-controllers. To this end, the
above mentioned sparsity of the solutions built by SVMs (see Subsection
\ref{subsec:method}) is an advantage. Anyway, the low percentages of SVs in the
best models found (Figure \ref{fig:SVs}) indicate that the problem, in general,
is easy from the point of view of machine learning: in case SVMs proved too
hard to implement in practice, even a less sophisticated method such as, e.g.,
nearest neighbours, could obtain acceptable performance values.

This consideration suggests that the rehabilitation community should probably stop
looking for more and more sophisticated features and machine learning methods,
and concentrate on other issues such as the prosthesis itself (dexterity, impedance
control, humanoid-ness) and sensorial feedback. As well, knowing the state-of-the-art of
hand prostheses, we claim that invasive techniques such as, e.g., electroneurography
\cite{cipriani} will not be necessary for a long time to come, at least for
trans-radial amputees. We have \emph{just begun} to exploit plain, old surface EMG.

\subsection{EMG patterns}

The form of feed-forward control enforced by this method is, we claim, ``more
natural'' than the standard single-muscle activation technique, since here
we detect real-time \emph{phantom movements} and \emph{forces}.
No claim is made about the physiological resemblance of the EMG patterns found
to muscular patterns elicited for the same grasps in a healthy arm. As long as
each subject can generate a distinct pattern for each grasp posture, this pattern
can be detected and the appropriate command can be sent to the prosthesis.
The net effect is that the amputee could obtain, e.g., a $20$N power grasp just
by doing it with the phantom limb --- what he used to do naturally before the trauma.

As a matter of fact, we find amazing that long-term amputees, one of which has
had almost \emph{no forearm} for $45$ years, can still produce such distinct signals
for anatomically similar phantom movements such as, e.g., a pinch grip and a
tripodal grip. (Indeed all subjects are users of conventional myoelectric
prostheses, but in that case movements are required which can be at
best defined gross, if compared to the precision obtained in this experiment.)
It is common understanding that, due to cortical plasticity, motor/sensory brain
areas devoted to amputated limbs are in the long run reassigned to other
functions. The wonderful accuracy we have seen seems to go against this belief,
and is in agreement with recent neurological studies on above/below elbow
amputees \cite{sirigu1,sirigu2}. Trans-cranial magnetic stimulation and EMG detection
on such subjects has revealed that stable, precise EMG patterns appear in subjects with long-term
amputations when they are required to move their phantom limbs. \cite{pascual} is
an earlier study of the same kind, performed on Braille-proficient blind.
In \cite{sirigu2} such a pattern is even observed in an \emph{above-elbow} amputee, asked to perform
phantom thumb flexion, due to residual muscle synergies. This lets us hope that
our technique could be applied, although probably not to the same extent, to more
severe amputees such as trans-humerals.

\subsection{On the training modalities}

The three modalities were devised in order to provide a simple and effective
way to obtain labels and force values for each EMG sample. For healthy
subjects, this is usually done using a force sensor gripped with the hand
\cite{2008.BioCyb} or a dataglove \cite{sebelius}; obviously these ideas
will not work with an amputee.
The competing use of three different modalities has proven effective in finding
the best way for each single subject to train the system; as it turns out,
almost consistently for classification and regression (consider Figure
\ref{fig:results} again), performances \emph{differ} significantly as
the modality changes for each subject. This is an important issue, since
a better and quicker training could be obtained if we could
know in advance what modality suits best an amputee. We have no general
answer to this question so far, but some considerations are at hand.

The teacher imitation and bilateral action
modalities were essentially suggested by common sense, while the mirror-box
is inspired by Ramachandran's experiments on amputees of the
mid-Nineties \cite{ramachandran96}, where it was noted that the illusion of
seeing one's hand moving would reinforce the visual feedback loop and decrease
the phantom limb pain in monolateral hand amputees. We figured out that such
a device could actually improve the subject's ability to control EMG
activation patterns at least with respect to the bilateral action modality;
but it turns out that there is little or no difference between their performances,
and indeed mirror-box is \emph{worse} than bilateral action for Subject $1$ in
classification. Probably, a certain amount of training with the mirror is required,
before a subject becomes proficient with the device.

Let us then restrict our attention to imitation and bilateral action.
Imitation is predictably the less accurate modality, since it is the
teacher who presses the force sensor in order to simulate the force applied by the
subject; as well, the only way the subject has to know when to start grasping
is visual feedback, which indeed introduces a delay. (Anyway let us notice that
teacher imitation is the only possibility if we want to extend this method to
\emph{bilateral} hand amputees, whose quality of life can really be low after
the amputation.)
Both regression and classification performances are worse in this modality for Subjects
$2$ and $3$; but it is best for Subject $1$, and it degrades
as we go to Subject $2$ and $3$. An interesting observation is that Subject $1$
has the shortest stump left, Subject $2$ has more, and Subject $3$ has it all;
this qualifies $1$ as having a high degree of de-afferentiation (sensorial
feedback deprivation). It has been noted \cite{lajoie02,miall07}
that de-afferented patients perform \emph{better than normal subjects}
in tasks involving a conflict between what they perceive from their
moving limb and what they see; and this might explain why Subject $1$
obtains better results then the others when imitating someone rather
than figuring out his own movements. This is, of course, just a
hypothesis so far.
