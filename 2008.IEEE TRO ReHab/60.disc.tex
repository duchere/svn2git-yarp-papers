The results hereby presented clearly show that the plain, old EMG
signal can be used to drive a mechanical hand / prosthetic hand in a
radically new way: finer, force-controlled, more dexterous. Our
results clearly show that a machine learning approach such as Support
Vector Machines will effectively detect well-separated grasping
patterns in real time, as required by an amputee; at the same time,
the system will be able to detect how much force is involved in the
grasp. If the prosthesis has a sufficient number of DOFs, that is, it
can be position-controlled to mimic the required grasps, and if it can
be force-controlled, then our system will be able to control it in a
totally natural way, that is, according to what the patient wants it
to do.

Notice that we are making here \emph{no statement at all} about the
physical plausibility of the patterns we deal with, with respect to
the muscular patterns elicited for the same grasps in a healthy
arm. Actually, we have no hint that what the patient's stump muscles
do when the patient is asked to imagine, e.g., a pinch grip, is
related with what a healthy arm would do in the same situation. Since
each and every amputation is different from one another, and therefore
each stump has different muscular conditions, what we can hope for is
that the system works fine for a reasonable range of amputees and
stumps. Although we have only $3$ patients here, their diversity as
far as age, age of operation and type of amputation lets us hope for
the best.

In the end, as long as for each patient, each grasp type or posture
corresponds to a different pattern, then we can detect it and send the
appropriate command to the prosthesis. The patient will then be able
to elicit from the prosthesis, say, a $5$-Newton pinch grip, or a
$30$-Newton power grasp, just by ``desiring'' it. This is what we mean
by a better quality of life for amputees and a shorter training
time. The keyword \emph{adaptive prosthetics} is meant, here, in the
machine-learning sense, that is: the prosthesis is trained upon the
patient's data and therefore adapts to her/him.

We actually find it surprising that so much fine muscular activity
can be elicited from elderly patients, such as our Subjects $1$ and
$2$, who, by the way, have been operated \emph{decades}
ago. Notwithstanding this, they are still perfectly able to produce,
e.g., two very different patterns for similar grips such as the index
/ thumb pinch grip and the tripodal grip. This lets us hope that
possibly even a simpler method than SVMs could be used without
degrading the performance too much, in case the miniaturisation of
this method proves too hard. But actually, the analysis performed in
Section \ref{sec:impl} is, again, positive.

To sum up, this paper can be seen as the conclusion of a theoretical
piece of research focussed upon myoelectric control of mechanical
hands, with a special emphasis on prostheses, of course. In this
research we have first demonstrated, obth theoretically and
practically, that EMG activity of a healthy subject in highly
controlled conditions can be used to position- and force-control a
non-prosthetic mechanical hand \cite{2008.ICRA,2008.BioCyb}. We have
then extended the work towards the multi-subect and non-controlled
case, showing that the same good results appear (see the poster
\cite{2008.GNB}; a detailed report is about to be submitted to a
journal and is confidentially available from the corrsponding
author). Here (see also the poster \cite{2008.Neurorob}) we extend, at
last, the method to amputees, showing that significantly diverse
patients can drive our system just as easily.
