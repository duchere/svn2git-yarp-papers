The state-of-the-art feed-forward control of active hand prostheses
is rather poor. Even dexterous, multi-fingered commercial prostheses are
controlled via surface electromyography (EMG) in a way that enforces
a few fixed grasping postures, or a very basic estimate of force.
Control is not natural, meaning that the amputee must learn to associate,
e.g., wrist flexion and hand closing. Nevertheless, recent literature
indicates that much more information can be gathered from plain, old
surface EMG. To check this issue we have performed an experiment in
which three amputees train a Support Vector Machine (SVM) using five
commercially available EMG electrodes while asked to perform various
grasping postures and forces with their phantom limbs. In agreement
with recent neurological studies on cortical plasticity, we show that
amputees operated decades ago can still produce distinct and stable
signals for each posture and force. The SVM classifies the posture up
to a precision of $95\%$ and approximates the force with an error of
as little as $7\%$ of the signal range, sample-by-sample at $25$Hz.
These values are in line with results previously obtained by healthy
subjects while feed-forward controlling a dexterous mechanical hand.
We then conclude that our subjects could finely feed-forward control
a dexterous prosthesis both in force and position, using standard EMG
in a natural way, that is, using the phantom limb.
