The state-of-the-art in active (myoelectric) hand prosthetics is
rather poor if considered from the point of view of control by the
patient. Even the most advanced commercial prostheses, gifetd with
several degrees of freedom and able to mimic a number of human-like
grips $(a)$ provide no force control and $(b)$ enforce a non-natural
form of control, so the patient has to learn how to drive them from
scratch. For instance, opening of the hand is generally actuated using
the remnant of the wrist flexor muscle.

In this paper we describe and discuss in detail the positive outcome
of an experiment in which three amputees were asked to train a machine
learning system in order to drive an advanced active hand prosthesis,
both $(a)$ controlling its force and $(b)$ naturally commanding the
required grip. We then expand briefly and discuss how this system
could be practically realised, miniaturised and embedded in a
prosthesis.

All in all we propose a framework for \emph{adaptive hand
prosthetics}, that is a scenario in which, rather than having to learn
how to use a prosthesis, a patient will train it upon its own data,
and enter a positive feedback loop of reciprocal learning, which will
lead to shorter training times and a better quality of life.
