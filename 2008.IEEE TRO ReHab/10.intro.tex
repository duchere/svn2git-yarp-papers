One of the most widely used ways of feed-forward controlling Active Hand
Prostheses (AHPs) is forearm surface electromyography
(EMG), a technique by which muscle activation potentials are gathered
from the patient's stump skin, and then used to drive the
prosthesis\footnote{Sensorial feedback is even less developed at the
time of writing; thermic and vibrotactile feedback has been experimented
with little success \cite{zecca}, so that visual feedback is so far the
only effective way to close the control loop.}. EMG \cite{deluca} is
relatively cheap, non-invasive and easy to use, and it has been around
for some 40 years. Still, as it stands today, it is used in a way that
is highly innatural for the patient, allows for a very limited number
of grasp types, and enforces no control of the \emph{force} involved
in a grasp. 

Force control, or at least a way to feed-forward modulate the force
involved in a grasp, is paramount for daily-life activities of amputees,
for example to hold an egg without breaking it and/or to hold a hammer
without letting it slip. Moreover, due to the limitations in the signal
processing techniques, so far large muscles, such as the wrist flexor
and extensor, are used to drive the prosthesis; in the typical case the
patient must learn to associate wrist flexing with hand opening, and this
implies long training times. Lastly, such a rough schema cannot control
more than a few standard grasping postures. This, notwithstanding recent
advances in hand prostheses mechatronics, such as Touch Bionics's i-LIMB
\cite{ilimb}, a real leap forward as far as dexterity is concerned
(even more dexterous non-prosthetic mechanical hands have already appeared,
such as the DLR II hand \cite{Hua2006}).

In this paper we show that plain, old surface EMG can be used to reach
a much better form of feed-forward control of mechanical hands: similar
grasp postures can be discerned easily, and the required force can be understood
almost perfectly. The interesting point is that no specialised hardware
is needed (five commercially available EMG electrodes suffice) and no
careful positioning of the electrodes on the patients' stumps is required.
The innovative bit is represented by the use of a machine learning technique,
which means that the situation can be potentially radically improved without
the need of surgery and complex electronics.

More in detail, three hand amputees were required to imagine performing
different grasps with different levels of force with their missing limb;
the obtained data have then been analysed using a Support Vector Machine (SVM)
both for classification of the grasp and for regression upon the desired force.
The machine is able to distinguish the required grasp in real-time with a
high precision, and to understand the required force with an equally excellent
performance. SVMs, as well as other machine learning techniques, have already been employed
on the EMG signal to classify grasp postures and force required by healthy
subjects (see, e.g., \cite{smagt,dunlop,2008.BioCyb}), but as far as we know
this is the first time such an analysis is carried out on patients with some
systematicity, both for grasp and force control.

The task seems hard at first sight, since each patient's anatomy and residual
muscle activity is different; moreover, one would expect people operated
decades ago not to be able to imagine fine movements and postures of their
missing limbs. Instead, rather surprisingly, our data analysis reveals that
similar grasps such as a pinch and a tripodal grip can still be perfectly
distinguished; that the requried force is delicately modulated and it can be
estimated well; and that no anatomical / functional inspection of the stump
is required.

This result hints at a scenario in which the prosthesis adapts to the patient
as well as the other way round, entering a positive feedback loop
of reciprocal learning, leading to shorter training times and a better
quality of life for the patients.

The paper is structured as follows: Section \ref{sec:m&ms} describes
the experiment; Section \ref{sec:pre} describes the data
pre-processing and gives an idea of how the data look like; Section
\ref{sec:exp} describes the experimental results; lastly, Section
\ref{sec:disc} contains a discussion of the results and the conclusions.
