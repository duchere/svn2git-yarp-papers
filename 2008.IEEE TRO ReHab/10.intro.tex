\section{Introduction}
\label{sec:intro}

As of today, the most common way of feed-forward controlling an active
hand prosthesis is via forearm surface electromyography (EMG),
a technique by which motor unit activation potentials are read from the
amputee's stump skin and then used to command the prosthesis \cite{deluca97}.
The clinical success of EMG is motivated by its non-invasiveness and
relatively low cost \cite{englehart06}.
Still, even though multifingered hand prostheses have appeared on the market
(for example Touch Bionics's i-LIMB, see \url{www.touchbionics.com}),
it seems that EMG-based control is not keeping the pace, being limited to a
few hand postures or a simple proportional estimate of force.
Moreover, this control is quite non-natural, meaning that the amputee must
learn to associate muscle remnants actions to unrelated postures of the
prosthesis, e.g., phantom wrist flexion and hand closing.

It would be desirable, for a dexterous prosthesis, to let the amputee command a
grasp posture and force just by performing the corresponding action with the
phantom limb, resulting in a movement or force elicited by "desiring"
it. Also, a way to finely modulate the force involved in a grasp is paramount
in daily life activities, for example to hold an egg without breaking it or to
grasp a hammer without letting it slip. Lastly, the system should be able to
work in real-time and to adapt continually to the changing nature of the EMG
signal.

Actually, recent scientific literature
indicates that plain, old EMG signals can be used in a better way by improving
the signal processing method, essentially by switching to machine learning.
In \cite{2008.BioCyb} it is shown that a healthy subject can finely feed-forward
control a dexterous, non-prosthetic mechanical hand, namely the DLR-II
\cite{ButFisGre2004}), over a long time using $10$ EMG electrodes; willing to
explore this issue further, in particular as far as amputees are concerned, we
have devised a new experiment.

Three below-elbow traumatic amputees, two of which
were operated decades ago, were asked to perform
grasp postures with their phantom limb at various speeds and forces, while their
EMG activity was recorded using $5$ commercially available electrodes, positioned
without clinical help, and the desired force was
estimated using a force sensor in three different ways.
The EMG and force signals were used to train a standard machine learning system
(a Support Vector Machine) in order to check how well phantom limb postures
could be discriminated and the required force approximated. This paper is
a report of the experiment, whose outcome is positive.

We now quickly revise related work highlighting the novel contributions of this
work, then we describe the experiment and results and discuss them.

\subsection{Related work}

\subsubsection{The EMG signal}

Myoelectric control, i.e., feed-forward control of prostheses using surface EMG,
has been the most popular form of (externally powered) control by amputees since
the Sixties, mostly due to its relatively low cost and non-invasiveness. It has
provided a simple yet effective way to control single degree-of-freedom (DOF) hand
prostheses such as OttoBock's SensorHand for decades, also allowing for a simple
force estimation, the so-called proportional control. Body-powererd prostheses,
actuated via a harness connected, e.g., to the shoulder, and cosmetic ones, i.e.,
passive prostheses, have been less interesting, but also cheaper, options.

With the appearance on the market of multiple-DOFs hand prostheses, myoelectric
control has gained interest since it represents essentially the only non-surgical
way of finely controlling such artifacts. Before 1975, the common
control schema was based upon the identification of active muscle remnants in the
amputee's stump and the coding of two, at most three levels of activity of each
remnant to prosthesis commands \cite{bottomley65,childress69}. It is from the mid-Seventies
on that probabilistic methods as well as pattern recognition techniques began to be used,
including Bayes methods, artificial neural networks and nearest neighbors. More recently
EMG has been subject to accurate feature extraction and classification,
e.g. \cite{englehart01,dunlop,smagt06} in order to discriminate hand postures, not necessarily
used for grasping (for instance, the act of wrist flexion) and not only in the field
of prosthetics, e.g. \cite{fukuda,yokoi}.

Some authors of this very work \cite{2008.ICRA,2008.BioCyb} have as well verified that the problem, from
the very point of view of machine learning, is easy, and that a Support Vector Machine
\cite{BGV92}, a standard multi-layer perceptron and an incremental local-approximation
method such as LWPR \cite{lwpr} obtain similar results when applied to the plain EMG
signal as extracted by Otto Bock's commercially standard electrodes, the Myobock
(see \url{www.ottobockus.com}).

\subsubsection{Artificial / prosthetic hands}

In \cite{zecca02} (Section 3) an interesting comparison among artificial hands, including
the human hand, is shown; remarkable mechanical hands such as the Utah/MIT hand, the
Stanford/JPL hand, the DLR hand and the Robonaut Hand are compared (see also the
references in the paper for more details); but their high dexterity makes them non
usable as prostheses, either because they are too large, too heavy or too power-demanding.
The DLR-II hand, appeared in 2004 \cite{ButFisGre2004}, is probably the most sophisticated
mechanical hand at the time of writing, but it is far from being usable as a prosthesis.

As of 2004, the only choice was the Otto Bock / SUVA hand, which only had $2$ DOFs and
no opposable thumb. More recently, Touch Bionics's i-LIMB (\url{www.touchbionics.com}) has appeared, which has
5 independent DOFs and a passive opposable thumb. Its myoelectric control system enables
the selection among $5$ hand postures using $2$ electrodes (information collected from the
website); nevertheless, the impression is that most of the grasping effort is achieved through
the underactuation of the hand, as it happens with the CyberHand \cite{cyberhand}.

It seems reasonable to say that surface myoelectric control will still be the standard for the
years to come, even when the dexterity of commercial prostheses increases. Given the
stunning progress of machine learning, it is no wonder that as many as $127$ EMG
electrodes are used simultaneously in Targeted
Muscle Reinnervation \cite{kuiken06}, probably the most spectacular use of myoelectric control
so far, although it is out of the scope of this work since it involves surgery.

\subsubsection{Clinical applications}

Interestingly, as far as we know, little work has been done on amputees --- actually,
all aforementioned works are experimentally checked on healthy subjects. In the framework of
Project SmartHand (\url{www.elmat.lth.se/~smarthand}) \cite{sebelius} is the only
test on amputees, as far as we know. Sebelius had $5$ below-elbow amputees, plus one subject
with congenital malformation of the forearm, perform up to
$10$ hand postures (not grasping-related) while recording a 16-channel EMG signal. His
results indicate that a rather high recognition rate is obtained by a neural network
aided by a sort of decision tree, although not uniformly for all subjects; and that
the age of amputation seems not to influence the subjects' ability.
