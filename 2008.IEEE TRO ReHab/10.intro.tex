\section{Introduction}
\label{sec:intro}

As of today, the most common way of feed-forward controlling an active
hand prosthesis is via forearm surface electromyography (EMG),
a technique by which motor unit activation potentials are read from the
amputee's stump skin and then used to command the prosthesis \cite{deluca97}.
The clinical success of EMG is motivated by its non-invasiveness and
relatively low cost \cite{englehart06}.
Still, even though multifingered hand prostheses have appeared on the market
(for example Touch Bionics's i-LIMB, see \url{www.touchbionics.com}),
it seems that EMG-based control is not keeping the pace, being limited to a
few hand postures or a simple proportional estimate of force.
Moreover, this control is quite non-natural, meaning that the amputee must
learn to associate muscle remnants actions to unrelated postures of the
prosthesis, e.g., phantom wrist flexion and hand closing.

It would be desirable, for a dexterous prosthesis, to let the amputee command a
grasp posture and force just by performing the corresponding action with the
phantom limb, resulting in a movement or force elicited by "desiring"
it. Also, a way to finely modulate the force involved in a grasp is paramount
in daily life activities, for example to hold an egg without breaking it or to
grasp a hammer without letting it slip. Lastly, the system should be able to
work in real-time and to adapt continually to the changing nature of the EMG
signal.

Actually, recent scientific literature indicates that plain, old EMG signals
can be used in a better way in prosthetics, by improving the signal processing
method, essentially by switching to machine learning. Excellent results have
been shown on healthy subjects, but surprisingly, as far as we know, there is
so far just one study on amputees \cite{sebelius}. Willing to explore the issue
deeper and further, we have devised a new experiment. Three below-elbow traumatic
amputees, two of which were operated decades ago, were asked to perform
grasp postures with their phantom limb at various speeds and forces, while their
EMG activity was recorded using $5$ commercially available electrodes, positioned
without clinical help, and the desired force was
estimated using a force sensor in three different ways.
The EMG and force signals were used to train a standard machine learning system
(a Support Vector Machine) in order to check how well phantom limb postures
could be discriminated and the required force approximated. This paper is
a report of the experiment, whose outcome is positive.

We now quickly revise related work highlighting the novel contributions of this
work, then we describe the experiment and results and discuss them.

\subsection{Related work}

\subsubsection{Myoelectric control}

Myoelectric control, i.e., feed-forward control of prostheses using surface EMG,
has been the most popular form of (externally powered) control by amputees since
the Sixties, mostly due to its relatively low cost and non-invasiveness. It has
provided a simple yet effective way to control single degree-of-freedom (DOF) hand
prostheses such as OttoBock's SensorHand for decades, also allowing for a simple
force estimation, the so-called proportional control. Before 1975, the common
control schema was based upon the identification of active muscle remnants in the
amputee's stump and the coding of two, at most three levels of activity of each
remnant to prosthesis commands, e.g. \cite{bottomley65,childress69}. From the mid-Seventies
on probabilistic methods and pattern recognition techniques began to be used,
including Bayes methods, artificial neural networks and nearest neighbors. 
A thorough and well-organised survey and historical account of myoelectric
control for hand prostheses can be found in \cite{englehart06}.

With the appearance on the market of multiple-DOFs hand prostheses, the myoelectric
signal gains even more interest, since it represents the only non-surgical
way of finely controlling such artifacts. \cite{englehart01,dunlop,smagt06} are
examples of better and better posture discrimination using EMG, always on healthy
subjects. The issue is important also outside the field of prosthetics, e.g., in
robotic control and teleoperation \cite{fukuda,yokoi}.

In \cite{2008.ICRA,2008.BioCyb} it has been verified on one single healthy subject
that that the problem, from the very point of view of machine learning,
is easy, and that a Support Vector Machine \cite{BGV92},
a standard multi-layer perceptron and an incremental local-approximation
method such as LWPR \cite{lwpr} obtain similar results when applied to the plain EMG
signal as extracted by Otto Bock's commercially standard electrodes, the Myobock
(see \url{www.ottobockus.com}).

It seems then reasonable to say that surface myoelectric control will still be the
standard for the years to come, even when the dexterity of commercial prostheses
increases. Single-DOF control via EMG is being studied \cite{englehart08} and looks
extremely promising. As many as $127$ EMG electrodes are used simultaneously in Targeted
Muscle Reinnervation \cite{kuiken06}, probably the most spectacular use of myoelectric control
so far, although it is out of the scope of this work since it involves surgery.

\subsubsection{Artificial / prosthetic hands}

In Section $3$ of \cite{zecca02} a full comparison among artificial hands, including
the human hand, is shown; remarkable mechanical hands such as the Utah/MIT hand, the
Stanford/JPL hand, the DLR hand and the Robonaut Hand are compared (see also the
references in the paper for more details); but in those cases high dexterity means they
are not usable as prostheses, either because they are too large, too heavy or too power-demanding.
As well the DLR-II hand, appeared in 2004 \cite{ButFisGre2004}, probably the most sophisticated
mechanical hand at the time of writing, is far from being usable as a prosthesis.

As of 2004, a widely employed prosthesis was the Otto Bock / SUVA hand, which only had $2$ DOFs and
no opposable thumb. More recently, Touch Bionics's i-LIMB (\url{www.touchbionics.com}) has appeared, which has
5 independent DOFs and a passive opposable thumb. Its myoelectric control system enables
the selection among $5$ hand postures using $2$ electrodes (information collected from the
website); nevertheless, the impression is that most of the grasping effort is achieved through
the underactuation of the hand, as it happens with the CyberHand \cite{cyberhand}, still at
the prototype level, and the DLR Prosthetic Hand \cite{Hua2006}. In all cases, force control
is still underdeveloped, notwithstanding its relevance, and notwithstanding the ongoing work
on impedance control \cite{ott} in robotics. There is a trend of tranferring technology from
mechatronics to prosthetics and we are probably going to see more and more dexterous
prostheses built and marketed in the upcoming years; still, control
lags behind, even without considering the crucial issue of sensorial feedback,
dealt with in, e.g., \cite{cipriani} but completely open as yet.

\subsubsection{Clinical applications}

All aforementioned works and systems have been tested on healthy subjects. At the time
of writing and to the best of our knowledge, \cite{sebelius} is the only test on amputees, done
in the framework of Project SmartHand (\url{www.elmat.lth.se/~smarthand}).
Sebelius had $5$ below-elbow amputees, plus one subject
with congenital malformation of the forearm, perform up to
$10$ hand postures (not grasping-related) while recording a 16-channel EMG signal. His
results indicate that a rather high recognition rate is obtained by a neural network
aided by a sort of decision tree, although not uniformly for all subjects; and that
the age of amputation seems not to influence the subjects' ability.

\subsubsection{Original contributions}

To the best of our knowledge, this is the first time a position / force estimation
is attempted on amputees somehow systematically. We apply a standard machine learning
system to the EMG of three amputees both to recognise the grasping posture and
to approximate the required force. Some considerations about the subjects' diversity
as patients (age, age of operation, type of amputation) are drawn, also with respect to
its impact on the results. Just $5$ commercially available EMG electrodes are
employed, and their signals are fed as-they-are to the machine at $25$Hz.
Classification / regression are performed sample-by-sample. The encouraging results,
and the simplicity of the tools used, indicates that surface EMG still has to be
fully exploited.
