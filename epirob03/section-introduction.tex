
%%\citep{AB86} \citet{AD94}
%%\begin{figure}[!ht] \begin{center}

\section{Introduction}

The presence of a body changes the nature of perception.  
%
The body is a source of constraint on interpretation, opportunities for
experimentation, and a medium for communication.
%
Hands in particular are very revealing, since they interact directly
and flexibly with objects.  In this paper, we demonstrate several
methods for simplifying visual processing by being attentive
to hands, either of humans or robots.


Segmentation is an enabler of lots of stuff.

There are two platforms.

Protocol-based.


\ifnote

(OLD) To imitate an action, it must first be perceived correctly.  
%
The actor must be located and identified, and then tracked 
throughout the action.  The same needs to be done for
any objects involved.
%
Perception is an interpretive process that endeavors to
capture the essentials of an agent's context and
discard incidental, irrelevant details.
%
This differentiation between essential and incidental
is crucial to imitation, but simply represents an 
extreme of the kind of decision the perceptual system
faces at every instant.  Hence it seems wise to address
imitation within a wider context of a fully integrated 
perceptual system.

%
%This context-dependent process is key to imitation.
%
%We believe that any theory of imitation must be 
%solidly grounded in
%a theory of perception.
%
%[Okay, I'm just spouting nonsense here so far].

We show how a robot can acquire the appropriate percepts for imitation
through a shared activity, one that can be done either by the robot or
a human.  This activity is simply striking objects and watching how
they move.

\fi


\begin{figure}[tb]
\centerline{
\includegraphics[height=3.5cm]{cckemp-block}
\includegraphics[height=3.5cm]{cog5}}
\caption[Another example]{ 
%
%
The platforms.
%
}
%%\label{fig:resegment-example2}
\end{figure}

