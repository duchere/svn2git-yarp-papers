
%%\citep{AB86} \citet{AD94}
%%\begin{figure}[!ht] \begin{center}

\section{Introduction}

The presence of a body changes the nature of perception.  
%
The body is a source of constraint on interpretation, opportunities for
experimentation, and a medium for communication.
%
Hands in particular are very revealing, since they interact directly
and flexibly with objects.  In this paper, we demonstrate several
methods for simplifying visual processing by being attentive
to hands, either of humans or robots.
%
Our first argument is that in a wide range of situations, there are
many cues that can be used to make object segmentation an easy task.
Object segmentation or figure/ground separation is a long-standing
problem in computer vision, due to the fundamental ambiguities
involved in interpreting the 2D projection of a 3D world.  No matter
how good a passive system is at segmentation, there will be times when
only an active approach will work, since visual appearance can be
arbitrarily deceptive.

Our second argument is that object segmentation is truly a key ability
worth getting excited about.  We support this by demonstrating that
when segmentation is available, several other important vision
problems can be dealt with successfully -- object recognition,
object localization, edge detection, etc.

Here are the three situations we look at:

%%\begin{itemize} \pflist

%%\item %
      {\em (i)} 
      Active segmentation for a robot viewing its own actions.
      A robot arm
      probes an area, seeking to trigger object motion and then
      identify the boundaries of an object through its motion.

%%\item %
      {\em (ii)} 
      Active segmentation for a wearable system viewing its wearer's
      actions.  The system monitors human action, issues requests, and
      uses active sensing to detect grasped objects held up to view.

%%\item %
      {\em (iii)} 
      Protocol-based segmentation for a robot viewing a human's actions.
      Segmentation is achieved by detecting and interpreting 
      natural human showing behavior such as finger tapping, arm
      waving, or object shaking.


%%\end{itemize}




\ifnote

(OLD) To imitate an action, it must first be perceived correctly.  
%
The actor must be located and identified, and then tracked 
throughout the action.  The same needs to be done for
any objects involved.
%
Perception is an interpretive process that endeavors to
capture the essentials of an agent's context and
discard incidental, irrelevant details.
%
This differentiation between essential and incidental
is crucial to imitation, but simply represents an 
extreme of the kind of decision the perceptual system
faces at every instant.  Hence it seems wise to address
imitation within a wider context of a fully integrated 
perceptual system.

%
%This context-dependent process is key to imitation.
%
%We believe that any theory of imitation must be 
%solidly grounded in
%a theory of perception.
%
%[Okay, I'm just spouting nonsense here so far].

We show how a robot can acquire the appropriate percepts for imitation
through a shared activity, one that can be done either by the robot or
a human.  This activity is simply striking objects and watching how
they move.

\fi


\begin{figure}[tb]
\centerline{
\includegraphics[height=3.5cm]{cckemp-block}
\includegraphics[height=3.5cm]{cog5}}
\caption[Another example]{ 
%
%
The platforms.
%
}
%%\label{fig:resegment-example2}
\end{figure}

