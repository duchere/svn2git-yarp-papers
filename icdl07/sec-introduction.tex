\section{Introduction}
Growing evidence in developmental psychology shows the importance 
of motor activity for cognitive development in humans. In particular 
it is through manipulation that infants gain direct access to objects 
and discover properties that otherwise would remain hidden. 
This concerns for example properties like weight, shape, texture and 
softness that are, if not impossible, at least extremly hard to perceive 
by using visual information alone. In adults information originating 
from motor activity and direct contact with the environment, helps 
perception \cite{klatzky87hand}; during development
the physical interaction with the environment might
provide infants with natural invariances useful for learning.
Interestingly, motor and perceptual development seem to follow a similar 
timeline as if new achievements in the motor system were enacting
the development of new perceptual skills 
\cite{bushnell93motor}.

Research in developmental robotics has demonstrated the importance of
motor activity (in particular manipulation) for visual and haptic 
perception \cite{fitzpatrick07shared}. One of the 
limitations of these approaches is that controlling the interaction between
the robot and the environment is difficult especially when precise models are
not available. Experiments with robots have thus focused on situations
in which the interaction with the objects is quite simple. For these reasons
to investigate perceptual development in robots we first need to 
address the problem of motor development and improve the way robots interact
with the world.

In this context we focus here on reaching, which is an ability required for 
grasping. If vision of the hand is available this problem is easily solved 
by employing the visual Jacobian of the arm (visual servoing, see 
\cite{hutchinson96tutorial} for a review). The 
eye-to-hand Jacobian transformation is a function of the arm and head joints 
and includes knowledge of the camera parameters; its estimation is in practice 
a hard task.
The advantage of this approach is that even inaccurate estimation of the Jacobian
allows reducing the error of the task to zero. But the visual servoing approach 
is not necessarely the best solution. On one hand delays in the control loop pose 
limitations on the speed of the arm, while on the other hand it
requires that sight of the hand is continuously available during the task. 
Reaching can also be performed open-loop by directly relating the joint of 
the head with that of the arm 
\cite{blackburn94learning,metta99developmental}. 
The drawback of this approach is that errors in the task cannot be reduced 
arbitrarily as in the visual servoing case.

Results in developmental psychology suggest that both solutions might be
adopted by the brain. Clifton et al. \cite{clifton93isvisually} 
tested whether infants require vision of their hand when reaching; they 
found that infants' ability to touch (and grasp) objects is indipendent 
of whether sight of the hand is available or not. On the other hand 
other experiments \cite{ashmead93visual}
show that later on in development there is an increase of visual guidance 
in reaching. Together these results suggest the hypothesis that there 
are two ``distinct'' reaching mechanism: one that relies on ``proprioceptive'' 
information alone and one that uses ``visual feedback'' to compensate 
for errors in the visual domain.
In this paper we integrate the two solutions. We use the open loop
controller to bring the hand close to the target. The closed loop controller 
is activated e to the target. The closed loop controller 
is activated when visua visual feedback from the hand is available. In this
way the robot manages to reduce the error of the task to arbitrarily small 
amounts. The problem of redundancy in solved in the first case by posing 
additional constrains on the task. Finally we describe the procedure by 
which the robot learns all the transformations required by the controllers 
(the open-loop mapping and the arm visual Jacobian).