\section{Introduction}
Growing evidence in developmental psychology shows the importance 
of motor activity for cognitive development in humans. In particular 
it is through manipulation that infants gain direct access to objects 
and discover properties that otherwise would remain hidden. 
Properties like weight, shape, texture and softness that 
are, if not impossible, at least extremly hard to perceive by using
visual information alone. In these cases haptic information originating 
from motor activity and direct contact with the environment, is 
obviously helping perception in adults (cite Klatzky); during development
the physical interaction between the infant and the environment might
provide natural invariances useful for learning.
Interestingly motor and perceptual development seem to follow a similar 
timeline as if new achievements in the motor system were enacting new
the development of new perceptual skills (cite bushnell and boudread 
and needham, sticky mitten).

Research in developmental robotics has shown how motor activity (in particular 
manipulation) can indeed be useful for either visual or haptic perception (cite 
SharedChallenges paper as a review and consider adding more). One of the 
limitations of these approaches is that controlling the interaction between
the robot and the environment is difficult especially in absence of precise
models of either one. Experiments with robots have thus focused on situations
in which the interaction with the objects is quite simple. For these reasons
we believe that to investigate perceptual development in robots we first need to 
address the problem of motor development and improve the way robots interact
with the environment.

In this paper we focus on reaching, which is an ability required for grasping. 
If vision of the hand is available this problem is easily solved if the 
visual Jacobian of the arm is available (visual servoing). Jacobian of the 
eye-to-hand transformation is a function of the arm and head joints and includes 
knowledge of the camera parameters; its estimation in practice is a hard task. 
The advantage of this approach is that even inaccurate estimation of the Jacobian
allows reducing the error of the task to zero. But the visual servoing approach 
is not necessarely the best solution. On one hand delays in the control loop pose 
limitations on the speed of the arm, while on the other hand the approach requires 
that sight of the hand is continuously available during the task. Reaching can 
also be performed open-loop by directly relating the joint of the head with that of
the arm (cite motor-motor mapping). The drawback of this approach is that errors 
in the task cannot be reduced arbitrarily as in the visual servoing case.

Results in developmental psyhcology suggest that both solutions might be
adopted by the brain. Clifton et al. (cite) tested whether infants require 
vision of their hand when reaching; they found that infants' ability to touch 
(and grasp) objects is indipendent of whether sight of the hand is available
or not. Other experiments (cite Ashmed) show on the other hand that later on 
in development there is an increase of visual guidance in reaching. These
results suggest the hypothesis that there are two ``distinct'' reaching 
mechanism: one that relays on ``proprioceptive'' information alone and 
one that uses ``visual feedback'' to compensate for errors in the visual domain.

In this paper we integrate the two solutions together. We use an open loop
controller to bring the hand close to the target. A closed loop controller 
employs visual information from the hand as soon as this is available. In this
way the robot manages to reduce the error of the task to arbitrarily small 
amounts. The problem of redundancy in solved in the first case by posing additional 
constrains on the task. Finally we describe the procedure by which the robot learns all the 
transformations required by the controllers (the open-loop mapping and the arm 
visual Jacobian).


\subsection{A sub introduction}
\subsubsection{A sub sub introduction}
a subsection.

