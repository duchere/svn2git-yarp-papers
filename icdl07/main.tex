%% This skeleton file requires IEEEtran.cls version 1.6 or later.
%%
\documentclass[conference,letterpaper]{IEEEtran}
% If the IEEEtran.cls has not been installed into the LaTeX system files,
% manually specify the path to it:
% \documentclass[conference]{../sty/IEEEtran}
\IEEEoverridecommandlockouts
\overrideIEEEmargins

% some very useful LaTeX packages include:

\usepackage{cite}      % Written by Donald Arseneau
                        % V1.6 and later of IEEEtran pre-defines the format
                        % of the cite.sty package \cite{} output to follow
                        % that of IEEE. Loading the cite package will
                        % result in citation numbers being automatically
                        % sorted and properly "ranged". i.e.,
                        % [1], [9], [2], [7], [5], [6]
                        % (without using cite.sty)
                        % will become:
                        % [1], [2], [5]--[7], [9] (using cite.sty)
                        % cite.sty's \cite will automatically add leading
                        % space, if needed. Use cite.sty's noadjust option
                        % (cite.sty V3.8 and later) if you want to turn this
                        % off. cite.sty is already installed on most LaTeX
                        % systems. The latest version can be obtained at:
                        % http://www.ctan.org/tex-archive/macros/latex/contrib
%/supported/cite/

\usepackage[dvips]{graphicx}  % Written by David Carlisle and Sebastian Rahtz
                        % Required if you want graphics, photos, etc.
                        % graphicx.sty is already installed on most LaTeX
                        % systems. The latest version and documentation can
                        % be obtained at:
                        % http://www.ctan.org/tex-archive/macros/latex/required/graphics/
                        % Another good source of documentation is "Using
                        % Imported Graphics in LaTeX2e" by Keith Reckdahl
                        % which can be found as esplatex.ps and epslatex.pdf
                        % at: http://www.ctan.org/tex-archive/info/

%\usepackage{amsmath}   % From the American Mathematical Society
                        % A popular package that provides many helpful commands
                        % for dealing with mathematics. Note that the AMSmath
                        % package sets \interdisplaylinepenalty to 10000 thus
                        % preventing page breaks from occurring within multiline
                        % equations. Use:
\usepackage{multirow}
\usepackage[left=0.71in,top=0.94in,right=0.71in,bottom=1.18in]{geometry}
\setlength{\columnsep}{0.24in}
% correct bad hyphenation here
%\hyphenation{op-tical net-works semi-conduc-tor IEEEtran}

\begin{document}
% paper title
\title{\huge Yet Another Paper}

% author names and affiliations
\author{\authorblockN{Aieie Brazov}
\authorblockA{
\textit{Aieie Institute of Technology}\\
\textit{Aieie Str.}\\
\textit{aieie@ait.it}\\}%
\and
\authorblockN{Tizio}
\authorblockA{
\textit{Tizio Institute of Technology}\\
\textit{via Morego}\\
\textit{tizio@tit.it}\\}}%

% make the title area
\maketitle
\begin{abstract}
In this paper we discuss the implementation of a precise reaching controller 
on a humanoid robot upper torso. The proposed solution is completely based on a learning 
strategy which does not rely on \emph{a priori} models of the arm 
and head kinematics. The only major simplification is represented by the 
assumption that a visual model of the hand is available (i.e. the robot 
can visually localize the hand). In fact, from a practical point of view, 
the problem of creating a visual model of the hand is a stand alone problem 
which falls outside the scope of this work.
\end{abstract}

% key words
\begin{keywords}
Development of perceptual and motor systems, machine learning, visual servoing, redundant systems.
\end{keywords}
%
\section{Introduction}
Put here the introduction.
%

\section{Previous Works}

[-] locating an observed object with respect to the robot requires the following data:
		[*] robot forward kinematic (analytical or estimated)
		[*] cameras calibration (self or grid calibration)
		[*] hand-eye calibration 
		
[-] visual servoing requires:
		[*] the manipulator jacobian dx_hand = J(q_arm) dq_arm
		[*] the interaction matrix ds = L(q_head, q_arm) dx_hand
\section{Conclusions}
In this paper we have described the implementation of a reaching
behavior that integrates together an open loop and a closed 
loop controller. The open loop controller
allows the robot to perform faster movements and does not require visual 
feedback from the hand. When sight of the hand is available the closed
loop controller allows for precise positioning of the hand in the 
image plane. 

We describe an explorative strategy by which the robot autonomously 
acquires the forward motor map and the visual Jacobian transformations. 
Among the other things this strategy 
allows the estimation of the eye-to-hand visual Jacobian of the robot. 
The estimation of the Jacobian is a well studied task for which several 
solutions have been proposed \cite{Hosoda94versatile,Mansard06jacobian,
Lapreste04efficient}. None of these works, however, addresses the 
problem of the redundancy of both the head and the arm. In the experiments 
reported here the estimation of the Jacobian is performed with good 
accuracy for a subset of the arm workspace and for 
\emph{different head postures}. We believe
this is an important contribution with respect to the state of the art.

We do not rely on any prior information about the 
kinematic structure of the robot. The only simplification was that we used 
a color marker to visually localize the hand of the robot. Our assumption
is that the hand localization/identification is a separate problem
that needs to be solved before learning reaching. Previous work
by the same and other authors have suggested procedures by which 
the robot could autonomously learn to solve this task 
(\cite{Natale05,edsinger06what}). It will be interesting to see
how these approaches can be integrated with the work described 
in this paper.

\section*{Acknowledgement}
% optional entry into table of contents (if used)
%\addcontentsline{toc}{section}{Acknowledgment}
The work presented in this paper
has been supported by the \textsc{RobotCub} project (IST-2004-004370), funded by the
European Commission through the Unit E5 ``Cognitive Systems''. Moreover, it has
been partially supported by \textsc{Neurobotics}, a European FP6 
project (IST-2003-511492) and \textsc{Contact} (NEST 5010).




%

% references section
%%% this must be IEEEtran
\bibliographystyle{IEEEtran}
\bibliography{main}

\end{document}




