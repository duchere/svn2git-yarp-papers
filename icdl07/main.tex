%% This skeleton file requires IEEEtran.cls version 1.6 or later.
%%
\documentclass[conference,letterpaper]{IEEEtran}
% If the IEEEtran.cls has not been installed into the LaTeX system files,
% manually specify the path to it:
% \documentclass[conference]{../sty/IEEEtran}
\IEEEoverridecommandlockouts
\overrideIEEEmargins

% some very useful LaTeX packages include:

\usepackage{cite}      % Written by Donald Arseneau
                        % V1.6 and later of IEEEtran pre-defines the format
                        % of the cite.sty package \cite{} output to follow
                        % that of IEEE. Loading the cite package will
                        % result in citation numbers being automatically
                        % sorted and properly "ranged". i.e.,
                        % [1], [9], [2], [7], [5], [6]
                        % (without using cite.sty)
                        % will become:
                        % [1], [2], [5]--[7], [9] (using cite.sty)
                        % cite.sty's \cite will automatically add leading
                        % space, if needed. Use cite.sty's noadjust option
                        % (cite.sty V3.8 and later) if you want to turn this
                        % off. cite.sty is already installed on most LaTeX
                        % systems. The latest version can be obtained at:
                        % http://www.ctan.org/tex-archive/macros/latex/contrib
%/supported/cite/

\usepackage[dvips]{graphicx}  % Written by David Carlisle and Sebastian Rahtz
                        % Required if you want graphics, photos, etc.
                        % graphicx.sty is already installed on most LaTeX
                        % systems. The latest version and documentation can
                        % be obtained at:
                        % http://www.ctan.org/tex-archive/macros/latex/required/graphics/
                        % Another good source of documentation is "Using
                        % Imported Graphics in LaTeX2e" by Keith Reckdahl
                        % which can be found as esplatex.ps and epslatex.pdf
                        % at: http://www.ctan.org/tex-archive/info/

%\usepackage{amsmath}   % From the American Mathematical Society
                        % A popular package that provides many helpful commands
                        % for dealing with mathematics. Note that the AMSmath
                        % package sets \interdisplaylinepenalty to 10000 thus
                        % preventing page breaks from occurring within multiline
                        % equations. Use:
\usepackage{multirow}
\usepackage[left=0.71in,top=0.94in,right=0.71in,bottom=1.18in]{geometry}
\setlength{\columnsep}{0.24in}
% correct bad hyphenation here
%\hyphenation{op-tical net-works semi-conduc-tor IEEEtran}

\begin{document}
% paper title
\title{\huge Yet Another Paper}

% author names and affiliations
\author{\authorblockN{Aieie Brazov}
\authorblockA{
\textit{Aieie Institute of Technology}\\
\textit{Aieie Str.}\\
\textit{aieie@ait.it}\\}%
\and
\authorblockN{Tizio}
\authorblockA{
\textit{Tizio Institute of Technology}\\
\textit{via Morego}\\
\textit{tizio@tit.it}\\}}%

% make the title area
\maketitle
\begin{abstract}
In this paper we discuss the implementation of a precise reaching controller 
on a humanoid robot upper torso. The proposed solution is completely based on a learning 
strategy which does not rely on \emph{a priori} models of the arm 
and head kinematics. The only major simplification is represented by the 
assumption that a visual model of the hand is available (i.e. the robot 
can visually localize the hand). In fact, from a practical point of view, 
the problem of creating a visual model of the hand is a stand alone problem 
which falls outside the scope of this work.
\end{abstract}

% key words
\begin{keywords}
Development of perceptual and motor systems, machine learning, visual servoing, redundant systems.
\end{keywords}
%
\section{Introduction}
Growing evidence in developmental psychology shows the importance 
of motor activity for cognitive development in humans \cite{gallese06mirror}. 
In particular it is through manipulation that infants gain direct access to objects 
and discover properties that otherwise would remain hidden. 
This concerns for example properties like weight, shape, texture and 
softness that are, if not impossible, at least extremely hard to perceive 
by using visual information alone. In adults information originating 
from motor activity and direct contact with the environment, supports 
perception \cite{klatzky87hand}; during development
the physical interaction with the environment provides infants 
with natural invariances that are useful occasions for learning.
Interestingly, motor and perceptual development seem to follow synchronous  
schedules as if new achievements in the motor system were promoting
the development of new perceptual skills \cite{bushnell93motor}.

Research in developmental robotics has demonstrated the importance of
motor activity (in particular manipulation) for visual and haptic 
perception \cite{fitzpatrick07shared}. One of the 
limitations of these approaches is that controlling the interaction between
the robot and the environment is difficult especially when precise models are
not available. Experiments with robots have thus focused on situations
in which the interaction with objects is relatively simple. For these reasons
the investigation of perceptual development in robots requires addressing 
the problem of motor development first and improving how robots interact
with the world.

In this context we focus on reaching, which is clear prerequisite for 
grasping. If vision of the hand and target were available then this problem 
would be easily solved by employing a visual servoing approach (see 
\cite{hutchinson96tutorial} for a review) which is typically based on 
the use of the visuo-motor Jacobian of the arm. The eye-to-hand Jacobian 
transformation is a function of the arm and head joints and includes knowledge 
of the camera parameters; its estimation is in practice a difficult task.
The fundamental advantage of this approach is that even an inaccurate estimation 
of the Jacobian allows reducing the error of the task to zero. But the visual 
servoing approach is not necessarily the best solution. On one hand delays in 
the control loop pose limitations on the speed of the arm, while on the other 
hand it requires that the hand and target are continuously visible for the 
duration of the movement. 
Reaching can also be performed open-loop by directly relating the joint angles of 
the head with those of the arm \cite{blackburn94learning,metta99developmental}. 
The drawback of this approach is that errors because of modeling inaccuracies and 
noise in the sensors, calculations, actuation, cannot be made arbitrarily small 
as in the visual servoing case.

Results in developmental psychology suggest that both solutions might be
adopted by the brain. Clifton et al. \cite{clifton93isvisually} 
tested whether infants require vision of their hand when reaching; they 
found that infants' ability to touch (and grasp) objects is independent 
of whether sight of the hand is available or not. On the other hand 
other experiments \cite{ashmead93visual} show that later on in development 
there is an increase of visual guidance in reaching. Together these results 
suggest the hypothesis that there are two ``distinct'' reaching mechanism: 
one that relies on ``proprioceptive'' information alone and one that uses 
``visual feedback'' to compensate for errors in the visual domain. Further,
there are studies that show the link of the control of the gaze in relation
to the precision of reaching \cite{flanders-daghestani-berthoz-1999}. 
In this paper we integrate the two modes of control with an approach based on
the hypothesis that the target object is fixated. We use the open loop
controller to bring the hand close to the target. The closed loop controller 
is activated when visual feedback from the hand is available. In practice,
we show that the error could be made arbitrarily small. The problem of 
redundancy in solved in the first case by imposing additional constraints to 
the task. Finally, we describe the procedure by which the robot learns all 
the transformations required by the controllers (the open-loop mapping and 
the arm visual Jacobian).
\section{Previous Works}

%[-] locating an observed object with respect to the robot requires the following data:
%		[*] robot forward kinematic (analytical or estimated)
%	[*] cameras calibration (self or grid calibration)
%[*] hand-eye calibration 

%[-] visual servoing requires:
%	[*] the manipulator jacobian dx_hand = J(q_arm) dq_arm
%[*] the interaction matrix ds = L(q_head, q_arm) dx_hand

In this section we briefly describe previous approaches to the problem of 
performing precise reaching with a robot arm. As stated above, the problem 
naturally splits down into two phases, open loop and closed loop, both well 
studied in literature.

The {\em open loop} phase requires a sensory-motor map encoding the relationship between hand visual localization and arm position. Following a classical procedure, this map can be practically decomposed into three parts: the robot forward kinematics (mapping the hand reference frame into a robot reference frame), the camera projective map (mapping the hand reference frame into a camera reference frame) and the hand/eye map (mapping the camera reference frame into the hand reference frame). Extensive literature has been produced to describe different calibration procedures for retrieving each of these basic maps. Suitable kinematic \cite{Hollerbach96calibration} and hand/eye \cite{Tsai88calibration} calibration procedures can be used to retrieve the forward kinematic and the hand/eye maps. Similarly, different algorithms and strategies have been proposed for cameras and stereo rigs calibration, which is a well known problem `per se', studied mainly in computer vision \cite{Soatto03vision} but with extensive application in robotics. 

Though the final result of these procedures can be extremely accurate, the standard calibration techniques require the robot to operate in an highly structured environment (typically represented by a calibrated grid or object) with a precisely calibrated hand pose sensor (typically a stereo rig), which is not desirable in certain applications. Therefore, alternative procedures have been proposed in order to relax some of the above assumptions. In \cite{AHE01} for example, an hand/eye calibration procedure which does not use any calibration object is proposed. Other approaches have introduced the possibility of performing a kinematic calibration without measuring the hand pose \cite{Bennett91calibration}, but only relying on proprioception and exploiting specific kinematic constraints (e.g. by keeping the hand fixed on the ground).

For certain applications the classical calibration procedures are not 
necessary. A simpler approach \cite{blackburn94learning} avoids the estimation 
of the three maps mentioned above by learning a single forward map. 
In this case
the map links the head joint position to the corresponding arm position that
keeps the hand in fixation. When the robot fixates the target reaching can be 
performed by inverting this map to retrieve the arm command which brings
the hand to the fixation point. Recently this approach has been successfully 
extended to redundant manipulators \cite{lopes06learning}, although in the 
case of a 2-dimensional visual space.

The {\em closed loop} phase requires knowledge of the Jacobian of the open loop map.
It can be derived analytically from 
mathematical differentiation of the function describing the forward map itself. 
Alternatively, some works have proposed techniques to directly estimate the Jacobian matrix
\cite{Hosoda94versatile,Mansard06jacobian} or its inverse 
\cite{Lapreste04efficient}.

In this paper we integrated together the 
open \cite{blackburn94learning,Mansard06jacobian} and closed 
\cite{Hosoda94versatile,lopes06learning} loop 
strategies, both performed in the 3-dimensional space.
We also propose a procedure to estimate 
the Jacobian of the manipulator in the case of a redundant head. All the 
transformations required to perform the task, are autonomously estimated by the robot 
 without relying on any \emph{a  without relying on any \emph{a priori} knowledge about the robot 
kinematic structure.

\section{Conclusions}
In this paper we have described the implementation of a reaching
behavior that integrates together an open loop and a closed 
loop controller. The open loop controller
allows the robot to perform faster movements and does not require visual 
feedback from the hand. When sight of the hand is available the closed
loop controller allows for precise positioning of the hand in the 
image plane. The procedure among the other things estimates the eye-to-hand
visual Jacobian of the robot. In this respect our method provides
similar results to the ones described in the literature \cite{}

We describe an explorative strategy by which the robot autonomously 
acquires the transformations required to control the hand to reach for a
visually identified target.

We do not rely on any prior information about the 
parameters of the robot. The only simplification was that we used 
a color mark to visual localize the hand of the robot. Our assumption
is that the hand localization/identification is a separate problem
that needs to be solved before learning reaching. Previous work
by the same and other authors have suggested procedures by which 
the robot could autonomously learn to solve this task 
(\cite{Natale05,edsinger06what}). It will be interesting to see
how these approaches can be integrated with the work described 
in this paper.

\section*{Acknowledgement}
% optional entry into table of contents (if used)
%\addcontentsline{toc}{section}{Acknowledgment}
The work presented in this paper
has been supported by the \textsc{RobotCub} project, funded by the
European Commission through Unit E5 ``Cognition''. Moreover, it has
been partially supported by \textsc{Neurobotics}, a European FP6 
project (IST-2003-511492).




%

\end{document}

% references section
\bibliographystyle{IEEETran} 
\bibliography{main}


