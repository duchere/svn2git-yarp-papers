\section{Previous Works}

%[-] locating an observed object with respect to the robot requires the following data:
%		[*] robot forward kinematic (analytical or estimated)
%	[*] cameras calibration (self or grid calibration)
%[*] hand-eye calibration 

%[-] visual servoing requires:
%	[*] the manipulator jacobian dx_hand = J(q_arm) dq_arm
%[*] the interaction matrix ds = L(q_head, q_arm) dx_hand

In this section we briefly describe previous approaches to the problem of performing precise reaching with a robot arm. As it was previously said, precision is usually achieved with a visual servoing control scheme which allows accurate hand positioning. Therefore, the problem naturally splits down into two phases, open loop and closed loop, both well studied in literature.

The {\em open loop} phase requires a sensory-motor map encoding the relationship between hand visual localization and arm position. Following a classical decomposition, this map can be practically decomposed into three main parts: the robot forward kinematics (mapping the hand reference frame into a robot reference frame), the camera projective map (mapping the hand reference frame into a camera reference frame) and the hand/eye map (mapping the camera reference frame into the hand reference frame). Extensive literature has been produced (and is still being produced) to describe different calibration procedures for retrieving each of these basic maps. Suitable kinematic \cite{Hollerbach96calibration} and hand/eye \cite{Tsai88calibration} calibration procedures can be used to retrieve the forward kinematic and the hand/eye maps. Similarly, different algorithms and strategies have been proposed for cameras and stereo rigs calibration, which is a well known problem `per se', studied mainly in computer vision \cite{Soatto03vision} but with extensive application in robotics. 

Though the final result of these procedures can be extremely accurate, the standard calibration techniques require the robot to operate in an highly structured environment (typically represented by a calibrated grid or object) with a precisely calibrated hand pose sensor (typically a stereo rig), which is not desirable in certain applications. Therefore, alternative procedures have been proposed, capable of relaxing the strongest assumptions. In \cite{AHE01} for example, an hand/eye calibration procedure which does not use any calibration object is proposed. Other approaches introduced the possibility of performing a kinematic calibration without measuring the hand pose \cite{Bennett91calibration}, but only relying on proprioception and exploiting the kinematic constraints obtained by keeping the hand fixed on the ground. 


%However, in certain applications, the accuracy in representing the intermediate maps is not an issue if the overall control %strategy is capable of achieving the desired reaching accuracy. 


%This problem is usually complicated by the head mobility  



%\begin{itemize}
%\item[-] {\em Open loop.} Suppose that the robot sees an object. 

%\item[-] {\em Closed loop}: 
%\end{itemize}

%first the hand should be brought into the visual field, second a suitable jacobian should be used to guarantee the stability of %the visual servoing scheme.   


%the problem is usually decomposed into two parts, 


%During the last decade, the problem of ``visuo-motor maps learning'' has been studied with a growing interest becoming recently %a valuable alternative to more classical approaches to the problem of 



%based on complex and delicate calibration processes. 