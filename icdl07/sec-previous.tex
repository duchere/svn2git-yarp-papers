\section{Previous Works}

%[-] locating an observed object with respect to the robot requires the following data:
%		[*] robot forward kinematic (analytical or estimated)
%	[*] cameras calibration (self or grid calibration)
%[*] hand-eye calibration 

%[-] visual servoing requires:
%	[*] the manipulator jacobian dx_hand = J(q_arm) dq_arm
%[*] the interaction matrix ds = L(q_head, q_arm) dx_hand

In this section we briefly describe previous approaches to the problem of 
learning how to perform precise reaching with a robot arm. As stated above, the problem 
naturally splits down into two phases, open loop and closed loop, both well 
studied in literature.

The {\em open loop} phase requires a sensory-motor map encoding the relationship between hand visual localization and arm position. Following a classical procedure, this map can be practically decomposed into three parts: the robot forward kinematics (mapping the hand reference frame into a robot reference frame), the camera projective map (mapping the hand reference frame into a camera reference frame) and the hand/eye map (mapping the camera reference frame into the hand reference frame). Extensive literature has been produced to describe different calibration procedures for retrieving each of these basic maps. Suitable kinematic \cite{Hollerbach96calibration} and hand/eye \cite{Tsai88calibration} calibration procedures can be used to retrieve the forward kinematic and the hand/eye maps. Similarly, different algorithms and strategies have been proposed for cameras and stereo rigs calibration, which is a well known problem `per se', studied mainly in computer vision \cite{Soatto03vision} but with extensive application in robotics. 

Though the final result of these procedures can be extremely accurate, the standard calibration techniques require the robot to operate in an highly structured environment (typically represented by a calibrated grid or object) with a precisely calibrated hand pose sensor (typically a stereo rig), which is not desirable in certain applications. Therefore, alternative procedures have been proposed in order to relax some of the above assumptions. In \cite{AHE01} for example, an hand/eye calibration procedure which does not use any calibration object is proposed. Other approaches have introduced the possibility of performing a kinematic calibration without measuring the hand pose \cite{Bennett91calibration}, but only relying on proprioception and exploiting specific kinematic constraints (e.g. by keeping the hand fixed on the ground).

For certain applications the classical calibration procedures are not 
necessary. A simpler approach \cite{blackburn94learning} avoids the estimation 
of the three maps mentioned above by learning a single forward map. 
In this case
the map links the head joint position to the corresponding arm position that
keeps the hand in fixation. When the robot fixates the target reaching can be 
performed by inverting this map to retrieve the arm command which brings
the hand to the fixation point. Recently this approach has been successfully 
extended to redundant manipulators \cite{lopes06learning}, although in the 
case of a 2-dimensional visual space.

The {\em closed loop} phase requires knowledge of the Jacobian of the open loop map.
It can be derived analytically from 
mathematical differentiation of the function describing the forward map itself. 
Alternatively, some works have proposed techniques to directly estimate the Jacobian matrix
\cite{Hosoda94versatile,Mansard06jacobian} or its inverse 
\cite{Lapreste04efficient}.

In this paper we integrated together the 
open \cite{blackburn94learning,Mansard06jacobian} and closed 
\cite{Hosoda94versatile,lopes06learning} loop 
strategies, both performed in the 3-dimensional space.
We also propose a procedure to estimate 
the Jacobian of the manipulator in the case of a redundant head and arm. All the 
transformations required to perform the task, are autonomously estimated by the robot 
 without relying on any \emph{a priori} knowledge about the robot 
kinematic structure.
