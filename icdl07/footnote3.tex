More formally, let $\bar {\mathbf q}_{head} = \begin{bmatrix} \bar 
{\mathbf q}_{eyes} & \bar {\mathbf q}_{neck} \end{bmatrix}$ 
be a fixating position, i.e. $f_{head}(\bar {\mathbf q}_{head}) = 0$. 
Moreover, let's assume that the given head configuration is non-singular, 
i.e. the Jacobian matrix: $$\frac{\partial f_{head}}{\partial \mathbf 
q_{head}}(\bar {\mathbf q}_{head}),$$ has full row rank. Then, by the 
implicit function theorem there exists a function 
$\mathbf q_{neck}(\cdot): \mathbb R^3 \longrightarrow \mathbb R^3$ 
(locally defined around $\bar {\mathbf q}_{eyes}$) such that 
$f_{head}({\mathbf q}_{eyes}, {\mathbf q}_{neck} ({\mathbf q}_{eyes}) ) = 0$ 
for all the configuration ${\mathbf q}_{eyes}$ belonging to the neighborhood 
of $\bar {\mathbf q}_{eyes}$.