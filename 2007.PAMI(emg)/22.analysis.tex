The gathered data was analysed both for classification and
regression. \emph{Classification} is the process by which one wants to
assign a label to each sample in the input space, whereas in
\emph{regression} the target is a real-valued function of the values
of the input samples. Throughout this Section, we will assume that a
set of $l$ points in the input space is available, for which the
target (label or force value) is known; this set will be denoted by
$\{\xx_i,y_i\}_{i=1}^l$ and called \emph{training set}. As well, for
each experiment, a separate set of points, for which the targets are
not used for training, is assumed to be available, and this will be
called the \emph{testing set}. In general, the performance of a
machine is evaluated by training it on a training set and then testing
it on a testing set, possibly employing standard measures of the
generalisation error, such as cross-validation.

Taking into account the considerations of the previous Section, we set
the input space to be $\RR^{10}$, that is, one coordinate for each EMG
electrode; therefore, $\xx_i \in \RR^{10}, i=1,\ldots,l$. In the case
of classification, each category representing a grasping type would be
represented as an integer value, that is, $y_i \in \{0,\ldots,4\}
\subset \NN, i=1,\ldots,l$. In the case of regression, the force value
would be directly encoded as a real number, that is, $y_i \in \RR,
i=1,\ldots,l$. Before any analysis, all samples were normalised, as is
customary, by subtracting the mean values and dividing by the standard
deviation, for each input space dimension. No filtering whatsoever was
applied to the input signals, in order to have a more realistic,
delay-free result.

\subsubsection{Neural Networks}

Artificial Neural Networks (NNs or ANNs for short; see, e.g.,
\cite{bishop} for a comprehensive introduction) are probably the most
popular machine learning algorithm nowadays available for both
classification and regression. An ANN is a directed graph in which,
for every node, the weighted sum of the input values is evaluated;
this sum is then used as the argument of an \emph{activation function}
to determine the output of the node. The nodes fed the input values to
the network form the \emph{input layer}, and the nodes whose output is
taken as the output of the network are called \emph{output
layer}. Besides this, in general, an ANN can further have an arbitrary
number of nodes organised in \emph{hidden layers}, gifted with an
arbitrary connection topology.

An ANN is initialised with random weights; then, for every sample in
the training set, the network output is evaluated and its error with
respect to the target is considered. In order to reduce the average
error on the training set, a minimisation algorithm is then employed
to change the weights of the network, until the desired precision is
reached. Normally, a separate testing set is used to evaluate the
generalisation error (average error on the testing set) and to check
that it does not increase (overfitting) as the training error
decreases. If the generalisation error has been kept small, the
network will then be able to \emph{predict} the targets of the testing
samples with a reasonable accuracy.

For our experiment we strived to keep the ANN as simple as possible.
We then chose a feed-forward neural network with $10$ units in the
input layer; one hidden layer with $10$ units with hyperbolic tangent
(sigmoidal) activation function; $5$ units in the output layer for
classification, each unit representing one category, and one unit in
the output layer for regression, the unit representing the target
force value. The network was trained using the gradient descent
learning function; backpropagation was enforced via the the
quasi-Newton algorithm (for classification) and the
Levenberg-Marquardt algorithm (for regression). The Mean Square Error
(MSE) was used as a measure of performance. The training phase was
stopped arbitrarily after $30$ epochs. For each experiment, we
repeated the training phase $10$ times and then gathered the best
model found, in order to overcome the well-known problem of local
minima. No measure of generalisation error was taken into account.

The network was implemented in Matlab, Windows version $7.1.0.246$
(R14) Service Pack 3, running on a bi-processor $1.8$GHz machine with
1GB on-board memory; we used the Matlab Neural Network Toolbox,
version $5.0.1$ (R2006b).

\subsubsection{Support Vector Machines}

Support Vector Machines (SVMs; see, e.g.,
\cite{BGV92,Burges98,Cristianini00}) are a machine learning method
able to determine the best candidate function for a classification or
regression problem, drawn from a functional space induced by the
choice of a binary function between points in the input space,
$K(\xx_1,\xx_2)$, with $\xx_1, \xx_2 \in \RR^{10}$ in this case. $K$
is called \emph{kernel}. In the most general setting, the function
found is

\begin{equation} \label{eqn:sol}
  f(\xx) = \sum_{i=1}^l \alpha_i y_i K(\xx,\xx_i) + b
\end{equation}

\noindent where $b \in \RR$, whereas the $\alpha_i \in \RR$s are
Lagrangian coefficients obtained by solving a minimisation problem
whose cost functional is guaranteed to be convex. Because of this,
SVMs do not suffer from the problem of local minima; but their
training time is cubic in the number of samples in the training set,
as opposed to ANNs, for which it is \textbf{[[PATRICK, can we say
something here? I guess anyway it is not strictly necessary.]]}.

In order to overcome this problem, which would have made our
experiment unfeasible, we have decided to use a \emph{uniformisation}
strategy on the training sets, before training the machines. The idea
is that, in a real-life set-up such as ours, there can be many input
samples located in the very same region of the input space, with very
similar target values. One obvious case is that of label $0$,
indicating no ongoing grasping: it is intuitively expected that a
large number of samples will be taken in that region of the input
space, since the subject will be in the $0$ condition for a longer
time than all other labels.

Since all functions involved in the experiment are due to human
motion, we can assume that they are continuous and, probably,
derivable up to any arbitrary order. Therefore it makes no sense for
an approach such as SVMs to sample the input space in a non-uniform
way such as that described above. The uniformisation procedure
consists of removing, from a training set, those samples which are too
close to each other, according to a suitable notion of inter-sample
distance.

In order to take into account the different variances of the EMG
electrode values, we have decided to adopt Mahalanobis's distance as
the inter-sample distance. Let $\xx_1, \xx_2 \in \RR^{10}$; then the
Mahalanobis distance between $\xx_1$ and $\xx_2$ is defined as
follows:

$$ MD(\xx_1,\xx_2) = \sqrt{(\xx_1-\xx_2)^T \Sigma^{-1} (\xx_1-\xx_2)} $$

\noindent where $\Sigma$ is the $10$x$10$ covariance matrix, evaluated
on the training set. $MD(\xx_1,\xx_2)$ is a distance in which each
summand is weighted inversely with respect to the variance of the
samples along that dimension of the input space: it is therefore a
measure of distance independent of the variance of the single
electrodes. Notice that if $\Sigma$ is replaced by the identity
matrix, $MD(\xx_1,\xx_2)$ is reduced to the usual notion of Euclidean
distance.

Since checking the inter-sample distance obviously takes a quadratic
time with respect to the number of samples, which was unfeasible, we
adopted an approximated method which was able to remove most, but not
all, samples with an insufficient Mahalanobis distance from any other
sample. After a few initial experiments we set the threshold distance
at $1$. We also decided to employ the Gaussian kernel. All our
experiments with SVMs were then performed on uniformised training
sets, using $5$-fold cross-validation and grid search to find the
optimal values of the standard Gaussian kernel hyperparameters, $C$
and $\sigma$.

On the other hand, notice that no \emph{testing} set was uniformised,
since it would probably be unfeasible to apply the same procedure in
an on-line setting. Notice, further, that applying uniformisation
resulted in training sets which were considerably smaller than the
original ones, up to about $100$ times smaller.

Lastly, we employed a well-known freely available SVM package,
\emph{libsvm} v2.83 \cite{ChangL01}, in the Matlab wrapped flavour.

\subsubsection{Locally Weighted Projection Regression}

Locally Weighted Projection Regression (LWPR --- see \cite{lwpr}) is a
regression method especially targeted for high-dimensional
spaces with redundant and irrelevant input dimensions. It employs
locally linear models, each of which performs univariate regressions
in selected directions in the input space. It has a computational
complexity that is linear in the number of inputs, but due to its
incrementality it can take long time to train (as we verified it was
the case). Therefore we used it for regression only, and trained it
with the uniformisation procedure.

We used the latest stable C version of LWPR, kindly made available by
Stefan Klanke, wrapped in a Matlab command interface. We chose to use
the Radial Basis Function kernel and meta-learning, and then
performed $5$-fold cross-validation and found the initial values of
the distance metric for receptive fields by grid search.
