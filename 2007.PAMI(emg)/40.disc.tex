As far as regression is concerned, the results presented above are, to
the best of our knowledge, totally novel. Surprinsingly, regression
from the forearm surcafe EMG signal to the force applied by the hand
had never been attempted before. Given the good performance obtained
by our models, we claim that the relationship between the EMG signal
and the force has been captured by the models, under variable
conditions of muscle fatigue (within one session) and electrode
displacement (within sessions belonging to different groups).

One remarkable point, to be further investigated, is the
\emph{correct} sampling of the EMG signal. The signal presents, as
already said, a wide variance due to a number of factors. Our work
shows clearly that, if the relevant data only is gathered, a
remarkable accuracy, both in classification and regression, can be
achieved. So far, this has been done using $(a)$ the uniformisation
procedure to \emph{reduce} the training sets, that is, to eliminate
form the training sets irrelevant data; and $(b)$ a careful
a-posteriori selection of the models which had performed best on a
thorough cross-sessions analysis.

Both procedures, actually, can be seen as the need to sample the
relevant portion of the input space in a uniform way. One can think of
an online version of the uniformisation procedure, in which new
samples deemed irrelevant (according to a suitable inter-sample
distance measure, such as the Mahalanobis distance) are rejected and
never used for training. The same procedure would, when presented with
samples sufficiently distant from the current training set, decide to
actually use them. This could possibly lead to an incrementally
growing training set, which would eventually reach a plateau, when all
relevant portions of the input space have been explored by the user.

The current setup and analysis can be improved in a number of ways,
among which:

\begin{enumerate}

  \item a better way of gathering the force at the finger joints, not
    only via the force/torque sensor which is unable to detect the
    onset of finger movements, but also using torque sensors applied
    at the finger joints directly;

  \item subsampling, pre- and post-filtering, which would probably
    lead to a better conditioned signal in input, and a more suitable
    control signal in output;

  \item sampling in different positions of the arm (not only the
    forearm), and/or while the arm is moving, as it happens in the
    real life of a disabled person wearing the prosthesis. This would
    heavily involve a good sampling strategy.

\end{enumerate}

As far as driving a prosthesis is concerned, \textbf{[[PATRICK will
you say something here???]}
