Our experimental results, performed on a large data set of about
$153000$ samples, clearly show that the Online Uniformisation
procedure can be used uniformly to obtain dramatically smaller
training sets with no \emph{qualitative} loss of information; in other
words, as more of the input space is sampled, OU keeps the training
set up-to-date and small. OU sets will result in small (and therefore,
fast) and accurate models of the sought-for EMG-to-hand
map. Remarkably, OU works fine for both grasp classification and force
regression, and with all three different machine learning
approaches. Moreover, it is extremely simple, being nothing more than
an online check of Euclidean distance in the input space. This check
is done so far by considering the new sample's distance from all
samples in the current training set, and therefore could become
unfeasibly heavy as the set grows; but the same check can be done in
constant time using an algorithmic optimisation, such a hash table.

The choice of the minimum inter-sample distance $d$ is obviously
crucial and depends on the required accuracy in classification and/or
regression; but as we have seen, as $d$ is increased, the machine's
performance degrades only linearly, whereas the training sets become
polynomially smaller.

We believe this is the first step toward the real application of
machine learning to an EMG-driven adaptive, dexterous hand
prosthesis. Let us consider the problems outlined in Subsection
\ref{subsubsec:electrodes}: in this paper we have solved problems
$3$ and $4$. Now, since OU lets us obtain good accuracy with extremely
small training sets, it is not too far-fetched to say that the
solution of problem $2$ is at hand --- in principle, the changing arm
posture can be taken into account simply by sampling more of the input
space.

All in all, problem $1$ is, really, of lesser interest, since one
patient only is supposed to ever wear a prosthesis. On the other hand,
a futher problem is that of training the system upon an amputee. First
of all, the patient must still have a good deal of muscular and
nervous plasticity in her arm stump; then, a smart way of collecting
training data must be devised. (A simple idea is that of gathering EMG
data from the patient's stump and grasping/force data from her healthy
hand, while instructing her to imagine doing the same actions with
both hands.)

As far as force regression is concerned, the results presented above
are, to the best of our knowledge, totally novel. Surprinsingly,
regression from the forearm surcafe EMG signal to the force applied by
the hand had never been attempted before. Given the good performance
obtained by our models, we claim that the relationship between the EMG
signal and the force has been captured by the models, under variable
conditions of muscle fatigue (within one session) and electrode
displacement (within sessions belonging to different groups).

All in all, in this paper we have presented a machine learning
approach to joint classification of grasping and regression on the
applied force, using forearm surface electromyography. The approach is
totally non-invasive, easy to set up and use and it can be applied
from scratch with no previous knowledge of the problem. The Online
Uniformisation procedure can be used to incrementally build a training
set which will result in small and accurate models of the problem.

Our experiments, carried out using a Support Vector Machine with
Gaussian kernel, a Neural Network with sigmoidal activation function
and Locally Weighted Projection Regression, indicate that the approach
achieves, using a training set of about $1800$ samples on a total of
$153000$ (for $d=021$), an average accuracy of around $90\%$ in
classification of grasp types and a normalised root MSE of $7.89\%$ in
prediciton of the force applied during the grasp. Of the tested
approaches, SVM is marginally better than the others, especially when
larger training sets are used. The OU procedure is able to find as
small a training set as $77.4$ samples on average (out of $153000$),
which will still result in a SVM having a remarkable NRMSE of
$12.12\%$.

Future work can be divided in three steps: firstly, problem $2$ of 
\ref{subsubsec:electrodes} must be taken into account by gathering
more samples, possibly with a more sophisticated EMG/force sensing
setup. Secondly, once a stable model has been built, the actual
position/force control of a mechanical hand must be realised. And
thirdly, in the medium-to-long term, we plan to implant the hand on a
disabled person with extensive experiments to test the real usability
of the proposed approach.
