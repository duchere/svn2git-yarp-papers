One of the biggest problems when dealing with advanced hand prostheses
is, nowadays, that of control by the patient. Assume we can build a
mechanical hand which is light and dexterous enough to be used by an
amputee with beneficial effects during everyday life; how is then the
patient supposed to command the prostheses what to do to the desired
degree of accuracy and velocity? Right now, commercially available
prostheses are not very dexterous, i.e., have too few degrees of
freedom, or cannot be finely controlled; but things are changing
rapidly, and therefore the need for good control of these mechanical
hands arises.

This paper deals with advanced hand prostheses control via surface
electromyography (EMG), probably the simplest and most accurate means
to do it. Building upon recent results, we show that machine learning,
together with a simple downsampling algorithm, can be effectively used
to control on-line, in real time, a highly dexterous robotic hand. The
system determines the type of grasp a human subject is willing to use,
and the required amount of force involved, with a remarkable degree of
accuracy.

This paves the way for a better control of advanced hand prostheses
and opens up a scenario in which life conditions of amputees can be
effectively improved with respect to the state of the art.
