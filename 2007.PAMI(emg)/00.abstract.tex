This paper is in the framework of \emph{active hand prostheses}, i.e.,
mechanical hands which are light enough to be implanted on disabled
persons; more precisely, it deals with the issue of \emph{controlling}
an AHP. We hereby describe an experiment conducted on a healthy
subject, aimed at establishing whether forearm surface
electromyography (EMG) can be used to reconstruct, to a resonable
degree of accuracy, the hand posture and force involved in
grasping.

Several machine learning methods were compared on a data set gathered
over two days; moreover, a simple sparsification strategy was devised,
which enabled the methods to work also in real time in an on-line
setting. The system was tested on a real mechanical hand (not a
prosthetic one so far, though). The results of the experiment confirm
that the problem can be solved with a remarkable precision. This paves
the way for a better control of AHPs and opens up a scenario in which
life conditions of amputees can be effectively improved with respect
to the state of the art.
