\section{Discussion} \label{Discussion}

With this experiment we aimed in first place at finding out whether a strategy based on some kind of force field model of a flying object could be involved in the prediction required to intercept it, and consequently on what sort of information it could be based.
To begin with the first question posed in the introduction, the proposed experiment suggests that dynamic information may play a role in prediction for interception. The trajectories subjects had to deal with were different in each trial, because of different initial ball positions and velocities, in all experimental conditions. In one of them a unique force field drove the ball during all the trials, in the other protocol, instead, different fields characterized each trial. The trajectories were however always parabolic and presented therefore always the same shape. The significant difference between performances in the two conditions, then, should be related the different dynamical characteristics (different kind of force fields) of the tasks. According to our view in the fixed field protocol subjects adjusted an internal model of the forces acting on the ball and adopted this model to score the measured better performance in prediction .
This kind of process is in a certain sense related to that used by subjects in the experiment described by Mussa-Ivaldi, who learned to correctly perform reaching movements immersed in a force fields modeling the force perturbing their arm and counterbalancing it. This hypothesis is then already an attempt to give an affirmative answer to the first question: subject relies also on models in object interception prediction.
This idea is also supported by the \textit{Transfer Effect} observed when the fixed force field is suddenly turned off, replaced by a vertical force that changes randomly its orientation at each trial. In particular subjects show no adaptation phases after the error peak and return immediately in trend with their previous performance, as they could simply switch on a model already learned.
Furthermore it seems that this process of dynamic modeling doesn't depend on a particular value of the force field. In the studied experiment, in fact, errors made in the fixed force field case, both when it was upward and when it was downward oriented were always lower then those performed when there was no possibility to model the force, because of its variability.

Given that the central nervous system effectively relies on dynamic cues to exploit this kind of prediction, it remains to understand how this information is collected. A further result of our experiments could give suggestions on this topic.
We observed that subjects who had to cope with balls moving in a force field similar to gravity (a vertical force downward oriented) performed always better than those who dealt with an identical but symmetrical force (upward oriented). People were asked to intercept the target by shifting up and down a small virtual paddle by actually moving their hand up and down. They had therefore to perform an action as at the same time they were experiencing the gravitational force. We assume that what affected the results of those who intercepted balls in the ``anti-gravitational'' field was the opposition between the force felt and that seen on the screen.  Afterward the ``pseudo - gravitational'' case could be also easier due to our everyday experience with gravity. Indeed the same results could be also explained by theories proposed in \citep{McIntyre}, \citep{Lacquaniti}, \citep{Lacquaniti_internal} and \citep{McIntyre2}. In order to prove our theory we have to design a new experimental procedure which will be part of our future work.

Practically we are postulating that performing a real action to intercept the target and thus actively involving the motor system are important to the prediction itself. In particular according to our view the motor system has a role in managing dynamics information, so that models used in interception could be different depending on whether the motor system is activated or not.
It has been stated that we can understand better what happens by mapping other people's movements to our own, and that this is obtained through the subthreshold activation of our muscles. Therefore we can suggest that the complete activation of our motor system could be the mean with which we can obtain the contextual dynamical structure.
% 
%This hypothesis seems to explain some results obtained during an experiment proposed by McIntyre and colleagues \citep{Lacquaniti}. They showed that if a subject is asked to punch a ball which appears at the end of a screen in concurrence with a virtual ball falling along the screen, he times his movements as if he predicted a gravitational acceleration, whatever the real motion is. However when the subject in the same condition must instead ''explode'' the virtual ball by pressing a button and not punching the ball, the timing is tuned to a constant velocity motion, irrespective of the real motion of the ball. According to our theory this difference can be accounted by the fact that the motor system in the second case is not involved as much as in the first. When it is activated to produce an action, it enters also the process to realize a prediction and force the system to take into account also force information, on line or previously estimated (in this particular case gravitational force.)

Our work doesn't aim at describing how this process is realized, but just at suggesting an investigation direction.
This model of prediction could also be seen as an effort to propose a unifying approach, in which prediction of inanimated objects interception is based on similar, though different, principles as prediction of human movements. 


% Prediction would find actually a place among all the other skills in which perception and action are strictly linked. in prediction. This skill, at least in object interception, could be actually somehow inserted in a particular ''mirror'' framework, in which action and perception become strictly linked. At the same time 
%One future development of the experiment proposed could be reproducing the same task but in force fields different from gravity, for instance with a manipolandum. It would be then possible to understand better how the motor information helps in this kind of prediction tasks.


%In this way, by comparing the performance of the subjects when the motor feedback is or is not available, it would be possible to detect if humans can infer a force field only on the  basis of the visual input and whether they take advantage of
%%which advantage rise from the possibility to feel the force.
%feeling the force to improve their prediction or not.
%It would also be interesting to prepare a test in which the visual and the motor information are in contrast, to detect which of the two sensory input play a leading role in the prediction task and if the interception of the flying object is yet possible.


%This could be an evidence of the role of the motor system in prediction, as conveyor of dynamical information to the brain.