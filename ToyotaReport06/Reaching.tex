\section{Reaching}
\label{sec:reaching}

We describe two approaches to solve the reaching task. A first method 
uses the forward mapping between the arm joint space and the position 
of the hand represented in the head reference frame \begin{bmatrix} 
\theta_y & \theta_p & \alpha_v^d\end{bmatrix}^\top \in \mathbb R^3. The 
second method controls the speeds of the arm so to minimize the position 
of the hand in the image plane with respect to the target (usually the 
center of the image plane).

\subsection{Open Loop Reaching}
This approach uses a mapping which links 
the position of the head in joint space to the arm joint configuration
which brings the hand to the fixation point:

\begin{equation} \label{Eq:reaching1}
\mathbf q_{head}=\tilde{f}_{ol}}(\mathbf q_{arm}), \qquad \tilde{f}_{ol} : \mathbb R^7 \longrightarrow \mathbb R^4.
\end{equation}

where $\mathbf q_{arm}$ and $\mathbf q_{head}$ are the configurations of the arm and head 
respectively. In the condition when the robot fixates the hand, $\mathbf q_{head}$
implicitly codes its spatial location. we use the subscript \emph{ol} to stress the
fact that this function allows the robot to reach for a target in open-loop, 
meaning that once the hand has started to move no further corrections can be 
made to the trajectory.

Given the kinematics of the robot $\mathbf q_{head}$ is made up of the following quantities:

\begin{eqnarray*}
\mathbf q_{head}=
\begin{bmatrix} \theta_y & \theta_p & \theta_r & \alpha_v^d & \alpha_v^c & \alpha_t^c \end{bmatrix}^\top \in \mathbb R^6.
\end{eqnarray*}

During tracking $\theta_r$ is maintained stationary to 0, while the head controller 
poses additional contraints on the head joints; in particular we know from section 
\ref{Sec:TrackerController} that the controller minimizes $\alpha_t^c$ and
$\alpha^c_v$ (see equation \ref{Eq:HeadEyeControl}) so that they asymptotically
converge to $\alpha_t^c \rightarrow 0$ and $\alpha_v^c \rightarrow 0$. In this 
case we have:

\begin{eqnarray*}
\mathbf q_{head}=
\begin{bmatrix} \theta_y & \theta_p & 0 & \alpha_v^d & 0 & 0 \end{bmatrix}^\top \in \mathbb R^6.
\end{eqnarray*}

Let us define:

\begin{eqnarray*}
\mathbf x_{ol_hand}=
\begin{bmatrix} \theta_y & \theta_p & \alpha_v^d\end{bmatrix}^\top \in \mathbb R^3.
\end{eqnarray*}

which, in our case, uniquely codes the position of the hand in a subset of the head joint space; we now rewrite \ref{Eq:reaching1} to get:

\begin{equation} \label{Eq:reaching2}
\mathbf q_{arm}=f_{ol}}(\mathbf x_{ol_hand}), \qquad f_{ol} : \mathbb R^3 \longrightarrow \mathbb R^4.
\end{equation}

in which the dimensionality of the mapping has been greatly reduced.

\subsection{Closed Loop Reaching}














