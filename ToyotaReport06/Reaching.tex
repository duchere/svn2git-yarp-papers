\section{Reaching}
\label{sec:reaching}

We describe two approaches to solve the reaching task. A first method 
uses the mapping between the head and arm joints when the
robot is fixating the hand (i.e. when the latter is at the center of
both cameras). The second method controls the speeds of the arm 
so to minimize the position of the hand in the image plane with respect
to the target (usually the center of the image plane).

In the first case the mapping can be represented as a function which links 
the position of the head in joint space to the arm joint configuration
which brings the hand to the fixation point:


\begin{equation} \label{Eq:reaching1}
\mathbf q_{arm}=\tilde{f}_{OL}}(\mathbf q_{head}), \qquad f_{CL} : \mathbb R^7 \longrightarrow \mathbb R^4.
\end{equation}

where $q_{arm}$ and $q_{head}$ are the configurations of the arm and head 
respectively. In the condition when the robot fixates the hand, ${q_head}$
implicitly codes its spatial location. we use the subscript OL to stress the
fact that this function allows the robot to reach for a target in open-loop, 
meaning that once the hand has started to move no other correction can be 
made to the trajectory.

Given the kinematics of the robot
${q_head}$ is made up of the following quantities:

\begin{eqnarray*}
\mathbf q_{head}=
\begin{bmatrix} \theta_y & \theta_p & \theta_r \alpha_v^d & \alpha_v^c & \alpha_t^c \end{bmatrix}^\top \in \mathbb R^6.
\end{eqnarray*}

During tracking ${\theta_r}$ is maintained stationary to 0, while the head controller 
poses additional contraints on the head joints; in particular we know from section 
\ref{Sec:TrackerController} that the controller minimizes ${\alfa_t^c}$ and
${\alfa^c_v}$ (see equation \ref{Eq:HeadEyeControl}) so that they asymptotically
convergece to ${\alfa_t^c \rightarrow 0}$ and ${\alfa_v^c \rightarrow 0}$. In this 
case we have:

\begin{eqnarray*}
\mathbf q_{head}=
\begin{bmatrix} \theta_y & \theta_p & 0 \alpha_v^d & 0 & 0 \end{bmatrix}^\top \in \mathbb R^6.
\end{eqnarray*}

Let us define:

\begin{eqnarray*}
\mathbf q_{target}=
\begin{bmatrix} \theta_y \theta_p & \alpha_v^d\end{bmatrix}^\top \in \mathbb R^3.
\end{eqnarray*}

which codes the position of the target the head is currently fixating at. In the
assumption that the robot is fixating the hand we can re-write equation 
\eq{Eq:reaching1} to get:

\begin{equation} \label{Eq:reaching2}
\mathbf q_{arm}=\tilde{f}_{OL}}(\mathbf q_{target}), \qquad f_{OL} : \mathbb R^3 \longrightarrow \mathbb R^4.
\end{equation}

in which the dimensionality of the mapping has been greatily reduced.













